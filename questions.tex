
\documentclass[12pt]{article}
\usepackage{xeCJK}
\usepackage{fontspec}
\usepackage{amsmath}
\usepackage{amssymb}
\usepackage{graphicx}
\usepackage{geometry}
\usepackage{titlesec}
\usepackage{enumitem}
\usepackage{caption}

\setCJKmainfont{SimSun}
\setmainfont{Times New Roman}
\setmathfont{Latin Modern Math}

\geometry{margin=1in}
\graphicspath{{images/}}

\title{{\Huge Physics Questions}}
\date{}

\begin{document}
\maketitle
\setlength{\parskip}{0.5em}


\noindent
\textbf{Q1.} A ball decelerates uniformly from +28.0m/s  to +14.0m/s  in 0.002s , then accelerates back to +20.0m/s in 0.003s .
What is the total displacement during contact?



\textbf{A.} 0.042m \\
\textbf{B.} 0.056m \\
\textbf{C.} 0.070m \\
\textbf{D.} 0.084m \\

\textbf{Answer:} E \\
\textbf{Explanation:} Stage 1:
[IMAGE:0]
. Stage 2:
[IMAGE:1]
. Total
[IMAGE:2]
.

\hrule
\vspace{1em}


\noindent
\textbf{Q2.} A ball(v=12.0m/s) compresses a racket string by 0.02m  before rebounding at 10.0m/s.
What is the peak deceleration?



\textbf{A.} -1100m/s
2 \\
\textbf{B.} -2200m/s
2 \\
\textbf{C.} -3600m/s
2 \\
\textbf{D.} -4400m/s
2 \\

\textbf{Answer:} C \\
\textbf{Explanation:} Energy loss implies non-constant force, but assuming average deceleration:
[IMAGE:0]

\hrule
\vspace{1em}


\noindent
\textbf{Q3.} A topspin ball slows horizontally from 30.0m/s  to 24.0m/s  while gaining 2.0m/s downward due to spin. Contact time is 0.006s.
What is the net acceleration?



\textbf{A.} 800m/s
2 \\
\textbf{B.} 1050m/s
2 \\
\textbf{C.} 2000m/s
2 \\
\textbf{D.} 2500m/s
2 \\

\textbf{Answer:} B \\
\textbf{Explanation:} [IMAGE:0]

\hrule
\vspace{1em}


\noindent
\textbf{Q4.} A tennis ball (v=25.0m/s) penetrates a net, slowing to 5.0m/s  over 0.010m.
What is the deceleration magnitude?



\textbf{A.} 30000m/s
2 \\
\textbf{B.} 25000m/s
2 \\
\textbf{C.} 20000m/s
2 \\
\textbf{D.} 15000m/s
2 \\

\textbf{Answer:} A \\
\textbf{Explanation:} [IMAGE:0]

\hrule
\vspace{1em}


\noindent
\textbf{Q5.} A tennis ball approaches at 22.0 m/s horizontally.
A racket strikes the ball with a constant force at an upward and backward angle, resulting in a horizontal speed of 17.0 m/s and a vertical speed of 12.0 m/s.
If the contact time is 0.004 s, what is the magnitude of the net acceleration?



\textbf{A.} 3250m/s
2 \\
\textbf{B.} 3600m/s
2 \\
\textbf{C.} 5000m/s
2 \\
\textbf{D.} 6100m/s
2 \\

\textbf{Answer:} A \\
\textbf{Explanation:} [IMAGE:0]
Net
[IMAGE:1]
Acceleration
[IMAGE:2]
.
(Adjust options to match calculations.)

\hrule
\vspace{1em}


\noindent
\textbf{Q6.} A tennis ball strikes the court surface at 18.0m/s at a
[IMAGE:0]
angle. The bounce reverses the vertical velocity component and reduces the horizontal speed by 20%. The contact time is 0.005s.
What is the magnitude of the average acceleration during impact?
[IMAGE:1]



\textbf{A.} 2400m/s
2 \\
\textbf{B.} 3600m/s
2 \\
\textbf{C.} 3700m/s
2 \\
\textbf{D.} 3800m/s
2 \\

\textbf{Answer:} B \\
\textbf{Explanation:} Vertical velocity changes from
[IMAGE:0]
to
[IMAGE:1]
(
[IMAGE:2]
), while horizontal velocity changes from
[IMAGE:3]
. Total
[IMAGE:4]
Acceleration
[IMAGE:5]

\hrule
\vspace{1em}


\noindent
\textbf{Q7.} A tennis ball travelling at 12.0m/s
is hit by a racket. As a result of the impact, the ball returns back along its original path having undergone a change in velocity of 36.0m/s
. The acceleration of the ball whilst in contact with the racket is constant with magnitude
[IMAGE:0]
.
What is the total distance travelled by the ball whilst in contact with the racket?



\textbf{A.} 0.00cm \\
\textbf{B.} 4.80cm \\
\textbf{C.} 7.50cm \\
\textbf{D.} 15.0cm \\

\textbf{Answer:} C \\
\textbf{Explanation:} the initial velocity is
[IMAGE:0]
; the terminal one is
[IMAGE:1]
, the process is not symmetrical in time;
the first half is
[IMAGE:2]
the second half is
[IMAGE:3]
Thus, the answer is
[IMAGE:4]

\hrule
\vspace{1em}


\noindent
\textbf{Q8.} A tennis ball travelling at 10.0m/s
is hit by a racket. As a result of the impact, the ball returns back along its original path having undergone a change in velocity of 30.0m/s
. The acceleration of the ball whilst in contact with the racket is constant with magnitude
[IMAGE:0]
.
What is the total distance travelled by the ball whilst in contact with the racket?



\textbf{A.} 0.00cm \\
\textbf{B.} 5.00cm \\
\textbf{C.} 9.60cm \\
\textbf{D.} 14.4cm \\

\textbf{Answer:} B \\
\textbf{Explanation:} the initial velocity is 10.0m/s ; the terminal one is -20.0m/s , the process is not symmetrical in time; the first half is 10.0*10.0/6000/2=0.01m=1cm; the second half is 20.0*20.0/6000/2=0.04m=4cm. Thus, the answer is 5cm.

\hrule
\vspace{1em}


\noindent
\textbf{Q9.} A tennis ball travelling at 10.0m/s
is hit by a racket. As a result of the impact, the ball returns back along its original path having undergone a change in velocity of 20.0m/s
. The acceleration of the ball whilst in contact with the racket is constant with magnitude
[IMAGE:0]
.
What is the total distance travelled by the ball whilst in contact with the racket?



\textbf{A.} 2.00cm \\
\textbf{B.} 4.80cm \\
\textbf{C.} 9.60cm \\
\textbf{D.} 14.4cm \\

\textbf{Answer:} A \\
\textbf{Explanation:} the initial velocity is 10.0m/s ; the terminal one is -10.0m/s , the process is symmetrical in time; the first half is 10.0*10.0/5000/2=0.01m=1cm. Thus, the answer is 2cm.

\hrule
\vspace{1em}


\noindent
\textbf{Q10.} A tennis ball travelling at 12.0m/s
is hit by a racket. As a result of the impact, the ball returns back along its original path having undergone a change in velocity of 24.0m/s
. The acceleration of the ball whilst in contact with the racket is constant with magnitude
[IMAGE:0]
.
What is the total distance travelled by the ball whilst in contact with the racket?



\textbf{A.} 2.4cm \\
\textbf{B.} 4.8cm \\
\textbf{C.} 9.6cm \\
\textbf{D.} 14.4cm \\

\textbf{Answer:} A \\
\textbf{Explanation:} the initial velocity is 12.0m/s ; the terminal one is -12.0m/s , the process is symmetrical in time; the first half is 12.0*12.0/6000/2=0.012m=1.20cm. Thus, the answer is 2.40cm .

\hrule
\vspace{1em}


\noindent
\textbf{Q11.} Liquid fuel causes effective mass to oscillate as m(t) = m₀ - kt + εsin(ωt) with constant thrust.
Acceleration behavior?



\textbf{A.} [IMAGE:0] \\
\textbf{B.} [IMAGE:1] \\
\textbf{C.} [IMAGE:2] \\
\textbf{D.} [IMAGE:3] \\

\textbf{Answer:} G \\
\textbf{Explanation:} a(t) = F/[m(t)] shows oscillation can affect the acceleration which depends on the parameter settings of the mass equation.

\hrule
\vspace{1em}


\noindent
\textbf{Q12.} During atmospheric ascent, rocket reduces thrust proportional to altitude (
[IMAGE:0]
) while losing mass at constant rate.
Check the acceleration profile?



\textbf{A.} Always increases \\
\textbf{B.} Reaches maximum then decreases \\
\textbf{C.} Sinusoidal \\
\textbf{D.} Piecewise linear \\

\textbf{Answer:} B \\
\textbf{Explanation:} Analysis Process:
Thrust Variation:
[IMAGE:0]
(Linear decrease with altitude)
Mass Variation:
[IMAGE:1]
(Constant mass depletion rate)
Acceleration Expression:
[IMAGE:2]
[IMAGE:3]
Initial Phase: At low altitudes (
[IMAGE:4]
), thrust reduction is negligible. Dominant effect: Rapid mass decrease \to  acceleration increases.
Later Phase: As
[IMAGE:5]
approaches
[IMAGE:6]
, thrust decays sharply. Thrust reduction outweighs mass loss \to  acceleration decreases.
Resulting Profile: Unimodal curve (initial rise \to  peak \to  decline) . Peak occurs when thrust and mass effects balance

\hrule
\vspace{1em}


\noindent
\textbf{Q13.} Rocket's exhaust velocity increases linearly with time (
[IMAGE:0]
) while mass flow rate is constant as
[IMAGE:1]
. The initial mass of rocket is
[IMAGE:2]
.
[IMAGE:3]
is a constant.
Check the acceleration behavior?



\textbf{A.} [IMAGE:0] \\
\textbf{B.} [IMAGE:1] \\
\textbf{C.} [IMAGE:2] \\
\textbf{D.} [IMAGE:3] \\

\textbf{Answer:} B \\
\textbf{Explanation:} Thrust formula:
[IMAGE:0]
(momentum theorem)
Given conditions:
[IMAGE:1]
(constant mass flow rate)
[IMAGE:2]
(exhaust velocity increases linearly with time)
Therefore, thrust:
[IMAGE:3]
(linear increase)
Rocket mass:
[IMAGE:4]
(linear decrease)
Acceleration:
[IMAGE:5]
When
[IMAGE:6]
, the acceleration exhibits precisely linear growth.

\hrule
\vspace{1em}


\noindent
\textbf{Q14.} A rokect with mass
. For first half of fuel (total mass is
[IMAGE:0]
): burns at rate R with thrust F. For second half: burns at 2R with thrust 1.5F.
What describes acceleration?



\textbf{A.} Constant throughout \\
\textbf{B.} Jumps up at midpoint \\
\textbf{C.} Jumps down at midpoint \\
\textbf{D.} Always increasing \\

\textbf{Answer:} B \\
\textbf{Explanation:} First phase:
[IMAGE:0]
with final value
[IMAGE:1]
Second phase:
[IMAGE:2]
, whose numerator is suddently increasing and denominator is decreasing faster.

\hrule
\vspace{1em}


\noindent
\textbf{Q15.} The ratio of mass of two solid spheres is 9:4, the ratio of density of these spheres is 2:3. Find the ratio of the radius of the first sphere and the diameter of the second sphere.



\textbf{A.} 3:1 \\
\textbf{B.} 2:3 \\
\textbf{C.} 3:2 \\
\textbf{D.} 3:4 \\

\textbf{Answer:} D \\
\textbf{Explanation:} The volume ratio is 27:8, the radius ratio is 3 :2. Therefore, the ratio of the radius of the first sphere and the diameter of the second sphere is 3:4 (aha, other option are all misleading term.)

\hrule
\vspace{1em}


\noindent
\textbf{Q16.} A rocket (mass is
[IMAGE:0]
) adjusts its thrust to always equal 10% of its instantaneous fuel mass (F = 0.1M). Fuel burns at constant rate(
[IMAGE:1]
, where C is a constant, and the inital mass of fuel is M0).
How does acceleration behave?



\textbf{A.} Constant at 0.1
[IMAGE:0] \\
\textbf{B.} Increases \\
\textbf{C.} Decreases \\
\textbf{D.} Sinusoidal variation \\

\textbf{Answer:} C \\
\textbf{Explanation:} Based on acceleration-force equation, it has
initial state:
[IMAGE:0]
operating state:
[IMAGE:1]
it can be proved that
[IMAGE:2]

\hrule
\vspace{1em}


\noindent
\textbf{Q17.} A pendulum bob of mass
[IMAGE:0]
swings from a height
[IMAGE:1]
to the lowest point of its arc. If the length of the pendulum is
[IMAGE:2]
, what is the speed of the bob at the lowest point?



\textbf{A.} [IMAGE:0] \\
\textbf{B.} [IMAGE:1] \\
\textbf{C.} [IMAGE:2] \\
\textbf{D.} [IMAGE:3] \\

\textbf{Answer:} A \\
\textbf{Explanation:} Using energy conservation is easier than using kinematics equations :
[IMAGE:0]
;
[IMAGE:1]
.

\hrule
\vspace{1em}


\noindent
\textbf{Q18.} The ratio of mass of two solid spheres is 9:4, the ratio of density of these spheres is 2:3. Find the ratio of the radius of these two solid spheres.



\textbf{A.} 4:9 \\
\textbf{B.} 2:3 \\
\textbf{C.} 3:2 \\
\textbf{D.} 32:81 \\

\textbf{Answer:} C \\
\textbf{Explanation:} The volume ratio is 27:8, the radius ratio is therefore: 3 :2.

\hrule
\vspace{1em}


\noindent
\textbf{Q19.} The ratio of radius of first solid spheres and diameter of second solid spheres is 3:4, the ratio of density of these spheres is 2:3. Find the ratio of the mass of these two solid spheres.



\textbf{A.} 9:4 \\
\textbf{B.} 27:16 \\
\textbf{C.} 27:32 \\
\textbf{D.} 27:648 \\

\textbf{Answer:} A \\
\textbf{Explanation:} The volume ratio is 81:8, the mass ratio is therefore: 9 :4. The option D is a misleading term (mix up radius and diameter).

\hrule
\vspace{1em}


\noindent
\textbf{Q20.} A rocket adjusts its thrust to always equal 10% of its instantaneous total mass (F = 0.1m). Fuel burns at constant rate.
How does acceleration behave?



\textbf{A.} Constant at 0.1
[IMAGE:0] \\
\textbf{B.} Increases linearly \\
\textbf{C.} Decreases exponentially \\
\textbf{D.} Proportional to 1/m \\

\textbf{Answer:} A \\
\textbf{Explanation:} The answer is A.
a = F/m = 0.1m/m = 0.1 m/s² always.

\hrule
\vspace{1em}


\noindent
\textbf{Q21.} A hovercraft of mass
[IMAGE:0]
moves at constant speed
[IMAGE:1]
on a horizontal surface. If the lift fan provides a force equal to the weight of the hovercraft which consumes power
[IMAGE:2]
, what is the power required to maintain this speed if friction is
[IMAGE:3]
?



\textbf{A.} [IMAGE:0] \\
\textbf{B.} [IMAGE:1] \\
\textbf{C.} [IMAGE:2] \\
\textbf{D.} [IMAGE:3] \\

\textbf{Answer:} E \\
\textbf{Explanation:} Power is force times velocity:
[IMAGE:0]
. P is not vector but scalar.

\hrule
\vspace{1em}


\noindent
\textbf{Q22.} The ratio of radius of two solid spheres is 3:4, the ratio of density of these spheres is 2:3. Find the ratio of the mass of these two solid spheres.



\textbf{A.} 4:9 \\
\textbf{B.} 27:16 \\
\textbf{C.} 27:
64 \\
\textbf{D.} 16:81 \\

\textbf{Answer:} C \\
\textbf{Explanation:} The volume ratio is 81:64, the mass ratio is therefore: 27 :64.

\hrule
\vspace{1em}


\noindent
\textbf{Q23.} A ball of mass
[IMAGE:0]
is thrown vertically upward with initial speed
[IMAGE:1]
. Ignoring air resistance, what is the maximum height reached?



\textbf{A.} [IMAGE:0] \\
\textbf{B.} [IMAGE:1] \\
\textbf{C.} [IMAGE:2] \\
\textbf{D.} [IMAGE:3] \\

\textbf{Answer:} D \\
\textbf{Explanation:} Using kinematic equation:
[IMAGE:0]
;
[IMAGE:1]
.

\hrule
\vspace{1em}


\noindent
\textbf{Q24.} A rocket travelling in space is burning its fuel at a increasing rate
[IMAGE:0]
, where
[IMAGE:1]
is inital rate of burning fuel,
[IMAGE:2]
denotes time and
[IMAGE:3]
is a constant. By expelling the burnt fuel through a nozzle, the engine is applying a constant force to the rocket.
What is happening to the magnitude of the velocity of the rocket?



\textbf{A.} It is increasing at an increasing rate. \\
\textbf{B.} It is increasing at a constant rate. \\
\textbf{C.} It is increasing at a decreasing rate. \\
\textbf{D.} It is not changing. \\

\textbf{Answer:} B \\
\textbf{Explanation:} The purposive force is a constant; the mass is decreasing.
Thus, the acceleration is therefore increasing;
the jerk(rate of change of acceleration) is
[IMAGE:0]
, which means the acceleration is increasing at an increasing rate in time.

\hrule
\vspace{1em}


\noindent
\textbf{Q25.} The ratio of radius of two solid spheres is 4:3, the ratio of density of these spheres is 1:3. Find the ratio of the mass of these two solid spheres.



\textbf{A.} 1:9 \\
\textbf{B.} 16:27 \\
\textbf{C.} 8:9 \\
\textbf{D.} 4:81 \\

\textbf{Answer:} E \\
\textbf{Explanation:} The volume ratio is 64:27, the mass ratio is therefore: 64:81.

\hrule
\vspace{1em}


\noindent
\textbf{Q26.} A rocket of mass
[IMAGE:0]
ejects mass at a rate
[IMAGE:1]
with velocity
[IMAGE:2]
relative to the rocket. If the rocket starts from rest, what is its speed after time
[IMAGE:3]
?



\textbf{A.} [IMAGE:0] \\
\textbf{B.} [IMAGE:1] \\
\textbf{C.} [IMAGE:2] \\
\textbf{D.} [IMAGE:3] \\

\textbf{Answer:} A \\
\textbf{Explanation:} Using rocket equation:
[IMAGE:0]
.

\hrule
\vspace{1em}


\noindent
\textbf{Q27.} A rocket travelling in space is burning its fuel at a decreasing rate
[IMAGE:0]
, where
[IMAGE:1]
is inital rate of burning fuel,
[IMAGE:2]
denotes time and
[IMAGE:3]
is a constant. By expelling the burnt fuel through a nozzle, the engine is applying a constant force to the rocket.
What is happening to the magnitude of the velocity of the rocket?



\textbf{A.} It is increasing at an increasing rate. \\
\textbf{B.} It is increasing at a constant rate. \\
\textbf{C.} It is increasing at a decreasing rate. \\
\textbf{D.} It is not changing. \\

\textbf{Answer:} B \\
\textbf{Explanation:} The purposive force is a constant; the mass is decreasing.
PS: Though the fuel consumption is at a decreasing rate, the mass is still decreasing.
Thus, the acceleration is therefore increasing;
the jerk(rate of change of acceleration) is
[IMAGE:0]
, which means the acceleration increasing is at a contant rate in time.

\hrule
\vspace{1em}


\noindent
\textbf{Q28.} A transverse wave propagates in a medium with a wave speed of 15 m/s, a frequency of 3 Hz, and an amplitude of 5 cm. During the wave's propagation, a particle starts vibrating from the equilibrium position. After 3 periods, what is the maximum distance of the particle from the equilibrium position?



\textbf{A.} 5cm \\
\textbf{B.} 10cm \\
\textbf{C.} 15cm \\
\textbf{D.} 20cm \\

\textbf{Answer:} A \\
\textbf{Explanation:} The particle is only vibrating in the direction perpendicular to the propagation direction (the option C is a misleading term). Thus, it can be the amplitude which is 5 cm.

\hrule
\vspace{1em}


\noindent
\textbf{Q29.} A bicycle of mass
[IMAGE:0]
traveling at speed
[IMAGE:1]
encounters a slope inclined at an angle
[IMAGE:2]
and coasts up it until it stops. If the coefficient of friction is
[IMAGE:3]
, what is the distance traveled up the slope?



\textbf{A.} [IMAGE:0] \\
\textbf{B.} [IMAGE:1] \\
\textbf{C.} [IMAGE:2] \\
\textbf{D.} [IMAGE:3] \\

\textbf{Answer:} A \\
\textbf{Explanation:} Deceleration is
[IMAGE:0]
;
[IMAGE:1]
;
[IMAGE:2]
.

\hrule
\vspace{1em}


\noindent
\textbf{Q30.} A transverse wave propagates in a medium with a wavelength of 8 meters, a frequency of 4 Hz, and an amplitude of 6 cm. During the wave's propagation, a particle starts vibrating from its maximum displacement. How long will it take for the particle to return to its maximum displacement (in the same direction) for the first time?



\textbf{A.} 0.125 seconds \\
\textbf{B.} 0.25 seconds \\
\textbf{C.} 0.5 seconds \\
\textbf{D.} 1 seconds \\

\textbf{Answer:} B \\
\textbf{Explanation:} [IMAGE:0]

\hrule
\vspace{1em}


\noindent
\textbf{Q31.} A rocket travelling in space is burning its fuel at a constant rate. By expelling the burnt fuel through a nozzle, the engine is applying a constant force to the rocket.
What is happening to the magnitude of the distance of the rocket?



\textbf{A.} It is increasing at an increasing rate. \\
\textbf{B.} It is increasing at a constant rate. \\
\textbf{C.} It is increasing at a decreasing rate. \\
\textbf{D.} It is not changing. \\

\textbf{Answer:} A \\
\textbf{Explanation:} The purposive force is a constant; the mass is decreasing; the acceleration is therefore increasing; so the velocity is creasing at an increasing rate and the distance is creasing at an increasing rate.

\hrule
\vspace{1em}


\noindent
\textbf{Q32.} A skateboarder of mass
[IMAGE:0]
starts from rest and reaches a speed
[IMAGE:1]
in time
[IMAGE:2]
by applying a constant force
[IMAGE:3]
. If friction is constant as
[IMAGE:4]
, what is the distance covered?



\textbf{A.} [IMAGE:0] \\
\textbf{B.} [IMAGE:1] \\
\textbf{C.} [IMAGE:2] \\
\textbf{D.} [IMAGE:3] \\

\textbf{Answer:} E \\
\textbf{Explanation:} Acceleration is
[IMAGE:0]
;
[IMAGE:1]
;
[IMAGE:2]
.

\hrule
\vspace{1em}


\noindent
\textbf{Q33.} A rocket travelling in space is burning its fuel at a constant rate. By expelling the burnt fuel through a nozzle, the engine is applying a constant force to the rocket.
What is happening to the magnitude of the velocity of the rocket?



\textbf{A.} It is increasing at an increasing rate. \\
\textbf{B.} It is increasing at a constant rate. \\
\textbf{C.} It is increasing at a decreasing rate. \\
\textbf{D.} It is not changing. \\

\textbf{Answer:} A \\
\textbf{Explanation:} The purposive force is a constant; the mass is decreasing; the acceleration is therefore increasing; so the velocity is creasing at an increasing rate.

\hrule
\vspace{1em}


\noindent
\textbf{Q34.} A transverse travels 10m within a period 5s. The amplitude of the wave is 10m. Find the distance travelled by one of the particles in the waveform during 16s.



\textbf{A.} 32m \\
\textbf{B.} 64m \\
\textbf{C.} 100m \\
\textbf{D.} 128m \\

\textbf{Answer:} D \\
\textbf{Explanation:} [IMAGE:0]

\hrule
\vspace{1em}


\noindent
\textbf{Q35.} A transverse travels 10m within a period 4s. The amplitude of the wave is 5m. Find the distance travelled by one of the particles in the waveform during 16s.



\textbf{A.} 40m \\
\textbf{B.} 100m \\
\textbf{C.} 80m \\
\textbf{D.} 200m \\

\textbf{Answer:} C \\
\textbf{Explanation:} The particle is only vibrating in the direction perpendicular to the propagation direction; In each cycle, the object passes 4 amplitudes; Therefore
[IMAGE:0]
.

\hrule
\vspace{1em}


\noindent
\textbf{Q36.} A train of mass 3m
moving at speed v/2
applies its brakes and stops in time 3t
. If the braking force is F/2
, what distance does the train travel before stopping?



\textbf{A.} [IMAGE:0] \\
\textbf{B.} [IMAGE:1] \\
\textbf{C.} [IMAGE:2] \\
\textbf{D.} [IMAGE:3] \\

\textbf{Answer:} B \\
\textbf{Explanation:} Deceleration is
[IMAGE:0]
;
[IMAGE:1]
;
[IMAGE:2]

\hrule
\vspace{1em}


\noindent
\textbf{Q37.} A future vehicle of mass 800 kg travels in a straight line along a horizontal road, as shown in the acceleration/deceleration–time graph.
What is the average resultant force acting on the vehicle over the time for which it is accelerating / decelerating?
[IMAGE:0]



\textbf{A.} -380N \\
\textbf{B.} 420N \\
\textbf{C.} -960N \\
\textbf{D.} -7500N \\

\textbf{Answer:} G \\
\textbf{Explanation:} What is the average resultant force acting on the vehicle over the time for which it is accelerating?
The vehicle is accelerating from 0s to 10s (all the time because acceleration is bigger than zero, which may be a trap).
0s~5s:
[IMAGE:0]
5s~10s:
[IMAGE:1]
Thus, the acceleration in average is therefore
[IMAGE:2]
;
[IMAGE:3]
(F option is a disturbance term)

\hrule
\vspace{1em}


\noindent
\textbf{Q38.} As the diagram indicates: the ground is frictionless; the force F(N)
is exerting on the left hand side of the body A; three bodies (A, B and C) have mass m(kg)
, 2m(kg)
, and 2m(kg)
seperately; Find the stiffness coefficient of the spring between B and C:



\textbf{A.} [IMAGE:0] \\
\textbf{B.} [IMAGE:1] \\
\textbf{C.} [IMAGE:2] \\
\textbf{D.} [IMAGE:3] \\

\textbf{Answer:} D \\
\textbf{Explanation:} [IMAGE:0]

\hrule
\vspace{1em}


\noindent
\textbf{Q39.} A sled of mass
[IMAGE:0]
slides down a frictionless incline which has angle
[IMAGE:1]
with the ground and reaches the bottom with speed
[IMAGE:2]
. If the vertical height of the incline is
[IMAGE:3]
, and the sled started from rest, what is the length of the incline?



\textbf{A.} [IMAGE:0] \\
\textbf{B.} [IMAGE:1] \\
\textbf{C.} [IMAGE:2] \\
\textbf{D.} [IMAGE:3] \\

\textbf{Answer:} D \\
\textbf{Explanation:} Using energy conservation:
[IMAGE:0]
; solve for
[IMAGE:1]
and relate to incline length
[IMAGE:2]
via
[IMAGE:3]
;
[IMAGE:4]
.

\hrule
\vspace{1em}


\noindent
\textbf{Q40.} A car of mass 800 kg travels in a straight line along a horizontal road, as shown in the deceleration–time graph.
What is the average resultant force acting on the car over the time for which it is decelerating?



\textbf{A.} -380N \\
\textbf{B.} -420N \\
\textbf{C.} -960N \\
\textbf{D.} -1190N \\

\textbf{Answer:} C \\
\textbf{Explanation:} By the gradient of the graph in the linear region; the velocity of 5s is
[IMAGE:0]
; the acceleration in average is therefore
[IMAGE:1]
;
[IMAGE:2]
.

\hrule
\vspace{1em}


\noindent
\textbf{Q41.} A future vehicle of mass 400 kg travels in a straight line along a horizontal road, as shown in the acceleration–time graph.
What is the final kinetic energy of the vehicle?
[IMAGE:0]



\textbf{A.} 50J \\
\textbf{B.} 125J \\
\textbf{C.} 150J \\
\textbf{D.} 450000J \\

\textbf{Answer:} F \\
\textbf{Explanation:} The vehicle is accelerating from 0s to 10s (all the time because acceleration is not zero).
0s~5s:
[IMAGE:0]
5s~10s:
[IMAGE:1]
Thus,
[IMAGE:2]

\hrule
\vspace{1em}


\noindent
\textbf{Q42.} As the diagram indicates: the ground is frictionless; the force
[IMAGE:0]
is exerting on the left hand side of the body A; three bodies (A, B and C) have mass m
, 2m
, and 3m
seperately; Find the force between A and B:



\textbf{A.} F/2 \\
\textbf{B.} 5F/6 \\
\textbf{C.} 7F/9 \\
\textbf{D.} F/3 \\

\textbf{Answer:} B \\
\textbf{Explanation:} [IMAGE:0]

\hrule
\vspace{1em}


\noindent
\textbf{Q43.} A car of mass
[IMAGE:0]
starts from rest and accelerates uniformly to a speed
[IMAGE:1]
in a time
[IMAGE:2]
on a horizontal road. If the average horizontal force applied is
[IMAGE:3]
, what is the distance covered by the car during this acceleration?



\textbf{A.} [IMAGE:0] \\
\textbf{B.} [IMAGE:1] \\
\textbf{C.} [IMAGE:2] \\
\textbf{D.} [IMAGE:3] \\

\textbf{Answer:} F \\
\textbf{Explanation:} The acceleration is
[IMAGE:0]
; by formula:
[IMAGE:1]
;
[IMAGE:2]
.

\hrule
\vspace{1em}


\noindent
\textbf{Q44.} As the diagram indicates: the ground is frictionless; the force
[IMAGE:0]
is exerting on the left hand side of the body A; three bodies (A, B and C) have mass 3m
, 2m
, and m
seperately; Find the force between A and B:



\textbf{A.} F \\
\textbf{B.} F/2 \\
\textbf{C.} 2F/3 \\
\textbf{D.} F/3 \\

\textbf{Answer:} B \\
\textbf{Explanation:} [IMAGE:0]

\hrule
\vspace{1em}


\noindent
\textbf{Q45.} A future vehicle of mass 600 kg travels in a straight line along a horizontal road, as shown in the acceleration–time graph.
What is the final velocity of the vehicle?
[IMAGE:0]



\textbf{A.} 25m/s \\
\textbf{B.} 50m/s \\
\textbf{C.} 100m/s \\
\textbf{D.} 125m/s \\

\textbf{Answer:} E,E \\
\textbf{Explanation:} The vehicle is accelerating from 0s to 10s (all the time because acceleration is not zero).
0s~5s:
[IMAGE:0]
5s~10s:
[IMAGE:1]

\hrule
\vspace{1em}


\noindent
\textbf{Q46.} A lorry of mass m, and travelling initially at speed 2v along a horizontal road, is brought to rest by an average horizontal braking force F in time t. Ignoring any other resistive forces, what distance is travelled by the lorry during this time? (gravitational field strength = 10 N kg–1)



\textbf{A.} [IMAGE:0] \\
\textbf{B.} [IMAGE:1] \\
\textbf{C.} [IMAGE:2] \\
\textbf{D.} [IMAGE:3] \\

\textbf{Answer:} E \\
\textbf{Explanation:} The deceleration is
[IMAGE:0]
; by formula:
[IMAGE:1]
;
[IMAGE:2]
.

\hrule
\vspace{1em}


\noindent
\textbf{Q47.} As the diagram indicates: the ground is frictionless; the force
F
is exerting on the left hand side of the body A; three bodies are with mass
[IMAGE:0]
; Find the force between A and B:



\textbf{A.} F \\
\textbf{B.} F/2 \\
\textbf{C.} 2F/3 \\
\textbf{D.} F/3 \\

\textbf{Answer:} C \\
\textbf{Explanation:} [IMAGE:0]

\hrule
\vspace{1em}


\noindent
\textbf{Q48.} A future vehicle of mass 600 kg travels in a straight line along a horizontal road, as shown in the acceleration–time graph.
What is the average resultant force acting on the vehicle over the time for which it is accelerating?
[IMAGE:0]



\textbf{A.} 380N \\
\textbf{B.} 420N \\
\textbf{C.} 550N \\
\textbf{D.} 5090N \\

\textbf{Answer:} G \\
\textbf{Explanation:} The vehicle is accelerating from 0s to 10s (all the time because acceleration is not zero).
0s~5s:
[IMAGE:0]
5s~10s:
[IMAGE:1]
[IMAGE:2]
[IMAGE:3]
[IMAGE:4]

\hrule
\vspace{1em}


\noindent
\textbf{Q49.} A shape is formed by drawing a triangle ABC inside the triangle ADE. BC is parallel to DE. The area of triangle ABC is 18 cm², and the area of triangle ADE is 72 cm². BC = s cm, DE = s + 8 cm.
Determine the height of triangle ADE to side DE.



\textbf{A.} [IMAGE:0] \\
\textbf{B.} [IMAGE:1] \\
\textbf{C.} [IMAGE:2] \\
\textbf{D.} [IMAGE:3] \\

\textbf{Answer:} A \\
\textbf{Explanation:} Area ratio is square of altitude (or side) ratio. Given
[IMAGE:0]
, sides ratio is
[IMAGE:1]
. So,
[IMAGE:2]
. Solving gives
[IMAGE:3]
, hence
[IMAGE:4]
cm. Thus, the height is
[IMAGE:5]
.

\hrule
\vspace{1em}


\noindent
\textbf{Q50.} As the diagram indicates: the ground is frictionless; the force F=5N
is exerting on the left hand side of the body A; three bodies are with mass m=2kg
; Find the force between B and C (round to one decimal place):



\textbf{A.} 5 \\
\textbf{B.} 2.5 \\
\textbf{C.} 3.3 \\
\textbf{D.} 1.7 \\

\textbf{Answer:} D \\
\textbf{Explanation:} The three bodies have the same acceleration; the acceleration is F/3m; the third one only has been pushed by B on the B-C boundary; the force is therefore:
[IMAGE:0]

\hrule
\vspace{1em}


\noindent
\textbf{Q51.} A car of mass 900 kg travels in a straight line along a horizontal road, as shown in the speed–time graph.
What is the average resultant force acting on the car over the time for which it is accelerating?



\textbf{A.} 380N \\
\textbf{B.} 420N \\
\textbf{C.} 550N \\
\textbf{D.} 1190N \\

\textbf{Answer:} F \\
\textbf{Explanation:} The terminal velocity is
[IMAGE:0]
; the acceleration in average is therefore
[IMAGE:1]
.

\hrule
\vspace{1em}


\noindent
\textbf{Q52.} A shape is formed by drawing a triangle ABC inside the triangle ADE. BC is parallel to DE. The median from A to BC in triangle ABC is 4 cm, and the median from A to DE in triangle ADE is 10 cm. The two medians are collinear. BC = r cm, DE = r + 9 cm.
Find the length of DE.



\textbf{A.} [IMAGE:0] \\
\textbf{B.} [IMAGE:1] \\
\textbf{C.} [IMAGE:2] \\
\textbf{D.} [IMAGE:3] \\

\textbf{Answer:} B \\
\textbf{Explanation:} Medians ratio equals sides ratio. So,
[IMAGE:0]
. Solving gives
[IMAGE:1]
, hence
[IMAGE:2]
cm.

\hrule
\vspace{1em}


\noindent
\textbf{Q53.} A car of mass 850 kg travels in a straight line along a horizontal road, as shown in the distance–time graph.
What is the average resultant force acting on the car over the time for which it is accelerating?



\textbf{A.} 380N \\
\textbf{B.} 420N \\
\textbf{C.} 550N \\
\textbf{D.} 1190N \\

\textbf{Answer:} D \\
\textbf{Explanation:} By the gradient of the graph in the linear region; the terminal velocity is
[IMAGE:0]
; the acceleration in average is therefore
[IMAGE:1]
;
[IMAGE:2]

\hrule
\vspace{1em}


\noindent
\textbf{Q54.} A object with zero intial velocity and zero initial acceleration is placed on a horizontal conveyor belt; find the correct statement:
1.
The gravity of object A and the normal reaction force of the conveyor belt on object A is an action and reaction force pair;
2.
When the conveyor belt moves at a constant speed, the gravity and the normal reaction force of the conveyor belt on the object are balanced;
3.
If the conveyor belt suddenly accelerates, the object will slide backward relative to the conveyor belt;



\textbf{A.} 1 only \\
\textbf{B.} 1 and 2 \\
\textbf{C.} 2 and 3 \\
\textbf{D.} 2 only \\

\textbf{Answer:} D \\
\textbf{Explanation:} The ground is fixed as can be seen in the graph. "1." The gravity of object A and the normal reaction force of the conveyor belt on object A is a balanced forces pair (which is not the same as "action and reaction force pair"). "2." is obviously right. "3." Object A may not necessarily slide. In deed, it depends on the maximum static friction force between the object A and the conveyor belt.

\hrule
\vspace{1em}


\noindent
\textbf{Q55.} A shape is formed by drawing a triangle ABC inside the triangle ADE. BC is parallel to DE.
[IMAGE:0]
, and AB=4cm, BD=6cm.
Calculate the length of DE.



\textbf{A.} [IMAGE:0] \\
\textbf{B.} [IMAGE:1] \\
\textbf{C.} [IMAGE:2] \\
\textbf{D.} [IMAGE:3] \\

\textbf{Answer:} C \\
\textbf{Explanation:} From similarity,
[IMAGE:0]
.
[IMAGE:1]
, thus,
[IMAGE:2]
.

\hrule
\vspace{1em}


\noindent
\textbf{Q56.} A car of mass 500 kg travels in a straight line along a horizontal road.
The car accelerates non-uniformly from rest for 5.0 seconds and then moves at constant speed, as shown in the distance–time graph.
What is the average resultant force acting on the car over the time for which it is accelerating?



\textbf{A.} 380N \\
\textbf{B.} 420N \\
\textbf{C.} 500N \\
\textbf{D.} 1200N \\

\textbf{Answer:} C \\
\textbf{Explanation:} By the gradient of the graph in the linear region; the terminal velocity is
[IMAGE:0]
; the acceleration in average is therefore
[IMAGE:1]
;
[IMAGE:2]
.

\hrule
\vspace{1em}


\noindent
\textbf{Q57.} The object A is placed on a circular disc rotating at a constant speed without sliding between them; the disc is rough and the object rotate with the disc simultaneously; find the correct statement:
1.
The gravity of object A and the normal reaction force of the disc on object A is an action and reaction force pair;
2.
When the disc rotates at a constant speed, the gravity and the normal reaction force of the disc on the object are balanced;
3.
If the disc suddenly stops rotating, the object will continue to move in the direction of the radius of the disk;



\textbf{A.} 1 only \\
\textbf{B.} 1 and 2 \\
\textbf{C.} 2 and 3 \\
\textbf{D.} 2 only \\

\textbf{Answer:} D \\
\textbf{Explanation:} The ground is fixed as can be seen in the graph. "1." The gravity of object A and the normal reaction force of the disc on object A is a balanced forces pair (which is not the same as "action and reaction force pair"). "2." is obviously right. "3." If the disc suddenly stops rotating, the object will continue to move in the tangential direction.

\hrule
\vspace{1em}


\noindent
\textbf{Q58.} A shape is formed by drawing a triangle ABC inside the triangle ADE. BC is parallel to DE. AB = 8 cm, BC = p cm, DE = p + 6 cm.
[IMAGE:0]
is the bigger root of roots to the equation
[IMAGE:1]
.
Determine the length of AD.



\textbf{A.} [IMAGE:0] \\
\textbf{B.} [IMAGE:1] \\
\textbf{C.} [IMAGE:2] \\
\textbf{D.} [IMAGE:3] \\

\textbf{Answer:} D \\
\textbf{Explanation:} [IMAGE:0]
is the bigger root of roots to the equation
[IMAGE:1]
. So,
[IMAGE:2]
.
Equal angles and BC || DE imply similarity. So,
[IMAGE:3]
. Solve
[IMAGE:4]
that
[IMAGE:5]

\hrule
\vspace{1em}


\noindent
\textbf{Q59.} A car of mass 600 kg travels in a straight line along a horizontal road.
The car accelerates non-uniformly from rest for 5.0 seconds and then moves at constant speed, as shown in the distance–time graph.
What is the average resultant force acting on the car over the time for which it is accelerating?



\textbf{A.} 320N \\
\textbf{B.} 480N \\
\textbf{C.} 640N \\
\textbf{D.} 1200N \\

\textbf{Answer:} D \\
\textbf{Explanation:} By the gradient of the graph in the linear region; the terminal velocity is
[IMAGE:0]
; the acceleration in average is therefore
[IMAGE:1]

\hrule
\vspace{1em}


\noindent
\textbf{Q60.} The object A with zero velocity and zero acceleration is hanging from a spring; find the correct statement:
1.
The gravity of object A and the tension force of the spring on object A is an action and reaction force pair;
2.
When the object is in equilibrium, the magnitude of gravity is equal to the tension force of the spring on the object;
3.
If the spring is now cut, the object will fall freely;



\textbf{A.} 1 only \\
\textbf{B.} 1 and 2 \\
\textbf{C.} 2 and 3 \\
\textbf{D.} 3 only \\

\textbf{Answer:} C \\
\textbf{Explanation:} The ground is fixed as can be seen in the graph. "1." The gravity of object A and the tension force of the spring on object A is a balanced forces pair (which is not the same as "action and reaction force pair"). "2." is obviously right. "3." is obviously right.

\hrule
\vspace{1em}


\noindent
\textbf{Q61.} A shape is formed by drawing a triangle ABC inside the triangle ADE. BC is parallel to DE. The height from A to BC is 3 cm, the height from A to DE is 9 cm, BC = n cm, and DE = n + 4 cm.
Find the length of DE.



\textbf{A.} [IMAGE:0] \\
\textbf{B.} [IMAGE:1] \\
\textbf{C.} [IMAGE:2] \\
\textbf{D.} [IMAGE:3] \\

\textbf{Answer:} A \\
\textbf{Explanation:} Heights ratio is
[IMAGE:0]
, so sides ratio is also
[IMAGE:1]
. Thus,
[IMAGE:2]
. Solving gives
[IMAGE:3]
, hence
[IMAGE:4]
cm.

\hrule
\vspace{1em}


\noindent
\textbf{Q62.} The object A with zero velocity and zero acceleration is placed on an inclined plane; find the correct statement:
1.
The gravity of object A and the normal reaction force of the inclined plane on object A is a balanced forces pair;
2.
The magnitude of normal component of gravity is equal to the magnitude of normal reaction force of the inclined plane on the object;
3.
If the inclined plane is now a smooth surface with no friction; the object will slide down the plane;



\textbf{A.} 1 only \\
\textbf{B.} 2 and 3 \\
\textbf{C.} 1 and 3 \\
\textbf{D.} 3 only \\

\textbf{Answer:} B \\
\textbf{Explanation:} The inclined plane is fixed as can be seen in the graph. "1." the two forces have different directions. "2." is obviously right. "3." is obviously right because of the component of gravity (parallel to the inclined plane).

\hrule
\vspace{1em}


\noindent
\textbf{Q63.} During the soldering process, precise control of the soldering iron tip temperature is crucial. To maintain a constant tip temperature, a soldering iron is equipped with a temperature sensor and feedback control system. Assuming the tip (mass 2.0 g, copper) needs to be kept at 250°C while the ambient temperature is 20°C, and the heat exchange rate between the tip and the environment is 0.5 W/°C (i.e., the tip loses 0.5 W of heat for every 1°C above the ambient temperature)
Calculate the thermal power required from the soldering iron to maintain the tip at 250°C.



\textbf{A.} 100W \\
\textbf{B.} 115W \\
\textbf{C.} 230W \\
\textbf{D.} 460W \\

\textbf{Answer:} B \\
\textbf{Explanation:} Temperature difference:
[IMAGE:0]
Heat loss power:
[IMAGE:1]
Thermal power required is equal to the hear loss power, which is
[IMAGE:2]
.

\hrule
\vspace{1em}


\noindent
\textbf{Q64.} A shape is formed by drawing a triangle ABC inside the triangle ADE. BC is parallel to DE. The perimeter of triangle ABC is 24 cm, BC = m cm, DE = m + 8 cm, and the perimeter of triangle ADE is 40 cm.
Calculate the length of AD+AE.



\textbf{A.} 14cm \\
\textbf{B.} 16cm \\
\textbf{C.} 18cm \\
\textbf{D.} 20cm \\

\textbf{Answer:} D \\
\textbf{Explanation:} Perimeters of similar triangles are in the ratio of their sides. So,
[IMAGE:0]
. Solving gives
[IMAGE:1]
, so
[IMAGE:2]
cm.
Thus,
[IMAGE:3]
,

\hrule
\vspace{1em}


\noindent
\textbf{Q65.} A soldering iron needs to adjust the tip temperature for welding different materials. To quickly reach the required welding temperature, a designer develops a new tip material whose specific heat capacity varies with temperature (relationship:
[IMAGE:0]
, where t is the temperature in °C). When the soldering iron heats this new tip (mass 1.0 g) with a thermal power of 50 W, the tip's temperature rises from 20°C to 200°C in 30 s.
Calculate the heat transferred to the surrounding environment during this process.



\textbf{A.} 111.6J \\
\textbf{B.} 240J \\
\textbf{C.} 360J \\
\textbf{D.} 1200J \\

\textbf{Answer:} E \\
\textbf{Explanation:} Heat provided by the soldering iron:
[IMAGE:0]
Heat absorbed by the tip (using average specific heat capacity because of the linearity of
[IMAGE:1]
equation):
Average specific heat capacity:
[IMAGE:2]
Heat absorbed:
[IMAGE:3]
Heat transferred to the environment:
[IMAGE:4]

\hrule
\vspace{1em}


\noindent
\textbf{Q66.} The object A with zero velocity and zero acceleration is placed on the ground; find the correct statement:
1.
The gravity of A and the normal reaction by the ground to A is an action and reaction force pair;
2.
The magnitude of the gravity equals the normal reaction force from the ground to the object when in equilibrium;
3.
The normal reaction by the ground to A can be a negative;



\textbf{A.} 1 only \\
\textbf{B.} 1 and 2 \\
\textbf{C.} 1 and 3 \\
\textbf{D.} 1 and 3 \\

\textbf{Answer:} E \\
\textbf{Explanation:} The ground is fixed as can be seen in the graph. "1." The gravity of A and the normal reaction by the ground to A is a balanced forces pair (which is not the same as "an action and reaction force pair"). "2." is obviously right. "3." is obviously wrong.

\hrule
\vspace{1em}


\noindent
\textbf{Q67.} A shape is formed by drawing a triangle ABC inside the triangle ADE. BC is parallel to DE. The area of triangle ABC is 12 cm², BC = 3w cm, DE = w + 5 cm, and the area of triangle ADE is 48 cm².
Determine the length of DE.



\textbf{A.} [IMAGE:0] \\
\textbf{B.} [IMAGE:1] \\
\textbf{C.} [IMAGE:2] \\
\textbf{D.} [IMAGE:3] \\

\textbf{Answer:} E \\
\textbf{Explanation:} The ratio of areas of similar triangles is the square of the ratio of their sides. So,
[IMAGE:0]
. Thus,
[IMAGE:1]
. Solving gives
[IMAGE:2]
, so
[IMAGE:3]
cm.

\hrule
\vspace{1em}


\noindent
\textbf{Q68.} During continuous operation, the copper tip of a soldering iron (mass 1.8 g, specific heat capacity
[IMAGE:0]
) experiences process of repeated heating and cooling. In a specific soldering task, the tip is first heated to 300°C from the room temperature(20°C), then rapidly cooled to 50°C, then heated again to 250°C, and finally cooled to room temperature (20°C). Assuming each heating and cooling process is linear and takes 20 s. The thermal power is 50 W when it comes to rising up the temperature of the copper tip.
Calculate the total heat transferred to the surrounding environment during the entire process.



\textbf{A.} 320J \\
\textbf{B.} 480J \\
\textbf{C.} 640J \\
\textbf{D.} 1200J \\

\textbf{Answer:} G \\
\textbf{Explanation:} Heat absorbed during first heating:
[IMAGE:0]
Heat lost during first cooling:
[IMAGE:1]
Heat absorbed during second heating:
[IMAGE:2]
Heat lost during second cooling:
[IMAGE:3]
Total heat transferred to the environment:
[IMAGE:4]
.
PS: If write down the
[IMAGE:5]
equation at the very beginning, the calculation process will become easier.

\hrule
\vspace{1em}


\noindent
\textbf{Q69.} A shape is formed by drawing a triangle ABC inside the triangle ADE. BC is parallel to DE. AC = 7 cm, BC = z cm, DE = 2z + 1 cm, CE = z + 1 cm.
Find the length of DE.



\textbf{A.} [IMAGE:0] \\
\textbf{B.} [IMAGE:1] \\
\textbf{C.} [IMAGE:2] \\
\textbf{D.} [IMAGE:3] \\

\textbf{Answer:} C \\
\textbf{Explanation:} Using similarity,
[IMAGE:0]
. Since
[IMAGE:1]
, we get
[IMAGE:2]
. Solving yields
[IMAGE:3]
(negative root -1 is discarded), hence
[IMAGE:4]
cm.

\hrule
\vspace{1em}


\noindent
\textbf{Q70.} As the diagram indicates: An object with weight 6N is on a slope with angle 45 degrees to the horizontal direction; The coefficient of friction is 0.5 and the maximum static friction
[IMAGE:0]
. An external force
[IMAGE:1]
is exerted to the object: Find the minimum force that is required to move the object downwards



\textbf{A.} [IMAGE:0] \\
\textbf{B.} [IMAGE:1] \\
\textbf{C.} [IMAGE:2] \\
\textbf{D.} [IMAGE:3] \\

\textbf{Answer:} B \\
\textbf{Explanation:} [IMAGE:0]
[IMAGE:1]

\hrule
\vspace{1em}


\noindent
\textbf{Q71.} A shape is formed by drawing a triangle ABC inside the triangle ADE. BC is parallel to DE. AC = 5 cm, BC = y cm, DE = y + 2 cm, EC = y - 3 cm.
Calculate the length of DE.



\textbf{A.} [IMAGE:0] \\
\textbf{B.} [IMAGE:1] \\
\textbf{C.} [IMAGE:2] \\
\textbf{D.} [IMAGE:3] \\

\textbf{Answer:} B \\
\textbf{Explanation:} Since BC || DE, triangles ABC and ADE are similar. So,
[IMAGE:0]
. Given
[IMAGE:1]
, we have
[IMAGE:2]
. Solving this gives
[IMAGE:3]
(discarding the negative root -2), so
[IMAGE:4]
.

\hrule
\vspace{1em}


\noindent
\textbf{Q72.} To improve soldering efficiency, a designer coats the surface of a copper tip (mass 2.0 g, specific heat capacity
[IMAGE:0]
) with a special material that significantly enhances the tip's thermal conductivity but also increases its heat capacity by 20%. When the soldering iron heats the tip with a thermal power of 50 W, the tip's temperature rises by 220°C in 10 s.
Calculate the heat lost to the environment. after coating.



\textbf{A.} 288.8J \\
\textbf{B.} 480J \\
\textbf{C.} 640J \\
\textbf{D.} 1200J \\

\textbf{Answer:} A \\
\textbf{Explanation:} Heat provided by the soldering iron:
[IMAGE:0]
Heat absorbed by the tip:
[IMAGE:1]
Heat lost to the environment:
[IMAGE:2]

\hrule
\vspace{1em}


\noindent
\textbf{Q73.} As the diagram indicates: An object with weight 6N is on a slope with angle 45 degrees to the horizontal direction; The coefficient of friction is 0.5. An external force
[IMAGE:0]
is exerted to the object: Find the minimum force that is required to move the object upwards



\textbf{A.} [IMAGE:0] \\
\textbf{B.} [IMAGE:1] \\
\textbf{C.} [IMAGE:2] \\
\textbf{D.} [IMAGE:3] \\

\textbf{Answer:} C \\
\textbf{Explanation:} [IMAGE:0]
[IMAGE:1]
[IMAGE:2]
[IMAGE:3]

\hrule
\vspace{1em}


\noindent
\textbf{Q74.} A shape is formed by drawing a triangle ABC inside the triangle ADE. BC is parallel to DE. AB = x-3 cm BC = x - 1 cm DE = x + 3 cm DB = 3 cm.
Calculate the length of DE.



\textbf{A.} [IMAGE:0] \\
\textbf{B.} [IMAGE:1] \\
\textbf{C.} [IMAGE:2] \\
\textbf{D.} [IMAGE:3] \\

\textbf{Answer:} E \\
\textbf{Explanation:} AB/AD=BC/DE.
[IMAGE:0]
with
[IMAGE:1]
[IMAGE:2]
.
[IMAGE:3]
.

\hrule
\vspace{1em}


\noindent
\textbf{Q75.} During the soldering process, the copper tip of a soldering iron (mass
[IMAGE:0]
, specific heat capacity
[IMAGE:1]
) accumulates heat due to prolonged use. When the soldering iron stops heating, the tip begins to cool and its temperature drops by
[IMAGE:2]
in
[IMAGE:3]
Assuming all the heat lost by the tip is absorbed by the surrounding environment and the cooling rate is constant
Calculate the heat lost by the tip per second during cooling.



\textbf{A.} 3J/s \\
\textbf{B.} 60J/s \\
\textbf{C.} 180J/s \\
\textbf{D.} 1000J/s \\

\textbf{Answer:} A \\
\textbf{Explanation:} Total heat lost by the tip:
[IMAGE:0]
Heat lost per second:
[IMAGE:1]
.

\hrule
\vspace{1em}


\noindent
\textbf{Q76.} As the diagram indicates: An object with weight 6N is on a slope with angle 45 degrees to the horizontal direction; The coefficient of friction is 0.5. An external force
[IMAGE:0]
is exerted to the object: Find the minimum force that is required to move the object upwards



\textbf{A.} 1N \\
\textbf{B.} 3N \\
\textbf{C.} 9N \\
\textbf{D.} 18N \\

\textbf{Answer:} D \\
\textbf{Explanation:} [IMAGE:0]
[IMAGE:1]
[IMAGE:2]
[IMAGE:3]

\hrule
\vspace{1em}


\noindent
\textbf{Q77.} A soldering iron is equipped with an aluminum tip of mass 3.0 g (specific heat capacity of aluminum =
[IMAGE:0]
). When the soldering iron heats the tip with a thermal power of 40 W, the tip's temperature rises to a certain temperature in 30 s. However, due to a heat sink on the tip's surface, some heat is lost to the environment. If only 80% of the heat provided by the heating power is actually absorbed by the tip.
Calculate the heat lost to the environment.



\textbf{A.} 120J \\
\textbf{B.} 200J \\
\textbf{C.} 240J \\
\textbf{D.} 1200J \\

\textbf{Answer:} C \\
\textbf{Explanation:} Heat provided by the soldering iron:
[IMAGE:0]
Heat absorbed by the tip:
[IMAGE:1]
Heat lost to the environment:
[IMAGE:2]

\hrule
\vspace{1em}


\noindent
\textbf{Q78.} As the diagram indicates: An object with weight 6N is on a slope with angle 30 degrees to the horizontal direction; The coefficient of friction is 0.6. An external force is exerted horizontally to the object: Find the minimum force that is required to move the object upwards



\textbf{A.} [IMAGE:0] \\
\textbf{B.} [IMAGE:1] \\
\textbf{C.} [IMAGE:2] \\
\textbf{D.} [IMAGE:3] \\

\textbf{Answer:} D \\
\textbf{Explanation:} [IMAGE:0]
[IMAGE:1]

\hrule
\vspace{1em}


\noindent
\textbf{Q79.} A soldering iron has a copper tip of mass 2.5g.
The tip is heated with 30W of thermal power. In 50s, the temperature of the tip increases by
[IMAGE:0]
How much energy is transferred from the tip to the surroundings in this time? (specific heat capacity of copper =
[IMAGE:1]
).



\textbf{A.} 320J \\
\textbf{B.} 480J \\
\textbf{C.} 640J \\
\textbf{D.} 1200J \\

\textbf{Answer:} D \\
\textbf{Explanation:} By the conservation of energy; during this time; the energy to heat the tip minus the energy dissipated into the air equals the energy to raise the temperature of the tip: therefore
[IMAGE:0]

\hrule
\vspace{1em}


\noindent
\textbf{Q80.} A soldering iron has a copper tip of mass 1.0g.
The tip is heated with 20W of thermal power. In 50s,the temperature of the tip increases by
[IMAGE:0]
.
How much energy is transferred from the tip to the surroundings in this time? (specific heat capacity of copper =
[IMAGE:1]
).



\textbf{A.} 320J \\
\textbf{B.} 480J \\
\textbf{C.} 640J \\
\textbf{D.} 800J \\

\textbf{Answer:} E \\
\textbf{Explanation:} By the conservation of energy; during this time; the energy to heat the tip minus the energy dissipated into the air equals the energy to raise the temperature of the tip: therefore
[IMAGE:0]

\hrule
\vspace{1em}


\noindent
\textbf{Q81.} A solid frustum of a cone with lower base radius
[IMAGE:0]
, upper base radius
[IMAGE:1]
, and height
[IMAGE:2]
fits inside a hollow cylinder. The cylinder has the same internal radius as the lower base radius of the frustum and a height equal to the height of the frustum. What fraction of the empty space is occupied in the cylinder?



\textbf{A.} [IMAGE:0] \\
\textbf{B.} [IMAGE:1] \\
\textbf{C.} [IMAGE:2] \\
\textbf{D.} [IMAGE:3] \\

\textbf{Answer:} C \\
\textbf{Explanation:} The volume of the frustum
[IMAGE:0]
. The volume of the cylinder
[IMAGE:1]
. The ratio
[IMAGE:2]
.
Thus, the answer is
[IMAGE:3]

\hrule
\vspace{1em}


\noindent
\textbf{Q82.} As the diagram indicates: An object with weight 6N is on a slope with angle 30 degrees to the horizontal direction; The coefficient of friction is 0.4. An external force is exerted horizontally to the object: Find the minimum force that is required to move the object upwards



\textbf{A.} [IMAGE:0] \\
\textbf{B.} [IMAGE:1] \\
\textbf{C.} [IMAGE:2] \\
\textbf{D.} [IMAGE:3] \\

\textbf{Answer:} A \\
\textbf{Explanation:} [IMAGE:0]
[IMAGE:1]

\hrule
\vspace{1em}


\noindent
\textbf{Q83.} A soldering iron has a copper tip of mass 2.0g.
The tip is heated with 30W of thermal power. In 40s, the temperature of the tip increases by
[IMAGE:0]
.
How much energy is transferred from the tip to the surroundings in this time? (specific heat capacity of copper =
[IMAGE:1]
).



\textbf{A.} 320J \\
\textbf{B.} 480J \\
\textbf{C.} 640J \\
\textbf{D.} 880J \\

\textbf{Answer:} D \\
\textbf{Explanation:} By the conservation of energy; during this time; the energy to heat the tip minus the energy dissipated into the air equals the energy to raise the temperature of the tip: therefore
[IMAGE:0]

\hrule
\vspace{1em}


\noindent
\textbf{Q84.} A solid frustum of a cone with lower base radius
[IMAGE:0]
, upper base radius
[IMAGE:1]
, and height
[IMAGE:2]
fits inside a hollow cylinder. The cylinder has the same internal radius as the lower base radius of the frustum and a height equal to the height of the frustum. What fraction of the space inside the cylinder is occupied by the frustum?



\textbf{A.} [IMAGE:0] \\
\textbf{B.} [IMAGE:1] \\
\textbf{C.} [IMAGE:2] \\
\textbf{D.} [IMAGE:3] \\

\textbf{Answer:} D \\
\textbf{Explanation:} The volume of the frustum
[IMAGE:0]
. The volume of the cylinder
[IMAGE:1]
. The ratio
[IMAGE:2]
.

\hrule
\vspace{1em}


\noindent
\textbf{Q85.} As the diagram indicates, the spring stiffness of each spring is 30N/mm; the box weight at the bottom is 36N; find the distance of the spring A descends:



\textbf{A.} 0.3mm \\
\textbf{B.} 0.6mm \\
\textbf{C.} 1.6mm \\
\textbf{D.} 3mm \\

\textbf{Answer:} B \\
\textbf{Explanation:} By parallel connection, the stiffness of the combined spring is 60N/mm, and the force is equal to the box weight which is 36N. The extension of the spring A is therefore 0.6mm.

\hrule
\vspace{1em}


\noindent
\textbf{Q86.} A soldering iron has a copper tip of mass 2.0g.
The tip is heated with 20W of thermal power. In 40s, the temperature of the tip increases by
[IMAGE:0]
.
How much energy is transferred from the tip to the surroundings in this time? (specific heat capacity of copper =
[IMAGE:1]
).



\textbf{A.} 320J \\
\textbf{B.} 480J \\
\textbf{C.} 640J \\
\textbf{D.} 1200J \\

\textbf{Answer:} C \\
\textbf{Explanation:} By the conservation of energy; during this time; the energy to heat the tip minus the energy dissipated into the air equals the energy to raise the temperature of the tip: therefore
[IMAGE:0]
.

\hrule
\vspace{1em}


\noindent
\textbf{Q87.} As the diagram indicates, the spring stiffness of each spring is 30N/mm; the weight at the bottom is 36N; find the distance of the weight descends:



\textbf{A.} 1mm \\
\textbf{B.} 3mm \\
\textbf{C.} 4mm \\
\textbf{D.} 7.5mm \\

\textbf{Answer:} B \\
\textbf{Explanation:} By parallel connection, the stiffness of the combined spring is 60N/mm, the below part is 15N/mm; the resultant stiffness is 12N/mm(1/(1/60+1/15)). The extension is therefore 3mm.

\hrule
\vspace{1em}


\noindent
\textbf{Q88.} A solid regular octahedron with edge length
[IMAGE:0]
fits inside a hollow sphere. The sphere has a diameter equal to the distance between two opposite vertices of the octahedron. What fraction of the space inside the sphere is taken up by the octahedron?



\textbf{A.} [IMAGE:0] \\
\textbf{B.} [IMAGE:1] \\
\textbf{C.} [IMAGE:2] \\
\textbf{D.} [IMAGE:3] \\

\textbf{Answer:} C \\
\textbf{Explanation:} The distance between two opposite vertices of a regular octahedron with edge length
[IMAGE:0]
is
[IMAGE:1]
. The volume of the octahedron
[IMAGE:2]
. The volume of the sphere
[IMAGE:3]
. The ratio
[IMAGE:4]
.

\hrule
\vspace{1em}


\noindent
\textbf{Q89.} Mike heats ice cubes and observes the physical changes of ice. During this process, he measures and draws a graph of temperature versus time, as shown in the figure below. Based on the figure, which of the following analysis is correct?



\textbf{A.} [IMAGE:0] \\
\textbf{B.} [IMAGE:1] \\
\textbf{C.} [IMAGE:2] \\
\textbf{D.} [IMAGE:3] \\

\textbf{Answer:} B \\
\textbf{Explanation:} A. The AB segment in the figure represents the temperature rise of ice, while the BC segment represents the melting process of ice; hence, A is incorrect.
B. From the figure, it can be seen that the temperature of ice remains constant at 0\circ C, indicating that ice is a crystal; hence, B is correct.
C. Since the mass of ice and water is the same and they are heated by the same alcohol lamp, the slow rise in water temperature indicates that the specific heat capacity of water is larger than that of ice; hence, C is incorrect.
D. The temperature of boiling water remains constant, but it requires continuous heat absorption. If heating is stopped, boiling will also stop; hence, D is incorrect.

\hrule
\vspace{1em}


\noindent
\textbf{Q90.} As the diagram indicates, the spring stiffness of each spring is 5N/mm; the weight at the bottom is 15N; find the distance of the weight descends:



\textbf{A.} 1mm \\
\textbf{B.} 2.5mm \\
\textbf{C.} 4mm \\
\textbf{D.} 7.5mm \\

\textbf{Answer:} D \\
\textbf{Explanation:} By parallel connection, the stiffness of the combined spring is 10N/mm, the below part is 2.5N/mm; the resultant stiffness is 2N/mm(1/(1/10+1/2.5)). The extension is therefore 7.5mm.

\hrule
\vspace{1em}


\noindent
\textbf{Q91.} A solid square pyramid with a square base of side length
[IMAGE:0]
and height
[IMAGE:1]
fits inside a hollow cube. The cube has an edge length equal to the slant height of the pyramid. And
[IMAGE:2]
. What fraction of the space inside the cube is occupied by the pyramid?



\textbf{A.} [IMAGE:0] \\
\textbf{B.} [IMAGE:1] \\
\textbf{C.} [IMAGE:2] \\
\textbf{D.} [IMAGE:3] \\

\textbf{Answer:} C \\
\textbf{Explanation:} The slant height of the pyramid
[IMAGE:0]
. The volume of the pyramid
[IMAGE:1]
. The volume of the cube
[IMAGE:2]
. , Thus, the ratio
[IMAGE:3]
.

\hrule
\vspace{1em}


\noindent
\textbf{Q92.} A glass of water is placed in a refrigerator; with a mass of 0.1 kg and initial temperature of 10 degrees; Find the equilibrium temperature: The latent heat of ice is 300kJ/kg; the specific heat capacity of water is 2.09 kJ/(kg\cdot degree); the specific heat capacity of water is 4200 kJ/(kg\cdot degree); Under standard atmospheric pressure , water can completely freeze and the freezing temperature (i.e., the ice point) of the water is -20 degrees Celsius.
Calculate the amount of heat that needs to be released for the water in this cup to turn into ice at 0 degrees Celsius.



\textbf{A.} 1631.42KJ \\
\textbf{B.} 3600.34KJ \\
\textbf{C.} 4234.18KJ \\
\textbf{D.} 4260.00KJ \\

\textbf{Answer:} C \\
\textbf{Explanation:} Assume that the equilibrium temperature is x; the heat released by the water equals the heat absorbs the ice:
[IMAGE:0]

\hrule
\vspace{1em}


\noindent
\textbf{Q93.} A 5 kg object is moving with an initial velocity of 6 m/s. It collides with a stationary object of mass 50 kg. After the collision, the 5 kg object moves with a velocity of -1 m/s. What is the velocity of the 3 kg object after the collision? The table is without friction.



\textbf{A.} 0.3m/s \\
\textbf{B.} 0.5m/s \\
\textbf{C.} 0.7m/s \\
\textbf{D.} 1m/s \\

\textbf{Answer:} C \\
\textbf{Explanation:} The answer is C.
By conservation of momentum, The velocity * mass of the entire system is conserved before and after the collision.
Pay attention to the direction, or the option B will be a disturbance term.

\hrule
\vspace{1em}


\noindent
\textbf{Q94.} A solid torus (doughnut - shaped) with inner radius
[IMAGE:0]
and outer radius
[IMAGE:1]
fits inside a hollow cylinder. And its height is
[IMAGE:2]
. The cylinder has a radius equal to the outer radius of the torus and a height equal to the (outer) diameter of the torus's tube. What fraction of the space inside the cylinder is taken up by the torus?



\textbf{A.} [IMAGE:0] \\
\textbf{B.} [IMAGE:1] \\
\textbf{C.} [IMAGE:2] \\
\textbf{D.} [IMAGE:3] \\

\textbf{Answer:} D \\
\textbf{Explanation:} The volume of the torus
[IMAGE:0]
. The volume of the cylinder
[IMAGE:1]
. The ratio
[IMAGE:2]
.

\hrule
\vspace{1em}


\noindent
\textbf{Q95.} A glass of water is placed in a refrigerator; with a mass of 0.2 kg and initial temperature of 5 degrees; Find the equilibrium temperature: The latent heat of ice is 300kJ/kg; the specific heat capacity of water is 2.09 kJ/(kg\cdot degree); the specific heat capacity of water is 4200 kJ/(kg\cdot degree); Under standard atmospheric pressure , water can completely freeze and the freezing temperature (i.e., the ice point) of the water is 0 degrees Celsius.
Calculate the amount of heat that needs to be released for the water in this cup to turn into ice at 0 degrees Celsius.



\textbf{A.} 1630KJ \\
\textbf{B.} 3600KJ \\
\textbf{C.} 4200KJ \\
\textbf{D.} 4260KJ \\

\textbf{Answer:} D \\
\textbf{Explanation:} Assume that the equilibrium temperature is x; the heat released by the water equals the heat absorbs the ice:
[IMAGE:0]

\hrule
\vspace{1em}


\noindent
\textbf{Q96.} A 5 kg object is moving with an initial velocity of 6 m/s. It collides with a stationary object of mass 2 kg. After the collision, the 5 kg object moves with a velocity of 4 m/s. What is the velocity of the 3 kg object after the collision? The table is without friction.



\textbf{A.} 3m/s \\
\textbf{B.} 5m/s \\
\textbf{C.} 7m/s \\
\textbf{D.} 8m/s \\

\textbf{Answer:} B \\
\textbf{Explanation:} By conservation of momentum, The velocity * mass of the entire system is conserved before and after the collision.

\hrule
\vspace{1em}


\noindent
\textbf{Q97.} A solid regular tetrahedron with edge length
[IMAGE:0]
fits inside a hollow cube. The cube has an edge length equal to the edge length of the tetrahedron. What fraction of the space inside the cube is occupied by the tetrahedron?



\textbf{A.} [IMAGE:0] \\
\textbf{B.} [IMAGE:1] \\
\textbf{C.} [IMAGE:2] \\
\textbf{D.} [IMAGE:3] \\

\textbf{Answer:} A \\
\textbf{Explanation:} he height of a regular tetrahedron with edge length
[IMAGE:0]
is
[IMAGE:1]
. The volume of the tetrahedron
[IMAGE:2]
and the volume of the cube
[IMAGE:3]
. The ratio
[IMAGE:4]
.

\hrule
\vspace{1em}


\noindent
\textbf{Q98.} A car goes to a rest with constant speed 50m/s. At a certain moment, it decelerates uniformly at -2 m/s² until it stops. Then, it continues to accelerate uniformly at 3m/s² to 30m/s .What is the total of the deceleration-acceleration process.



\textbf{A.} 10s \\
\textbf{B.} 25s \\
\textbf{C.} 35s \\
\textbf{D.} 45s \\

\textbf{Answer:} C \\
\textbf{Explanation:} [IMAGE:0]

\hrule
\vspace{1em}


\noindent
\textbf{Q99.} A piece of ice undergoes the following three processes:
1.
Ice at -10°C is heated to 0°C, absorbing heat Q1;
2.
Ice at 0°C melts into water at 0°C, absorbing heat Q2;
3.
Water at 10°C is heated to 20°C, absorbing heat Q3.
It is known that the specific heat capacity of ice is less than that of water. The mass remains constant throughout the entire process. Which of the following is true?



\textbf{A.} [IMAGE:0] \\
\textbf{B.} [IMAGE:1] \\
\textbf{C.} [IMAGE:2] \\
\textbf{D.} [IMAGE:3] \\

\textbf{Answer:} C \\
\textbf{Explanation:} To compare the heat absorbed in each process, we need to use the formulas for heat absorption:
1.
Heat absorbed by ice from -10°C to 0°C:
[IMAGE:0]
2.
Heat absorbed by ice melting at 0°C:
[IMAGE:1]
where
[IMAGE:2]
is the latent heat of fusion for ice.
3.
Heat absorbed by water from 10°C to 20°C:
[IMAGE:3]
Given that the specific heat capacity of ice
[IMAGE:4]
is less than that of water
[IMAGE:5]
, we have:
[IMAGE:6]
Thus,
[IMAGE:7]
.
The latent heat of fusion
[IMAGE:8]
for ice is generally much larger than the specific heat capacities
[IMAGE:9]
and
[IMAGE:10]
multiplied by the temperature change. Therefore:
[IMAGE:11]
Combining these results, we get:
[IMAGE:12]

\hrule
\vspace{1em}


\noindent
\textbf{Q100.} A solid hemisphere of radius R
is inside a hollow cone. The cone has a circular base with radius 2R
and a height equal to 3R
. What fraction of the space inside the cone is taken up by the hemisphere?



\textbf{A.} [IMAGE:0] \\
\textbf{B.} [IMAGE:1] \\
\textbf{C.} [IMAGE:2] \\
\textbf{D.} [IMAGE:3] \\

\textbf{Answer:} D \\
\textbf{Explanation:} [IMAGE:0]
.

\hrule
\vspace{1em}


\noindent
\textbf{Q101.} A car goes to a rest with constant speed 50m/s. At a certain moment, it decelerates uniformly at -2 m/s² for 10 s. What is the final velocity of the car?



\textbf{A.} 10m/s \\
\textbf{B.} 30m/s \\
\textbf{C.} 40m/s \\
\textbf{D.} 70m/s \\

\textbf{Answer:} B \\
\textbf{Explanation:} [IMAGE:0]

\hrule
\vspace{1em}


\noindent
\textbf{Q102.} A solid right - circular cone with radius r=3R
and height h=4R
fits inside a hollow cylinder. The cylinder has the same internal radius as the cone and a height equal to the slant height of the cone. What fraction of the space inside the cylinder is occupied by the cone?



\textbf{A.} [IMAGE:0] \\
\textbf{B.} [IMAGE:1] \\
\textbf{C.} [IMAGE:2] \\
\textbf{D.} [IMAGE:3] \\

\textbf{Answer:} D \\
\textbf{Explanation:} [IMAGE:0]
.

\hrule
\vspace{1em}


\noindent
\textbf{Q103.} A solid cube of side length
[IMAGE:0]
fits perfectly inside a hollow rectangular prism. The rectangular prism has the same internal length and width as the side length of the cube, and its height is equal to the diagonal of the cube's base. What fraction of the space inside the rectangular prism is taken up by the cube?



\textbf{A.} [IMAGE:0] \\
\textbf{B.} [IMAGE:1] \\
\textbf{C.} [IMAGE:2] \\
\textbf{D.} [IMAGE:3] \\

\textbf{Answer:} E \\
\textbf{Explanation:} [IMAGE:0]
.

\hrule
\vspace{1em}


\noindent
\textbf{Q104.} A car starts from rest and accelerates uniformly at 1 m/s² for 1 min. What is the final velocity of the car?



\textbf{A.} 1m/s \\
\textbf{B.} 30m/s \\
\textbf{C.} 60m/s \\
\textbf{D.} 90m/s \\

\textbf{Answer:} C \\
\textbf{Explanation:} Using the SUVAT equation:
[IMAGE:0]
, where: u
is the final velocity,
[IMAGE:1]
is the initial velocity (0 m/s, as the car starts from rest), a
is the acceleration (1 m/s²), and
[IMAGE:2]
is the time (1minute = 60 seconds).
Substituting the values:
[IMAGE:3]
Thus, the final velocity is 60 m/s. And option A is a disturbance term.

\hrule
\vspace{1em}


\noindent
\textbf{Q105.} The ice is submerged into a glass of water; the 1.0kg ice is at -10 degrees; 0.2 kg water is at 5 degrees; Find the equilibrium temperature: The latent heat of ice is 300kJ/kg; the specific heat capacity of water is 2.09 kJ/(kg\cdot degree); the specific heat capacity of water is 4200 kJ/(kg\cdot degree); The ice can completely melt when the temperature between 0 degrees and 5 degrees; the melting point of ice under standard atmospheric pressure is 0 degrees:



\textbf{A.} 5.23 \\
\textbf{B.} 8.7 \\
\textbf{C.} 7.7 \\
\textbf{D.} 0.87 \\

\textbf{Answer:} E \\
\textbf{Explanation:} Assume that the equilibrium temperature is x; the heat released by the water equals the heat absorbs the ice:
[IMAGE:0]
[IMAGE:1]

\hrule
\vspace{1em}


\noindent
\textbf{Q106.} A solid sphere of radius r fits inside a hollow cylinder. The cylinder has the same internal diameter and length as the diameter of the sphere. What fraction of the empty space inside the cylinder is taken up?



\textbf{A.} [IMAGE:0] \\
\textbf{B.} [IMAGE:1] \\
\textbf{C.} [IMAGE:2] \\
\textbf{D.} [IMAGE:3] \\

\textbf{Answer:} A \\
\textbf{Explanation:} [IMAGE:0]
.

\hrule
\vspace{1em}


\noindent
\textbf{Q107.} A rectangular prism has dimensions 2, 4, and 3. What is the perimeter of a triangular ABC (the dashed line triangular in the diagram below)? B is in the center of the bottom face. A and C lie in the vertices of the rectangular prism.



\textbf{A.} [IMAGE:0] \\
\textbf{B.} [IMAGE:1] \\
\textbf{C.} [IMAGE:2] \\
\textbf{D.} [IMAGE:3] \\

\textbf{Answer:} B \\
\textbf{Explanation:} [IMAGE:0]
[IMAGE:1]
Thus,
[IMAGE:2]

\hrule
\vspace{1em}


\noindent
\textbf{Q108.} The ice is submerged into a glass of water; the 1.0kg ice is at -30 degrees; 1 kg water is at 5 degrees; Find the equilibrium temperature: The latent heat of ice is 300kJ/kg; the specific heat capacity of water is 2.09 kJ/(kg\cdot degree); the specific heat capacity of water is 4200 kJ/(kg\cdot degree); The ice can completely melt when the temperature between 0 degrees and 5 degrees; the melting point of ice under standard atmospheric pressure is 0 degrees:



\textbf{A.} 5.23 \\
\textbf{B.} 10.00 \\
\textbf{C.} 3.42 \\
\textbf{D.} 2.46 \\

\textbf{Answer:} D \\
\textbf{Explanation:} Assume that the equilibrium temperature is x; the heat released by the water equals the heat absorbs the ice:
[IMAGE:0]
[IMAGE:1]

\hrule
\vspace{1em}


\noindent
\textbf{Q109.} Which of the following is correctly classified as a scalar quantity?



\textbf{A.} Current \\
\textbf{B.} Friction \\
\textbf{C.} Electric field intensity \\
\textbf{D.} Momentum \\

\textbf{Answer:} A \\
\textbf{Explanation:} Current is a scalar quantity because it has only magnitude. (It is noted that the current has "fake direction" because it does not meet the parallelogram law.)
Friction is a vector quantity, as it has both magnitude and direction.
Electric field intensity is a vector quantity, as it has both magnitude and direction.
Momentum is a vector quantity, as it has both magnitude and direction.

\hrule
\vspace{1em}


\noindent
\textbf{Q110.} A cube has sides of length 2. What is the area of a triangular ABC (the dashed line triangular in the diagram below)?



\textbf{A.} [IMAGE:0] \\
\textbf{B.} [IMAGE:1] \\
\textbf{C.} [IMAGE:2] \\
\textbf{D.} [IMAGE:3] \\

\textbf{Answer:} D \\
\textbf{Explanation:} [IMAGE:0]
.

\hrule
\vspace{1em}


\noindent
\textbf{Q111.} The ice is submerged into a glass of water; the 1.0kg ice is at 0 degrees; 1 kg water is at 5 degrees; Find the equilibrium temperature: The latent heat of ice is 300kJ/kg; the specific heat capacity of water is 4200 kJ/Kg*degree; The ice can completely melt when the temperature between 0 degrees and 5 degrees; the melting point of ice under standard atmospheric pressure is 0 degrees



\textbf{A.} 5.2 \\
\textbf{B.} 2.5 \\
\textbf{C.} 3.2 \\
\textbf{D.} 10.0 \\

\textbf{Answer:} B \\
\textbf{Explanation:} Assume that the equilibrium temperature is x; the heat released by the water equals the heat absorbs the ice:
[IMAGE:0]
where
[IMAGE:1]
corresponds to the process of ice at 0 degrees turning into water at 0 degrees,
[IMAGE:2]
corresponds to the process of the water formed from the ice absorbing heat, and
[IMAGE:3]
corresponds to the process of the original water in the glass releasing heat.

\hrule
\vspace{1em}


\noindent
\textbf{Q112.} A cube has sides of length 2. What is the perimeter of a triangular ABC (the dashed line triangular in the diagram below)?
[IMAGE:0]



\textbf{A.} [IMAGE:0] \\
\textbf{B.} [IMAGE:1] \\
\textbf{C.} [IMAGE:2] \\
\textbf{D.} [IMAGE:3] \\

\textbf{Answer:} B \\
\textbf{Explanation:} [IMAGE:0]
.

\hrule
\vspace{1em}


\noindent
\textbf{Q113.} Which of the following is correctly classified as a vector quantity?



\textbf{A.} Distance \\
\textbf{B.} Speed \\
\textbf{C.} Time \\
\textbf{D.} Momentum \\

\textbf{Answer:} D \\
\textbf{Explanation:} Momentum is a vector quantity because it has both magnitude and direction.
Distance is a scalar quantity, as it only has magnitude.
Speed is a scalar quantity, as it only has magnitude.
Time is a scalar quantity, as it only has magnitude.

\hrule
\vspace{1em}


\noindent
\textbf{Q114.} The ice is submerged into a glass of water; the 2.0kg ice is at 0 degrees; 1 kg water is at 30 degrees; Find the equilibrium temperature: The latent heat of ice is 300kJ/kg; the specific heat capacity of water is 4200 kJ/Kg*degree; The ice can completely melt when the temperature between 0 degrees and 30 degrees; the melting point of ice under standard atmospheric pressure is 0 degrees:



\textbf{A.} 5.6 \\
\textbf{B.} 7.4 \\
\textbf{C.} 6.6 \\
\textbf{D.} 10.0 \\

\textbf{Answer:} D \\
\textbf{Explanation:} Assume that the equilibrium temperature is x; the heat released by the water equals the heat absorbs the ice:
[IMAGE:0]
where
[IMAGE:1]
corresponds to the process of ice at 0 degrees turning into water at 0 degrees,
[IMAGE:2]
corresponds to the process of the water formed from the ice absorbing heat, and
[IMAGE:3]
corresponds to the process of the original water in the glass releasing heat.

\hrule
\vspace{1em}


\noindent
\textbf{Q115.} A cube has sides of length 5. What is the length of a line joining the midpoint of one face to the midpoint of an adjacent face (the dashed line in the diagram below)?



\textbf{A.} [IMAGE:0] \\
\textbf{B.} [IMAGE:1] \\
\textbf{C.} [IMAGE:2] \\
\textbf{D.} [IMAGE:3] \\

\textbf{Answer:} A \\
\textbf{Explanation:} [IMAGE:0]
.

\hrule
\vspace{1em}


\noindent
\textbf{Q116.} Which of the following is correctly classified as a vector quantity?



\textbf{A.} Distance \\
\textbf{B.} Velocity \\
\textbf{C.} Time \\
\textbf{D.} Humidity \\

\textbf{Answer:} B \\
\textbf{Explanation:} Velocity is a vector quantity because it has both magnitude and direction (e.g., 10 m/s to the north), which is different from the "Speed".
Distance is a scalar quantity, as it only has magnitude.
Time is a scalar quantity, as it only has magnitude.
Humidity is a scalar quantity, as it only has magnitude.

\hrule
\vspace{1em}


\noindent
\textbf{Q117.} A rectangular prism has dimensions 1, 2, and 3. What is the length of a line joining a
center point of a face
to the midpoint of
an adjacent
middle edge (the dashed line in the diagram below)?
[IMAGE:0]



\textbf{A.} [IMAGE:0] \\
\textbf{B.} [IMAGE:1] \\
\textbf{C.} [IMAGE:2] \\
\textbf{D.} [IMAGE:3] \\

\textbf{Answer:} C \\
\textbf{Explanation:} [IMAGE:0]

\hrule
\vspace{1em}


\noindent
\textbf{Q118.} The ice is submerged into a glass of water; the 0.5kg ice is at 0 degrees; 0.5 kg water is at 20 degrees; Find the equilibrium temperature: The latent heat of ice is 300kJ/kg; the specific heat capacity of water is 4200 kJ/Kg*degree; The ice can completely melt when the temperature between 0 degrees and 20 degrees; the melting point of ice under standard atmospheric pressure is 0 degrees:



\textbf{A.} 5.6 \\
\textbf{B.} 7.4 \\
\textbf{C.} 3.6 \\
\textbf{D.} 10.0 \\

\textbf{Answer:} D \\
\textbf{Explanation:} Assume that the equilibrium temperature is x; the heat released by the water equals the heat absorbs the ice:
[IMAGE:0]
where
[IMAGE:1]
corresponds to the process of ice at 0 degrees turning into water at 0 degrees,
[IMAGE:2]
corresponds to the process of the water formed from the ice absorbing heat, and
[IMAGE:3]
corresponds to the process of the original water in the glass releasing heat.

\hrule
\vspace{1em}


\noindent
\textbf{Q119.} A cube has sides of length 4. What is the length of a line joining a vertex to the midpoint of a face diagonal on the opposite face (the dashed line in the diagram below)?



\textbf{A.} [IMAGE:0] \\
\textbf{B.} [IMAGE:1] \\
\textbf{C.} [IMAGE:2] \\
\textbf{D.} [IMAGE:3] \\

\textbf{Answer:} B \\
\textbf{Explanation:} [IMAGE:0]
.

\hrule
\vspace{1em}


\noindent
\textbf{Q120.} The ice is submerged into a glass of water; the 0.5kg ice is at 0 degrees; 1 kg water is at 15 degrees; Find the equilibrium temperature: The latent heat of ice is 300kJ/kg; the specific heat capacity of water is 4200 kJ/Kg*degree; The ice can completely melt when the temperature between 0 degrees and 15 degrees; the melting point of ice under standard atmospheric pressure is 0 degrees:



\textbf{A.} 5.6 \\
\textbf{B.} 7.4 \\
\textbf{C.} 3.2 \\
\textbf{D.} 10 \\

\textbf{Answer:} D \\
\textbf{Explanation:} Assume that the equilibrium temperature is x; the heat released by the water equals the heat absorbs the ice:
[IMAGE:0]
where
[IMAGE:1]
corresponds to the process of ice at 0 degrees turning into water at 0 degrees,
[IMAGE:2]
corresponds to the process of the water formed from the ice absorbing heat, and
[IMAGE:3]
corresponds to the process of the original water in the glass releasing heat.

\hrule
\vspace{1em}


\noindent
\textbf{Q121.} A cube has sides of length 3. What is the length of a line joining the midpoint of one edge to the midpoint of an opposite edge (the dashed line in the diagram below)?



\textbf{A.} [IMAGE:0] \\
\textbf{B.} [IMAGE:1] \\
\textbf{C.} [IMAGE:2] \\
\textbf{D.} [IMAGE:3] \\

\textbf{Answer:} C \\
\textbf{Explanation:} [IMAGE:0]
.

\hrule
\vspace{1em}


\noindent
\textbf{Q122.} A transformer with 81% efficiency has its 400-turn primary connected to 200V AC. The 100-turn secondary supplies power to parallel resistors of 50Ω and 200Ω through 50Ω cables.
What is the total power consumed by the resistors?



\textbf{A.} 4W \\
\textbf{B.} 8W \\
\textbf{C.} 10W \\
\textbf{D.} 20W \\

\textbf{Answer:} C \\
\textbf{Explanation:} Turns ratio = 400:100 = 4:1. Ideal secondary voltage = 200V/4 = 50V. Actual voltage = \sqrt{}0.81×50=45V. Parallel resistance = (50×200)/(50+200) =40Ω. Total resistance = 40+50 = 90Ω. Current = 45V/90Ω =0.5A. Resistor power = (0.5)²×40 =10W

\hrule
\vspace{1em}


\noindent
\textbf{Q123.} A stone with mass 4kg is rising against air resistance with the initial velocity
[IMAGE:0]
(the rising direction is the positive direction); the air resistance value is shown by:
[IMAGE:1]
.
Where v is the velocity of the stone; g, gravitational acceleration is taken as
[IMAGE:2]
. Find the terminal velocity:



\textbf{A.} 0.5m/s \\
\textbf{B.} 1m/s \\
\textbf{C.} 1.5m/s \\
\textbf{D.} 2m/s \\

\textbf{Answer:} D \\
\textbf{Explanation:} The rising process is a misleading information and the initial velocity is unused. And the terminal velocity is found when air resistance equals the weight, the object is no longer to accelerate or decelerate.

\hrule
\vspace{1em}


\noindent
\textbf{Q124.} A rectangular prism has dimensions 2, 4, and 3. What is the length of a line joining a vertex to the center of the opposite face (the dashed line in the diagram below)?



\textbf{A.} [IMAGE:0] \\
\textbf{B.} [IMAGE:1] \\
\textbf{C.} [IMAGE:2] \\
\textbf{D.} [IMAGE:3] \\

\textbf{Answer:} A \\
\textbf{Explanation:} [IMAGE:0]

\hrule
\vspace{1em}


\noindent
\textbf{Q125.} A 120V AC source powers a 64%-efficient transformer with 500 primary turns. The 2000-turn secondary is connected to a device with 400Ω resistance through cables adding 100Ω.
What voltage is actually applied to the device?



\textbf{A.} 240.5V \\
\textbf{B.} 307.2V \\
\textbf{C.} 480.6V \\
\textbf{D.} 540.5V \\

\textbf{Answer:} B \\
\textbf{Explanation:} Turns ratio = 500:2000 = 1:4. Ideal secondary voltage = 120V×4 = 480V. Actual voltage = \sqrt{}0.64×480 =384V. Voltage divider gives device voltage = (400/500)×384 =307.2V.

\hrule
\vspace{1em}


\noindent
\textbf{Q126.} A cube has sides of length 2. What is the length of a line joining a vertex to the midpoint of one of the opposite edges (the dashed line in the diagram below)?



\textbf{A.} [IMAGE:0] \\
\textbf{B.} [IMAGE:1] \\
\textbf{C.} [IMAGE:2] \\
\textbf{D.} [IMAGE:3] \\

\textbf{Answer:} D \\
\textbf{Explanation:} [IMAGE:0]
.

\hrule
\vspace{1em}


\noindent
\textbf{Q127.} A step-down transformer (3000 turns primary, 150 turns secondary) with 81% efficiency powers a 20Ω light bulb through cables of resistance R. The primary voltage is 600V. The power of the light bulb is 10W.
What is the value of R?



\textbf{A.} 3Ω \\
\textbf{B.} 7Ω \\
\textbf{C.} 8Ω \\
\textbf{D.} 10Ω \\

\textbf{Answer:} B \\
\textbf{Explanation:} Turns ratio = 3000:150 = 20:1. Ideal secondary voltage = 600V/20 = 30V. Actual voltage = \sqrt{}0.81×30 =27V. Total resistance = 10+R Ω. Total power = (27)²/(10+R) W.
Thus,
[IMAGE:0]
[IMAGE:1]

\hrule
\vspace{1em}


\noindent
\textbf{Q128.} A cube has sides of unit length. What is the length of a line joining a vertex to the center of cube (the dashed line in the diagram below)?



\textbf{A.} [IMAGE:0] \\
\textbf{B.} [IMAGE:1] \\
\textbf{C.} [IMAGE:2] \\
\textbf{D.} [IMAGE:3] \\

\textbf{Answer:} C \\
\textbf{Explanation:} [IMAGE:0]
.

\hrule
\vspace{1em}


\noindent
\textbf{Q129.} A stone with mass 0.2kg is falling against air resistance with falling direction being the positive direction; the air resistance is shown by:
[IMAGE:0]
. Where v is the velocity of the stone; g, gravitational acceleration is taken as
[IMAGE:1]
. Find the terminal velocity:



\textbf{A.} [IMAGE:0] \\
\textbf{B.} [IMAGE:1] \\
\textbf{C.} [IMAGE:2] \\
\textbf{D.} [IMAGE:3] \\

\textbf{Answer:} B \\
\textbf{Explanation:} [IMAGE:0]

\hrule
\vspace{1em}


\noindent
\textbf{Q130.} The secondary coil of an 81% efficient transformer is connected to a resistor by cables of total resistance 1000 Ω. The voltage in the primary coil is 1000 V. There are 120 turns in the primary coil and 480 turns in the secondary coil.
What is the power produced as heat in the cables?



\textbf{A.} 810W \\
\textbf{B.} 1000W \\
\textbf{C.} 1300W \\
\textbf{D.} 12960W \\

\textbf{Answer:} D \\
\textbf{Explanation:} If the transformer is 100% efficient; then, the primary coil is 1000V, the ratio of voltage is 1:4, the voltage in the secondary coil is therefore: 4000V. However, the transformer is 81% efficient. Thus, the actual voltage in the secondary coil is therefore:
[IMAGE:0]
. the power of the resistor is therefore
[IMAGE:1]
.

\hrule
\vspace{1em}


\noindent
\textbf{Q131.} Which one of the following about nuclear stability is true?



\textbf{A.} [IMAGE:0] \\
\textbf{B.} [IMAGE:1]
 \\
\textbf{C.} [IMAGE:2] \\
\textbf{D.} [IMAGE:3] \\

\textbf{Answer:} C \\
\textbf{Explanation:} Nuclei with magic numbers of protons and neutrons (2, 8, 20, 28, 50, 82, 126) have closed shells and are more stable.

\hrule
\vspace{1em}


\noindent
\textbf{Q132.} Which statement about nuclear fusion is correct?



\textbf{A.} Nuclear fusion occurs at room temperature and pressure. \\
\textbf{B.} The fuel for nuclear fusion is hard to obtain and store. \\
\textbf{C.} The energy released in a nuclear fusion reaction is less than that in a nuclear fission reaction. \\
\textbf{D.} Nuclear fusion is the process that powers the sun and other stars. \\

\textbf{Answer:} D \\
\textbf{Explanation:} The sun and stars generate energy through nuclear fusion of hydrogen into helium.

\hrule
\vspace{1em}


\noindent
\textbf{Q133.} The secondary coil of an 100% efficient transformer is connected to a resistor by cables of total resistance 1000 Ω. The voltage in the primary coil is 1000 V. There are 120 turns in the primary coil and 480 turns in the secondary coil.
What is the power produced as heat in the cables?



\textbf{A.} 810W \\
\textbf{B.} 1000W \\
\textbf{C.} 1300W \\
\textbf{D.} 12960W \\

\textbf{Answer:} E \\
\textbf{Explanation:} The transformer is 100% efficient; then, the primary coil is 1000V, the ratio of voltage is 1:4, the voltage in the secondary coil is: 4000V. The power of the resistor is therefore
[IMAGE:0]
.

\hrule
\vspace{1em}


\noindent
\textbf{Q134.} A stone with mass 2kg is falling against air resistance; the air resistance is shown by:
[IMAGE:0]
.
Where v is the velocity of the stone; g, gravitational acceleration is taken as
[IMAGE:1]
. Find the terminal velocity:



\textbf{A.} 0.5 m/s \\
\textbf{B.} 1 m/s \\
\textbf{C.} 1.5 m/s \\
\textbf{D.} 2 m/s \\

\textbf{Answer:} A \\
\textbf{Explanation:} The terminal velocity is found when air resistance equals the weight, the object is no longer to accelerate or decelerate.

\hrule
\vspace{1em}


\noindent
\textbf{Q135.} The secondary coil of an 81% efficient transformer is connected to a resistor by cables of total resistance 1000 Ω. The current in the primary coil is 4.0 A. There are 120 turns in the primary coil and 480 turns in the secondary coil.
What is the power produced as heat in the cables?



\textbf{A.} 810W \\
\textbf{B.} 900W \\
\textbf{C.} 1000W \\
\textbf{D.} 1210W \\

\textbf{Answer:} A \\
\textbf{Explanation:} If the transformer is 100% efficient; then, the primary coil is 4A, the ratio of current is 4:1, the current in the secondary coil is therefore: 1A. However, the transformer is 81% efficient. Thus, the actual current in the secondary coil is therefore:
[IMAGE:0]
. the power of the resistor is therefore
[IMAGE:1]
.

\hrule
\vspace{1em}


\noindent
\textbf{Q136.} Resultant force and force component:
F
1
is with magnitude +1N, pointing northward; F
2
is with magnitude
[IMAGE:0]
N, pointing east-ward; Find the component of the resulting force along the direction on the bearing of 60 degrees:



\textbf{A.} 2N \\
\textbf{B.} -2N \\
\textbf{C.} [IMAGE:0] \\
\textbf{D.} 4N \\

\textbf{Answer:} A \\
\textbf{Explanation:} [IMAGE:0]

\hrule
\vspace{1em}


\noindent
\textbf{Q137.} The secondary coil of an ideal, 100% efficient transformer is connected to a resistor by cables of total resistance 9000 Ω. The current in the primary coil is 4.0 A. There are 200 turns in the primary coil and 1200 turns in the secondary coil.
What is the power produced as heat in the cables?



\textbf{A.} 600W \\
\textbf{B.} 3000W \\
\textbf{C.} 4000W \\
\textbf{D.} 6000W \\

\textbf{Answer:} C \\
\textbf{Explanation:} The transformer is 100% efficient; the primary coil is 4A; the ratio of current is 6:1; the current in the secondary coil is therefore: 2/3
[IMAGE:0]
A, the power of the resistor is therefore
[IMAGE:1]

\hrule
\vspace{1em}


\noindent
\textbf{Q138.} Resultant force and force component:
F
1
is with magnitude +1N, pointing northward; F
2
is with magnitude
[IMAGE:0]
N, pointing east-ward; Find the component of the resulting force along the direction on the bearing of 30 degrees:



\textbf{A.} 2N \\
\textbf{B.} -2N \\
\textbf{C.} [IMAGE:0] \\
\textbf{D.} 4N \\

\textbf{Answer:} A \\
\textbf{Explanation:} The resultant itself is on the direction reguired, the magnitude is also: 2N

\hrule
\vspace{1em}


\noindent
\textbf{Q139.} The secondary coil of an ideal, 100% efficient transformer is connected to a resistor by cables of total resistance 500 Ω. The current in the primary coil is 4.0 A. There are 960 turns in the primary coil and 480 turns in the secondary coil.
What is the power produced as heat in the cables?



\textbf{A.} 8000w \\
\textbf{B.} 16000w \\
\textbf{C.} 32000w \\
\textbf{D.} 48000w \\

\textbf{Answer:} C \\
\textbf{Explanation:} The transformer is 100% efficient; the primary coil is 4A; the ratio of current is 1:2; the current in the secondary coil is therefore: 8A, the power of the resistor is therefore 8*8*500=32000W

\hrule
\vspace{1em}


\noindent
\textbf{Q140.} Resultant force and force component:
F
1
is with magnitude +10N, pointing northward; F
2
is with magnitude +10N, pointing east-ward; Find the component of the resulting force along the direction on the bearing of 225 degrees:



\textbf{A.} 10N \\
\textbf{B.} -10N \\
\textbf{C.} [IMAGE:0] \\
\textbf{D.} [IMAGE:1] \\

\textbf{Answer:} D \\
\textbf{Explanation:} The resultant itself is on the direction required, the magnitude is also:
[IMAGE:0]

\hrule
\vspace{1em}


\noindent
\textbf{Q141.} The secondary coil of an ideal, 100% efficient transformer is connected to a resistor by cables of total resistance 1000 Ω. The current in the primary coil is 8.0 A. There are 360 turns in the primary coil and 480 turns in the secondary coil.
What is the power produced as heat in the cables?



\textbf{A.} 1600W \\
\textbf{B.} 3000W \\
\textbf{C.} 6000W \\
\textbf{D.} 18000W \\

\textbf{Answer:} E \\
\textbf{Explanation:} The transformer is 100% efficient; the primary coil is 4A; the ratio of current is 4:3; the current in the secondary coil is therefore: 6A, the power of the resistor is therefore 6*6*1000=36000W

\hrule
\vspace{1em}


\noindent
\textbf{Q142.} Which one of the following statements about nuclear reactions is true?



\textbf{A.} In a nuclear fusion reaction,heavy nuclei combine to form lighter nuclei \\
\textbf{B.} The energy released in a nuclear fission reaction comes from the conversion of mass into energy according to
[IMAGE:0] \\
\textbf{C.} A nuclear chain reaction can only occur in a nuclear bomb \\
\textbf{D.} All radioactive isotopes have the same half - life \\

\textbf{Answer:} B \\
\textbf{Explanation:} The famous Einstein's mass - energy equivalence formula explains the energy source in nuclear fission.

\hrule
\vspace{1em}


\noindent
\textbf{Q143.} Vectoral addition:
[IMAGE:0]
These are two vectors representing two forces, they both exert on one object with mass 2kg; Find the acceleration of the object:



\textbf{A.} [IMAGE:0] \\
\textbf{B.} [IMAGE:1] \\
\textbf{C.} [IMAGE:2] \\
\textbf{D.} [IMAGE:3] \\

\textbf{Answer:} E \\
\textbf{Explanation:} The magnitude of the resultant force is
[IMAGE:0]
by equation
F=ma;acceleration is
[IMAGE:1]

\hrule
\vspace{1em}


\noindent
\textbf{Q144.} The secondary coil of an ideal, 100% efficient transformer is connected to a resistor by cables of total resistance 6000 Ω. The current in the primary coil is 4.0 A. There are 120 turns in the primary coil and 480 turns in the secondary coil.
What is the power produced as heat in the cables?



\textbf{A.} 60w \\
\textbf{B.} 300w \\
\textbf{C.} 1500w \\
\textbf{D.} 6000w \\

\textbf{Answer:} D \\
\textbf{Explanation:} The transformer is 100% efficient; the primary coil is 4A; the ratio of current is 4:1; the current in the secondary coil is therefore: 1A, the power of the resistor is therefore 1*1*6000=6000W.

\hrule
\vspace{1em}


\noindent
\textbf{Q145.} A small train is undergoing uniformly accelerated linear motion, and its displacement as a function of time is given by
[IMAGE:0]
, where x is in meters (m) and  is in seconds (s). Which of the following statements are correct?
1.
The initial velocity of the object is
[IMAGE:1]
.
2.
The acceleration of the object is
[IMAGE:2]
.
3.
At
[IMAGE:3]
, the velocity of the object is
[IMAGE:4]
.



\textbf{A.} 1 only \\
\textbf{B.} 2 only \\
\textbf{C.} 3 only \\
\textbf{D.} 1 and 2 only \\

\textbf{Answer:} G \\
\textbf{Explanation:} For uniformly accelerated linear motion, the displacement x as a function of time t can be expressed as:
[IMAGE:0]
, where v0 is the initial velocity and a is the acceleration.
The given displacement formula is:
[IMAGE:1]
Therefore, all statements are correct, and the answer is G.

\hrule
\vspace{1em}


\noindent
\textbf{Q146.} Which one of the following statements about nuclear physics is wrong?



\textbf{A.} The process of emission of a gamma ray from a nucleus is called gamma decay \\
\textbf{B.} The half-life of a radioactive substance is the time taken for half of its nuclei to decay \\
\textbf{C.} The number of neutrons in a nucleus is its mass number minus its atomic number (proton number) \\
\textbf{D.} When a nucleus emits a beta particle, the number of practices changes \\

\textbf{Answer:} D \\
\textbf{Explanation:} Number of particles is conserved since the only change is between neutrons and protons.

\hrule
\vspace{1em}


\noindent
\textbf{Q147.} Vectoral addition:
[IMAGE:0]
These are two vectors representing two forces, they both exert on one object with mass 2kg; Find the acceleration of the object:



\textbf{A.} [IMAGE:0] \\
\textbf{B.} [IMAGE:1] \\
\textbf{C.} [IMAGE:2] \\
\textbf{D.} [IMAGE:3] \\

\textbf{Answer:} B \\
\textbf{Explanation:} The magnitude of the resultant force is 6N,by equation F=ma;acceleration is
[IMAGE:0]

\hrule
\vspace{1em}


\noindent
\textbf{Q148.} A train's acceleration is given by
[IMAGE:0]
, where
[IMAGE:1]
and
[IMAGE:2]
are constants. Which of the following is true?
1: The train's acceleration approaches a constant value as time increases.
2: The train's velocity approaches a constant value as time increases.
3: The train's displacement approaches a constant value as time increases.



\textbf{A.} none of them \\
\textbf{B.} 1 only \\
\textbf{C.} 2 only \\
\textbf{D.} 3 only \\

\textbf{Answer:} E \\
\textbf{Explanation:} The train's acceleration approaches a constant value 0 as time increases. The train's velocity approaches a constant value as time increases. The train's displacement keep increasing as time increases.

\hrule
\vspace{1em}


\noindent
\textbf{Q149.} Vectoral addition:
[IMAGE:0]
These are two vectors representing two forces, they both exert on one object with mass 2kg; Find the acceleration of the object:



\textbf{A.} [IMAGE:0] \\
\textbf{B.} [IMAGE:1] \\
\textbf{C.} [IMAGE:2] \\
\textbf{D.} [IMAGE:3] \\

\textbf{Answer:} D \\
\textbf{Explanation:} The magnitude of the resultant force is 13N,by equation F=ma; acceleration is
[IMAGE:0]

\hrule
\vspace{1em}


\noindent
\textbf{Q150.} A train's velocity is modeled by
[IMAGE:0]
, where a, b, and c are constants. Which statements hold?
1: The train's acceleration oscillates with time.
2: The train's kinetic energy oscillates with time.
3: The net force on the train is always directed opposite to its velocity.



\textbf{A.} none of them \\
\textbf{B.} 1 only \\
\textbf{C.} 2 only \\
\textbf{D.} 3 only \\

\textbf{Answer:} E \\
\textbf{Explanation:} Term 1 and term 2 is obviously right. v is periodically up and down. Thus, the net force on the train can not always directed opposite to its velocity.

\hrule
\vspace{1em}


\noindent
\textbf{Q151.} A train's position x along a track is described by
[IMAGE:0]
, where
[IMAGE:1]
are constants. Which statements are correct?
1: The train's acceleration is linear in time.
2: The train's momentum is cubic in time (assuming constant mass).
3: The parameter d can only be 0.



\textbf{A.} none of them \\
\textbf{B.} 1 only \\
\textbf{C.} 2 only \\
\textbf{D.} 3 only \\

\textbf{Answer:} B \\
\textbf{Explanation:} The train's acceleration is linear in time. The train's velocity is quartic in time so that the train's momentum is also quartic in time. The parameter d can only be any constant.

\hrule
\vspace{1em}


\noindent
\textbf{Q152.} As the diagram indicates, the spring stiffness of each spring is 12N/mm; the Force at the top is 48N; find the distance of the spring descends:



\textbf{A.} 0.25mm \\
\textbf{B.} 10mm \\
\textbf{C.} 4mm \\
\textbf{D.} 10m \\

\textbf{Answer:} C \\
\textbf{Explanation:} By parallel connection, the stiffness of the combined spring is 24N/mm, The extension is therefore 4mm.

\hrule
\vspace{1em}


\noindent
\textbf{Q153.} A train moves along a straight track such that its acceleration a varies with time as
[IMAGE:0]
, where c is a constant. Which of the following is true?
1: The train's velocity is quartic in time.
2: The train's kinetic energy is five power in time.
3: The net force on the train is cubic in time.



\textbf{A.} none of them \\
\textbf{B.} 1 only \\
\textbf{C.} 2 only \\
\textbf{D.} 3 only \\

\textbf{Answer:} A \\
\textbf{Explanation:} The train's velocity is cubic in time and The train's kinetic energy is six power in time. The net force on the train is quartic in time.

\hrule
\vspace{1em}


\noindent
\textbf{Q154.} As the diagram indicates, the spring stiffness of each spring is 5N/mm; the weight at the bottom is 20N; find the distance of the weight descends:



\textbf{A.} 10mm \\
\textbf{B.} 2mm \\
\textbf{C.} 5mm \\
\textbf{D.} 20m \\

\textbf{Answer:} A \\
\textbf{Explanation:} By parallel connection, the stiffness of the combined spring is 40N/mm, the below part is 2.5N/mm; the resultant stiffness is 2N/mm(1/10+1/2.5). The extension is therefore 5mm.

\hrule
\vspace{1em}


\noindent
\textbf{Q155.} An electric train is traveling along a straight horizontal track. Its velocity v as a function of time t is given by
[IMAGE:0]
, where a and b are constants. Which of the following statements is/are correct?
1: The train's acceleration is constant.
2: The train's displacement from its initial position is cubic in time.
3: The power required to maintain the train's motion increases with time.



\textbf{A.} none of them \\
\textbf{B.} 1 only \\
\textbf{C.} 2 only \\
\textbf{D.} 3 only \\

\textbf{Answer:} G \\
\textbf{Explanation:} The train's acceleration is linear to time. The train's displacement from its initial position is cubic in time. The power required to maintain the train's motion increases with time because of
[IMAGE:0]
, where velocity and acceleration increases with time.

\hrule
\vspace{1em}


\noindent
\textbf{Q156.} Force and spring,
As the diagram indicates, the spring stiffness of each spring is 20N/mm; the weight at the bottom is 40N; find the distance of the weight descends:



\textbf{A.} 1mm \\
\textbf{B.} 2.5mm \\
\textbf{C.} 5mm \\
\textbf{D.} 1.25m \\

\textbf{Answer:} C \\
\textbf{Explanation:} By parallel connection, the stiffness of the combined spring is 40N/mm, the below part is 10N/mm; the resultant stiffness is 8N/mm(1/40+1/10). The extension is therefore 5mm.

\hrule
\vspace{1em}


\noindent
\textbf{Q157.} A 5kg firework explodes in mid-air, splitting into two parts. One part with a mass of 2kg moves horizontally to the left at 10m/s after the explosion. What is the horizontal velocity of the other part?



\textbf{A.} 5 m/s to the right \\
\textbf{B.} 8 m/s to the right \\
\textbf{C.} 4 m/s to the right \\
\textbf{D.} 6.67 m/s to the right \\

\textbf{Answer:} D \\
\textbf{Explanation:} Momentum is conserved during the explosion, with the total momentum before the explosion being zero (as the firework was stationary). After the explosion, the momenta of the two parts are equal in magnitude but opposite in direction.
[IMAGE:0]

\hrule
\vspace{1em}


\noindent
\textbf{Q158.} An electric train is travelling along a straight horizontal track. It passes a point Q on the track at time t = 0. The distance x that it then travels away from Q is given by the equation:
[IMAGE:0]
where a and b are constants.
Which of the following statements is/are correct?
1: The kinetic energy of the train is linear to time.
2: The resultant force acting on the train increases with time.
3: The rate at which energy is transferred to the train increases with time.



\textbf{A.} none of them \\
\textbf{B.} 1 only \\
\textbf{C.} 2 only \\
\textbf{D.} 3 only \\

\textbf{Answer:} D \\
\textbf{Explanation:} The acceleration is constant and power maintained its velocity is also raising. Velocity is linear to time. So
[IMAGE:0]
means the kinetic energy of the train is quadratic to time.

\hrule
\vspace{1em}


\noindent
\textbf{Q159.} An electric train is travelling along a straight horizontal track. It passes a point Q on the track at time t = 0. The distance x that it then travels away from Q is given by the equation:
[IMAGE:0]
where a and b are constants.
Which of the following statements is/are correct?
1: The speed of the train is a constant.
2: The a can be any constant.
3: The rate at which energy is transferred to the train increases with time.



\textbf{A.} none of them \\
\textbf{B.} 1 only \\
\textbf{C.} 2 only \\
\textbf{D.} 3 only \\

\textbf{Answer:} B \\
\textbf{Explanation:} The acceleration is 0 and power maintained its velocity is constant.
The a can only be constant 0, because x=0 when t=0.

\hrule
\vspace{1em}


\noindent
\textbf{Q160.} A 2kg object A is moving to the right at 6m/s and collides with a stationary 3kg object B. After the collision, object A moves to the left at 3m/s. What is the velocity of object B after the collision?



\textbf{A.} 3 m/s \\
\textbf{B.} 5 m/s \\
\textbf{C.} 4 m/s \\
\textbf{D.} 6 m/s \\

\textbf{Answer:} B \\
\textbf{Explanation:} According to the law of conservation of momentum, the total momentum of the system remains constant before and after the collision.
Solving:
[IMAGE:0]

\hrule
\vspace{1em}


\noindent
\textbf{Q161.} A 4 kg object is moving with an initial velocity of 6 m/s. It collides with a stationary object of mass 2 kg. After the collision, the 4 kg object moves with a velocity of 0 m/s. What is the velocity of the 2 kg object after the collision? The table is without friction



\textbf{A.} 6 m/s \\
\textbf{B.} 8 m/s \\
\textbf{C.} 12 m/s \\
\textbf{D.} 3.5 m/s \\

\textbf{Answer:} C \\
\textbf{Explanation:} By conservation of momentum; The velocity * mass of the entire system is conserved before and after the collision

\hrule
\vspace{1em}


\noindent
\textbf{Q162.} An electric train is travelling along a straight horizontal track. It passes a point Q on the track at time t = 0. The distance x that it then travels away from Q is given by the equation:
[IMAGE:0]
where a and b are constants.
Which of the following statements is/are correct?
1: The speed of the train increases with time at a constant rate.
2: The resultant force acting on the train increases with time.
3: The rate at which energy is transferred to the train increases with time.



\textbf{A.} none of them \\
\textbf{B.} 1 only \\
\textbf{C.} 2 only \\
\textbf{D.} 3 only \\

\textbf{Answer:} A \\
\textbf{Explanation:} The acceleration is 0 and power maintained its velocity is constant.

\hrule
\vspace{1em}


\noindent
\textbf{Q163.} An object is horizontally launched with an initial velocity of 10 m/s, and air resistance is neglected. What is the vertical component of the object's velocity at the end of the fifth second? gravitational acceleration is taken as 10
$𝑚$
/
$𝑠$.



\textbf{A.} 15 m/s \\
\textbf{B.} 30 m/s \\
\textbf{C.} 50 m/s \\
\textbf{D.} 20 m/s \\

\textbf{Answer:} D \\
\textbf{Explanation:} In horizontal projectile motion, the vertical component of velocity is only affected by gravitational acceleration. Using the SUVAT equation: v = u + at Where:
• v is the final vertical velocity
• u is the initial vertical velocity (0 m/s, as the object is horizontally launched)
• a is the gravitational acceleration (10 m/s²)
• t is the time (5 seconds) Substituting the values: v = 0 + 10 * 5 = 50 m/s Thus, the vertical component of velocity at the end of the fifth second is 50 m/s.

\hrule
\vspace{1em}


\noindent
\textbf{Q164.} An electric train is travelling along a straight horizontal track. It passes a point Q on the track at time t = 0. The distance x that it then travels away from Q is given by the equation:
[IMAGE:0]
where a and b are constants.
Which of the following statements is/are correct?
1: The speed of the train increases with time at a constant rate.
2: The resultant force acting on the train increases with time.
3: The rate at which energy is transferred to the train increases with time.



\textbf{A.} none of them \\
\textbf{B.} 1 only \\
\textbf{C.} 2 only \\
\textbf{D.} 3 only \\

\textbf{Answer:} F \\
\textbf{Explanation:} The acceleration is constant and power maintained its velocity is also raising.

\hrule
\vspace{1em}


\noindent
\textbf{Q165.} An object is dropped from a height with an initial velocity of 0. It falls for 3 seconds with a gravitational acceleration of 10 m/s
²
. What is the velocity of the object at the end of these 3 seconds?



\textbf{A.} 15 m/s \\
\textbf{B.} 30 m/s \\
\textbf{C.} 25 m/s \\
\textbf{D.} 20 m/s \\

\textbf{Answer:} B \\
\textbf{Explanation:} Using the SUVAT equation: v = u + at Where:
• v is the final velocity
• u is the initial velocity (0 m/s)
• a is the acceleration (10 m/s²)
• t is the time (3 seconds) Substituting the values: v = 0 + 10 * 3 = 30 m/s Thus, the velocity at the end of 3 seconds is 30 m/s

\hrule
\vspace{1em}


\noindent
\textbf{Q166.} A car starts from 3 m/s and accelerates uniformly at 4 m/s² for 5 seconds. What is the final velocity of the car?



\textbf{A.} 20 m/s \\
\textbf{B.} 17 m/s \\
\textbf{C.} 25 m/s \\
\textbf{D.} 23 m/s \\

\textbf{Answer:} D \\
\textbf{Explanation:} Using the SUVAT equation: v=u+at
Where:
• v is the final velocity
• u is the initial velocity (3 m/s)
• a is the acceleration (4 m/s²)
• t is the time (5 seconds)
Substituting the values: v=3+(4)*(5)=23 m/s Thus, the final velocity is 23 m/s.

\hrule
\vspace{1em}


\noindent
\textbf{Q167.} Anna keeps a record of what she earns each day.
Her income increased or decreased by a random variable percent
[IMAGE:0]
everyday from Sunday to Friday.
On Friday, she earned £1.25. How much did Emma probably earn on Sunday?



\textbf{A.} £0.1 \\
\textbf{B.} £0.4 \\
\textbf{C.} £1000 \\
\textbf{D.} £10000 \\

\textbf{Answer:} C \\
\textbf{Explanation:} Assumer she earns x
pounds on Sunday.
[IMAGE:0]

\hrule
\vspace{1em}


\noindent
\textbf{Q168.} Which of the following is correctly classified as a vector quantity?



\textbf{A.} Electric potential \\
\textbf{B.} Electric field strength \\
\textbf{C.} Resistance \\
\textbf{D.} Electric charge \\

\textbf{Answer:} B \\
\textbf{Explanation:} Electric field strength is a vector quantity because it has both magnitude and direction
Electric potential is a scalar quantity as it only has magnitude.
Resistance is a scalar quantity as it only has magnitude.
Electric charge is a scalar quantity as it only has magnitude.

\hrule
\vspace{1em}


\noindent
\textbf{Q169.} Anna keeps a record of what she earns each day.
Her income increased or decreased by a random variable percent
[IMAGE:0]
everyday from Sunday to Friday.
On Friday, she earned £640. How much did Emma probably earn on Sunday?



\textbf{A.} £1 \\
\textbf{B.} £2 \\
\textbf{C.} £10 \\
\textbf{D.} £10000 \\

\textbf{Answer:} E \\
\textbf{Explanation:} Assumer she earns x
pounds on Sunday.
[IMAGE:0]

\hrule
\vspace{1em}


\noindent
\textbf{Q170.} Which of the following is correctly classified as a vector quantity?



\textbf{A.} Work \\
\textbf{B.} Force \\
\textbf{C.} Time \\
\textbf{D.} Volume \\

\textbf{Answer:} B \\
\textbf{Explanation:} Force is a vector quantity because it has both magnitude and direction (e.g., 5 Newtons east).
• Work is a scalar quantity as it only has magnitude.
• Time is a scalar quantity as it only has magnitude.
• Volume is a scalar quantity as it only has magnitude.

\hrule
\vspace{1em}


\noindent
\textbf{Q171.} Emma keeps a record of what she earns each day.
Her income increased or decreased by a variable percent  P=25i%(Sunday i=1, Monday i=-2, Tuesday i=3, Wednesday i=0, Thursday i=-1, Friday i=-2, Saturday i=3 ) everyday from Sunday to Wednesday.
On Wednesday, she earned £1400. How much did Emma earn on Sunday?



\textbf{A.} £10 \\
\textbf{B.} £32 \\
\textbf{C.} £80 \\
\textbf{D.} £110 \\

\textbf{Answer:} C \\
\textbf{Explanation:} Assumer she earns
[IMAGE:0]
pounds on Sunday.
[IMAGE:1]

\hrule
\vspace{1em}


\noindent
\textbf{Q172.} Which of the following is correctly classified as a vector quantity?



\textbf{A.} Speed \\
\textbf{B.} Mass \\
\textbf{C.} Velocity \\
\textbf{D.} Time \\

\textbf{Answer:} C \\
\textbf{Explanation:} Velocity is a vector quantity because it has both magnitude and direction (e.g.,
10 meters to the north).
• Speed is a scalar quantity, as it only has magnitude.
• Mass is a scalar quantity, as it only has magnitude.
• Time is a scalar quantity, as it only has magnitude.

\hrule
\vspace{1em}


\noindent
\textbf{Q173.} An object is in free fall from a height, with air resistance neglected. During the fall, which of the following groups correctly matches scalar and vector quantities?



\textbf{A.} Scalar: Velocity, Vector: Average velocity \\
\textbf{B.} Scalar: Displacement, Vector: Acceleration \\
\textbf{C.} Scalar: Time of fall, Vector: Velocity \\
\textbf{D.} Scalar: Displacement, Vector: Change in position \\

\textbf{Answer:} C \\
\textbf{Explanation:} Time of fall
、
speed
、
Distance traveled are scalars as they has only magnitude.
•Acceleration
、
velocity
、
Average velocity
、
Change in position are vectors as they has both magnitude and direction.

\hrule
\vspace{1em}


\noindent
\textbf{Q174.} Emma keeps a record of what she earns each day.
Her income increased by a variable percent
[IMAGE:0]
(Sunday i=1
, Monday i=2
, Tuesday i=3
, Wednesday i=4
, Thursday i=5
, Friday i=6
, Saturday i=7
) everyday from Sunday to Wednesday.
On Wednesday, she earned £75. How much did Emma earn on Sunday?



\textbf{A.} £10 \\
\textbf{B.} £32 \\
\textbf{C.} £117.60 \\
\textbf{D.} £110 \\

\textbf{Answer:} A \\
\textbf{Explanation:} Assumer she earns
[IMAGE:0]
pounds on Sunday.
[IMAGE:1]

\hrule
\vspace{1em}


\noindent
\textbf{Q175.} An object is moving north at 5m/s and comes to rest after 10 seconds. During this process, which of the following physical quantities are scalar and which are vector?



\textbf{A.} Scalar: Time, Vector: Change in velocity \\
\textbf{B.} Scalar: Change in velocity, Vector: Displacement \\
\textbf{C.} Scalar: Acceleration, Vector: Force \\
\textbf{D.} Scalar: Distance traveled, Vector: speed \\

\textbf{Answer:} A \\
\textbf{Explanation:} In this problem, we need to distinguish between scalar and vector quantities. Scalars have only magnitude, while vectors have both magnitude and direction.
•Time
、
speed
、
Distance traveled are scalars as they has only magnitude.
•Acceleration
、
Change in velocity
、
Force are vectors as they has both magnitude and direction.

\hrule
\vspace{1em}


\noindent
\textbf{Q176.} Emma keeps a record of what she earns each day.
Her income increased by a constant percent P everyday from Sunday to Wednesday. On Sunday, she only earn £12, but she earned £40.5 on Wednesday. What is the P ?



\textbf{A.} 10% \\
\textbf{B.} 20% \\
\textbf{C.} 40% \\
\textbf{D.} 50% \\

\textbf{Answer:} D \\
\textbf{Explanation:} [IMAGE:0]

\hrule
\vspace{1em}


\noindent
\textbf{Q177.} A car is traveling at a speed of 50 km/h to the east
Identify the scalar and vector quantities:



\textbf{A.} Scalar: Displacement, Vector: Velocity \\
\textbf{B.} Scalar: Acceleration, Vector: Time \\
\textbf{C.} Scalar: Temperature, Vector: Mass \\
\textbf{D.} Scalar: Speed, Vector: Force \\

\textbf{Answer:} D \\
\textbf{Explanation:} Speed is a scalar quantity (it has only magnitude).
• Velocity is a vector quantity (it has both magnitude and direction)

\hrule
\vspace{1em}


\noindent
\textbf{Q178.} An object with a mass of 2kg is moving to the right on a horizontal surface with an initial velocity of 10m/s. The air resistance acting on the object is a constant 4N, opposite to the direction of motion. What is the magnitude of the object's acceleration during the motion?



\textbf{A.} 3m/s² \\
\textbf{B.} 5m/s² \\
\textbf{C.} 0m/s² \\
\textbf{D.} 1m/s² \\

\textbf{Answer:} E \\
\textbf{Explanation:} According to Newton's second law, the net force acting on the object in the horizontal direction is the air resistance. The air resistance is 4N opposite to the motion. Using
F
=
ma
, the acceleration is
a
​=−2
m
/
s
2
. The magnitude of the acceleration is 2m/s², opposite to the direction of motion. Therefore, the correct answer is option E.

\hrule
\vspace{1em}


\noindent
\textbf{Q179.} Emma keeps a record of what she earns each day.
On Monday, her income increased by 1/3 compared to Sunday. On Tuesday, her income decreased by 1/4 compared to Monday. On Wednesday, her income decreased by 1/2 compared to Tuesday.
On Wednesday, she earned £48. How much did Emma earn on Sunday?



\textbf{A.} £15.12 \\
\textbf{B.} £96 \\
\textbf{C.} £117.6 \\
\textbf{D.} £110 \\

\textbf{Answer:} B \\
\textbf{Explanation:} Assumer she earns x pounds on Sunday; the money earned on Monday is 4/3 x, on Tuesday is x; on Wednesday is 1.5x which is 48; x is therefore 32.

\hrule
\vspace{1em}


\noindent
\textbf{Q180.} Rob keeps a record of what he earns each day.
On Monday, he earned 20% more than he earned on Sunday. On Tuesday, he earned 55% less than he earned on Monday. On Wednesday, he earned 50% less than he earned on Tuesday.
On Wednesday, he earned £16.2. How much did Rob earn on Sunday?



\textbf{A.} £15.12 \\
\textbf{B.} £60 \\
\textbf{C.} £117.6 \\
\textbf{D.} £110 \\

\textbf{Answer:} B \\
\textbf{Explanation:} Assumer he earns x pounds on Sunday; the money earned on Monday is 1.2x, on Tuesday is 0.54x; on Wednesday is 0.27x which is 16.2; x is therefore 60.

\hrule
\vspace{1em}


\noindent
\textbf{Q181.} An object with a mass of 1kg is dropped from a height and starts falling from rest. Assume that the air resistance is a constant 2N, opposite to the direction of motion. What is the magnitude of the object's acceleration during the fall? gravitational acceleration is taken as 9.8
$𝑚$
/
$𝑠$.



\textbf{A.} 10m/s² \\
\textbf{B.} 8m/s² \\
\textbf{C.} 0m/s² \\
\textbf{D.} 2m/s² \\

\textbf{Answer:} B \\
\textbf{Explanation:} According to Newton's second law, the net force acting on the object during the fall is the gravitational force minus the air resistance. Using F=ma, the acceleration is a=8m/s2. Therefore, the correct answer is option B.

\hrule
\vspace{1em}


\noindent
\textbf{Q182.} Rob keeps a record of what he earns each day.
On Monday, he earned 30% more than he earned on Sunday. On Tuesday, he earned 10% less than he earned on Monday. On Wednesday, he earned 40% less than he earned on Tuesday.
On Wednesday, he earned £351. How much did Rob earn on Sunday?



\textbf{A.} £15.12 \\
\textbf{B.} £35.28 \\
\textbf{C.} £117.60 \\
\textbf{D.} £200 \\

\textbf{Answer:} G \\
\textbf{Explanation:} Assumer he earns x pounds on Sunday; the money earned on Monday is 1.3x, on Tuesday is 1.17x; on Wednesday is 0.702x which is 351; x is therefore 500.

\hrule
\vspace{1em}


\noindent
\textbf{Q183.} A stone with mass 1kg is falling against air resistance; the air resistance is shown by:
$𝑓$
= 9.8
∗
$𝑣$
Where v is the velocity of the stone; g, gravitational acceleration is taken as 9.8
$𝑚$
/
$𝑠$.
Find the terminal velocity:



\textbf{A.} 1
$𝑚$
/
$𝑠$ \\
\textbf{B.} 2
$𝑚$
/
$𝑠$ \\
\textbf{C.} 2.5
$𝑚$
/
$𝑠$ \\
\textbf{D.} 9.8
$𝑚$
/
$𝑠$ \\

\textbf{Answer:} A \\
\textbf{Explanation:} The terminal velocity is found when air resistance equals the weight, the object is no
longer to accelerate or decelerate.

\hrule
\vspace{1em}


\noindent
\textbf{Q184.} Rob keeps a record of what he earns each day.
On Monday, he earned 40% more than he earned on Sunday. On Tuesday, he earned 25% less than he earned on Monday. On Wednesday, he earned 100% more than he earned on Tuesday.
On Wednesday, he earned £77. How much did Rob earn on Sunday?



\textbf{A.} £60 \\
\textbf{B.} £90 \\
\textbf{C.} £36.67 \\
\textbf{D.} £200 \\

\textbf{Answer:} C \\
\textbf{Explanation:} Assumer he earns x pounds on Sunday; the money earned on Monday is 1.4x, on Tuesday is 1.05x; on Wednesday is 2.1x which is 77; x is therefore around 36.67.

\hrule
\vspace{1em}


\noindent
\textbf{Q185.} Rob keeps a record of what he earns each day.
On Monday, he earned 50% more than he earned on Sunday. On Tuesday, he earned 20% less than he earned on Monday. On Wednesday, he earned 60% less than he earned on Tuesday.
On Wednesday, he earned £96. How much did Rob earn on Sunday?



\textbf{A.} 15.12 \\
\textbf{B.} 35.28 \\
\textbf{C.} 117.60 \\
\textbf{D.} 200 \\

\textbf{Answer:} D \\
\textbf{Explanation:} Assumer he earns x pounds on Sunday; the money earned on Monday is 1.5x, on Tuesday is 1.2x; on Wednesday is 0.48x which is 96; x is therefore 200.

\hrule
\vspace{1em}


\noindent
\textbf{Q186.} A force F=10 N is applied to an object at an angle of 30
\circ 
above the horizontal. Resolve this force into its horizontal and vertical components and determine their magnitudes:



\textbf{A.} 4
$𝑁$
; 6N \\
\textbf{B.} [IMAGE:0]
$𝑁$; 5N \\
\textbf{C.} 5
$𝑁$; 5N \\
\textbf{D.} [IMAGE:1]
$𝑁$
; 5N \\

\textbf{Answer:} D \\
\textbf{Explanation:} According to the decomposition of forces, the horizontal component of force is
[IMAGE:0]
$𝑁$
; the vertical component of force is 5N

\hrule
\vspace{1em}


\noindent
\textbf{Q187.} Two isolated spheres have masses
[IMAGE:0]
and
[IMAGE:1]
. They are moving towards each other along the same straight line with speeds
[IMAGE:2]
and
[IMAGE:3]
respectively as shown:
The spheres collide with each other under elastic collision case. How many times can two spheres collide at most?



\textbf{A.} 1 \\
\textbf{B.} 2 \\
\textbf{C.} 3 \\
\textbf{D.} 4 \\

\textbf{Answer:} A \\
\textbf{Explanation:} By the conservation of momentum; the velocity of the spheres is:
[IMAGE:0]
;
[IMAGE:1]
, where
[IMAGE:2]
and
[IMAGE:3]
are the velocity of two spheres after
[IMAGE:4]
collision, respectively.
[IMAGE:5]
Thus, the formulas can be derived as follows:
[IMAGE:6]
Thus, if we assume that the two spheres can collide more than once time, it has
[IMAGE:7]
So
[IMAGE:8]
which is contradictory to the assumption "they can collide at least once time".
Thus, only once.

\hrule
\vspace{1em}


\noindent
\textbf{Q188.} On a horizontal ground, an object is acted upon by two forces. The first force F
1
has a magnitude of 4N and makes an angle of 30° with the horizontal direction; the second force F
2
has a magnitude of
[IMAGE:0]
$𝑁$
and points horizontally to the left. Find the magnitude of the resultant force of these two forces and the angle between it and the horizontal direction.



\textbf{A.} 2
$𝑁$
; 0
° \\
\textbf{B.} [IMAGE:0]
$𝑁$
; 90
° \\
\textbf{C.} 2
$𝑁$
;
90
° \\
\textbf{D.} [IMAGE:1]
$𝑁$
; 60
° \\

\textbf{Answer:} C \\
\textbf{Explanation:} According to the synthesis and decomposition of forces, it can be known that F
2
cancels out the horizontal rightward component of F
1
, and the vertical component of F1 is 2N in magnitude.

\hrule
\vspace{1em}


\noindent
\textbf{Q189.} F
1
is with magnitude 30 N, pointing south-ward; F
2
is with magnitude 40 N, pointing west-ward; Find the magnitude of the component of the resulting force:



\textbf{A.} 40
$𝑁$ \\
\textbf{B.} [IMAGE:0]
$𝑁$ \\
\textbf{C.} 50
$𝑁$ \\
\textbf{D.} [IMAGE:1]
$𝑁$ \\

\textbf{Answer:} C \\
\textbf{Explanation:} According to the Pythagorean theorem, the magnitude of the resultant force is 50N

\hrule
\vspace{1em}


\noindent
\textbf{Q190.} In a Cartesian coordinate system, an object starts at the origin. It first moves along the vector
A
=(2m,3m), then moves along the vector
B
=(4m,5m). Find the magnitude of the object's final displacement vector
R
relative to the origin.



\textbf{A.} 9m \\
\textbf{B.} 2m \\
\textbf{C.} 8m \\
\textbf{D.} 4m \\

\textbf{Answer:} E \\
\textbf{Explanation:} According to the rules of vector addition; the magnitude of the resultant Displacement is 10m

\hrule
\vspace{1em}


\noindent
\textbf{Q191.} Two isolated spheres have masses
[IMAGE:0]
and
[IMAGE:1]
. They are moving towards each other along the same straight line with speeds
[IMAGE:2]
and
[IMAGE:3]
respectively as shown:



\textbf{A.} 1 \\
\textbf{B.} 2 \\
\textbf{C.} 3 \\
\textbf{D.} 4 \\

\textbf{Answer:} A \\
\textbf{Explanation:} By the conservation of momentum; the velocity of the spheres is:
[IMAGE:0]
;
[IMAGE:1]
, where
[IMAGE:2]
and
[IMAGE:3]
are the velocity of two spheres after
[IMAGE:4]
collision, respectively.
[IMAGE:5]
or
[IMAGE:6]
.
Thus, the formulas can be derived as follows:
[IMAGE:7]
Thus, if the mass are equal, that is
[IMAGE:8]
[IMAGE:9]
, which means they exchanges the velociy.
Thus, only once.

\hrule
\vspace{1em}


\noindent
\textbf{Q192.} A pendulum bob of mass m swings from a height 2h to the lowest point of its arc. If the length of the pendulum is L, what is the speed of the bob at the lowest point?



\textbf{A.} [IMAGE:0] \\
\textbf{B.} [IMAGE:1] \\
\textbf{C.} [IMAGE:2] \\
\textbf{D.} [IMAGE:3] \\

\textbf{Answer:} E \\
\textbf{Explanation:} Using energy conservation is easier than using kinematics equations:
[IMAGE:0]

\hrule
\vspace{1em}


\noindent
\textbf{Q193.} An object is acted upon by two concurrent forces. One force
F
1​​=(3N,0N) points in the positive x-direction, and the other force
F
2​​=(0N,4N) points in the positive y-direction. Find the magnitude of the resultant force
F
:



\textbf{A.} 5
N \\
\textbf{B.} 3N \\
\textbf{C.} 0N \\
\textbf{D.} 4
N \\

\textbf{Answer:} A \\
\textbf{Explanation:} According to the rules of vector addition; the magnitude of the resultant force is 5N

\hrule
\vspace{1em}


\noindent
\textbf{Q194.} Two isolated spheres have masses
[IMAGE:0]
and
[IMAGE:1]
. They are moving towards each other along the same straight line with speeds
[IMAGE:2]
and
[IMAGE:3]
respectively. During the collision, the spheres coalesce but then immediately explode into two fragments of equal mass, moving in velocity
[IMAGE:4]
and -2V. Can the scenarios described in the problems actually occur? If your answer is "can", please calculate the variation of kinetic energy during the coalesce-explosion process?



\textbf{A.} [IMAGE:0] \\
\textbf{B.} [IMAGE:1] \\
\textbf{C.} [IMAGE:2] \\
\textbf{D.} [IMAGE:3] \\

\textbf{Answer:} D \\
\textbf{Explanation:} By the conservation of momentum; the velocity of the spheres is:
[IMAGE:0]
.
Conservation of momentum:
[IMAGE:1]
is equal to
[IMAGE:2]
Increasing of the kinetic energy:
[IMAGE:3]
Thus, increase
[IMAGE:4]
It conforms to the explosion model and represents a possible physical scenario.

\hrule
\vspace{1em}


\noindent
\textbf{Q195.} A hovercraft of mass m moves at constant speed v on a horizontal surface. If the lift fan provides a force equal to the weight of the hovercraft which consumes power
[IMAGE:0]
, what is the power required to maintain this speed if friction is
[IMAGE:1]
?



\textbf{A.} [IMAGE:0] \\
\textbf{B.} [IMAGE:1] \\
\textbf{C.} [IMAGE:2] \\
\textbf{D.} [IMAGE:3] \\

\textbf{Answer:} E \\
\textbf{Explanation:} Power is force times velocity:
[IMAGE:0]
P is not vector  but scalar, so
[IMAGE:1]

\hrule
\vspace{1em}


\noindent
\textbf{Q196.} [IMAGE:0]
[IMAGE:1]
These are two vectors representing two forces as shown above, they both exert on one object with mass 1kg; Find the acceleration of the object:



\textbf{A.} 4
$𝑚$
/
$𝑠$
2 \\
\textbf{B.} 0
$𝑚$
/
$𝑠$
2 \\
\textbf{C.} 2.5
$𝑚$
/
$𝑠$
2 \\
\textbf{D.} 2
$𝑚$
/
$𝑠$
2 \\

\textbf{Answer:} B \\
\textbf{Explanation:} The magnitude of the resultant force is 0N, by equation F=ma; m=1kg; acceleration is 0N

\hrule
\vspace{1em}


\noindent
\textbf{Q197.} A ball of mass 100m is thrown vertically upward with initial speed
[IMAGE:0]
. Ignoring air resistance, what is the maximum height reached?



\textbf{A.} [IMAGE:0] \\
\textbf{B.} [IMAGE:1] \\
\textbf{C.} [IMAGE:2] \\
\textbf{D.} [IMAGE:3] \\

\textbf{Answer:} E \\
\textbf{Explanation:} [IMAGE:0]

\hrule
\vspace{1em}


\noindent
\textbf{Q198.} A small ball A with a mass of v0
moves at a speed of v0
on a smooth horizontal surface and collides head-on with a stationary small ball B with a mass of 3m
. After the collision, the speed of ball A is va
and the speed of ball B is vb
. Which of the following could be the values of va
and vb
?



\textbf{A.} [IMAGE:0] \\
\textbf{B.} [IMAGE:1] \\
\textbf{C.} [IMAGE:2] \\
\textbf{D.} [IMAGE:3] \\

\textbf{Answer:} B \\
\textbf{Explanation:} This question belongs to the type of collision result possibility judgment. The problem-solving method involves the conditions that must be met during a collision. Two conditions are: firstly, the conservation of momentum during the collision; secondly, the non-increase of kinetic energy.
Based on these two conditions, we can derive the following two formulas:
[IMAGE:0]
which implies
[IMAGE:1]
;
[IMAGE:2]
which implies
[IMAGE:3]
;
After obtaining these two formulas, we can substitute each of the four options into these two formulas one by one. If the result does not hold, then that option can be eliminated.

\hrule
\vspace{1em}


\noindent
\textbf{Q199.} Two isolated spheres have masses
[IMAGE:0]
and
[IMAGE:1]
. They are moving towards each other along the same straight line with speeds
[IMAGE:2]
and
[IMAGE:3]
respectively as shown:
The spheres collide with each other and coalesce. Then the combination sphere move to collide with another static sphere with mass
[IMAGE:4]
and coalesce again. What is the loss of kinetic energy during the collision processes?



\textbf{A.} 15/4mv
2 \\
\textbf{B.} 29/3mv
2 \\
\textbf{C.} 51/4mv
2 \\
\textbf{D.} 12mv
2 \\

\textbf{Answer:} C \\
\textbf{Explanation:} By the conservation of momentum; the velocity of the spheres is:
[IMAGE:0]
; the total deficit in kinetic energy is energy before collision minus the one after twice collision; which is:
[IMAGE:1]

\hrule
\vspace{1em}


\noindent
\textbf{Q200.} A rocket of mass
[IMAGE:0]
ejects mass at a rate
[IMAGE:1]
with velocity u
relative to the rocket. If the rocket starts from rest, what is its speed after time t
?



\textbf{A.} uln2 \\
\textbf{B.} uln3 \\
\textbf{C.} u \\
\textbf{D.} [IMAGE:0] \\

\textbf{Answer:} B \\
\textbf{Explanation:} [IMAGE:0]

\hrule
\vspace{1em}


\noindent
\textbf{Q201.} A bicycle of mass
[IMAGE:0]
traveling at speed
[IMAGE:1]
encounters a slope inclined at an angle
[IMAGE:2]
and coasts up it until it stops. If the coefficient of friction is
[IMAGE:3]
, what is the distance traveled up the slope?



\textbf{A.} [IMAGE:0] \\
\textbf{B.} [IMAGE:1] \\
\textbf{C.} [IMAGE:2] \\
\textbf{D.} [IMAGE:3] \\

\textbf{Answer:} E \\
\textbf{Explanation:} [IMAGE:0]

\hrule
\vspace{1em}


\noindent
\textbf{Q202.} Two isolated spheres have masses
[IMAGE:0]
and
[IMAGE:1]
. They are moving towards each other along the same straight line with speeds
[IMAGE:2]
and
[IMAGE:3]
respectively. During the collision, the spheres coalesce but then immediately explode into two fragments of equal mass, moving in velocity
[IMAGE:4]
and
[IMAGE:5]
. What is the variation of kinetic energy during the entire process?



\textbf{A.} [IMAGE:0] \\
\textbf{B.} [IMAGE:1] \\
\textbf{C.} [IMAGE:2] \\
\textbf{D.} [IMAGE:3] \\

\textbf{Answer:} B \\
\textbf{Explanation:} By the conservation of momentum; the velocity of the spheres is:v0=2v
[IMAGE:0]
Thus, increase
[IMAGE:1]

\hrule
\vspace{1em}


\noindent
\textbf{Q203.} A skateboarder of mass m
starts from rest and reaches a speed
[IMAGE:0]
in time
[IMAGE:1]
by applying a constant force
[IMAGE:2]
. If friction is constant as
[IMAGE:3]
, what is the distance covered?



\textbf{A.} [IMAGE:0] \\
\textbf{B.} [IMAGE:1] \\
\textbf{C.} [IMAGE:2] \\
\textbf{D.} [IMAGE:3] \\

\textbf{Answer:} E \\
\textbf{Explanation:} 

\hrule
\vspace{1em}


\noindent
\textbf{Q204.} Two isolated spheres have masses
[IMAGE:0]
and
[IMAGE:1]
. They are moving towards each other along the same straight line with speeds
[IMAGE:2]
and
[IMAGE:3]
respectively as shown:
The spheres collide with each other and coalesce. What is the loss of kinetic energy during the collision?



\textbf{A.} 0 \\
\textbf{B.} [IMAGE:0] \\
\textbf{C.} [IMAGE:1] \\
\textbf{D.} [IMAGE:2] \\

\textbf{Answer:} D \\
\textbf{Explanation:} By the conservation of momentum; the velocity of the spheres is: v0=0; the total deficit in kinetic energy is energy before collision minus the one after collision; which is:
[IMAGE:0]

\hrule
\vspace{1em}


\noindent
\textbf{Q205.} A train of mass 3m moving at speed v/2 applies its brakes and stops in time 3t. If the braking force is F, what distance does the train travel before stopping?



\textbf{A.} [IMAGE:0] \\
\textbf{B.} [IMAGE:1] \\
\textbf{C.} [IMAGE:2] \\
\textbf{D.} [IMAGE:3] \\

\textbf{Answer:} C \\
\textbf{Explanation:} [IMAGE:0]

\hrule
\vspace{1em}


\noindent
\textbf{Q206.} Two isolated spheres have masses
[IMAGE:0]
and
[IMAGE:1]
. They are moving towards each other along the same straight line with speeds
[IMAGE:2]
and
[IMAGE:3]
respectively as shown:



\textbf{A.} [IMAGE:0] \\
\textbf{B.} [IMAGE:1] \\
\textbf{C.} [IMAGE:2] \\
\textbf{D.} [IMAGE:3] \\

\textbf{Answer:} D \\
\textbf{Explanation:} By the conservation of momentum; the velocity of the spheres is: v0=-v; the total deficit in kinetic energy is energy before collision minus the one after collision; which is:
[IMAGE:0]

\hrule
\vspace{1em}


\noindent
\textbf{Q207.} Two isolated spheres have masses
[IMAGE:0]
and
[IMAGE:1]
. They are moving towards each other along the same straight line with speeds
[IMAGE:2]
and
[IMAGE:3]
respectively as shown:
The spheres collide with each other and coalesce. What is the loss of kinetic energy during the collision?



\textbf{A.} 0 \\
\textbf{B.} [IMAGE:0] \\
\textbf{C.} [IMAGE:1] \\
\textbf{D.} [IMAGE:2] \\

\textbf{Answer:} D \\
\textbf{Explanation:} By the conservation of momentum; the velocity of the spheres is: v0=0; the total deficit in kinetic energy is energy before collision minus the one after collision; which is:
[IMAGE:0]

\hrule
\vspace{1em}


\noindent
\textbf{Q208.} A sled of mass 5m slides down a frictionless incline which has angle
[IMAGE:0]
with the ground and reaches the bottom with speed V. If the vertical height of the incline is h, and the sled started from rest, what is the length of the incline?



\textbf{A.} [IMAGE:0] \\
\textbf{B.} [IMAGE:1] \\
\textbf{C.} [IMAGE:2] \\
\textbf{D.} [IMAGE:3] \\

\textbf{Answer:} D \\
\textbf{Explanation:} [IMAGE:0]

\hrule
\vspace{1em}


\noindent
\textbf{Q209.} Two isolated spheres have masses
[IMAGE:0]
and
[IMAGE:1]
. They are moving towards each other along the same straight line with speeds
[IMAGE:2]
and
[IMAGE:3]
respectively as shown:
The spheres collide with each other and coalesce. What is the loss of kinetic energy during the collision?



\textbf{A.} [IMAGE:0] \\
\textbf{B.} [IMAGE:1] \\
\textbf{C.} [IMAGE:2] \\
\textbf{D.} [IMAGE:3] \\

\textbf{Answer:} C \\
\textbf{Explanation:} By the conservation of momentum; the velocity of the spheres is:
[IMAGE:0]
; the total deficit in kinetic energy is energy before collision minus the one after collision; which is:
[IMAGE:1]

\hrule
\vspace{1em}


\noindent
\textbf{Q210.} A small cart of mass 200 g is at rest on a track and in contact with a light uncompressed spring. The spring is compressed by 50 cm and the cart remains in contact with it. The spring is released and causes the cart to move and compress another spring placed at a distance. Two springs have same parameters. If the second spring is compressed by 25 cm, what is the percentage of mechanical energy loss compared to the original energy in this process?
These problems incorporate variations in mass, spring compression, track types, and additional elements like collisions and energy loss, providing a comprehensive set of challenges.



\textbf{A.} 25% \\
\textbf{B.} 50% \\
\textbf{C.} 55% \\
\textbf{D.} 75% \\

\textbf{Answer:} D \\
\textbf{Explanation:} Using the law of conservation of energy with energy loss:
Initial elastic potential energy:
[IMAGE:0]
Final elastic potential energy:
[IMAGE:1]
Energy loss:
[IMAGE:2]
(based on proportion and simplification)

\hrule
\vspace{1em}


\noindent
\textbf{Q211.} A car of mass m starts from rest and accelerates uniformly to a speed 3v in a time 2t on a horizontal road. If the average horizontal force applied is 2F, what is the distance covered by the car during this acceleration?



\textbf{A.} [IMAGE:0] \\
\textbf{B.} [IMAGE:1] \\
\textbf{C.} [IMAGE:2] \\
\textbf{D.} [IMAGE:3] \\

\textbf{Answer:} C \\
\textbf{Explanation:} [IMAGE:0]

\hrule
\vspace{1em}


\noindent
\textbf{Q212.} Select the correct statement(s) regarding the motion of an object with constant acceleration:
The displacement of an object with constant acceleration is always a straight line.
The acceleration of an object with constant acceleration must be zero.
The velocity of an object with constant acceleration changes by the same amount in each equal time interval.



\textbf{A.} All of above \\
\textbf{B.} 1 and 2 \\
\textbf{C.} 2 only \\
\textbf{D.} 3 only \\

\textbf{Answer:} D \\
\textbf{Explanation:} Displacement can be a curve if the initial velocity is not in the direction of acceleration (e.g., projectile motion). Acceleration is not zero if it is constant and non-zero.

\hrule
\vspace{1em}


\noindent
\textbf{Q213.} Which statement(s) correctly describes the effect of acceleration on an object's motion?
Acceleration always increases the speed of an object.
Acceleration can change the direction of an object's velocity without changing its speed.
If an object has a negative acceleration, it must be slowing down.



\textbf{A.} All of above \\
\textbf{B.} 2 and 3 \\
\textbf{C.} 2 only \\
\textbf{D.} 3 only \\

\textbf{Answer:} C \\
\textbf{Explanation:} Acceleration can change direction without changing speed (e.g., circular motion). Negative acceleration (deceleration) means the object is slowing down if its velocity is positive, but it could also mean speeding up if its velocity is negative.

\hrule
\vspace{1em}


\noindent
\textbf{Q214.} A small ball of mass 150 g is at rest on a frictionless surface and in contact with a light uncompressed spring. The spring is compressed by 40 cm and the ball remains in contact with it. The spring is released and causes the ball to move and strike a vertical wall, where it rebounds with half its initial speed. What is the compression of the spring after the ball rebounds and comes to rest?



\textbf{A.} 10cm \\
\textbf{B.} 20cm \\
\textbf{C.} 30cm \\
\textbf{D.} 40cm \\

\textbf{Answer:} B \\
\textbf{Explanation:} before:
[IMAGE:0]
after:
[IMAGE:1]
Thus,
[IMAGE:2]

\hrule
\vspace{1em}


\noindent
\textbf{Q215.} Choose the true statement(s) about the relationship between speed and velocity:
Speed is a vector quantity that includes direction.
Velocity is a scalar quantity that only depends on magnitude.
Speed is the magnitude of velocity.



\textbf{A.} All of above \\
\textbf{B.} 1 only \\
\textbf{C.} 2 only \\
\textbf{D.} 3 only \\

\textbf{Answer:} D \\
\textbf{Explanation:} Speed is a scalar quantity (magnitude only), while velocity is a vector quantity (magnitude and direction).

\hrule
\vspace{1em}


\noindent
\textbf{Q216.} A small cart of mass 80 g is at rest on a frictionless track and in contact with a light uncompressed spring. The spring is compressed by 20 cm and the cart remains in contact with it. The spring is released and causes the cart to move and compress another spring placed at a distance. If the second spring is compressed by 10 cm, what is the spring constant of the second spring if the first spring constant is known to be 100 N/m?



\textbf{A.} 50N/m \\
\textbf{B.} 100N/m \\
\textbf{C.} 200N/m \\
\textbf{D.} 300N/m \\

\textbf{Answer:} E \\
\textbf{Explanation:} Using the law of conservation of energy:
[IMAGE:0]
Given
[IMAGE:1]
solve for k2
:
[IMAGE:2]
[IMAGE:3]

\hrule
\vspace{1em}


\noindent
\textbf{Q217.} Select the correct statements(s) about the motion graphs:
The slope of a position-time graph represents the velocity of the object.
The area under a velocity-time graph represents the displacement of the object.
A horizontal line on a velocity-time graph indicates that the object is accelerating.



\textbf{A.} All of above \\
\textbf{B.} 1 and 2 \\
\textbf{C.} 2 only \\
\textbf{D.} 3 only \\

\textbf{Answer:} B \\
\textbf{Explanation:} The slope of a position-time graph is velocity, and the area under a velocity-time graph is displacement. A horizontal line on a velocity-time graph indicates constant velocity, not acceleration.

\hrule
\vspace{1em}


\noindent
\textbf{Q218.} A lorry of mass m, and travelling initially at speed 2v along a horizontal road, is brought to rest by an average horizontal braking force 2F in time t. Ignoring any other resistive forces, what distance is travelled by the lorry during this time? (gravitational field strength = 10 N kg–1)



\textbf{A.} [IMAGE:0] \\
\textbf{B.} [IMAGE:1] \\
\textbf{C.} [IMAGE:2] \\
\textbf{D.} [IMAGE:3] \\

\textbf{Answer:} C \\
\textbf{Explanation:} [IMAGE:0]

\hrule
\vspace{1em}


\noindent
\textbf{Q219.} Which statement(s) accurately describes the concept of instantaneous velocity?
Instantaneous velocity is the average velocity of an object over a long period of time.
Instantaneous velocity is the velocity of an object at a specific instant in time.
Instantaneous velocity is always equal to the average velocity.



\textbf{A.} All of above \\
\textbf{B.} 1 only \\
\textbf{C.} 2 only \\
\textbf{D.} 3 only \\

\textbf{Answer:} C \\
\textbf{Explanation:} Instantaneous velocity is the velocity at a specific instant, not an average over time.

\hrule
\vspace{1em}


\noindent
\textbf{Q220.} A small puck of mass 80 g is at rest on a frictionless air hockey table and in contact with a light uncompressed spring whose
[IMAGE:0]
. The spring is compressed by 20 cm and the puck remains in contact with it. The spring is released and causes the puck to slide and hit a stationary puck of unknown mass, causing both to move together. If the combined mass moves with a speed of 0.5 m/s, what is the mass of the second puck?



\textbf{A.} 35g \\
\textbf{B.} 70g \\
\textbf{C.} 105g \\
\textbf{D.} 140g \\

\textbf{Answer:} B \\
\textbf{Explanation:} Using the law of conservation of momentum:
[IMAGE:0]
Given
[IMAGE:1]
,
[IMAGE:2]
, and assuming v1 is the velocity of the first puck before collision, which can be calculated as:
[IMAGE:3]
[IMAGE:4]
Thus, we solve for m2:
[IMAGE:5]
Since the velocity halves after collision,
[IMAGE:6]
.

\hrule
\vspace{1em}


\noindent
\textbf{Q221.} A shape is formed by drawing a triangle ABC inside the triangle ADE. BC is parallel to DE. The area of triangle ABC is 18 cm², and the area of triangle ADE is 72 cm². BC = 2s cm, DE = s + 9 cm.
Determine the height of triangle ADE to side DE.



\textbf{A.} 9cm \\
\textbf{B.} 12cm \\
\textbf{C.} 15cm \\
\textbf{D.} [IMAGE:0] \\

\textbf{Answer:} B \\
\textbf{Explanation:} [IMAGE:0]

\hrule
\vspace{1em}


\noindent
\textbf{Q222.} Choose the correct statement(s) regarding the motion of an object:
An object with a constant velocity has no acceleration.
An object with a changing velocity must have a constant acceleration.
Acceleration is always in the same direction as the velocity.



\textbf{A.} All of above \\
\textbf{B.} 1 only \\
\textbf{C.} 2 only \\
\textbf{D.} 3 only \\

\textbf{Answer:} B \\
\textbf{Explanation:} An object with constant velocity has zero acceleration. Acceleration can change if velocity changes in a non-uniform way. Acceleration can be in the opposite direction of velocity (e.g., deceleration).

\hrule
\vspace{1em}


\noindent
\textbf{Q223.} A shape is formed by drawing a triangle ABC inside the triangle ADE. BC is parallel to DE. The median from A to BC in triangle ABC is 4 cm, and the median from A to DE in triangle ADE is 10 cm. The two medians are collinear. BC = 2r cm, DE = 2r + 9 cm.
Find the length of DE.



\textbf{A.} 12cm \\
\textbf{B.} 15cm \\
\textbf{C.} 18cm \\
\textbf{D.} [IMAGE:0] \\

\textbf{Answer:} A \\
\textbf{Explanation:} [IMAGE:0]

\hrule
\vspace{1em}


\noindent
\textbf{Q224.} A small slider of mass 30 g is at rest near the bottom of a frictionless inclined plane and in contact with a light uncompressed spring. The spring is compressed by 20 cm and the slider remains in contact with it. The spring is released and causes the slider to rise up the plane to a maximum vertical height of 80 cm. The inclined plane has a different angle 60°. The slider is replaced with one of mass 20 g and the inclined plane change its angle to 30°. The spring is now compressed by 10 cm, and the new slider remains in contact with it. To what maximum vertical height does this new slider rise after it is released?



\textbf{A.} 15cm \\
\textbf{B.} 30cm \\
\textbf{C.} 40cm \\
\textbf{D.} 30
[IMAGE:0]
cm \\

\textbf{Answer:} B \\
\textbf{Explanation:} The height is not affected by the angle. Because that
:
[IMAGE:0]
[IMAGE:1]
The energy is stored 1/4 as before; The mass is 2/3 as before; the height is therefore 3/8 times as before; which is 30cm.

\hrule
\vspace{1em}


\noindent
\textbf{Q225.} Select the true statement(s) about displacement and distance:
Displacement is always equal to the distance traveled by an object.
Displacement can be zero even if the distance traveled is not zero.
Distance is a vector quantity that includes direction.



\textbf{A.} All of above \\
\textbf{B.} 1 only \\
\textbf{C.} 2 only \\
\textbf{D.} 3 only \\

\textbf{Answer:} C \\
\textbf{Explanation:} Displacement is the straight-line distance between initial and final positions, which can be zero if the object returns to its starting point. Distance is a scalar quantity.

\hrule
\vspace{1em}


\noindent
\textbf{Q226.} Which of the following statements correctly describes the relationship between velocity and acceleration?
If an object's velocity is constant, its acceleration must be zero.
If an object's acceleration is zero, its velocity must be changing.
Acceleration can only be positive if the object is speeding up.



\textbf{A.} All of above \\
\textbf{B.} 1 only \\
\textbf{C.} 2 only \\
\textbf{D.} 3 only \\

\textbf{Answer:} B \\
\textbf{Explanation:} If velocity is constant, acceleration is zero. If acceleration is zero, velocity is constant. Acceleration can be negative if the object is slowing down.

\hrule
\vspace{1em}


\noindent
\textbf{Q227.} A shape is formed by drawing a triangle ABC inside the triangle ADE. BC is parallel to DE. BC+DE=26cm and AB=3cm, BD=7cm.
Calculate the length of DE.



\textbf{A.} 12cm \\
\textbf{B.} 15cm \\
\textbf{C.} 20cm \\
\textbf{D.} [IMAGE:0] \\

\textbf{Answer:} C \\
\textbf{Explanation:} [IMAGE:0]

\hrule
\vspace{1em}


\noindent
\textbf{Q228.} 5
A small block of mass 3 g is at rest on a horizontal frictionless surface and in contact with a light uncompressed spring. The spring is compressed by 10.0 cm and the block remains in contact with it. The spring is released and causes the block to move and compress another identical spring placed at a distance (height is 1cm). If the parameters of two springs are the same including
[IMAGE:0]
,
[IMAGE:1]
, what is the compression of the second spring?



\textbf{A.} 0.2cm \\
\textbf{B.} 1.0cm \\
\textbf{C.} 2.0cm \\
\textbf{D.} 3.0cm \\

\textbf{Answer:} C \\
\textbf{Explanation:} [IMAGE:0]

\hrule
\vspace{1em}


\noindent
\textbf{Q229.} Choose the correct statement below:
Velocity is the rate of change of displacement.
Distance is a vector quantity.
Acceleration can be positive or negative depending on direction.



\textbf{A.} All of above \\
\textbf{B.} 1 and 2 \\
\textbf{C.} 3 only \\
\textbf{D.} 1 and 3 \\

\textbf{Answer:} D \\
\textbf{Explanation:} Distance is a scalar quantity.

\hrule
\vspace{1em}


\noindent
\textbf{Q230.} A shape is formed by drawing a triangle ABC inside the triangle ADE. BC is parallel to DE. AB = 8 cm, BC = p cm, DE = p + 6 cm.
[IMAGE:0]
is the smaller root of roots to the equation
[IMAGE:1]
.
Determine the length of AD.



\textbf{A.} 12cm \\
\textbf{B.} 15cm \\
\textbf{C.} 18cm \\
\textbf{D.} 20cm \\

\textbf{Answer:} D \\
\textbf{Explanation:} [IMAGE:0]

\hrule
\vspace{1em}


\noindent
\textbf{Q231.} A small slider of mass 30 g is at rest near the bottom of a frictionless slope and in contact with a light uncompressed spring as shown
The spring is compressed by 30 cm and the slider remains in contact with it.
The spring is released and causes the slider to rise up the slope to a maximum vertical height of 160 cm.
The slider is replaced with one of mass 20 g.
The spring is now compressed by 15 cm, and the new slider remains in contact with it. To what maximum vertical height does this new slider rise after it is released?
(the spring obeys Hooke’s law; assume that air resistance is negligible)



\textbf{A.} 40cm \\
\textbf{B.} 60cm \\
\textbf{C.} 90cm \\
\textbf{D.} 120cm \\

\textbf{Answer:} B \\
\textbf{Explanation:} [IMAGE:0]
[IMAGE:1]
The energy is stored 1/4 as before; The mass is 2/3 as before; the height is therefore 3/8 times as before; which is 60cm.

\hrule
\vspace{1em}


\noindent
\textbf{Q232.} Choose the correct statement below:
1. Acceleration is the change rate of the velocity.
2. Velocity is a vector.
3. Distance has not only magnitude but also direction.



\textbf{A.} [IMAGE:0] \\
\textbf{B.} [IMAGE:1] \\
\textbf{C.} [IMAGE:2] \\
\textbf{D.} [IMAGE:3] \\

\textbf{Answer:} B \\
\textbf{Explanation:} Distance is a scalar quantity.

\hrule
\vspace{1em}


\noindent
\textbf{Q233.} Choose the correct statement below:
Elastic collisions conserve both momentum and kinetic energy.
Inelastic collisions conserve momentum but not kinetic energy.
Perfectly inelastic collisions result in objects moving together post-collision.



\textbf{A.} All of above \\
\textbf{B.} 1 and 2 \\
\textbf{C.} 3 only \\
\textbf{D.} 1 only \\

\textbf{Answer:} A \\
\textbf{Explanation:} Elastic collisions conserve both; inelastic conserve momentum; perfectly inelastic result in combined motion.

\hrule
\vspace{1em}


\noindent
\textbf{Q234.} A shape is formed by drawing a triangle ABC inside the triangle ADE. BC is parallel to DE. The height from A to BC is 3 cm, the height from A to DE is 9 cm, BC = 2n cm, and DE = n + 5 cm.
Find the length of DE.



\textbf{A.} 6cm \\
\textbf{B.} 8cm \\
\textbf{C.} 10cm \\
\textbf{D.} [IMAGE:0] \\

\textbf{Answer:} A \\
\textbf{Explanation:} [IMAGE:0]

\hrule
\vspace{1em}


\noindent
\textbf{Q235.} A small slider of mass 30 g is at rest near the bottom of a frictionless slope and in contact with a light uncompressed spring as shown
The spring is compressed by 5.0 cm and the slider remains in contact with it.
The spring is released and causes the slider to rise up the slope to a maximum vertical height of 10 cm.
The slider is replaced with one of mass 50 g.
The spring is now compressed by 25 cm, and the new slider remains in contact with it. To what maximum vertical height does this new slider rise after it is released?
(the spring obeys Hooke’s law; assume that air resistance is negligible)



\textbf{A.} 40cm \\
\textbf{B.} 60cm \\
\textbf{C.} 90cm \\
\textbf{D.} 120cm \\

\textbf{Answer:} E \\
\textbf{Explanation:} [IMAGE:0]
The energy is stored 25 times as before; The mass is 5/3 as before; the height is therefore 15 times as before; which is 150cm.

\hrule
\vspace{1em}


\noindent
\textbf{Q236.} Choose the correct statement below:
Energy is conserved in all physical processes.
Non-conservative forces, like friction, convert mechanical energy into other forms.
Mechanical energy is always conserved in closed systems.



\textbf{A.} All of above \\
\textbf{B.} 1 and 2 \\
\textbf{C.} 3 only \\
\textbf{D.} 1 only \\

\textbf{Answer:} E \\
\textbf{Explanation:} Energy conservation depends on forces; non-conservative forces alter mechanical energy.

\hrule
\vspace{1em}


\noindent
\textbf{Q237.} A shape is formed by drawing a triangle ABC inside the triangle ADE. BC is parallel to DE. The perimeter of triangle ABC is 24 cm, BC = m cm, DE = m + 8 cm, and the perimeter of triangle ADE is 48 cm.
Calculate the length of AD+AE.



\textbf{A.} 24cm \\
\textbf{B.} 26cm \\
\textbf{C.} 28cm \\
\textbf{D.} 30cm \\

\textbf{Answer:} E \\
\textbf{Explanation:} [IMAGE:0]

\hrule
\vspace{1em}


\noindent
\textbf{Q238.} A small slider of mass 30 g is at rest near the bottom of a frictionless slope and in contact with a light uncompressed spring as shown
The spring is compressed by 5.0 cm and the slider remains in contact with it.
The spring is released and causes the slider to rise up the slope to a maximum vertical height of 10 cm.
The slider is replaced with one of mass 10 g.
The spring is now compressed by 15 cm, and the new slider remains in contact with it. To what maximum vertical height does this new slider rise after it is released?
(the spring obeys Hooke’s law; assume that air resistance is negligible)



\textbf{A.} 40cm \\
\textbf{B.} 60cm \\
\textbf{C.} 90cm \\
\textbf{D.} 120cm \\

\textbf{Answer:} F \\
\textbf{Explanation:} [IMAGE:0]
The energy is stored 9 times as before; The mass is 1/3 as before; the height is therefore 27 times as before; which is 270cm.

\hrule
\vspace{1em}


\noindent
\textbf{Q239.} Choose the correct statement below:
Impulse is equal to the change in momentum.
A larger force always produces a larger impulse.
Impulse only depends on the time over which a force acts.



\textbf{A.} [IMAGE:0] \\
\textbf{B.} [IMAGE:1] \\
\textbf{C.} [IMAGE:2] \\
\textbf{D.} [IMAGE:3] \\

\textbf{Answer:} D \\
\textbf{Explanation:} Impulse equals momentum change; force and time both affect impulse.

\hrule
\vspace{1em}


\noindent
\textbf{Q240.} A shape is formed by drawing a triangle ABC inside the triangle ADE. BC is parallel to DE. The area of triangle ABC is 3 cm², BC = 3w cm, DE = w + 22 cm, and the area of triangle ADE is 48 cm².
Determine the length of DE.



\textbf{A.} 16cm \\
\textbf{B.} 12cm \\
\textbf{C.} [IMAGE:0] \\
\textbf{D.} 8cm \\

\textbf{Answer:} E \\
\textbf{Explanation:} [IMAGE:0]

\hrule
\vspace{1em}


\noindent
\textbf{Q241.} A small slider of mass 30 g is at rest near the bottom of a frictionless slope and in contact with a light uncompressed spring as shown
The spring is compressed by 5.0 cm and the slider remains in contact with it.
The spring is released and causes the slider to rise up the slope to a maximum vertical height of 20 cm.
The slider is replaced with one of mass 20g.
The spring is now compressed by 15 cm, and the new slider remains in contact with it. To what maximum vertical height does this new slider rise after it is released?
(the spring obeys Hooke’s law; assume that air resistance is negligible)



\textbf{A.} 40cm \\
\textbf{B.} 60cm \\
\textbf{C.} 90cm \\
\textbf{D.} 120cm \\

\textbf{Answer:} F \\
\textbf{Explanation:} The answer is F.
[IMAGE:0]
, so
[IMAGE:1]
The energy is stored 9 times as before; The mass is 2/3 as before; the height is therefore 27/2 times as before; which is 270cm.

\hrule
\vspace{1em}


\noindent
\textbf{Q242.} Choose the correct statement below:
Energy can neither be created nor destroyed.
Kinetic energy is the energy of motion.
Potential energy is not always associated with height.



\textbf{A.} [IMAGE:0] \\
\textbf{B.} [IMAGE:1] \\
\textbf{C.} [IMAGE:2] \\
\textbf{D.} [IMAGE:3] \\

\textbf{Answer:} A \\
\textbf{Explanation:} Potential energy can be associated with other factors like elasticity.

\hrule
\vspace{1em}


\noindent
\textbf{Q243.} Choose the correct statement below:
Friction always opposes the direction of motion.
Static friction is greater than kinetic friction for the same surfaces.
Friction can be beneficial, such as in walking.



\textbf{A.} [IMAGE:0] \\
\textbf{B.} [IMAGE:1] \\
\textbf{C.} [IMAGE:2] \\
\textbf{D.} [IMAGE:3] \\

\textbf{Answer:} A \\
\textbf{Explanation:} Friction opposes motion; static friction > kinetic friction; friction aids walking.

\hrule
\vspace{1em}


\noindent
\textbf{Q244.} A shape is formed by drawing a triangle ABC inside the triangle ADE. BC is parallel to DE. AC = 4 cm, BC = z cm, DE = 2z + 1 cm, CE = z + 1 cm.
Find the length of DE.



\textbf{A.} 11cm \\
\textbf{B.} 13cm \\
\textbf{C.} 15cm \\
\textbf{D.} [IMAGE:0] \\

\textbf{Answer:} C \\
\textbf{Explanation:} [IMAGE:0]

\hrule
\vspace{1em}


\noindent
\textbf{Q245.} Choose the correct statement below:
Work is done when a force causes displacement.
Power is the rate at which work is done.
Work can be done without any energy transfer.



\textbf{A.} [IMAGE:0] \\
\textbf{B.} [IMAGE:1] \\
\textbf{C.} [IMAGE:2] \\
\textbf{D.} [IMAGE:3] \\

\textbf{Answer:} B \\
\textbf{Explanation:} Work involves energy transfer.

\hrule
\vspace{1em}


\noindent
\textbf{Q246.} A shape is formed by drawing a triangle ABC inside the triangle ADE. BC is parallel to DE. AC = 5 cm, BC = 5y cm, DE = y + 2 cm, EC = y - 3 cm.
Calculate the length of DE.



\textbf{A.} 1cm \\
\textbf{B.} 2cm \\
\textbf{C.} 3cm \\
\textbf{D.} 5cm \\

\textbf{Answer:} C \\
\textbf{Explanation:} [IMAGE:0]

\hrule
\vspace{1em}


\noindent
\textbf{Q247.} A ball decelerates uniformly from +28.0m/s to +14.0m/s in 0.006s, then accelerates back to +20.0m/s in 0.008s.
What is the total displacement during contact?



\textbf{A.} 0.220m \\
\textbf{B.} 0.620m \\
\textbf{C.} 0.226m \\
\textbf{D.} 0.262m \\

\textbf{Answer:} D \\
\textbf{Explanation:} Stage 1:
[IMAGE:0]
. Stage 2:
[IMAGE:1]
. Total
[IMAGE:2]
.

\hrule
\vspace{1em}


\noindent
\textbf{Q248.} A ball (v=12.0m/s) compresses a racket string by 0.032m  before rebounding at 8.0m/s.
What is the peak deceleration?



\textbf{A.} -1800m/s
2 \\
\textbf{B.} -2250m/s
2 \\
\textbf{C.} -3600m/s
2 \\
\textbf{D.} -4400m/s
2 \\

\textbf{Answer:} B \\
\textbf{Explanation:} Energy loss implies non-constant force, but assuming average deceleration:
[IMAGE:0]
.
(pay attention to the words "peak" and "deceleration")

\hrule
\vspace{1em}


\noindent
\textbf{Q249.} A topspin ball slows horizontally from 30.0m/s  to 24.0m/s  while gaining 6.0m/s  downward due to spin. Contact time is 0.0034s.
What is the net acceleration?



\textbf{A.} 800m/s
2 \\
\textbf{B.} 1050m/s
2 \\
\textbf{C.} 2000m/s
2 \\
\textbf{D.} 2500m/s
2 \\

\textbf{Answer:} D \\
\textbf{Explanation:} [IMAGE:0]
. Net
[IMAGE:1]
.
[IMAGE:2]
.

\hrule
\vspace{1em}


\noindent
\textbf{Q250.} A tennis ball (v=25.0m/s) penetrates a net, slowing to 5.0m/s  over 0.020m.
What is the deceleration magnitude?



\textbf{A.} 30000m/s
2 \\
\textbf{B.} 25000m/s
2 \\
\textbf{C.} 20000m/s
2 \\
\textbf{D.} 15000m/s
2 \\

\textbf{Answer:} D \\
\textbf{Explanation:} [IMAGE:0]

\hrule
\vspace{1em}


\noindent
\textbf{Q251.} A tennis ball approaches at 32.0m/s  horizontally. The racket applies a constant upward force, adding a vertical velocity of 8.0m/s while reducing horizontal speed to 17.0m/s . The contact lasts 0.005s .
What is the magnitude of the net acceleration?



\textbf{A.} 3250m/s
2 \\
\textbf{B.} 3600m/s
2 \\
\textbf{C.} 4250m/s
2 \\
\textbf{D.} 5600m/s
2 \\

\textbf{Answer:} C \\
\textbf{Explanation:} [IMAGE:0]
. Net
[IMAGE:1]
. Acceleration
[IMAGE:2]
.

\hrule
\vspace{1em}


\noindent
\textbf{Q252.} A tennis ball strikes the court surface at 18.0m/s at a
[IMAGE:0]
angle. The bounce reverses the vertical velocity component and reduces the horizontal speed by 30%. The contact time is 0.015s.
What is the magnitude of the average acceleration during impact? (
[IMAGE:1]
)



\textbf{A.} 830m/s
2 \\
\textbf{B.} 915m/s
2 \\
\textbf{C.} 1020m/s
2 \\
\textbf{D.} 1240m/s
2 \\

\textbf{Answer:} D \\
\textbf{Explanation:} Vertical velocity changes from
[IMAGE:0]
to
[IMAGE:1]
(
[IMAGE:2]
), while horizontal velocity changes from
[IMAGE:3]
to
[IMAGE:4]
(
[IMAGE:5]
). Total
[IMAGE:6]
. Acceleration
[IMAGE:7]
.

\hrule
\vspace{1em}


\noindent
\textbf{Q253.} A tennis ball travelling at 24.0m/s is hit by a racket. As a result of the impact, the ball returns back along its original path having undergone a change in velocity of 36.0m/s. The acceleration of the ball whilst in contact with the racket is constant with magnitude
[IMAGE:0]
.
What is the total distance travelled by the ball whilst in contact with the racket?



\textbf{A.} 3.00cm \\
\textbf{B.} 7.50cm \\
\textbf{C.} 8.00cm \\
\textbf{D.} 15.2cm \\

\textbf{Answer:} D \\
\textbf{Explanation:} the initial velocity is 24.0m/s; the terminal one is -12.0m/s, the process is not symmetrical in time;
the first half is  24.0*24.0/3600/2=0.08m=8.00cm.
the second half is 12.0*12.0/3600/2=0.020m=2.00cm.
Thus, the answer is 10.00cm.

\hrule
\vspace{1em}


\noindent
\textbf{Q254.} A tennis ball travelling at 10.0m/s is hit by a racket. As a result of the impact, the ball returns back along its original path having undergone a change in velocity of 40.0m/s. The acceleration of the ball whilst in contact with the racket is constant with magnitude
[IMAGE:0]
.
What is the total distance travelled by the ball whilst in contact with the racket?



\textbf{A.} 1.25cm \\
\textbf{B.} 5.00cm \\
\textbf{C.} 10.00cm \\
\textbf{D.} 11.25cm \\

\textbf{Answer:} E \\
\textbf{Explanation:} the initial velocity is 10.0m/s; the terminal one is  -30m/s, the process is not symmetrical in time; the first half is
[IMAGE:0]
; the second half is
[IMAGE:1]
. Thus, the answer is
[IMAGE:2]
.

\hrule
\vspace{1em}


\noindent
\textbf{Q255.} A tennis ball travelling at 10.0m/s is hit by a racket. As a result of the impact, the ball returns back along its original path having undergone a change in velocity of 20.0m/s. The acceleration of the ball whilst in contact with the racket is constant with magnitude
[IMAGE:0]
.
What is the total distance travelled by the ball whilst in contact with the racket?



\textbf{A.} 2.00cm \\
\textbf{B.} 2.50cm \\
\textbf{C.} 3.00cm \\
\textbf{D.} 14.4cm \\

\textbf{Answer:} C \\
\textbf{Explanation:} the initial velocity is 10.0m/s
; the terminal one is -10.0m/s
, the process is symmetrical in time; the first half is
[IMAGE:0]
. Thus, the answer is 3.00cm
.
(原题的改题题目有问题应该为5000)

\hrule
\vspace{1em}


\noindent
\textbf{Q256.} A tennis ball travelling at 5.5m/s is hit by a racket. As a result of the impact, the ball returns back along its original path having undergone a change in velocity of  11.0m/s. The acceleration of the ball whilst in contact with the racket is constant with magnitude
[IMAGE:0]
.
What is the total distance travelled by the ball whilst in contact with the racket?



\textbf{A.} 1.00cm \\
\textbf{B.} 2.00cm \\
\textbf{C.} 4.00cm \\
\textbf{D.} 4.4cm \\

\textbf{Answer:} A \\
\textbf{Explanation:} the initial velocity is 5.5m/s
; the terminal one is -5.5m/s
, the process is symmetrical in time; the first half is
[IMAGE:0]
. Thus, the answer is 1.00cm
.

\hrule
\vspace{1em}


\noindent
\textbf{Q257.} Rocket's exhaust velocity decreases linearly with time (
[IMAGE:0]
,
[IMAGE:1]
) while mass flow rate is constant as
[IMAGE:2]
. The initial mass of rocket is m0
.
[IMAGE:3]
is a constant.
Check the acceleration behavior?



\textbf{A.} [IMAGE:0] \\
\textbf{B.} [IMAGE:1] \\
\textbf{C.} [IMAGE:2] \\
\textbf{D.} [IMAGE:3] \\

\textbf{Answer:} F \\
\textbf{Explanation:} Thrust formula:
[IMAGE:0]
(momentum theorem)
Given conditions:
[IMAGE:1]
(constant mass flow rate)
[IMAGE:2]
(exhaust velocity decreases linearly with time)
Therefore, thrust:
[IMAGE:3]
(linear increase)
Rocket mass:
[IMAGE:4]
(linear decrease)
Acceleration:
[IMAGE:5]

\hrule
\vspace{1em}


\noindent
\textbf{Q258.} Rocket's exhaust velocity decreases linearly with time (
[IMAGE:0]
,
[IMAGE:1]
) while mass flow rate is constant as
[IMAGE:2]
. The initial mass of rocket is m0
[IMAGE:3]
.
[IMAGE:4]
is a constant.
Check the acceleration behavior?



\textbf{A.} [IMAGE:0] \\
\textbf{B.} [IMAGE:1] \\
\textbf{C.} [IMAGE:2] \\
\textbf{D.} [IMAGE:3] \\

\textbf{Answer:} D \\
\textbf{Explanation:} Thrust formula:
[IMAGE:0]
(momentum theorem)
Given conditions:
[IMAGE:1]
(constant mass flow rate)
[IMAGE:2]
(exhaust velocity decreases linearly with time)
Therefore, thrust:
[IMAGE:3]
(linear increase)
Rocket mass:
[IMAGE:4]
(linear decrease)
Acceleration:
[IMAGE:5]

\hrule
\vspace{1em}


\noindent
\textbf{Q259.} Choose the correct statement below:
Potential energy is always associated with gravitational fields.
Elastic potential energy is stored in deformed objects.
Kinetic energy can be negative in certain reference frames.



\textbf{A.} [IMAGE:0] \\
\textbf{B.} [IMAGE:1] \\
\textbf{C.} [IMAGE:2] \\
\textbf{D.} [IMAGE:3] \\

\textbf{Answer:} E \\
\textbf{Explanation:} Potential energy types vary; kinetic energy is non-negative.

\hrule
\vspace{1em}


\noindent
\textbf{Q260.} Choose the correct statement below:
Work done is equal to the change in kinetic energy according to the work-energy theorem.
Power is the rate at which energy is transferred or converted.
Work can be done without any displacement occurring.



\textbf{A.} [IMAGE:0] \\
\textbf{B.} [IMAGE:1] \\
\textbf{C.} [IMAGE:2] \\
\textbf{D.} [IMAGE:3] \\

\textbf{Answer:} B \\
\textbf{Explanation:} Work requires displacement; power is energy transfer rate.

\hrule
\vspace{1em}


\noindent
\textbf{Q261.} A shape is formed by drawing a triangle ABC inside the triangle ADE. BC is parallel to DE. AB = x-3 cm BC = x - 3 cm DE = x + 3 cm DB = 3 cm.
Calculate the length of DE.



\textbf{A.} 5cm \\
\textbf{B.} 6cm \\
\textbf{C.} 9cm \\
\textbf{D.} 10cm \\

\textbf{Answer:} B \\
\textbf{Explanation:} [IMAGE:0]

\hrule
\vspace{1em}


\noindent
\textbf{Q262.} Rocket's exhaust velocity decreases linearly with time (
[IMAGE:0]
,
[IMAGE:1]
) while mass flow rate is constant as
[IMAGE:2]
. The initial mass of rocket is m0
[IMAGE:3]
.
[IMAGE:4]
is a constant.
Check the acceleration behavior?



\textbf{A.} [IMAGE:0] \\
\textbf{B.} [IMAGE:1] \\
\textbf{C.} [IMAGE:2] \\
\textbf{D.} [IMAGE:3] \\

\textbf{Answer:} E \\
\textbf{Explanation:} Thrust formula:
[IMAGE:0]
(momentum theorem)
Given conditions:
[IMAGE:1]
(constant mass flow rate)
[IMAGE:2]
(exhaust velocity decreases linearly with time)
Therefore, thrust:
[IMAGE:3]
(linear increase)
Rocket mass:
[IMAGE:4]
(linear decrease)
Acceleration:
[IMAGE:5]

\hrule
\vspace{1em}


\noindent
\textbf{Q263.} Choose the correct statement below:
Momentum is conserved in all types of collisions if no external forces act.
In an inelastic collision, kinetic energy is partially converted into other forms of energy.
Elastic collisions occur only between perfectly rigid bodies.



\textbf{A.} [IMAGE:0] \\
\textbf{B.} [IMAGE:1] \\
\textbf{C.} [IMAGE:2] \\
\textbf{D.} [IMAGE:3] \\

\textbf{Answer:} B \\
\textbf{Explanation:} Momentum conservation depends on external forces; inelastic collisions involve energy conversion.

\hrule
\vspace{1em}


\noindent
\textbf{Q264.} A solid frustum of a cone with lower base radius
[IMAGE:0]
[IMAGE:1]
, upper base radius
[IMAGE:2]
, and height
[IMAGE:3]
fits inside a hollow cylinder. The cylinder has the an internal radius equal to
[IMAGE:4]
times the upper base radius of the frustum and a height equal to the height of the frustum. What fraction of the empty space is occupied in the cylinder?



\textbf{A.} [IMAGE:0] \\
\textbf{B.} [IMAGE:1] \\
\textbf{C.} [IMAGE:2] \\
\textbf{D.} [IMAGE:3] \\

\textbf{Answer:} E \\
\textbf{Explanation:} [IMAGE:0]

\hrule
\vspace{1em}


\noindent
\textbf{Q265.} Choose the correct statement below:
1. The kinetic energy is always conserved when no external forces and non-conservative forces are on then system.
2. The objects stop during a perfect inelastic collision.
3. The kinetic energy is conserved in a perfect elastic collision.



\textbf{A.} [IMAGE:0] \\
\textbf{B.} [IMAGE:1] \\
\textbf{C.} [IMAGE:2] \\
\textbf{D.} [IMAGE:3] \\

\textbf{Answer:} C \\
\textbf{Explanation:} Perfectly inelastic collisions result in stuck objects moving at a common velocity, not necessarily zero.

\hrule
\vspace{1em}


\noindent
\textbf{Q266.} Choose the correct statement below concerning the conservation of momentum:
The law of conservation of momentum states that the total momentum of an isolated system remains constant.
In a closed system, the momentum of individual particles may change, but the total momentum remains the same.
Momentum conservation does not apply if there are external forces acting on the system.



\textbf{A.} [IMAGE:0] \\
\textbf{B.} [IMAGE:1] \\
\textbf{C.} [IMAGE:2] \\
\textbf{D.} [IMAGE:3] \\

\textbf{Answer:} A \\
\textbf{Explanation:} Momentum conservation applies only in the absence of external forces.

\hrule
\vspace{1em}


\noindent
\textbf{Q267.} A rokect with mass
[IMAGE:0]
. For first half of fuel (total mass is mf
): burns at rate R with thrust
[IMAGE:1]
. For second half: burns at 2R with thrust
[IMAGE:2]
.
What describes acceleration?



\textbf{A.} Constant throughout \\
\textbf{B.} Jumps up at midpoint \\
\textbf{C.} Jumps down midpoint \\
\textbf{D.} Always increasing \\

\textbf{Answer:} A \\
\textbf{Explanation:} First phase:
[IMAGE:0]
with final value
[IMAGE:1]
Second phase:
[IMAGE:2]
, which does not change.

\hrule
\vspace{1em}


\noindent
\textbf{Q268.} Choose the correct statement below about momentum and impulse:
Impulse is equal to the change in momentum of an object.
A large force applied for a short time can produce the same impulse as a small force applied for a long time.
Impulse is a vector quantity, and its direction is the same as the direction of the force.



\textbf{A.} [IMAGE:0] \\
\textbf{B.} [IMAGE:1] \\
\textbf{C.} [IMAGE:2] \\
\textbf{D.} [IMAGE:3] \\

\textbf{Answer:} A \\
\textbf{Explanation:} All statements are correct regarding impulse and momentum.

\hrule
\vspace{1em}


\noindent
\textbf{Q269.} A solid frustum of a cone with lower base radius R
, upper base radius
[IMAGE:0]
, and height h
fits inside a hollow cylinder. The cylinder has the an internal radius equal to
[IMAGE:1]
times the upper base radius of the frustum and a height equal to the height of the frustum. What fraction of the space inside the cylinder is occupied by the frustum?



\textbf{A.} [IMAGE:0] \\
\textbf{B.} [IMAGE:1] \\
\textbf{C.} [IMAGE:2] \\
\textbf{D.} [IMAGE:3] \\

\textbf{Answer:} A \\
\textbf{Explanation:} [IMAGE:0]

\hrule
\vspace{1em}


\noindent
\textbf{Q270.} Choose the correct statement below regarding momentum and collisions:
In an elastic collision, both momentum and kinetic energy are conserved.
In an inelastic collision, only momentum is conserved, and kinetic energy is partially converted to other forms.
The total momentum of a system of particles is always zero if there are no external forces.



\textbf{A.} [IMAGE:0] \\
\textbf{B.} [IMAGE:1] \\
\textbf{C.} [IMAGE:2] \\
\textbf{D.} [IMAGE:3] \\

\textbf{Answer:} B \\
\textbf{Explanation:} The total momentum of a system is conserved if there are no external forces, but it is not necessarily zero.

\hrule
\vspace{1em}


\noindent
\textbf{Q271.} Choose the correct statement below:
Elastic collisions conserve both momentum and kinetic energy.
Perfectly inelastic collisions result in maximum kinetic energy loss.
Inelastic collisions conserve momentum but not kinetic energy.



\textbf{A.} [IMAGE:0] \\
\textbf{B.} [IMAGE:1] \\
\textbf{C.} [IMAGE:2] \\
\textbf{D.} [IMAGE:3] \\

\textbf{Answer:} A \\
\textbf{Explanation:} Elastic collisions conserve both; perfectly inelastic maximize energy loss.

\hrule
\vspace{1em}


\noindent
\textbf{Q272.} A rocket (mass is m) adjusts its thrust to always equal 5% of its instantaneous fuel mass (F = 0.2M). Fuel burns at constant rate(
[IMAGE:0]
, where
[IMAGE:1]
, and the inital mass of fuel is
[IMAGE:2]
).
How does acceleration behave?



\textbf{A.} Constant at 0.1
[IMAGE:0] \\
\textbf{B.} Increases \\
\textbf{C.} Decreases \\
\textbf{D.} Proportional to 1/m \\

\textbf{Answer:} C \\
\textbf{Explanation:} Based on acceleration-force equation, it has
initial state:
[IMAGE:0]
operating state:
[IMAGE:1]
it can be proved that
[IMAGE:2]

\hrule
\vspace{1em}


\noindent
\textbf{Q273.} A solid regular octahedron with edge length
[IMAGE:0]
fits inside a hollow sphere. The sphere has a diameter equal to the distance between two opposite vertices of the octahedron. What fraction of the empty space inside the sphere is?



\textbf{A.} [IMAGE:0] \\
\textbf{B.} [IMAGE:1] \\
\textbf{C.} [IMAGE:2] \\
\textbf{D.} [IMAGE:3] \\

\textbf{Answer:} C \\
\textbf{Explanation:} [IMAGE:0]

\hrule
\vspace{1em}


\noindent
\textbf{Q274.} Choose the correct statement below:
Impulse equals the change in momentum of an object.
A longer collision time reduces the force experienced during impact.
Impulse is independent of the time over which a force acts.



\textbf{A.} [IMAGE:0] \\
\textbf{B.} [IMAGE:1] \\
\textbf{C.} [IMAGE:2] \\
\textbf{D.} [IMAGE:3] \\

\textbf{Answer:} B \\
\textbf{Explanation:} Impulse depends on force and time; longer time reduces force.

\hrule
\vspace{1em}


\noindent
\textbf{Q275.} Choose the correct statement below:
Center of mass motion is unaffected by internal forces.
Exploding a bomb fragments the mass but doesn’t change the system’s center of mass velocity.
Friction is an internal force in a system of sliding blocks.



\textbf{A.} [IMAGE:0] \\
\textbf{B.} [IMAGE:1] \\
\textbf{C.} [IMAGE:2] \\
\textbf{D.} [IMAGE:3] \\

\textbf{Answer:} B \\
\textbf{Explanation:} Center of mass motion ignores internal forces; friction here is external.

\hrule
\vspace{1em}


\noindent
\textbf{Q276.} A rocket adjusts its thrust in discrete steps to always equal 20% of its instantaneous total mass at the time of each adjustment (F=0.2m). The fuel burns at a constant rate, but the thrust is only recalculated and adjusted at specific, equally spaced time intervals (e.g., every 10 seconds).
How does the thrust behave?



\textbf{A.} [IMAGE:0] \\
\textbf{B.} [IMAGE:1] \\
\textbf{C.} [IMAGE:2] \\
\textbf{D.} [IMAGE:3] \\

\textbf{Answer:} F \\
\textbf{Explanation:} Discrete recalculation and adjustment makes the piecewise constant.
(原题数字有点问题0.1改为0.2)

\hrule
\vspace{1em}


\noindent
\textbf{Q277.} A solid square pyramid with a square base of side length a
and height h
fits inside a hollow cube. The cube has an edge length equal to the slant height of the pyramid. And
[IMAGE:0]
. What is the empty fraction of the space inside the cube?



\textbf{A.} [IMAGE:0] \\
\textbf{B.} [IMAGE:1] \\
\textbf{C.} [IMAGE:2] \\
\textbf{D.} [IMAGE:3] \\

\textbf{Answer:} D \\
\textbf{Explanation:} [IMAGE:0]

\hrule
\vspace{1em}


\noindent
\textbf{Q278.} Choose the correct statement below:
A rocket’s thrust propels it forward by expelling mass backward.
The total momentum of a rocket and its expelled fuel remains constant in a vacuum.
A rocket’s speed increases continuously in space due to constant thrust.



\textbf{A.} [IMAGE:0] \\
\textbf{B.} [IMAGE:1] \\
\textbf{C.} [IMAGE:2] \\
\textbf{D.} [IMAGE:3] \\

\textbf{Answer:} B \\
\textbf{Explanation:} Thrust depends on mass expulsion; momentum is conserved, but the acceleration or deceleration is uncertain.

\hrule
\vspace{1em}


\noindent
\textbf{Q279.} A rocket travelling in space is burning its fuel at a increasing rate
[IMAGE:0]
, where c0 is inital rate of burning fuel,
[IMAGE:1]
t denotes time and a>0 is a constant. By expelling the burnt fuel through a nozzle, the engine is applying a constant force to the rocket.
What is happening to the magnitude of the velocity of the rocket?



\textbf{A.} It is increasing at an increasing rate. \\
\textbf{B.} It is increasing at a constant rate. \\
\textbf{C.} It is increasing at a decreasing rate. \\
\textbf{D.} It is not changing \\

\textbf{Answer:} B \\
\textbf{Explanation:} The purposive force is a constant; the mass is decreasing.
Thus, the acceleration is therefore increasing;
the jerk(rate of change of acceleration) is
[IMAGE:0]
, which means the acceleration is increasing at an increasing rate in time.

\hrule
\vspace{1em}


\noindent
\textbf{Q280.} Choose the correct statement below:
Kinetic energy is always conserved in elastic collisions.
In inelastic collisions, total kinetic energy decreases.
Momentum is conserved in both elastic and inelastic collisions.



\textbf{A.} [IMAGE:0] \\
\textbf{B.} [IMAGE:1] \\
\textbf{C.} [IMAGE:2] \\
\textbf{D.} [IMAGE:3] \\

\textbf{Answer:} A \\
\textbf{Explanation:} Kinetic energy conservation varies by collision type, but momentum is always conserved.

\hrule
\vspace{1em}


\noindent
\textbf{Q281.} A solid torus (doughnut - shaped) with inner radius
[IMAGE:0]
and outer radius 2r
fits inside a hollow cylinder. And its height is r
. The cylinder has a radius equal to the outer radius of the torus and a height equal to the (outer) diameter of the torus's tube. What fraction of the space inside the cylinder is taken up by the torus?



\textbf{A.} [IMAGE:0] \\
\textbf{B.} [IMAGE:1] \\
\textbf{C.} [IMAGE:2] \\
\textbf{D.} [IMAGE:3] \\

\textbf{Answer:} C \\
\textbf{Explanation:} [IMAGE:0]

\hrule
\vspace{1em}


\noindent
\textbf{Q282.} Choose the correct statement below:
Angular momentum is conserved in a system with no external torques.
When a figure skater pulls in their arms, their angular velocity decreases.
The total angular momentum of a closed system remains constant if no external torques act.



\textbf{A.} [IMAGE:0] \\
\textbf{B.} [IMAGE:1] \\
\textbf{C.} [IMAGE:2] \\
\textbf{D.} [IMAGE:3] \\

\textbf{Answer:} B \\
\textbf{Explanation:} Angular momentum conservation depends on external torques; pulling in arms increases angular velocity.

\hrule
\vspace{1em}


\noindent
\textbf{Q283.} Choose the correct statement below:
The momentum of a system is always conserved.
When two spheres collides, the momentum of two spheres system is not conserved.
When two spheres collides, the momentum of each sphere is conserved if treated individually.



\textbf{A.} [IMAGE:0] \\
\textbf{B.} [IMAGE:1] \\
\textbf{C.} [IMAGE:2] \\
\textbf{D.} [IMAGE:3] \\

\textbf{Answer:} E \\
\textbf{Explanation:} The momentum of the system is conserved when no external force(friction etc) is not allied on the system; internal force is not considered, however, the internal force affects the momentum of each object.

\hrule
\vspace{1em}


\noindent
\textbf{Q284.} A rocket travelling in space is burning its fuel at a decreasing rate
[IMAGE:0]
, where C0
is inital rate of burning fuel, t
denotes time and a>0
is a constant. By expelling the burnt fuel through a nozzle, the engine is applying a constant force to the rocket.
What is happening to the magnitude of the velocity of the rocket?



\textbf{A.} [IMAGE:0] \\
\textbf{B.} [IMAGE:1] \\
\textbf{C.} [IMAGE:2] \\
\textbf{D.} [IMAGE:3] \\

\textbf{Answer:} B,E \\
\textbf{Explanation:} The purposive force is a constant; the mass is decreasing.
PS: Though the fuel consumption is at a decreasing rate, the mass is still decreasing.
Thus, the acceleration is therefore increasing;
the jerk(rate of change of acceleration) is
[IMAGE:0]
, which means the acceleration increasing is at a contant rate in time.

\hrule
\vspace{1em}


\noindent
\textbf{Q285.} A solid regular tetrahedron with edge length
[IMAGE:0]
fits inside a hollow cube. The cube has an edge length equal to the height of the tetrahedron. What fraction of the space inside the cube is occupied by the tetrahedron?



\textbf{A.} [IMAGE:0] \\
\textbf{B.} [IMAGE:1] \\
\textbf{C.} [IMAGE:2] \\
\textbf{D.} [IMAGE:3] \\

\textbf{Answer:} D \\
\textbf{Explanation:} [IMAGE:0]

\hrule
\vspace{1em}


\noindent
\textbf{Q286.} Choose the correct statement below:
A glider’s lift force depends on its wing area and airspeed.
Air resistance (drag) increases with the glider’s speed.
A glider’s acceleration is always downward due to gravity.



\textbf{A.} [IMAGE:0] \\
\textbf{B.} [IMAGE:1] \\
\textbf{C.} [IMAGE:2] \\
\textbf{D.} [IMAGE:3] \\

\textbf{Answer:} B \\
\textbf{Explanation:} Lift and drag depend on speed and area; the total acceleration isn’t always downward.

\hrule
\vspace{1em}


\noindent
\textbf{Q287.} A solid hemisphere of radius
[IMAGE:0]
is inside a hollow cone. The cone has a circular base with radius 2R
and a height equal to 3R
. What fraction of the empty space inside the cone?



\textbf{A.} [IMAGE:0] \\
\textbf{B.} [IMAGE:1] \\
\textbf{C.} [IMAGE:2] \\
\textbf{D.} [IMAGE:3] \\

\textbf{Answer:} E \\
\textbf{Explanation:} [IMAGE:0]

\hrule
\vspace{1em}


\noindent
\textbf{Q288.} A rocket is initially at rest in deep space. The rocket's engines are firing, applying a constant thrust force in one direction. However, an equal and opposite force is simultaneously applied to the rocket (e.g., via a tether or some external mechanism), resulting in no net force.
What is happening to the magnitude of the distance of the rocket?



\textbf{A.} It is decreasing at an increasing rate. \\
\textbf{B.} It is decreasing at a constant rate. \\
\textbf{C.} It is decreasing at a decreasing rate. \\
\textbf{D.} It is not changing. \\

\textbf{Answer:} C \\
\textbf{Explanation:} The net force on the rocket is zero, so the acceleration is also zero. And the velocity remains constant.

\hrule
\vspace{1em}


\noindent
\textbf{Q289.} A solid right - circular cone with radius r=5R and height h=12R fits inside a hollow cylinder. The cylinder has the same internal radius as the cone and a height equal to the slant height of the cone. What fraction of the space inside the cylinder is occupied by the cone?



\textbf{A.} [IMAGE:0] \\
\textbf{B.} [IMAGE:1] \\
\textbf{C.} [IMAGE:2] \\
\textbf{D.} [IMAGE:3] \\

\textbf{Answer:} C \\
\textbf{Explanation:} [IMAGE:0]

\hrule
\vspace{1em}


\noindent
\textbf{Q290.} A rocket travelling in space is burning its fuel at a constant rate. By expelling the burnt fuel through a nozzle, the engine is applying a constant force to the rocket. However, the rocket is also experiencing an equal and opposite drag force from the extremely thin interstellar medium, resulting in no net force acting on the rocket.
What is happening to the magnitude of the velocity of the rocket?



\textbf{A.} It is decreasing at an increasing rate. \\
\textbf{B.} It is decreasing at a constant rate. \\
\textbf{C.} It is decreasing at a decreasing rate. \\
\textbf{D.} It is not changing. \\

\textbf{Answer:} D \\
\textbf{Explanation:} The net force on the rocket is zero, so the acceleration is also zero. And the velocity remains constant.

\hrule
\vspace{1em}


\noindent
\textbf{Q291.} A solid cube of side length
[IMAGE:0]
fits perfectly inside a hollow rectangular prism. The rectangular prism has the same internal length and width as the side length of the cube, and its height is equal to the diagonal of the cube's body. What fraction of the space inside the rectangular prism is taken up by the cube?



\textbf{A.} [IMAGE:0] \\
\textbf{B.} [IMAGE:1] \\
\textbf{C.} [IMAGE:2] \\
\textbf{D.} [IMAGE:3] \\

\textbf{Answer:} C \\
\textbf{Explanation:} [IMAGE:0]

\hrule
\vspace{1em}


\noindent
\textbf{Q292.} A solid sphere of radius r fits inside a hollow cylinder. The cylinder has the same internal diameter and as the diameter of the sphere and has the height twice as the diameter of the sphere. What fraction of the empty space inside the cylinder is taken up?



\textbf{A.} [IMAGE:0] \\
\textbf{B.} [IMAGE:1] \\
\textbf{C.} [IMAGE:2] \\
\textbf{D.} [IMAGE:3] \\

\textbf{Answer:} D \\
\textbf{Explanation:} [IMAGE:0]

\hrule
\vspace{1em}


\noindent
\textbf{Q293.} A future vehicle of mass 560 kg travels in a straight line along a horizontal road, as shown in the acceleration/deceleration–time graph.
What is the average resultant force acting on the vehicle over the time for which it is accelerating / decelerating?



\textbf{A.} -380N \\
\textbf{B.} 420N \\
\textbf{C.} -960N \\
\textbf{D.} -7500N \\

\textbf{Answer:} G \\
\textbf{Explanation:} What is the average resultant force acting on the vehicle over the time for which it is accelerating?
The vehicle is accelerating from 0s to 10s (all the time because acceleration is bigger than zero, which may be a trap).
0s~5s:
[IMAGE:0]
5s~10s:
[IMAGE:1]
Thus, the acceleration in average is therefore
[IMAGE:2]
;
[IMAGE:3]
. (F option is a disturbance term)

\hrule
\vspace{1em}


\noindent
\textbf{Q294.} A rectangular prism has dimensions 2, 4, and
[IMAGE:0]
. What is the perimeter of a triangular ABC (the dashed line triangular in the diagram below)? B is in the center of the bottom face. A and C lie in the vertices of the rectangular prism.



\textbf{A.} [IMAGE:0] \\
\textbf{B.} [IMAGE:1] \\
\textbf{C.} [IMAGE:2] \\
\textbf{D.} [IMAGE:3] \\

\textbf{Answer:} D \\
\textbf{Explanation:} [IMAGE:0]

\hrule
\vspace{1em}


\noindent
\textbf{Q295.} A car of mass 660 kg travels in a straight line along a horizontal road, as shown in the deceleration–time graph.
What is the average resultant force acting on the car over the time for which it is decelerating?



\textbf{A.} -699N \\
\textbf{B.} -792N \\
\textbf{C.} -840N \\
\textbf{D.} -1190N \\

\textbf{Answer:} B \\
\textbf{Explanation:} By the gradient of the graph in the linear region; the velocity of 5s is
[IMAGE:0]
; the acceleration in average is therefore
[IMAGE:1]
;
[IMAGE:2]
.

\hrule
\vspace{1em}


\noindent
\textbf{Q296.} A cube has sides of length
[IMAGE:0]
. What is the area of a triangular ABC (the dashed line triangular in the diagram below)?



\textbf{A.} [IMAGE:0] \\
\textbf{B.} [IMAGE:1] \\
\textbf{C.} [IMAGE:2] \\
\textbf{D.} [IMAGE:3] \\

\textbf{Answer:} D \\
\textbf{Explanation:} 

\hrule
\vspace{1em}


\noindent
\textbf{Q297.} A future vehicle of mass 250 kg travels in a straight line along a horizontal road, as shown in the acceleration–time graph.
What is the final kinetic energy of the vehicle?



\textbf{A.} 120J \\
\textbf{B.} 180000J \\
\textbf{C.} 360000J \\
\textbf{D.} 720000J \\

\textbf{Answer:} E \\
\textbf{Explanation:} The vehicle is accelerating from 0s to 10s (all the time because acceleration is not zero).
0s~5s:
[IMAGE:0]
5s~10s:
[IMAGE:1]
Thus,
[IMAGE:2]

\hrule
\vspace{1em}


\noindent
\textbf{Q298.} A cube has sides of length
[IMAGE:0]
. What is the perimeter of a triangular ABC (the dashed line triangular in the diagram below)?



\textbf{A.} [IMAGE:0] \\
\textbf{B.} [IMAGE:1] \\
\textbf{C.} [IMAGE:2] \\
\textbf{D.} [IMAGE:3] \\

\textbf{Answer:} B \\
\textbf{Explanation:} [IMAGE:0]

\hrule
\vspace{1em}


\noindent
\textbf{Q299.} A future vehicle of mass 600 kg travels in a straight line along a horizontal road, as shown in the acceleration–time graph.
What is the final velocity of the vehicle?



\textbf{A.} 25m/s \\
\textbf{B.} 50m/s \\
\textbf{C.} 100m/s \\
\textbf{D.} 120m/s \\

\textbf{Answer:} F \\
\textbf{Explanation:} The vehicle is accelerating from 0s to 10s (all the time because acceleration is not zero).
0-3s:
[IMAGE:0]
3-6s:
[IMAGE:1]
so the final velocity is:
[IMAGE:2]

\hrule
\vspace{1em}


\noindent
\textbf{Q300.} Choose the correct statement below:
A hot air balloon rises because the buoyant force exceeds its mass.
Air resistance slows the balloon’s ascent.
The balloon’s acceleration is constant during flight.



\textbf{A.} [IMAGE:0] \\
\textbf{B.} [IMAGE:1] \\
\textbf{C.} [IMAGE:2] \\
\textbf{D.} [IMAGE:3] \\

\textbf{Answer:} E \\
\textbf{Explanation:} A hot air balloon rises because the buoyant force exceeds its gravity or weight (gravity or weight are not equal to the mass). Buoyancy drives ascent, but drag opposes it; acceleration varies.

\hrule
\vspace{1em}


\noindent
\textbf{Q301.} A cube has sides of length 6. What is the length of a line joining the midpoint of one face to the midpoint of an adjacent face (the dashed line in the diagram below)?



\textbf{A.} [IMAGE:0] \\
\textbf{B.} [IMAGE:1] \\
\textbf{C.} [IMAGE:2] \\
\textbf{D.} [IMAGE:3] \\

\textbf{Answer:} B \\
\textbf{Explanation:} [IMAGE:0]

\hrule
\vspace{1em}


\noindent
\textbf{Q302.} A future vehicle of mass 340 kg travels in a straight line along a horizontal road, as shown in the acceleration–time graph.
What is the average resultant force acting on the vehicle over the time for which it is accelerating?



\textbf{A.} 1080N \\
\textbf{B.} 3070N \\
\textbf{C.} 4090N \\
\textbf{D.} 5100N \\

\textbf{Answer:} D \\
\textbf{Explanation:} The vehicle is accelerating from 0s to 10s (all the time because acceleration is not zero).
0s~5s:
[IMAGE:0]
5s~10s:
[IMAGE:1]
Thus, the acceleration in average is therefore
[IMAGE:2]

\hrule
\vspace{1em}


\noindent
\textbf{Q303.} Choose the correct statement below:
A raindrop’s terminal velocity depends on its size and shape.
Larger raindrops experience greater air resistance and fall faster.
All raindrops fall at the same speed regardless of size due to Galileo's experiments and conclusions.



\textbf{A.} [IMAGE:0] \\
\textbf{B.} [IMAGE:1] \\
\textbf{C.} [IMAGE:2] \\
\textbf{D.} [IMAGE:3] \\

\textbf{Answer:} B \\
\textbf{Explanation:} Terminal velocity varies with size and shape; larger drops fall faster which depends on the terminal velocity.

\hrule
\vspace{1em}


\noindent
\textbf{Q304.} A rectangular prism has dimensions 1, 2, and 3. What is the length of a line joining a vertex to the midpoint of the middle opposite edge (the dashed line in the diagram below)?



\textbf{A.} [IMAGE:0] \\
\textbf{B.} [IMAGE:1] \\
\textbf{C.} [IMAGE:2] \\
\textbf{D.} [IMAGE:3] \\

\textbf{Answer:} D \\
\textbf{Explanation:} [IMAGE:0]

\hrule
\vspace{1em}


\noindent
\textbf{Q305.} Choose the correct statement below:
A paper airplane with a larger wing area glides longer due to lift.
Air resistance opposes the motion of the paper airplane.
The paper airplane’s acceleration is constant during flight.



\textbf{A.} [IMAGE:0] \\
\textbf{B.} [IMAGE:1] \\
\textbf{C.} [IMAGE:2] \\
\textbf{D.} [IMAGE:3] \\

\textbf{Answer:} B \\
\textbf{Explanation:} Lift and drag govern airplane motion, but acceleration varies.

\hrule
\vspace{1em}


\noindent
\textbf{Q306.} A car of mass 645 kg travels in a straight line along a horizontal road, as shown in the speed–time graph.
What is the average resultant force acting on the car over the time for which it is accelerating?



\textbf{A.} 380N \\
\textbf{B.} 420N \\
\textbf{C.} 550N \\
\textbf{D.} 1290N \\

\textbf{Answer:} D \\
\textbf{Explanation:} The terminal velocity is
[IMAGE:0]
; the acceleration in average is therefore
[IMAGE:1]
;
[IMAGE:2]
.

\hrule
\vspace{1em}


\noindent
\textbf{Q307.} Choose the correct statement below:
A bicycle’s rolling resistance is reduced with smoother tires.
At terminal velocity, the falling object has no forces at all.
Air resistance does not affect horizontal motion.



\textbf{A.} [IMAGE:0] \\
\textbf{B.} [IMAGE:1] \\
\textbf{C.} [IMAGE:2] \\
\textbf{D.} [IMAGE:3] \\

\textbf{Answer:} D \\
\textbf{Explanation:} Smoother tires reduce friction, and terminal velocity balances forces which only means the net force is zero.

\hrule
\vspace{1em}


\noindent
\textbf{Q308.} A cube has sides of length 5. What is the length of a line joining a vertex to the midpoint of a face diagonal on the opposite face (the dashed line in the diagram below)?



\textbf{A.} [IMAGE:0] \\
\textbf{B.} [IMAGE:1] \\
\textbf{C.} [IMAGE:2] \\
\textbf{D.} [IMAGE:3] \\

\textbf{Answer:} C \\
\textbf{Explanation:} [IMAGE:0]

\hrule
\vspace{1em}


\noindent
\textbf{Q309.} Choose the correct statement below:
A skydiver’s terminal velocity increases with body mass.
Air resistance acting on a skydiver is proportional to velocity squared.
A skydiver accelerates indefinitely until impact.



\textbf{A.} [IMAGE:0] \\
\textbf{B.} [IMAGE:1] \\
\textbf{C.} [IMAGE:2] \\
\textbf{D.} [IMAGE:3] \\

\textbf{Answer:} B \\
\textbf{Explanation:} Terminal velocity depends on mass, and drag often scales with (
[IMAGE:0]
).

\hrule
\vspace{1em}


\noindent
\textbf{Q310.} A car of mass 710 kg travels in a straight line along a horizontal road, as shown in the distance–time graph.
What is the average resultant force acting on the car over the time for which it is accelerating?



\textbf{A.} 994N \\
\textbf{B.} 1148N \\
\textbf{C.} 1150N \\
\textbf{D.} 1190N \\

\textbf{Answer:} A \\
\textbf{Explanation:} By the gradient of the graph in the linear region; the terminal velocity is
[IMAGE:0]
; the acceleration in average is therefore
[IMAGE:1]
;
[IMAGE:2]
.

\hrule
\vspace{1em}


\noindent
\textbf{Q311.} A cube has sides of length 5. What is the length of a line joining the midpoint of one edge to the midpoint of an opposite edge (the dashed line in the diagram below)?



\textbf{A.} [IMAGE:0] \\
\textbf{B.} [IMAGE:1] \\
\textbf{C.} [IMAGE:2] \\
\textbf{D.} [IMAGE:3] \\

\textbf{Answer:} B \\
\textbf{Explanation:} [IMAGE:0]

\hrule
\vspace{1em}


\noindent
\textbf{Q312.} Choose the correct statement below:
The falling speed of a feather is equal to a stone.
In a vacuum, all objects fall at the same rate regardless of mass.
Air resistance has no effect on lightweight objects.



\textbf{A.} [IMAGE:0] \\
\textbf{B.} [IMAGE:1] \\
\textbf{C.} [IMAGE:2] \\
\textbf{D.} [IMAGE:3] \\

\textbf{Answer:} E \\
\textbf{Explanation:} A feather falls slower than a stone due to greater air resistance. Air resistance varies with shape and mass, but vacuum eliminates it.

\hrule
\vspace{1em}


\noindent
\textbf{Q313.} A car of mass 520 kg travels in a straight line along a horizontal road.
The car accelerates non-uniformly from rest for 5.0 seconds and then moves at constant speed, as shown in the distance–time graph.
What is the average resultant force acting on the car over the time for which it is accelerating?



\textbf{A.} 380N \\
\textbf{B.} 420N \\
\textbf{C.} 520N \\
\textbf{D.} 1200N \\

\textbf{Answer:} C \\
\textbf{Explanation:} By the gradient of the graph in the linear region; the terminal velocity is
[IMAGE:0]
; the acceleration in average is therefore
[IMAGE:1]
;
[IMAGE:2]
.

\hrule
\vspace{1em}


\noindent
\textbf{Q314.} Choose the correct statement below:
A parachute opens to increase air resistance, slowing descent.
Terminal velocity is reached when gravitational force equals air resistance.
A falling object’s acceleration varies.



\textbf{A.} [IMAGE:0] \\
\textbf{B.} [IMAGE:1] \\
\textbf{C.} [IMAGE:2] \\
\textbf{D.} [IMAGE:3] \\

\textbf{Answer:} A \\
\textbf{Explanation:} The above statements are all right.

\hrule
\vspace{1em}


\noindent
\textbf{Q315.} A rectangular prism has dimensions 2, 3, and 4. What is the length of a line joining a vertex to the center of the opposite face (the dashed line in the diagram below)?



\textbf{A.} [IMAGE:0] \\
\textbf{B.} [IMAGE:1] \\
\textbf{C.} [IMAGE:2] \\
\textbf{D.} [IMAGE:3] \\

\textbf{Answer:} C \\
\textbf{Explanation:} [IMAGE:0]

\hrule
\vspace{1em}


\noindent
\textbf{Q316.} Choose the correct statement below:
A larger surface area decreases the frictional force on an object sliding on a surface.
A car moving at constant speed on a flat road experiences zero net force.
The acceleration of the car is always positive.



\textbf{A.} [IMAGE:0] \\
\textbf{B.} [IMAGE:1] \\
\textbf{C.} [IMAGE:2] \\
\textbf{D.} [IMAGE:3] \\

\textbf{Answer:} E \\
\textbf{Explanation:} Friction depends on surface area, and a larger surface area increases the frictional force on an object sliding on a surface. . Constant speed implies balanced forces.

\hrule
\vspace{1em}


\noindent
\textbf{Q317.} A cube has sides of length 4. What is the length of a line joining a vertex to the midpoint of one of the opposite edges (the dashed line in the diagram below)?



\textbf{A.} [IMAGE:0] \\
\textbf{B.} [IMAGE:1] \\
\textbf{C.} [IMAGE:2] \\
\textbf{D.} 6 \\

\textbf{Answer:} D \\
\textbf{Explanation:} [IMAGE:0]

\hrule
\vspace{1em}


\noindent
\textbf{Q318.} A car of mass 810 kg travels in a straight line along a horizontal road.
The car accelerates non-uniformly from rest for 5.0 seconds and then moves at constant speed, as shown in the distance–time graph.
What is the average resultant force acting on the car over the time for which it is accelerating?



\textbf{A.} 380N \\
\textbf{B.} 420N \\
\textbf{C.} 426N \\
\textbf{D.} 486N \\

\textbf{Answer:} D \\
\textbf{Explanation:} By the gradient of the graph in the linear region; the terminal velocity is
[IMAGE:0]
; the acceleration in average is therefore
[IMAGE:1]
;
[IMAGE:2]
.

\hrule
\vspace{1em}


\noindent
\textbf{Q319.} Choose the correct statement below:
1. The air resistance is smaller on an object with larger cross-sectional area.
2. After a droplet falling for a long time, the air resistance can exceed the gravity.
3. For a droplet falling, consider air resistance, it’s acceleration is always 0.5g;



\textbf{A.} [IMAGE:0] \\
\textbf{B.} [IMAGE:1] \\
\textbf{C.} [IMAGE:2] \\
\textbf{D.} [IMAGE:3] \\

\textbf{Answer:} C \\
\textbf{Explanation:} The air resistance is larger on an object with larger cross-sectional area.
After a droplet falling for a long time, the gravity equals air resistance.
The acceleration is changing, instead of a constant 0.5g.

\hrule
\vspace{1em}


\noindent
\textbf{Q320.} Choose the correct statement below concerning the terminal velocity of a falling object:
Terminal velocity is the maximum speed that a falling object can reach when the force of air resistance equals the force of gravity.
The terminal velocity of an object depends on its mass, shape, and the density of the fluid through which it is falling.
A parachutist reaches a much lower terminal velocity compared to a stone of the same mass because the parachutist has a larger surface area and experiences more air resistance.



\textbf{A.} [IMAGE:0] \\
\textbf{B.} [IMAGE:1] \\
\textbf{C.} [IMAGE:2] \\
\textbf{D.} [IMAGE:3] \\

\textbf{Answer:} A \\
\textbf{Explanation:} All statements are correct.

\hrule
\vspace{1em}


\noindent
\textbf{Q321.} A cube has sides of length of 2
. What is the length of a line joining a vertex to the center of cube (the dashed line in the diagram below)?



\textbf{A.} [IMAGE:0] \\
\textbf{B.} [IMAGE:1] \\
\textbf{C.} [IMAGE:2] \\
\textbf{D.} [IMAGE:3] \\

\textbf{Answer:} D \\
\textbf{Explanation:} [IMAGE:0]

\hrule
\vspace{1em}


\noindent
\textbf{Q322.} A car of mass 500 kg travels in a straight line along a horizontal road.
The car accelerates non-uniformly from rest for 5.0 seconds and then moves at constant speed, as shown in the distance–time graph.
What is the average resultant force acting on the car over the time for which it is accelerating?



\textbf{A.} 320N \\
\textbf{B.} 480N \\
\textbf{C.} 1000N \\
\textbf{D.} 1200N \\

\textbf{Answer:} C \\
\textbf{Explanation:} By the gradient of the graph in the linear region; the terminal velocity is
[IMAGE:0]
; the acceleration in average is therefore
[IMAGE:1]
;
[IMAGE:2]
.

\hrule
\vspace{1em}


\noindent
\textbf{Q323.} Choose the correct statement below about the motion of objects influenced by air resistance:
In the absence of air resistance, all objects fall with the same acceleration due to gravity, which is approximately
[IMAGE:0]
on Earth. (assumed that the g value same in different places on Earth)
Air resistance depends on the shape, size, and speed of the object, and it acts in the same direction to the motion.
When an object reaches terminal velocity, the net force acting on it is zero because the force of air resistance equals the force of gravity.



\textbf{A.} [IMAGE:0] \\
\textbf{B.} [IMAGE:1] \\
\textbf{C.} [IMAGE:2] \\
\textbf{D.} [IMAGE:3] \\

\textbf{Answer:} D \\
\textbf{Explanation:} Air resistance depends on the shape, size, and speed of the object, and it acts in the opposite direction to the motion.

\hrule
\vspace{1em}


\noindent
\textbf{Q324.} Choose the correct statement below regarding the effects of air resistance on a falling object:
Air resistance increases as the speed of the falling object increases, eventually balancing the force of gravity to reach a constant speed known as terminal velocity.
The acceleration of a falling object decreases over time due to air resistance until it becomes zero at terminal velocity.
Objects with larger surface areas experience a greater air resistance force compared to objects with smaller surface areas, assuming the same shape and speed.



\textbf{A.} [IMAGE:0] \\
\textbf{B.} [IMAGE:1] \\
\textbf{C.} [IMAGE:2] \\
\textbf{D.} [IMAGE:3] \\

\textbf{Answer:} A \\
\textbf{Explanation:} All statements are correct.

\hrule
\vspace{1em}


\noindent
\textbf{Q325.} Which one of the following about nuclear stability is true?



\textbf{A.} [IMAGE:0] \\
\textbf{B.} [IMAGE:1] \\
\textbf{C.} [IMAGE:2] \\
\textbf{D.} [IMAGE:3] \\

\textbf{Answer:} C \\
\textbf{Explanation:} Not all light atomic nuclei are unstable. For example, helium - 4 and other light atomic nuclei are very stable.

\hrule
\vspace{1em}


\noindent
\textbf{Q326.} During the soldering process, precise control of the soldering iron tip temperature is crucial. To maintain a constant tip temperature, a soldering iron is equipped with a temperature sensor and feedback control system. Assuming the tip (mass 2.0 g, copper) needs to be kept at 250°C while the ambient temperature is 20°C, and the heat exchange rate between the tip and the environment is 0.3 W/°C (i.e., the tip loses 0.5 W of heat for every 1°C above the ambient temperature)
Calculate the thermal power required from the soldering iron to maintain the tip at 250°C.



\textbf{A.} 10W \\
\textbf{B.} 11.5W \\
\textbf{C.} 23W \\
\textbf{D.} 46W \\

\textbf{Answer:} E \\
\textbf{Explanation:} Temperature difference:
[IMAGE:0]
Heat loss power:
[IMAGE:1]
Thermal power required is equal to the heat loss power, which is
[IMAGE:2]
.

\hrule
\vspace{1em}


\noindent
\textbf{Q327.} Choose the correct statement below:
A satellite in orbit experiences microgravity due to freefall.
Orbital velocity depends on the mass of the Earth and the satellite’s altitude.
A satellite’s speed remains constant in a circular orbit.



\textbf{A.} [IMAGE:0] \\
\textbf{B.} [IMAGE:1] \\
\textbf{C.} [IMAGE:2] \\
\textbf{D.} [IMAGE:3] \\

\textbf{Answer:} A \\
\textbf{Explanation:} All describe satellite orbital mechanics.

\hrule
\vspace{1em}


\noindent
\textbf{Q328.} A soldering iron needs to adjust the tip temperature for welding different materials. To quickly reach the required welding temperature, a designer develops a new tip material whose specific heat capacity varies with temperature (relationship:
[IMAGE:0]
, where t is the temperature in °C). When the soldering iron heats this new tip (mass 1.0 g) with a thermal power of 30 W, the tip's temperature rises from 20°C to 200°C in 30 s.
Calculate the heat transferred to the surrounding environment during this process.



\textbf{A.} 111.6J \\
\textbf{B.} 648.4J \\
\textbf{C.} 788.4J \\
\textbf{D.} 748.8J \\

\textbf{Answer:} D \\
\textbf{Explanation:} Heat provided by the soldering iron:
[IMAGE:0]
Heat absorbed by the tip (using average specific heat capacity because of the linearity of
[IMAGE:1]
equation):
Average specific heat capacity:
[IMAGE:2]
Heat absorbed:
[IMAGE:3]
Heat transferred to the environment:
[IMAGE:4]
(原题改题最后数字有点问题不过答案没问题)

\hrule
\vspace{1em}


\noindent
\textbf{Q329.} Choose the correct statement below:
A rocket accelerates upward by expelling gas downward (Newton’s third law).
The thrust of a rocket depends only on the mass.
A rocket’s acceleration increases as fuel mass decreases.



\textbf{A.} [IMAGE:0] \\
\textbf{B.} [IMAGE:1] \\
\textbf{C.} [IMAGE:2] \\
\textbf{D.} [IMAGE:3] \\

\textbf{Answer:} D \\
\textbf{Explanation:} The thrust of a rocket depends on the mass and velocity of expelled gas. Thrust depends on gas properties, but acceleration trends are complex.

\hrule
\vspace{1em}


\noindent
\textbf{Q330.} Choose the correct statement below:
A pendulum swings with decreasing amplitude due to air resistance.
Energy in a freely-motion pendulum is conserved in the absence of friction.
The period of a pendulum depends only on its length and gravity.



\textbf{A.} [IMAGE:0] \\
\textbf{B.} [IMAGE:1] \\
\textbf{C.} [IMAGE:2] \\
\textbf{D.} [IMAGE:3] \\

\textbf{Answer:} A \\
\textbf{Explanation:} All statements align with pendulum physics.

\hrule
\vspace{1em}


\noindent
\textbf{Q331.} During continuous operation, the copper tip of a soldering iron (mass 1.8 g, specific heat capacity
[IMAGE:0]
) experiences process of repeated heating and cooling. In a specific soldering task, the tip is first heated to 300°C from the room temperature(20°C), then rapidly cooled to 50°C, then heated again to 250°C, and finally cooled to room temperature (20°C). Assuming each heating and cooling process is linear and takes 10 s. The thermal power is 25 W when it comes to rising up the temperature of the copper tip.
Calculate the total heat transferred to the surrounding environment during the entire process.



\textbf{A.} 320J \\
\textbf{B.} 480J \\
\textbf{C.} 840J \\
\textbf{D.} 1800J \\

\textbf{Answer:} E \\
\textbf{Explanation:} Heat absorbed during first heating:
[IMAGE:0]
Heat lost during first cooling:
[IMAGE:1]
Heat absorbed during second heating:
[IMAGE:2]
Heat lost during second cooling:
[IMAGE:3]
Total heat transferred to the environment:
[IMAGE:4]
.
PS: If write down the
[IMAGE:5]
equation at the very beginning, the calculation process will become easier.

\hrule
\vspace{1em}


\noindent
\textbf{Q332.} Choose the correct statement below:
A car accelerating on a smooth road experiences constant acceleration.
If air resistance is ignored, a falling object’s displacement increases linearly with time.
In vacuum, all objects fall with the same acceleration regardless of mass.



\textbf{A.} [IMAGE:0] \\
\textbf{B.} [IMAGE:1] \\
\textbf{C.} [IMAGE:2] \\
\textbf{D.} [IMAGE:3] \\

\textbf{Answer:} C \\
\textbf{Explanation:} If air resistance is ignored, a falling object’s velocity increases linearly with time. If air resistance is ignored, a falling object’s displacement increases and is quadratic to time.

\hrule
\vspace{1em}


\noindent
\textbf{Q333.} Choose the correct statement below:
A ball rolling on a rough surface eventually stops due to friction.
Friction always opposes the motion of an object.
The speed of the ball moving on a rough surface is constant until it stops.



\textbf{A.} [IMAGE:0] \\
\textbf{B.} [IMAGE:1] \\
\textbf{C.} [IMAGE:2] \\
\textbf{D.} [IMAGE:3] \\

\textbf{Answer:} B \\
\textbf{Explanation:} Friction causes deceleration, and its direction opposes motion.

\hrule
\vspace{1em}


\noindent
\textbf{Q334.} Which statement about nuclear fusion is correct?



\textbf{A.} Nuclear fusion occurs under normal room temperature and high pressure conditions. \\
\textbf{B.} The fuel for nuclear fusion is difficult to obtain and store. \\
\textbf{C.} The energy released in nuclear fusion is equal to that in nuclear fission. \\
\textbf{D.} Nuclear fusion is the process that powers the sun and other stara. \\

\textbf{Answer:} D \\
\textbf{Explanation:} The sun and stars generate energy through the fusion of hydrogen into helium.

\hrule
\vspace{1em}


\noindent
\textbf{Q335.} To improve soldering efficiency, a designer coats the surface of a copper tip (mass 5.0 g, specific heat capacity
[IMAGE:0]
) with a special material that significantly enhances the tip's thermal conductivity but also increases its heat capacity by 20%. When the soldering iron heats the tip with a thermal power of 70 W, the tip's temperature rises by 220°C in 8 s.
Calculate the heat lost to the environment. after coating.



\textbf{A.} 6J \\
\textbf{B.} 12J \\
\textbf{C.} 20J \\
\textbf{D.} 26J \\

\textbf{Answer:} E \\
\textbf{Explanation:} Heat provided by the soldering iron:
[IMAGE:0]
Heat absorbed by the tip:
[IMAGE:1]
Heat lost to the environment:
[IMAGE:2]

\hrule
\vspace{1em}


\noindent
\textbf{Q336.} Choose the correct statement below:
A skydiver reaches terminal velocity due to air resistance, preventing infinite acceleration.
Terminal velocity for a skydiver is reached when air drag balances gravitational force.
A skydiver’s acceleration remains constant at ( g ) throughout the fall.



\textbf{A.} [IMAGE:0] \\
\textbf{B.} [IMAGE:1] \\
\textbf{C.} [IMAGE:2] \\
\textbf{D.} [IMAGE:3] \\

\textbf{Answer:} B \\
\textbf{Explanation:} The acceleration decreases until terminal velocity is reached.

\hrule
\vspace{1em}


\noindent
\textbf{Q337.} Choose the correct statement below:
1. The terminal velocity will always accelerate.
2. For a droplet falling, considering air resistance, its terminal velocity is achieved when air drag equals gravity;
3. For a droplet falling, consider air resistance, its acceleration is changing and does not exceed g;



\textbf{A.} [IMAGE:0] \\
\textbf{B.} [IMAGE:1] \\
\textbf{C.} [IMAGE:2] \\
\textbf{D.} [IMAGE:3] \\

\textbf{Answer:} E \\
\textbf{Explanation:} The terminal velocity will finally be constant.

\hrule
\vspace{1em}


\noindent
\textbf{Q338.} Which of the following about nuclear fission is true?



\textbf{A.} [IMAGE:0] \\
\textbf{B.} [IMAGE:1] \\
\textbf{C.} [IMAGE:2] \\
\textbf{D.} [IMAGE:3] \\

\textbf{Answer:} C \\
\textbf{Explanation:} Nuclear fission generate lots of radioactive waste, which is an issue to be tackle with.

\hrule
\vspace{1em}


\noindent
\textbf{Q339.} During the soldering process, the copper tip of a soldering iron (mass
[IMAGE:0]
, specific heat capacity
[IMAGE:1]
) accumulates heat due to prolonged use. When the soldering iron stops heating, the tip begins to cool and its temperature drops by
[IMAGE:2]
in
[IMAGE:3]
. Assuming all the heat lost by the tip is absorbed by the surrounding environment and the cooling rate is constant
Calculate the heat lost by the tip per second during cooling.



\textbf{A.} 6J/s \\
\textbf{B.} 12J/s \\
\textbf{C.} 60J/s \\
\textbf{D.} 120J/s \\

\textbf{Answer:} B \\
\textbf{Explanation:} Total heat lost by the tip:
[IMAGE:0]
Heat lost per second:
[IMAGE:1]

\hrule
\vspace{1em}


\noindent
\textbf{Q340.} Which statement about nuclear forces is correct?



\textbf{A.} The electromagnetic force is the primary force holding the nucleus together. \\
\textbf{B.} The strong nuclear force acte exclusively between protons. \\
\textbf{C.} The strong electromagnetic force has a much greater range than the nuclear force. \\
\textbf{D.} The strong electromagnetic force is responsible for beta decay. \\

\textbf{Answer:} C \\
\textbf{Explanation:} The strong electromagnetic force has a much greater range than the nuclear force.

\hrule
\vspace{1em}


\noindent
\textbf{Q341.} A soldering iron is equipped with an aluminum tip of mass 3.0 g (specific heat capacity of aluminum =
[IMAGE:0]
). When the soldering iron heats the tip with a thermal power of 40 W, the tip's temperature rises to a certain temperature in 30 s. However, due to a heat sink on the tip's surface, some heat is lost to the environment. If only 88% of the heat provided by the heating power is actually absorbed by the tip.
Calculate the heat lost to the environment.



\textbf{A.} 124J \\
\textbf{B.} 144J \\
\textbf{C.} 180J \\
\textbf{D.} 1200J \\

\textbf{Answer:} B \\
\textbf{Explanation:} Heat provided by the soldering iron:
[IMAGE:0]
Heat absorbed by the tip:
[IMAGE:1]
Heat lost to the environment:
[IMAGE:2]

\hrule
\vspace{1em}


\noindent
\textbf{Q342.} Which one of the following about nuclear isotopes is accurate?



\textbf{A.} [IMAGE:0] \\
\textbf{B.} [IMAGE:1] \\
\textbf{C.} [IMAGE:2] \\
\textbf{D.} [IMAGE:3] \\

\textbf{Answer:} D \\
\textbf{Explanation:} Radioisotopes are widely used in medicine. For example, they are used for diagnosis (such as the radioactive tracers used in PET - CT examinations) and treatment (such as using radioactive iodine to treat thyroid diseases).

\hrule
\vspace{1em}


\noindent
\textbf{Q343.} A soldering iron has a copper tip of mass 2.5g.
The tip is heated with 22W of thermal power. In 60s, the temperature of the tip increases by
[IMAGE:0]
.
How much energy is transferred from the tip to the surroundings in this time? (specific heat capacity of copper =
[IMAGE:1]
).



\textbf{A.} 320J \\
\textbf{B.} 480J \\
\textbf{C.} 1020J \\
\textbf{D.} 1200J \\

\textbf{Answer:} C \\
\textbf{Explanation:} By the conservation of energy; during this time; the energy to heat the tip minus the energy dissipated into the air equals the energy to raise the temperature of the tip: therefore
[IMAGE:0]

\hrule
\vspace{1em}


\noindent
\textbf{Q344.} Which statement about nuclear energy production is true?



\textbf{A.} [IMAGE:0] \\
\textbf{B.} [IMAGE:1] \\
\textbf{C.} [IMAGE:2] \\
\textbf{D.} [IMAGE:3] \\

\textbf{Answer:} C \\
\textbf{Explanation:} The waste from nuclear power plants is radioactive.

\hrule
\vspace{1em}


\noindent
\textbf{Q345.} A soldering iron has a copper tip of mass 1.0g.
The tip is heated with 20W of thermal power. In 43s, the temperature of the tip increases by
[IMAGE:0]
.
How much energy is transferred from the tip to the surroundings in this time? (specific heat capacity of copper =
[IMAGE:1]
).



\textbf{A.} 320J \\
\textbf{B.} 480J \\
\textbf{C.} 640J \\
\textbf{D.} 700J \\

\textbf{Answer:} D \\
\textbf{Explanation:} By the conservation of energy; during this time; the energy to heat the tip minus the energy dissipated into the air equals the energy to raise the temperature of the tip: therefore
[IMAGE:0]

\hrule
\vspace{1em}


\noindent
\textbf{Q346.} In the following circuit, the voltmeter records a voltage value of 0V.
What is the possible resistance of resistor R?



\textbf{A.} [IMAGE:0] \\
\textbf{B.} [IMAGE:1] \\
\textbf{C.} [IMAGE:2] \\
\textbf{D.} [IMAGE:3] \\

\textbf{Answer:} C \\
\textbf{Explanation:} [IMAGE:0]
[IMAGE:1]
[IMAGE:2]
[IMAGE:3]

\hrule
\vspace{1em}


\noindent
\textbf{Q347.} Which of the following about radioactive decay is correct?



\textbf{A.} [IMAGE:0] \\
\textbf{B.} [IMAGE:1] \\
\textbf{C.} [IMAGE:2] \\
\textbf{D.} [IMAGE:3] \\

\textbf{Answer:} E \\
\textbf{Explanation:} The activity of a radioactive sample is proportional to the number of nuclei present.

\hrule
\vspace{1em}


\noindent
\textbf{Q348.} The circuit shown in Figure is called a Wheatstone bridge, which can be used to measure resistance more accurately.
[IMAGE:0]
is the resistance to be measured,
[IMAGE:1]
and
[IMAGE:2]
are two fixed resistors with known values,
[IMAGE:3]
is a resistance box whose resistance can be adjusted, and
[IMAGE:4]
is a sensitive galvanometer. The specific operation steps are as follows:
① Connect the circuit according to the circuit diagram;
② Close the switch
[IMAGE:5]
, adjust
[IMAGE:6]
until the reading of the sensitive galvanometer is zero, and record the reading of
[IMAGE:7]
as
[IMAGE:8]
,
[IMAGE:9]
is the number of recording times for different structure to the original circuit;
③ Substitute
[IMAGE:10]
into the formula:
[IMAGE:11]
. Calculate the result of
[IMAGE:12]
.
[IMAGE:13]
and
[IMAGE:14]
are unknown.
What is the formula
[IMAGE:15]
?



\textbf{A.} [IMAGE:0] \\
\textbf{B.} [IMAGE:1] \\
\textbf{C.} [IMAGE:2] \\
\textbf{D.} [IMAGE:3] \\

\textbf{Answer:} D \\
\textbf{Explanation:} Swap the positions of
[IMAGE:0]
and
[IMAGE:1]
and follow the operational steps described in the problem statement to measure twice. Let the readings of
[IMAGE:2]
before and after the swap be
[IMAGE:3]
and
[IMAGE:4]
respectively. According to the above results, we have
[IMAGE:5]
and
[IMAGE:6]
. Rearranging these equations, we get
[IMAGE:7]
.

\hrule
\vspace{1em}


\noindent
\textbf{Q349.} A soldering iron has a copper tip of mass 2.0g
The tip is heated with 27.5W of thermal power. In 40s, the temperature of the tip increases by
[IMAGE:0]
.



\textbf{A.} 320J \\
\textbf{B.} 760J \\
\textbf{C.} 780J \\
\textbf{D.} 880J \\

\textbf{Answer:} C \\
\textbf{Explanation:} By the conservation of energy; during this time; the energy to heat the tip minus the energy dissipated into the air equals the energy to raise the temperature of the tip: therefore
[IMAGE:0]

\hrule
\vspace{1em}


\noindent
\textbf{Q350.} Which statement about nuclear structure is valid?



\textbf{A.} [IMAGE:0] \\
\textbf{B.} [IMAGE:1] \\
\textbf{C.} [IMAGE:2] \\
\textbf{D.} [IMAGE:3] \\

\textbf{Answer:} D \\
\textbf{Explanation:} The proton count in a nucleus influences a lot of things including its chemical properties and its stability.

\hrule
\vspace{1em}


\noindent
\textbf{Q351.} A soldering iron has a copper tip of mass 2.0g.
The tip is heated with 23W of thermal power. In 33s, the temperature of the tip increases by
[IMAGE:0]
.
How much energy is transferred from the tip to the surroundings in this time? (specific heat capacity of copper =
[IMAGE:1]
).



\textbf{A.} 399J \\
\textbf{B.} 599J \\
\textbf{C.} 699J \\
\textbf{D.} 1299J \\

\textbf{Answer:} B \\
\textbf{Explanation:} By the conservation of energy; during this time; the energy to heat the tip minus the energy dissipated into the air equals the energy to raise the temperature of the tip: therefore
[IMAGE:0]

\hrule
\vspace{1em}


\noindent
\textbf{Q352.} The circuit shown in Figure is called a Wheatstone bridge, which can be used to measure resistance more accurately.
[IMAGE:0]
is the resistance to be measured,
[IMAGE:1]
and
[IMAGE:2]
are two fixed resistors with known values,
[IMAGE:3]
is a resistance box whose resistance can be adjusted, and
[IMAGE:4]
is a sensitive galvanometer. The specific operation steps are as follows:
① Connect the circuit according to the circuit diagram;
② Close the switch
[IMAGE:5]
, adjust
[IMAGE:6]
until the reading of the sensitive galvanometer is zero, and record the reading of
[IMAGE:7]
;
③ Substitute
[IMAGE:8]
into the formula:
[IMAGE:9]
. Calculate the result of
[IMAGE:10]
.
[IMAGE:11]
and
[IMAGE:12]
are known.
What is the formula
[IMAGE:13]
?



\textbf{A.} [IMAGE:0] \\
\textbf{B.} [IMAGE:1] \\
\textbf{C.} [IMAGE:2] \\
\textbf{D.} [IMAGE:3] \\

\textbf{Answer:} E \\
\textbf{Explanation:} Let the branch currents flowing through
[IMAGE:0]
and
[IMAGE:1]
be
[IMAGE:2]
and
[IMAGE:3]
respectively. Since no current flows through the galvanometer, i.e.,
[IMAGE:4]
, the currents flowing through
[IMAGE:5]
and
[IMAGE:6]
are also
[IMAGE:7]
and
[IMAGE:8]
respectively. Therefore, we have
[IMAGE:9]
From these, we get
[IMAGE:10]
. Substituting this into equation (2) and rearranging, we obtain
[IMAGE:11]
, which simplifies to
[IMAGE:12]
.

\hrule
\vspace{1em}


\noindent
\textbf{Q353.} Which one of the following statements about nuclear reactions is accurate?



\textbf{A.} [IMAGE:0] \\
\textbf{B.} [IMAGE:1] \\
\textbf{C.} [IMAGE:2] \\
\textbf{D.} [IMAGE:3] \\

\textbf{Answer:} B \\
\textbf{Explanation:} The mass-energy equivalence principle explains the energy released in fission.

\hrule
\vspace{1em}


\noindent
\textbf{Q354.} Mike heats ice cubes and observes the physical changes of ice. During this process, he measures and draws a graph of temperature versus time, as shown in the figure below. Based on the figure, which of the following analysis is correct?
[IMAGE:0]



\textbf{A.} The AB segment in the figure represents the temperature rise of ice, while the BC segment represents the melting process of ice; hence, A is incorrect. \\
\textbf{B.} From the figure, it can be seen that the temperature of ice remains constant at 0\circ C, indicating that ice is a crystal; hence, B is incorrect. \\
\textbf{C.} Since the mass of ice and water is the same and they are heated by the same alcohol lamp, the slow rise in water temperature indicates that the specific heat capacity of water is larger than that of ice; hence, C is correct. \\
\textbf{D.} The temperature of boiling water remains constant, but it requires continuous heat absorption. If heating is stopped, boiling will also stop; hence, D is incorrect. \\

\textbf{Answer:} C \\
\textbf{Explanation:} 

\hrule
\vspace{1em}


\noindent
\textbf{Q355.} In the following circuit, the voltmeter records a voltage value of 2V.
What is the possible resistance of resistor R?



\textbf{A.} [IMAGE:0] \\
\textbf{B.} [IMAGE:1] \\
\textbf{C.} [IMAGE:2] \\
\textbf{D.} [IMAGE:3] \\

\textbf{Answer:} C \\
\textbf{Explanation:} Case 1:
[IMAGE:0]
[IMAGE:1]
no such a R. (infinity)
Case 2:
[IMAGE:2]
[IMAGE:3]
[IMAGE:4]
Thus, only 250ohm can be possible.

\hrule
\vspace{1em}


\noindent
\textbf{Q356.} Which one of the following statements about nuclear physics is incorrect?



\textbf{A.} [IMAGE:0] \\
\textbf{B.} [IMAGE:1] \\
\textbf{C.} [IMAGE:2] \\
\textbf{D.} [IMAGE:3] \\

\textbf{Answer:} E \\
\textbf{Explanation:} Alpha decay results in the loss of two protons and two neutrons from the nucleus.

\hrule
\vspace{1em}


\noindent
\textbf{Q357.} A glass of water is placed in a refrigerator; with a mass of 0.2 kg and initial temperature of 5 degrees; Find the equilibrium temperature: The latent heat of ice is 300kJ/kg; the specific heat capacity of water is 2.09 kJ/(kg\cdot degree); the specific heat capacity of water is 4200 kJ/(kg\cdot degree); Under standard atmospheric pressure , water can completely freeze and the freezing temperature (i.e., the ice point) of the water is -30 degrees Celsius.
Calculate the amount of heat that needs to be released for the water in this cup to turn into ice at 0 degrees Celsius.



\textbf{A.} 3631.42KJ \\
\textbf{B.} 4272.54KJ \\
\textbf{C.} 8468.36KJ \\
\textbf{D.} 8520.00KJ \\

\textbf{Answer:} B \\
\textbf{Explanation:} Assume that the equilibrium temperature is x; the heat released by the water equals the heat absorbs the ice:
[IMAGE:0]
[IMAGE:1]

\hrule
\vspace{1em}


\noindent
\textbf{Q358.} In the following circuit, the voltmeter records a voltage value of 1V.
What is the possible resistance of resistor R?



\textbf{A.} [IMAGE:0] \\
\textbf{B.} [IMAGE:1] \\
\textbf{C.} [IMAGE:2] \\
\textbf{D.} [IMAGE:3] \\

\textbf{Answer:} H \\
\textbf{Explanation:} Case 1:
[IMAGE:0]
[IMAGE:1]
[IMAGE:2]
Case 2:
[IMAGE:3]
[IMAGE:4]
[IMAGE:5]
Thus, 2500ohm or 500ohm can be possible.

\hrule
\vspace{1em}


\noindent
\textbf{Q359.} In the following circuit, the voltmeter records a voltage of zero.
What is the resistance of resistor R?



\textbf{A.} [IMAGE:0] \\
\textbf{B.} [IMAGE:1] \\
\textbf{C.} [IMAGE:2] \\
\textbf{D.} [IMAGE:3] \\

\textbf{Answer:} F \\
\textbf{Explanation:} [IMAGE:0]
[IMAGE:1]
[IMAGE:2]

\hrule
\vspace{1em}


\noindent
\textbf{Q360.} A glass of water is placed in a refrigerator; with a mass of 0.4 kg and initial temperature of 10 degrees; Find the equilibrium temperature: The latent heat of ice is 300kJ/kg; the specific heat capacity of ice is 2.09 kJ/(kg\cdot degree); the specific heat capacity of water is 4200 kJ/(kg\cdot degree); Under standard atmospheric pressure , water can completely freeze and the freezing temperature (i.e., the ice point) of the water is 0 degrees Celsius.
Calculate the amount of heat that needs to be released for the water in this cup to turn into ice at 0 degrees Celsius.



\textbf{A.} 11630KJ \\
\textbf{B.} 13600KJ \\
\textbf{C.} 14200KJ \\
\textbf{D.} 16920KJ \\

\textbf{Answer:} D \\
\textbf{Explanation:} Assume that the equilibrium temperature is x; the heat released by the water equals the heat absorbs the ice:
[IMAGE:0]
PS: some formulas are listed as follows
[IMAGE:1]

\hrule
\vspace{1em}


\noindent
\textbf{Q361.} In the following circuit, the ammeter records a current of zero.
What is the resistance of resistor R?



\textbf{A.} [IMAGE:0] \\
\textbf{B.} [IMAGE:1] \\
\textbf{C.} [IMAGE:2] \\
\textbf{D.} [IMAGE:3] \\

\textbf{Answer:} H \\
\textbf{Explanation:} When there is no current in the ammeter, those two terminals are equipotential.
[IMAGE:0]
[IMAGE:1]
[IMAGE:2]
[IMAGE:3]

\hrule
\vspace{1em}


\noindent
\textbf{Q362.} A piece of ice undergoes the following three processes:
1.
Ice at -30°C is heated to 0°C, absorbing heat Q1
;
2.
Ice at 0°C melts into water at 0°C, absorbing heat Q2
;
3.
Water at 30°C is heated to 60°C, absorbing heat Q3
.
It is known that the specific heat capacity of ice is less than that of water. The mass remains constant throughout the entire process. Which of the following is true?



\textbf{A.} [IMAGE:0] \\
\textbf{B.} [IMAGE:1] \\
\textbf{C.} [IMAGE:2] \\
\textbf{D.} [IMAGE:3] \\

\textbf{Answer:} C \\
\textbf{Explanation:} To compare the heat absorbed in each process, we need to use the formulas for heat absorption:
1.
Heat absorbed by ice from -30°C to 0°C:
[IMAGE:0]
2.
Heat absorbed by ice melting at 0°C:
[IMAGE:1]
where
[IMAGE:2]
is the latent heat of fusion for ice.
3.
Heat absorbed by water from 30°C to 60°C:
[IMAGE:3]
Given that the specific heat capacity of ice
[IMAGE:4]
is less than that of water
[IMAGE:5]
, we have:
[IMAGE:6]
Thus,
[IMAGE:7]
.
The latent heat of fusion
[IMAGE:8]
for ice is generally much larger than the specific heat capacities
[IMAGE:9]
multiplied by the temperature change. Therefore:
[IMAGE:10]
Combining these results, we get:
[IMAGE:11]

\hrule
\vspace{1em}


\noindent
\textbf{Q363.} A cyclist travels at a constant speed of
[IMAGE:0]
. A second cyclist starts from rest and accelerates at
[IMAGE:1]
. However, after 10 seconds, the second cyclist's bike breaks down and decelerate at
[IMAGE:2]
. Will the second cyclist ever catch up to the first cyclist, and if so, when?



\textbf{A.} [IMAGE:0] \\
\textbf{B.} [IMAGE:1] \\
\textbf{C.} [IMAGE:2] \\
\textbf{D.} [IMAGE:3] \\

\textbf{Answer:} C \\
\textbf{Explanation:} [IMAGE:0]

\hrule
\vspace{1em}


\noindent
\textbf{Q364.} In the following circuit, the ammeter records a current of zero.
What is the resistance of resistor R?



\textbf{A.} [IMAGE:0] \\
\textbf{B.} [IMAGE:1] \\
\textbf{C.} [IMAGE:2] \\
\textbf{D.} [IMAGE:3] \\

\textbf{Answer:} D \\
\textbf{Explanation:} When there is no current in the ammeter, those two terminals are equipotential.
[IMAGE:0]
[IMAGE:1]
[IMAGE:2]
[IMAGE:3]
[IMAGE:4]

\hrule
\vspace{1em}


\noindent
\textbf{Q365.} The ice is submerged into a glass of water; the 1.0kg ice is at -14 degrees; 0.2 kg water is at 15 degrees; Find the equilibrium temperature: The latent heat of ice is 300kJ/kg; the specific heat capacity of ice is 2.09 kJ/(kg\cdot degree); the specific heat capacity of water is 4200 kJ/(kg\cdot degree); The ice can completely melt when the temperature between 0 degrees and 5 degrees; the melting point of ice under standard atmospheric pressure is 0 degrees:



\textbf{A.} 5.23 \\
\textbf{B.} 3.3 \\
\textbf{C.} 2.4 \\
\textbf{D.} 0.87 \\

\textbf{Answer:} C \\
\textbf{Explanation:} Assume that the equilibrium temperature is x; the heat released by the water equals the heat absorbs the ice:
[IMAGE:0]
PS: some formulas are listed as follows
[IMAGE:1]

\hrule
\vspace{1em}


\noindent
\textbf{Q366.} In the following circuit, the ammeter records a current of zero.
What is the resistance of resistor R?



\textbf{A.} [IMAGE:0] \\
\textbf{B.} [IMAGE:1] \\
\textbf{C.} [IMAGE:2] \\
\textbf{D.} [IMAGE:3] \\

\textbf{Answer:} B \\
\textbf{Explanation:} When there is no current in the ammeter, those two terminals are equipotential.
[IMAGE:0]
[IMAGE:1]
[IMAGE:2]
[IMAGE:3]
[IMAGE:4]

\hrule
\vspace{1em}


\noindent
\textbf{Q367.} A cyclist travels at a constant speed of
[IMAGE:0]
. A second cyclist starts from rest and accelerates at
[IMAGE:1]
. However, after 20 seconds, the second cyclist's bike breaks down and decelerate at
[IMAGE:2]
. Will the second cyclist ever catch up to the first cyclist, and if so, when is the second time?



\textbf{A.} [IMAGE:0] \\
\textbf{B.} [IMAGE:1] \\
\textbf{C.} [IMAGE:2] \\
\textbf{D.} [IMAGE:3] \\

\textbf{Answer:} A \\
\textbf{Explanation:} [IMAGE:0]

\hrule
\vspace{1em}


\noindent
\textbf{Q368.} In the following circuit, the ammeter records a current of zero.
What is the resistance of resistor R?



\textbf{A.} [IMAGE:0] \\
\textbf{B.} [IMAGE:1] \\
\textbf{C.} [IMAGE:2] \\
\textbf{D.} [IMAGE:3] \\

\textbf{Answer:} C \\
\textbf{Explanation:} When there is no current in the ammeter, those two terminals are equipotential.
[IMAGE:0]
[IMAGE:1]
[IMAGE:2]
[IMAGE:3]
[IMAGE:4]

\hrule
\vspace{1em}


\noindent
\textbf{Q369.} The ice is submerged into a glass of water; the 1.0kg ice is at -60 degrees; 1 kg water is at 50 degrees; Find the equilibrium temperature: The latent heat of ice is 300kJ/kg; the specific heat capacity of ice is 2.09 kJ/(kg\cdot degree); the specific heat capacity of water is 4200 kJ/(kg\cdot degree); The ice can completely melt when the temperature between 0 degrees and 5 degrees; the melting point of ice under standard atmospheric pressure is 0 degrees:



\textbf{A.} 5.3 \\
\textbf{B.} 10.0 \\
\textbf{C.} 13.12 \\
\textbf{D.} 25.0 \\

\textbf{Answer:} D \\
\textbf{Explanation:} Assume that the equilibrium temperature is x; the heat released by the water equals the heat absorbs the ice:
[IMAGE:0]
PS: some formulas are listed as follows
[IMAGE:1]

\hrule
\vspace{1em}


\noindent
\textbf{Q370.} A planetarium has different admission prices for adults and children, with a family discount.
Admission for 4 adults and 1 child costs £29.
Admission for 2 adults and 3 children costs £27.
A family of 5 adults and 2 children gets a 15% discount. What is the admission cost after the discount?



\textbf{A.} [IMAGE:0] \\
\textbf{B.} [IMAGE:1] \\
\textbf{C.} [IMAGE:2] \\
\textbf{D.} [IMAGE:3] \\

\textbf{Answer:} A \\
\textbf{Explanation:} Let a be the adult admission price, c be the child admission price;
4a + c = 29; 2a + 3c = 27.
24+5=29; 12+15=27;
Solving gives a = 6, c = 5.
Original price for 5a + 2c = 30 + 10 = 40.
After 15% discount: 40 * 0.85 = 34.

\hrule
\vspace{1em}


\noindent
\textbf{Q371.} A boat is traveling upstream at
[IMAGE:0]
relative to the water. A second boat moving with the water starts from the same point and accelerates downstream at
[IMAGE:1]
relative to the water when the first boat passes. If the river current is
[IMAGE:2]
, how long will it take for the distance of the second boat to be bigger than the the distance of first boat?



\textbf{A.} 20s \\
\textbf{B.} 40s \\
\textbf{C.} 60s \\
\textbf{D.} 80s \\

\textbf{Answer:} B \\
\textbf{Explanation:} [IMAGE:0]

\hrule
\vspace{1em}


\noindent
\textbf{Q372.} The ice is submerged into a glass of water; the 1.0kg ice is at 0 degrees; 1 kg water is at 23 degrees; Find the equilibrium temperature: The latent heat of ice is 300kJ/kg; the specific heat capacity of water is 4200 kJ/Kg*degree; The ice can completely melt when the temperature between 0 degrees and 6 degrees; the melting point of ice under standard atmospheric pressure is 0 degrees:



\textbf{A.} 5.2 \\
\textbf{B.} 7.5 \\
\textbf{C.} 10.0 \\
\textbf{D.} 11.5 \\

\textbf{Answer:} D \\
\textbf{Explanation:} Assume that the equilibrium temperature is x; the heat released by the water equals the heat absorbs the ice:
.
[IMAGE:0]
where 1.0*300
corresponds to the process of ice at 0 degrees turning into water at 0 degrees, 1.0*4200*x
corresponds to the process of the water formed from the ice absorbing heat, and 4200*1*(23-x)
corresponds to the process of the original water in the glass releasing heat.

\hrule
\vspace{1em}


\noindent
\textbf{Q373.} An amusement park charges different prices for adult and child rides, with a discount for multiple rides.
A ride for 2 adults and 5 children costs £43.
A ride for 3 adults and 3 children costs £42.
If a group buys 4 adult rides and 4 child rides, they get a 10% discount on the total price. What is the cost after the discount?



\textbf{A.} [IMAGE:0] \\
\textbf{B.} [IMAGE:1] \\
\textbf{C.} [IMAGE:2] \\
\textbf{D.} [IMAGE:3] \\

\textbf{Answer:} C \\
\textbf{Explanation:} Let a be the adult ride price, c be the child ride price;
2a + 5c = 43; 3a + 3c = 42.
18+25=43; 27+15=42.
Solving gives a = 9, c = 5.
Original price for 4a + 4c = 36 + 20 = 56.
After 10% discount: 56 * 0.9 = 50.4.

\hrule
\vspace{1em}


\noindent
\textbf{Q374.} Two runners start from the same point. Runner A runs at a constant speed of
[IMAGE:0]
. Runner B starts from rest and accelerates at
[IMAGE:1]
for the first 8
seconds, then runs at a constant speed. How long does it take for Runner B to catch up to Runner A for the first time, and how far have they traveled?



\textbf{A.} 18s,90m \\
\textbf{B.} 18s,104m \\
\textbf{C.} 14s,75m \\
\textbf{D.} 10s,60m \\

\textbf{Answer:} B \\
\textbf{Explanation:} [IMAGE:0]

\hrule
\vspace{1em}


\noindent
\textbf{Q375.} A science center has different entry fees for adults, students, and seniors.
Entry for 1 adult, 3 students, and 2 seniors costs £29.
Entry for 4 adults, 2 students, and 1 senior costs £41.
Entry for 2 adults, 5 students, and 1 senior costs £39.
What is the entry fee for 1 adult, 1 student, and 1 senior?



\textbf{A.} [IMAGE:0] \\
\textbf{B.} [IMAGE:1] \\
\textbf{C.} [IMAGE:2] \\
\textbf{D.} [IMAGE:3] \\

\textbf{Answer:} F \\
\textbf{Explanation:} Let a be the adult entry fee, s be the student entry fee, sr be the senior entry fee;
a + 3s + 2sr = 29; 4a + 2s + sr = 41; 2a + 5s + sr = 39.
7+12+10=29; 28+8+5=41; 14+20+5=39.
Solving gives a = 7, s = 4, sr = 5.
a + s + sr = 7 + 4 + 5 = 16.

\hrule
\vspace{1em}


\noindent
\textbf{Q376.} A bowling alley charges different prices for adult, child, and student games.
A game for 3 adults, 2 children, and 1 student costs £30.
A game for 1 adult, 4 children, and 2 students costs £30.
A game for 2 adults, 3 children, and 1 student costs £28.
What is the cost for 1 adult, 1 child, and 1 student game?



\textbf{A.} [IMAGE:0] \\
\textbf{B.} [IMAGE:1] \\
\textbf{C.} [IMAGE:2] \\
\textbf{D.} [IMAGE:3] \\

\textbf{Answer:} G \\
\textbf{Explanation:} Let a be the adult game price, c be the child game price, s be the student game price;
3a + 2c + s = 30; a + 4c + 2s = 30; 2a + 3c + s = 28.
18+8+4=30; 6+16+8=30; 12+12+4=28;
Solving gives a = 6, c = 4, s = 4.
a + c + s = 6 + 4 + 4 = 14.

\hrule
\vspace{1em}


\noindent
\textbf{Q377.} The ice is submerged into a glass of water; the 2.0kg ice is at 0 degrees; 1 kg water is at 50 degrees; Find the equilibrium temperature: The latent heat of ice is 300kJ/kg; the specific heat capacity of water is 4200 kJ/Kg*degree; The ice can completely melt when the temperature between 0 degrees and 25 degrees; the melting point of ice under standard atmospheric pressure is 0 degrees:



\textbf{A.} 5.6 \\
\textbf{B.} 7.4 \\
\textbf{C.} 8.3 \\
\textbf{D.} 16.6 \\

\textbf{Answer:} D \\
\textbf{Explanation:} [IMAGE:0]
where 2.0*300
corresponds to the process of ice at 0 degrees turning into water at 0 degrees, 2.0*4200*x
corresponds to the process of the water formed from the ice absorbing heat, and 4200*1*(50-x)
corresponds to the process of the original water in the glass releasing heat.

\hrule
\vspace{1em}


\noindent
\textbf{Q378.} A car is traveling at a constant speed of
[IMAGE:0]
on a straight road. A motorcycle starts from rest and accelerates at
[IMAGE:1]
. The motorcycle continues to accelerate until it reaches a maximum speed of
[IMAGE:2]
, after which it decelerate at
[IMAGE:3]
until it reaches the speed
[IMAGE:4]
. How many times will the motorcycle meet the car, and at what times?



\textbf{A.} [IMAGE:0] \\
\textbf{B.} [IMAGE:1] \\
\textbf{C.} [IMAGE:2] \\
\textbf{D.} [IMAGE:3] \\

\textbf{Answer:} D \\
\textbf{Explanation:} [IMAGE:0]

\hrule
\vspace{1em}


\noindent
\textbf{Q379.} A skating rink has different admission prices for adults, children, and seniors.
Admission for 2 adults, 4 children, and 1 senior costs £35.
Admission for 5 adults, 1 child, and 2 seniors costs £49.
Admission for 3 adults, 3 children, and 1 senior costs £38.
What is the admission cost for 1 adult, 1 child, and 1 senior?



\textbf{A.} [IMAGE:0] \\
\textbf{B.} [IMAGE:1] \\
\textbf{C.} [IMAGE:2] \\
\textbf{D.} [IMAGE:3] \\

\textbf{Answer:} C \\
\textbf{Explanation:} Let a be the adult admission price, c be the child admission price, s be the senior admission price;
2a + 4c + s = 35; 5a + c + 2s = 49; 3a + 3c + s = 38.
14+16+5=35; 35+4+10=49; 21+12+5=38.
Solving gives a = 7, c = 4, s = 5.
a + c + s = 7 + 4 + 5 = 16.

\hrule
\vspace{1em}


\noindent
\textbf{Q380.} The ice is submerged into a glass of water; the 0.5kg ice is at 0 degrees; 0.5 kg water is at 50 degrees; Find the equilibrium temperature: The latent heat of ice is 300kJ/kg; the specific heat capacity of water is 4200 kJ/Kg*degree; The ice can completely melt when the temperature between 0 degrees and 50 degrees; the melting point of ice under standard atmospheric pressure is 0 degrees:



\textbf{A.} 5.6 \\
\textbf{B.} 7.4 \\
\textbf{C.} 13.6 \\
\textbf{D.} 20.5 \\

\textbf{Answer:} E \\
\textbf{Explanation:} Assume that the equilibrium temperature is x; the heat released by the water equals the heat absorbs the ice:
[IMAGE:0]
where 0.5*300
corresponds to the process of ice at 0 degrees turning into water at 0 degrees, 0.5*4200*x
corresponds to the process of the water formed from the ice absorbing heat, and 4200*0.5*(50-x)
corresponds to the process of the original water in the glass releasing heat.

\hrule
\vspace{1em}


\noindent
\textbf{Q381.} A water park charges different prices for adult and child tickets.
A ticket for 1 adult and 5 children costs £33.
A ticket for 3 adults and 2 children costs £34.
What is the cost for 2 adults and 4 children?



\textbf{A.} [IMAGE:0] \\
\textbf{B.} [IMAGE:1] \\
\textbf{C.} [IMAGE:2] \\
\textbf{D.} [IMAGE:3] \\

\textbf{Answer:} E \\
\textbf{Explanation:} Let a be the adult ticket price, c be the child ticket price;
a + 5c = 33; 3a + 2c = 34.
8+25=33; 24+10=34.
Solving gives a = 8, c = 5.
2a + 4c = 16 + 20 = 36.

\hrule
\vspace{1em}


\noindent
\textbf{Q382.} A boat is traveling upstream at
[IMAGE:0]
relative to the water. A second boat starts from the same point and accelerates downstream at
[IMAGE:1]
relative to the water when the first boat passes. If the river current is
[IMAGE:2]
, how long will it take for the speed of the second boat to be bigger than the the distance of first boat?



\textbf{A.} 20s \\
\textbf{B.} 40s \\
\textbf{C.} 60s \\
\textbf{D.} 80s \\

\textbf{Answer:} A \\
\textbf{Explanation:} [IMAGE:0]

\hrule
\vspace{1em}


\noindent
\textbf{Q383.} A swimming pool has different entry fees for adults and children.
Entry for 4 adults and 1 child costs £29.
Entry for 2 adults and 3 children costs £27.
What is the entry fee for 6 adults and 2 children?



\textbf{A.} [IMAGE:0] \\
\textbf{B.} [IMAGE:1] \\
\textbf{C.} [IMAGE:2] \\
\textbf{D.} [IMAGE:3] \\

\textbf{Answer:} E \\
\textbf{Explanation:} Let a be the adult entry fee, c be the child entry fee;
4a + c = 29; 2a + 3c = 27.
24+5=29; 12+15=27.
Solving gives a = 6, c = 5.
6a + 2c = 36 + 10 = 46.

\hrule
\vspace{1em}


\noindent
\textbf{Q384.} The ice is submerged into a glass of water; the 0.5kg ice is at 0 degrees; 0.25 kg water is at 16 degrees; Find the equilibrium temperature: The latent heat of ice is 300kJ/kg; the specific heat capacity of water is 4200 kJ/Kg*degree; The ice can completely melt when the temperature between 0 degrees and 16 degrees; the melting point of ice under standard atmospheric pressure is 0 degrees:



\textbf{A.} 4.99 \\
\textbf{B.} 5.29 \\
\textbf{C.} 10 \\
\textbf{D.} 10.6 \\

\textbf{Answer:} B \\
\textbf{Explanation:} Assume that the equilibrium temperature is x; the heat released by the water equals the heat absorbs the ice:
[IMAGE:0]
where 0.5*300
corresponds to the process of ice at 0 degrees turning into water at 0 degrees, 0.5*4200*x
corresponds to the process of the water formed from the ice absorbing heat, and 4200*0.25*(16-x)
corresponds to the process of the original water in the glass releasing heat.

\hrule
\vspace{1em}


\noindent
\textbf{Q385.} A car is traveling at a constant speed of
[IMAGE:0]
on a straight road. It overtakes a truck moving at
[IMAGE:1]
in the same direction. At the moment the car overtakes the truck, the truck starts to accelerate at
[IMAGE:2]
. How long will it take for the truck to overtake the car again?



\textbf{A.} 10s \\
\textbf{B.} 16.67s \\
\textbf{C.} 25s \\
\textbf{D.} 33.33s \\

\textbf{Answer:} C \\
\textbf{Explanation:} [IMAGE:0]

\hrule
\vspace{1em}


\noindent
\textbf{Q386.} A zoo charges different prices for adult and child admission.
Admission for 3 adults and 2 children costs £38.
Admission for 1 adult and 4 children costs £26.
What is the admission cost for 5 adults and 3 children?



\textbf{A.} [IMAGE:0] \\
\textbf{B.} [IMAGE:1] \\
\textbf{C.} [IMAGE:2] \\
\textbf{D.} [IMAGE:3] \\

\textbf{Answer:} F \\
\textbf{Explanation:} Let a be the adult admission price, c be the child admission price;
3a + 2c = 38; a + 4c = 26.
30+8=38; 10+16=26.
Solving gives a = 10, c = 4.
5a + 3c = 50 + 12 = 62

\hrule
\vspace{1em}


\noindent
\textbf{Q387.} A museum has different entry fees for adults and students.
Entry for 2 adults and 5 students costs £29.
Entry for 4 adults and 3 students costs £37.
What is the entry fee for 3 adults and 4 students?



\textbf{A.} [IMAGE:0] \\
\textbf{B.} [IMAGE:1] \\
\textbf{C.} [IMAGE:2] \\
\textbf{D.} [IMAGE:3] \\

\textbf{Answer:} E \\
\textbf{Explanation:} Let a be the adult entry fee, s be the student entry fee;
2a + 5s = 29; 4a + 3s = 37.
14+15=29; 28+9=37.
Solving gives a = 7, s = 3.
3a + 4s = 21 + 12 = 33.

\hrule
\vspace{1em}


\noindent
\textbf{Q388.} A man is cycling along a straight horizontal road at a constant speed of
[IMAGE:0]
. He passes a boy who is cycling at
[IMAGE:1]
in the same direction. When the man is level with the boy, the boy begins to accelerate at a constant rate of
[IMAGE:2]
. The boy maintains this constant acceleration and the man continues at constant speed until the boy passes the man. What is the time interval between the two instances when the man and the boy are level?



\textbf{A.} 5s \\
\textbf{B.} 10s \\
\textbf{C.} 22.5s \\
\textbf{D.} 35s \\

\textbf{Answer:} E \\
\textbf{Explanation:} [IMAGE:0]

\hrule
\vspace{1em}


\noindent
\textbf{Q389.} A theme park charges different prices for adult and child tickets.
A ticket for 2 adults and 3 children costs £29.
A ticket for 3 adults and 2 children costs £31.
What is the cost for 4 adults and 4 children?



\textbf{A.} [IMAGE:0] \\
\textbf{B.} [IMAGE:1] \\
\textbf{C.} [IMAGE:2] \\
\textbf{D.} [IMAGE:3] \\

\textbf{Answer:} G \\
\textbf{Explanation:} Let a be the adult ticket price, c be the child ticket price;
2a + 3c = 29; 3a + 2c = 31.
14+15=29; 21+10=31.
Solving gives a = 7, c = 5.
4a + 4c = 28 + 20 = 48.

\hrule
\vspace{1em}


\noindent
\textbf{Q390.} From the equation for the power in an electrical circuit:
[IMAGE:0]
, where
[IMAGE:1]
is current and
[IMAGE:2]
is voltage. Find the correct statements:
Power is directly proportional to both current and voltage.
Doubling the current doubles the voltage if power remains constant.
Power increases only if current and voltage increase at the same time.



\textbf{A.} [IMAGE:0] \\
\textbf{B.} [IMAGE:1] \\
\textbf{C.} [IMAGE:2] \\
\textbf{D.} [IMAGE:3] \\

\textbf{Answer:} B \\
\textbf{Explanation:} Doubling the current doubles the power if voltage remains constant. Doubling the current bisects the voltage if power remains constant.

\hrule
\vspace{1em}


\noindent
\textbf{Q391.} From the equation for the work done:
[IMAGE:0]
, where
[IMAGE:1]
is force and
[IMAGE:2]
is distance. Find the correct statements:
Work done is directly proportional to the force applied.
Work done is directly proportional to the distance moved.
No work is done if there is no displacement.



\textbf{A.} [IMAGE:0] \\
\textbf{B.} [IMAGE:1] \\
\textbf{C.} [IMAGE:2] \\
\textbf{D.} [IMAGE:3] \\

\textbf{Answer:} A \\
\textbf{Explanation:} All statements are correct based on the work done formula.

\hrule
\vspace{1em}


\noindent
\textbf{Q392.} A man is cycling along a straight horizontal road at a constant speed of
[IMAGE:0]
. He passes a boy who is cycling at
[IMAGE:1]
in the same direction. When the man is level with the boy, the boy begins to accelerate at a constant rate of
[IMAGE:2]
. The boy maintains this constant acceleration and the man continues at constant speed until the boy passes the man. What is the time interval between the two instances when the man and the boy are level?



\textbf{A.} 5s \\
\textbf{B.} 10s \\
\textbf{C.} 20s \\
\textbf{D.} 35s \\

\textbf{Answer:} B \\
\textbf{Explanation:} [IMAGE:0]

\hrule
\vspace{1em}


\noindent
\textbf{Q393.} From the equation for the pressure of a gas:
[IMAGE:0]
, where
[IMAGE:1]
is force and
[IMAGE:2]
is area. Find the correct statements:
Pressure is inversely proportional to the area over which the force is applied.
Pressure increases with an increase in force (A is fixed).
Doubling the area halves the pressure for the same force.



\textbf{A.} [IMAGE:0] \\
\textbf{B.} [IMAGE:1] \\
\textbf{C.} [IMAGE:2] \\
\textbf{D.} [IMAGE:3] \\

\textbf{Answer:} A \\
\textbf{Explanation:} The answer is A. All statements are correct based on the pressure formula.

\hrule
\vspace{1em}


\noindent
\textbf{Q394.} A man is cycling along a straight horizontal road at a constant speed of
[IMAGE:0]
. He passes a boy who is cycling at
[IMAGE:1]
in the same direction. When t
A
he man is level with the boy, the boy begins to accelerate at a constant rate of
[IMAGE:2]
. The boy maintains this constant acceleration and the man continues at constant speed until the boy passes the man. What is the time interval between the two instances when the man and the boy are level?



\textbf{A.} 5s \\
\textbf{B.} 10s \\
\textbf{C.} 22.5s \\
\textbf{D.} 40s \\

\textbf{Answer:} B \\
\textbf{Explanation:} [IMAGE:0]

\hrule
\vspace{1em}


\noindent
\textbf{Q395.} From the equation for the volume of a cylinder:
[IMAGE:0]
, where
[IMAGE:1]
is the radius and
[IMAGE:2]
is the height. Find the correct statements:
The volume is proportional to the square of the radius.
The volume is directly proportional to the height.
Doubling the radius quadruples the volume.



\textbf{A.} [IMAGE:0] \\
\textbf{B.} [IMAGE:1] \\
\textbf{C.} [IMAGE:2] \\
\textbf{D.} [IMAGE:3] \\

\textbf{Answer:} A \\
\textbf{Explanation:} The answer is A. All statements are correct based on the volume formula.

\hrule
\vspace{1em}


\noindent
\textbf{Q396.} An advanced proton cyclotron increases particle energy by ΔE per revolution while experiencing the
[IMAGE:0]
time radiative energy loss
[IMAGE:1]
, where N is the number of revolutions. After N revolutions, what's its kinetic energy?



\textbf{A.} [IMAGE:0] \\
\textbf{B.} [IMAGE:1] \\
\textbf{C.} [IMAGE:2] \\
\textbf{D.} [IMAGE:3] \\

\textbf{Answer:} D \\
\textbf{Explanation:} [IMAGE:0]

\hrule
\vspace{1em}


\noindent
\textbf{Q397.} From the equation for the gravitational force between two masses:
[IMAGE:0]
, where
[IMAGE:1]
is the gravitational constant,
[IMAGE:2]
and
[IMAGE:3]
are masses, and
[IMAGE:4]
is the distance. Find the correct statements:
The gravitational force is proportional to the square of the distance.
The force is directly proportional to the product of the two masses.
Increasing the distance between masses decreases the force.



\textbf{A.} [IMAGE:0] \\
\textbf{B.} [IMAGE:1] \\
\textbf{C.} [IMAGE:2] \\
\textbf{D.} [IMAGE:3] \\

\textbf{Answer:} A \\
\textbf{Explanation:} The gravitational force is inversely proportional to the square of the distance.

\hrule
\vspace{1em}


\noindent
\textbf{Q398.} From the equation for the electric field due to a point charge:
[IMAGE:0]
, where
[IMAGE:1]
is Coulomb's constant,
[IMAGE:2]
is the charge, and
[IMAGE:3]
is the distance. Find the correct statements:
The electric field strength is inversely proportional to the square of the distance.
The electric field is directly proportional to the charge.
Doubling the distance from the charge reduces the electric field by a factor of four.



\textbf{A.} [IMAGE:0] \\
\textbf{B.} [IMAGE:1] \\
\textbf{C.} [IMAGE:2] \\
\textbf{D.} [IMAGE:3] \\

\textbf{Answer:} A \\
\textbf{Explanation:} All statements are correct based on the electric field formula.

\hrule
\vspace{1em}


\noindent
\textbf{Q399.} A bungee jumper (m) falls distance L before cord (2k) stretches, with air resistance F. What's maximum stretch distance?



\textbf{A.} [IMAGE:0] \\
\textbf{B.} [IMAGE:1] \\
\textbf{C.} [IMAGE:2] \\
\textbf{D.} [IMAGE:3] \\

\textbf{Answer:} A \\
\textbf{Explanation:} [IMAGE:0]

\hrule
\vspace{1em}


\noindent
\textbf{Q400.} From the equation for the potential energy of a spring:
[IMAGE:0]
, where
[IMAGE:1]
is the spring constant and
[IMAGE:2]
is the displacement. Find the correct statements:
The spring constant
[IMAGE:3]
represents the stiffness of the spring.
The potential energy increases with the square of the displacement.
A stiffer spring (larger
[IMAGE:4]
) stores less energy for the same displacement.



\textbf{A.} [IMAGE:0] \\
\textbf{B.} [IMAGE:1] \\
\textbf{C.} [IMAGE:2] \\
\textbf{D.} [IMAGE:3] \\

\textbf{Answer:} B \\
\textbf{Explanation:} A stiffer spring (larger
[IMAGE:0]
) stores more energy for the same displacement.

\hrule
\vspace{1em}


\noindent
\textbf{Q401.} From the equation for the kinetic energy of an object:
[IMAGE:0]
, where
[IMAGE:1]
is mass and
[IMAGE:2]
is velocity. Find the correct statements:
Kinetic energy is directly proportional to mass.
Kinetic energy is proportional to the square of velocity.
Kinetic energy increases with an increase in both mass and velocity.



\textbf{A.} [IMAGE:0] \\
\textbf{B.} [IMAGE:1] \\
\textbf{C.} [IMAGE:2] \\
\textbf{D.} [IMAGE:3] \\

\textbf{Answer:} A \\
\textbf{Explanation:} All statements are correct based on the kinetic energy formula.

\hrule
\vspace{1em}


\noindent
\textbf{Q402.} A charged particle (2q) with mass 2m accelerates through potential difference V while experiencing constant drag force F. What's its final speed after moving with distance d?



\textbf{A.} [IMAGE:0] \\
\textbf{B.} [IMAGE:1] \\
\textbf{C.} [IMAGE:2] \\
\textbf{D.} [IMAGE:3] \\

\textbf{Answer:} D \\
\textbf{Explanation:} Solution: 2qV = mv² + Fd.

\hrule
\vspace{1em}


\noindent
\textbf{Q403.} From the equation for the resistance of a wire:
[IMAGE:0]
, where
[IMAGE:1]
is the resistivity,
[IMAGE:2]
is the length, and
[IMAGE:3]
is the cross-sectional area. Find the correct statements:
The resistivity
[IMAGE:4]
is dependent of the wire's dimensions.
The longer the wire, the larger the resistance.
The resistance decreases as the cross-sectional area increases.



\textbf{A.} [IMAGE:0] \\
\textbf{B.} [IMAGE:1] \\
\textbf{C.} [IMAGE:2] \\
\textbf{D.} [IMAGE:3] \\

\textbf{Answer:} E \\
\textbf{Explanation:} The resistivity
[IMAGE:0]
is a material property and independent of the wire's dimensions.

\hrule
\vspace{1em}


\noindent
\textbf{Q404.} A small steel ball of mass m is released from the top of a semi-circular ramp of radius r as shown in the diagram:
After being released, the ball moves around the semi-circle to the lowest point at position P and then rises to a maximum height on the other side at position Q before falling down again. Assume that the friction force acting on the ball has a constant magnitude whilst the ball is moving. What is the equation of angle
[IMAGE:0]
of the ball with half kinetic energy (of the ball as it passes the position P at the second time) after passing through the P twice (gravitational field strength = g) (for example, the angle of Q position is 45°)



\textbf{A.} [IMAGE:0] \\
\textbf{B.} [IMAGE:1] \\
\textbf{C.} [IMAGE:2] \\
\textbf{D.} [IMAGE:3] \\

\textbf{Answer:} A \\
\textbf{Explanation:} [IMAGE:0]

\hrule
\vspace{1em}


\noindent
\textbf{Q405.} From equation for the capacitance of a parallel plate:
[IMAGE:0]
, where A is the projected area and d is the distance between two plates. Find the correct statements:
The constant stands for a quantity related to relative permittivity between two plates.
The longer the distance between two plates, the larger the capacitance is.
The capacitance is affected by the voltage applied.



\textbf{A.} [IMAGE:0] \\
\textbf{B.} [IMAGE:1] \\
\textbf{C.} [IMAGE:2] \\
\textbf{D.} [IMAGE:3] \\

\textbf{Answer:} B \\
\textbf{Explanation:} The shorter the distance between two plates, the larger the capacitance is due to
[IMAGE:0]
.
The capacitance is irrelevant to the voltage applied.

\hrule
\vspace{1em}


\noindent
\textbf{Q406.} Choose the correct statements below:
The gravitational potential energy of an object is zero at the center of the Earth.
A satellite in a circular orbit experiences no net force.
The orbital speed of a satellite decreases as it moves to a higher orbit.



\textbf{A.} [IMAGE:0] \\
\textbf{B.} [IMAGE:1] \\
\textbf{C.} [IMAGE:2] \\
\textbf{D.} [IMAGE:3] \\

\textbf{Answer:} E \\
\textbf{Explanation:} Potential energy is not zero at Earth's center which is a little bit complex (may be relevant to the reference points). Satellites in orbit experience centripetal force. Orbital speed decreases with higher orbits.

\hrule
\vspace{1em}


\noindent
\textbf{Q407.} Choose the correct statements below:
The gravitational field strength on a planet's surface is inversely proportional to the square of its radius.
A satellite's kinetic energy is greatest at its apogee.
The escape velocity from a planet is directly proportional to the square root of its radius.



\textbf{A.} [IMAGE:0] \\
\textbf{B.} [IMAGE:1] \\
\textbf{C.} [IMAGE:2] \\
\textbf{D.} [IMAGE:3] \\

\textbf{Answer:} B \\
\textbf{Explanation:} Field strength is inversely proportional to radius squared. Kinetic energy is greatest at perigee. Escape velocity is proportional to the square root of mass, not radius.

\hrule
\vspace{1em}


\noindent
\textbf{Q408.} Choose the correct statements below:
The gravitational force between two objects increases as the mass of either object increases.
A satellite in a lower orbit has a longer orbital period than one in a higher orbit.
The gravitational potential energy of an object decreases as it moves away from the Earth.



\textbf{A.} [IMAGE:0] \\
\textbf{B.} [IMAGE:1] \\
\textbf{C.} [IMAGE:2] \\
\textbf{D.} [IMAGE:3] \\

\textbf{Answer:} B \\
\textbf{Explanation:} The answer is B. Gravitational force increases with mass. Lower orbits have shorter periods. Potential energy increases (becomes less negative) with distance.

\hrule
\vspace{1em}


\noindent
\textbf{Q409.} Choose the correct statements below:
The gravitational field strength at the center of the Earth is zero.
A satellite's orbital velocity is directly proportional to the square root of the central body's mass.
The total energy of a satellite in a bound orbit is positive.



\textbf{A.} [IMAGE:0] \\
\textbf{B.} [IMAGE:1] \\
\textbf{C.} [IMAGE:2] \\
\textbf{D.} [IMAGE:3] \\

\textbf{Answer:} D \\
\textbf{Explanation:} The answer is D. Field strength at Earth's center is zero. Orbital velocity is proportional to the square root of mass. Total energy in bound orbit (which is different from the unbound orbit or escape orbit) is negative.

\hrule
\vspace{1em}


\noindent
\textbf{Q410.} Choose the correct statements below:
The weight of an object decreases as it moves away from the Earth's surface.
A satellite in a geostationary orbit has a period of 24 hours.
The gravitational potential energy of an object is always negative.



\textbf{A.} [IMAGE:0] \\
\textbf{B.} [IMAGE:1] \\
\textbf{C.} [IMAGE:2] \\
\textbf{D.} [IMAGE:3] \\

\textbf{Answer:} A \\
\textbf{Explanation:} Weight decreases with distance(due to the decreasing g). Geostationary orbits match Earth's rotation period. Gravitational potential energy is negative relative to infinity.

\hrule
\vspace{1em}


\noindent
\textbf{Q411.} Choose the correct statements below:
The gravitational force exerted by the Earth on the Moon is greater than that exerted by the Moon on the Earth.
The kinetic energy of a satellite in orbit is greatest at its apogee.
The gravitational field inside a uniform spherical shell is zero.



\textbf{A.} [IMAGE:0] \\
\textbf{B.} [IMAGE:1] \\
\textbf{C.} [IMAGE:2] \\
\textbf{D.} [IMAGE:3] \\

\textbf{Answer:} E \\
\textbf{Explanation:} The answer is E. Gravitational forces are equal and opposite. Kinetic energy is greatest at perigee. The field inside a uniform spherical shell is zero.

\hrule
\vspace{1em}


\noindent
\textbf{Q412.} A satellite in circular orbit at altitude h experiences atmospheric drag force F=
[IMAGE:0]
. The radius of the Earth is R. What's its orbital speed after completing half an orbit?



\textbf{A.} [IMAGE:0] \\
\textbf{B.} [IMAGE:1] \\
\textbf{C.} [IMAGE:2] \\
\textbf{D.} [IMAGE:3] \\

\textbf{Answer:} B \\
\textbf{Explanation:} [IMAGE:0]
[IMAGE:1]

\hrule
\vspace{1em}


\noindent
\textbf{Q413.} Choose the correct statements below:
The total mechanical energy of a satellite in a stable orbit is zero.
The gravitational potential energy of an object increases as it moves away from the Earth.
A satellite's orbital period is independent of its mass.



\textbf{A.} [IMAGE:0] \\
\textbf{B.} [IMAGE:1] \\
\textbf{C.} [IMAGE:2] \\
\textbf{D.} [IMAGE:3] \\

\textbf{Answer:} E \\
\textbf{Explanation:} Total mechanical energy in orbit is negative. Potential energy increases (becomes less negative) as distance increases. Orbital period depends only on the central body's mass and orbit radius.

\hrule
\vspace{1em}


\noindent
\textbf{Q414.} Choose the correct statements below:
The escape velocity from a planet is independent of the mass of the escaping object.
The gravitational field strength on the surface of a planet is directly proportional to its mass.
A satellite in an elliptical orbit has its greatest speed at the point closest to the planet.



\textbf{A.} [IMAGE:0] \\
\textbf{B.} [IMAGE:1] \\
\textbf{C.} [IMAGE:2] \\
\textbf{D.} [IMAGE:3] \\

\textbf{Answer:} A \\
\textbf{Explanation:} The answer is A. Escape velocity depends only on the planet's mass and radius. Gravitational field strength is proportional to mass. In an elliptical orbit, speed is greatest at periapsis (closest approach).

\hrule
\vspace{1em}


\noindent
\textbf{Q415.} Choose the correct statements below:
The gravitational force between two objects is inversely proportional to the square of the distance between their centers.
A satellite in a circular orbit has a constant speed.
The potential energy of an object in a gravitational field decreases as it moves closer to the source of the field.



\textbf{A.} [IMAGE:0] \\
\textbf{B.} [IMAGE:1] \\
\textbf{C.} [IMAGE:2] \\
\textbf{D.} [IMAGE:3] \\

\textbf{Answer:} A \\
\textbf{Explanation:} The gravitational force formula supports statement 1. In a circular orbit, speed remains constant due to balanced forces, supporting statement 2. Potential energy decreases as an object moves closer to the gravitational source, supporting statement 3.

\hrule
\vspace{1em}


\noindent
\textbf{Q416.} A spring (k) launches a block (m) across a rough surface (μ). If compressed distance d, what's proportion of the block's stopping distance to the compressed distanced?



\textbf{A.} [IMAGE:0] \\
\textbf{B.} [IMAGE:1] \\
\textbf{C.} [IMAGE:2] \\
\textbf{D.} [IMAGE:3] \\

\textbf{Answer:} D \\
\textbf{Explanation:} [IMAGE:0]

\hrule
\vspace{1em}


\noindent
\textbf{Q417.} Choose the correct statements below:
Gravitational field is not uniform between earth and moon.
The gravitational field strength on the surface of a planet depends on both its mass and radius; a larger planet does not necessarily have a smaller gravitational field strength on its surface compared to a smaller planet.
When the satellite transfers to a higher orbit, its kinetic energy must be increased.



\textbf{A.} [IMAGE:0] \\
\textbf{B.} [IMAGE:1] \\
\textbf{C.} [IMAGE:2] \\
\textbf{D.} [IMAGE:3] \\

\textbf{Answer:} A \\
\textbf{Explanation:} Statements 1 and 2 are correct.
[IMAGE:0]
; and
[IMAGE:1]
, when r is increased; kinetic energy is increased. Thus, Statement 3 is also correct.

\hrule
\vspace{1em}


\noindent
\textbf{Q418.} The unit of impulse in physics is equivalent to:



\textbf{A.} [IMAGE:0] \\
\textbf{B.} [IMAGE:1] \\
\textbf{C.} [IMAGE:2] \\
\textbf{D.} [IMAGE:3] \\

\textbf{Answer:} A \\
\textbf{Explanation:} Impulse = Force × Time; Unit = kgm/s.

\hrule
\vspace{1em}


\noindent
\textbf{Q419.} A roller coaster car starts from rest at height H, completes a vertical loop of diameter D (H>D), with constant friction force f. The friction before the roller coaster car entering the ring can be neglected. What's its speed at the top of the loop?



\textbf{A.} [IMAGE:0] \\
\textbf{B.} [IMAGE:1] \\
\textbf{C.} [IMAGE:2] \\
\textbf{D.} [IMAGE:3] \\

\textbf{Answer:} B \\
\textbf{Explanation:} [IMAGE:0]

\hrule
\vspace{1em}


\noindent
\textbf{Q420.} The unit for energy in the SI system is:



\textbf{A.} [IMAGE:0] \\
\textbf{B.} [IMAGE:1] \\
\textbf{C.} [IMAGE:2] \\
\textbf{D.} [IMAGE:3] \\

\textbf{Answer:} B \\
\textbf{Explanation:} Energy is measured in Joules; Unit = kgm²/s².

\hrule
\vspace{1em}


\noindent
\textbf{Q421.} When calculating work done, the unit used is:



\textbf{A.} [IMAGE:0] \\
\textbf{B.} [IMAGE:1] \\
\textbf{C.} [IMAGE:2] \\
\textbf{D.} [IMAGE:3] \\

\textbf{Answer:} B \\
\textbf{Explanation:} Work = Force × Distance; Unit = kgm²/s² (also known as Joule(J)).

\hrule
\vspace{1em}


\noindent
\textbf{Q422.} A block slides down a 60° incline of length L with coefficient of kinetic friction μ. If released from rest, what's its speed at the bottom?



\textbf{A.} [IMAGE:0] \\
\textbf{B.} [IMAGE:1] \\
\textbf{C.} [IMAGE:2] \\
\textbf{D.} [IMAGE:3] \\

\textbf{Answer:} D \\
\textbf{Explanation:} Solution: Net acceleration = gsin60° - μgcos60° = \sqrt{}3/2g - 0.5μg. And
[IMAGE:0]

\hrule
\vspace{1em}


\noindent
\textbf{Q423.} The SI unit for measuring mass is:



\textbf{A.} [IMAGE:0] \\
\textbf{B.} [IMAGE:1] \\
\textbf{C.} [IMAGE:2] \\
\textbf{D.} [IMAGE:3] \\

\textbf{Answer:} B \\
\textbf{Explanation:} Mass is measured in kilograms; Unit = kg or g or t(ton).
But the SI unit is only kg.

\hrule
\vspace{1em}


\noindent
\textbf{Q424.} The unit of momentum in physics is given by:



\textbf{A.} [IMAGE:0] \\
\textbf{B.} [IMAGE:1] \\
\textbf{C.} [IMAGE:2] \\
\textbf{D.} [IMAGE:3] \\

\textbf{Answer:} A \\
\textbf{Explanation:} Momentum = mass × velocity; Unit = kgm/s.

\hrule
\vspace{1em}


\noindent
\textbf{Q425.} In the context of motion, the SI unit for time is universally recognized as:



\textbf{A.} [IMAGE:0] \\
\textbf{B.} [IMAGE:1] \\
\textbf{C.} [IMAGE:2] \\
\textbf{D.} [IMAGE:3] \\

\textbf{Answer:} B \\
\textbf{Explanation:} Time is measured in seconds; Unit = s or ms or min or h.
But, SI unit is only s.

\hrule
\vspace{1em}


\noindent
\textbf{Q426.} A pendulum bob of mass 0.5m is released from height h above its lowest point. The string encounters constant air resistance force F during its swing. What is the bob's speed at the lowest point (at first time)?



\textbf{A.} [IMAGE:0] \\
\textbf{B.} [IMAGE:1] \\
\textbf{C.} [IMAGE:2] \\
\textbf{D.} [IMAGE:3] \\

\textbf{Answer:} B \\
\textbf{Explanation:} Solution: Energy loss = F×2πh/4 = F×πh/2. Apply energy conservation: 0.5mgh = ½×0.5mv² + F×πh/2.

\hrule
\vspace{1em}


\noindent
\textbf{Q427.} The SI unit for expressing displacement is:



\textbf{A.} [IMAGE:0] \\
\textbf{B.} [IMAGE:1] \\
\textbf{C.} [IMAGE:2] \\
\textbf{D.} [IMAGE:3] \\

\textbf{Answer:} B \\
\textbf{Explanation:} Displacement is a measure of length; Unit = m or km or cm.
But, SI unit is only m.

\hrule
\vspace{1em}


\noindent
\textbf{Q428.} A small steel ball of mass m is released from the top of a semi-circular ramp of radius r as shown in the diagram:
After being released, the ball moves around the semi-circle to the lowest point at position P and then rises to a maximum height on the other side at position Q before falling down again. Assume that the friction force acting on the ball has a constant magnitude whilst the ball is moving. What is the additional energy E after it passes position P at the second time to make the ball achieve the same height of Q on the left side? (gravitational field strength = g)



\textbf{A.} [IMAGE:0] \\
\textbf{B.} [IMAGE:1] \\
\textbf{C.} [IMAGE:2] \\
\textbf{D.} [IMAGE:3] \\

\textbf{Answer:} D \\
\textbf{Explanation:} [IMAGE:0]

\hrule
\vspace{1em}


\noindent
\textbf{Q429.} The unit representing acceleration in the International System of Units (SI) is:



\textbf{A.} [IMAGE:0] \\
\textbf{B.} [IMAGE:1] \\
\textbf{C.} [IMAGE:2] \\
\textbf{D.} [IMAGE:3] \\

\textbf{Answer:} B \\
\textbf{Explanation:} According to the formula: a = Δv/Δt; Unit = m/s² or km/s².
But the SI unit is only m/s².

\hrule
\vspace{1em}


\noindent
\textbf{Q430.} The standard unit for measuring velocity is:



\textbf{A.} [IMAGE:0] \\
\textbf{B.} [IMAGE:1] \\
\textbf{C.} [IMAGE:2] \\
\textbf{D.} [IMAGE:3] \\

\textbf{Answer:} B \\
\textbf{Explanation:} According to the definition: Velocity is displacement per unit time; Unit = m/s or cm/s or cm/ms.
But, standard unit is only m/s.

\hrule
\vspace{1em}


\noindent
\textbf{Q431.} The correct unit of the force is:



\textbf{A.} [IMAGE:0] \\
\textbf{B.} [IMAGE:1] \\
\textbf{C.} [IMAGE:2] \\
\textbf{D.} [IMAGE:3] \\

\textbf{Answer:} C \\
\textbf{Explanation:} According to the formula: F=ma; F=gm/$s^2$.

\hrule
\vspace{1em}


\noindent
\textbf{Q432.} The circuit shown in the diagram contains six resistors and an ideal volt ammeter. And the resistance of one of the six resistors is unknown (xΩ) . The reading scope on the ammeter is 4A.
What is the power dissipated in the "unknown resistor" (xΩ)?



\textbf{A.} 0W \\
\textbf{B.} 2W \\
\textbf{C.} 4W \\
\textbf{D.} 6W \\

\textbf{Answer:} G \\
\textbf{Explanation:} [IMAGE:0]

\hrule
\vspace{1em}


\noindent
\textbf{Q433.} The circuit shown in the diagram contains six resistors and an ideal voltmeter. And the resistance of one of the six resistors is unknown (xΩ) . The reading scope on the voltmeter is 5.5V.
What is the power dissipated in the "unknown resistor" (xΩ)?



\textbf{A.} 0W \\
\textbf{B.} 2W \\
\textbf{C.} 4W \\
\textbf{D.} 6W \\

\textbf{Answer:} F \\
\textbf{Explanation:} [IMAGE:0]

\hrule
\vspace{1em}


\noindent
\textbf{Q434.} In the diagram, QS is perpendicular to PR. PS = x cm. PQ = y cm. QR = z cm. ∠QRS = 61°. PSR is a straight line.
Which one of the following is an expression for the height to QR in triangular QRS, in
[IMAGE:0]
?



\textbf{A.} [IMAGE:0] \\
\textbf{B.} [IMAGE:1] \\
\textbf{C.} [IMAGE:2] \\
\textbf{D.} [IMAGE:3]
[IMAGE:4] \\

\textbf{Answer:} F \\
\textbf{Explanation:} [IMAGE:0]
,
[IMAGE:1]
[IMAGE:2]
[IMAGE:3]
.
Thus,
[IMAGE:4]
.

\hrule
\vspace{1em}


\noindent
\textbf{Q435.} The circuit shown in the diagram contains six resistors and an ideal digital ammeter. And one of the six resistors is variable (xΩ) whose scope is from 1Ω to 3Ω.
What is the reading scope on the ammeter?



\textbf{A.} [IMAGE:0] \\
\textbf{B.} [IMAGE:1] \\
\textbf{C.} [IMAGE:2] \\
\textbf{D.} [IMAGE:3] \\

\textbf{Answer:} G \\
\textbf{Explanation:} [IMAGE:0]

\hrule
\vspace{1em}


\noindent
\textbf{Q436.} In the diagram, QS is perpendicular to PR. PS = x cm. PQ = y cm. QR = z cm. ∠QRS = 61°. PSR is a straight line.
Which one of the following is an expression for perimeter of triangular QRS, in
[IMAGE:0]
?



\textbf{A.} [IMAGE:0] \\
\textbf{B.} [IMAGE:1] \\
\textbf{C.} [IMAGE:2] \\
\textbf{D.} [IMAGE:3] \\

\textbf{Answer:} D \\
\textbf{Explanation:} [IMAGE:0]
,
[IMAGE:1]
[IMAGE:2]
[IMAGE:3]
[IMAGE:4]
.

\hrule
\vspace{1em}


\noindent
\textbf{Q437.} In the diagram, QS is perpendicular to PR. PS = x cm. PQ = y cm. QR = z cm. ∠QRS = 61°.
[IMAGE:0]
. PSR is a straight line.
Which one of the following is an expression for area of triangular SQP, in
[IMAGE:1]
?



\textbf{A.} [IMAGE:0] \\
\textbf{B.} [IMAGE:1] \\
\textbf{C.} [IMAGE:2] \\
\textbf{D.} [IMAGE:3] \\

\textbf{Answer:} H \\
\textbf{Explanation:} [IMAGE:0]
[IMAGE:1]
[IMAGE:2]
.
[IMAGE:3]
and
[IMAGE:4]
Thus,
[IMAGE:5]
Thus,
[IMAGE:6]
.

\hrule
\vspace{1em}


\noindent
\textbf{Q438.} The circuit shown in the diagram contains six resistors and an ideal volt ammeter. And one of the six resistors is variable (xΩ) whose maximum is 6Ω.
What is the reading scope on the voltmeter?



\textbf{A.} [IMAGE:0] \\
\textbf{B.} [IMAGE:1] \\
\textbf{C.} [IMAGE:2] \\
\textbf{D.} [IMAGE:3] \\

\textbf{Answer:} D \\
\textbf{Explanation:} [IMAGE:0]

\hrule
\vspace{1em}


\noindent
\textbf{Q439.} In the diagram, QS is perpendicular to PR. PS = x cm. PQ = y cm. QR = z cm. ∠QRS = 61°. PSR is a straight line.
Which one of the following is an expression for area of triangular SQR, in
[IMAGE:0]
?



\textbf{A.} [IMAGE:0] \\
\textbf{B.} [IMAGE:1] \\
\textbf{C.} [IMAGE:2] \\
\textbf{D.} [IMAGE:3] \\

\textbf{Answer:} G \\
\textbf{Explanation:} [IMAGE:0]
,
[IMAGE:1]
[IMAGE:2]
[IMAGE:3]
.

\hrule
\vspace{1em}


\noindent
\textbf{Q440.} In the diagram, QS is perpendicular to PR. PS = x cm. PQ = y cm. QR = z cm. ∠QRS = 61°. PSR is a straight line.
Which one of the following is an expression for the length y, in cm?



\textbf{A.} [IMAGE:0] \\
\textbf{B.} [IMAGE:1] \\
\textbf{C.} [IMAGE:2] \\
\textbf{D.} [IMAGE:3] \\

\textbf{Answer:} A \\
\textbf{Explanation:} [IMAGE:0]
[IMAGE:1]
Thus,
[IMAGE:2]
.

\hrule
\vspace{1em}


\noindent
\textbf{Q441.} The circuit shown in the diagram contains six resistors and an ideal digital ammeter.
What is the reading on the ammeter?



\textbf{A.} 0V \\
\textbf{B.} 2V \\
\textbf{C.} 4V \\
\textbf{D.} 6V \\

\textbf{Answer:} C \\
\textbf{Explanation:} [IMAGE:0]

\hrule
\vspace{1em}


\noindent
\textbf{Q442.} In the diagram, QS is perpendicular to PR. PS = x cm. PQ = y cm. QR = z cm. ∠QRS = 61°. PSR is a straight line.
Which one of the following is an expression for the length x, in cm?



\textbf{A.} [IMAGE:0] \\
\textbf{B.} [IMAGE:1] \\
\textbf{C.} [IMAGE:2] \\
\textbf{D.} [IMAGE:3] \\

\textbf{Answer:} D \\
\textbf{Explanation:} [IMAGE:0]
[IMAGE:1]
Thus,
[IMAGE:2]
.

\hrule
\vspace{1em}


\noindent
\textbf{Q443.} The circuit shown in the diagram contains six resistors and an ideal digital ammeter.
What is the reading on the ammeter?



\textbf{A.} 0V \\
\textbf{B.} 2V \\
\textbf{C.} 4V \\
\textbf{D.} 6V \\

\textbf{Answer:} B \\
\textbf{Explanation:} [IMAGE:0]

\hrule
\vspace{1em}


\noindent
\textbf{Q444.} The circuit shown in the diagram contains six resistors and an ideal digital voltmeter.
What is the reading on the voltmeter?



\textbf{A.} 0V \\
\textbf{B.} 2V \\
\textbf{C.} 4V \\
\textbf{D.} 6V \\

\textbf{Answer:} C \\
\textbf{Explanation:} The voltmeter records the potential difference between the measured points, which is 13V and 9V. The difference is 4V.

\hrule
\vspace{1em}


\noindent
\textbf{Q445.} The circuit shown in the diagram contains six resistors and an ideal digital voltmeter.
What is the reading on the voltmeter?



\textbf{A.} 1V \\
\textbf{B.} 2V \\
\textbf{C.} 4V \\
\textbf{D.} 6V \\

\textbf{Answer:} A \\
\textbf{Explanation:} The voltmeter records the potential difference between the measured points, which is 10V and 9V. The difference is 1V.

\hrule
\vspace{1em}


\noindent
\textbf{Q446.} The circuit shown in the diagram contains six resistors and an ideal digital voltmeter.
What is the reading on the voltmeter?



\textbf{A.} 0V \\
\textbf{B.} 2.4V \\
\textbf{C.} 4V \\
\textbf{D.} 5.2V \\

\textbf{Answer:} C \\
\textbf{Explanation:} The voltmeter records the potential difference between the measured points, which is 10V and 6V. The difference is 4V.

\hrule
\vspace{1em}


\noindent
\textbf{Q447.} Calculate the sum of roots of:
[IMAGE:0]



\textbf{A.} [IMAGE:0] \\
\textbf{B.} [IMAGE:1] \\
\textbf{C.} [IMAGE:2] \\
\textbf{D.} [IMAGE:3] \\

\textbf{Answer:} F \\
\textbf{Explanation:} Let a = 1/(x-1), solve the quadratic equation, then find x values.
[IMAGE:0]
,
[IMAGE:1]
[IMAGE:2]
,
[IMAGE:3]
Thus,
[IMAGE:4]
.

\hrule
\vspace{1em}


\noindent
\textbf{Q448.} The circuit shown in the diagram contains six resistors and an ideal digital voltmeter.
What is the reading on the voltmeter?



\textbf{A.} 0V \\
\textbf{B.} 2V \\
\textbf{C.} 4V \\
\textbf{D.} 6V \\

\textbf{Answer:} A \\
\textbf{Explanation:} The voltmeter records the potential difference between the measured points, which is 10V and 10V. The difference is 0V.

\hrule
\vspace{1em}


\noindent
\textbf{Q449.} Find the product of all real solutions to:
[IMAGE:0]



\textbf{A.} [IMAGE:0] \\
\textbf{B.} [IMAGE:1] \\
\textbf{C.} [IMAGE:2] \\
\textbf{D.} [IMAGE:3] \\

\textbf{Answer:} C \\
\textbf{Explanation:} Let a = x²+3x, solve the quadratic equation, then find x values.
[IMAGE:0]
or
[IMAGE:1]
Then, the product of all real solutions can be -4 or "no real solutions". Thus, -4.

\hrule
\vspace{1em}


\noindent
\textbf{Q450.} Calculate the positive difference between solutions of:
[IMAGE:0]



\textbf{A.} [IMAGE:0] \\
\textbf{B.} [IMAGE:1] \\
\textbf{C.} [IMAGE:2] \\
\textbf{D.} [IMAGE:3] \\

\textbf{Answer:} C \\
\textbf{Explanation:} Let a = eˣ, solve the quadratic equation, then find x values and their positive difference.
The option B is their negative difference.

\hrule
\vspace{1em}


\noindent
\textbf{Q451.} A circuit contains a fixed capacitor Y, a fixed resistor X, and a variable resistor W. The power supply has no internal resistance.
The resistance of W decreases. What is the process of the charge stored in Y at steady state?



\textbf{A.} decreases \\
\textbf{B.} stays constant \\
\textbf{C.} increases \\
\textbf{D.} uncertain \\

\textbf{Answer:} C \\
\textbf{Explanation:} The charge stored in a capacitor is directly proportional to the voltage across it (fixed capacitor). When the resistance of W decreases, the voltage across Y increases, resulting in more stored charge.

\hrule
\vspace{1em}


\noindent
\textbf{Q452.} Determine the sum of all solutions to:
[IMAGE:0]



\textbf{A.} [IMAGE:0] \\
\textbf{B.} [IMAGE:1] \\
\textbf{C.} [IMAGE:2] \\
\textbf{D.} [IMAGE:3] \\

\textbf{Answer:} B \\
\textbf{Explanation:} [IMAGE:0]
[IMAGE:1]
or
[IMAGE:2]
Solution 1:
[IMAGE:3]
or
[IMAGE:4]
and
[IMAGE:5]
Thus,
[IMAGE:6]
Solution 2:
[IMAGE:7]
or
[IMAGE:8]
Thus,
[IMAGE:9]
and
[IMAGE:10]
.

\hrule
\vspace{1em}


\noindent
\textbf{Q453.} Determine the sum of all solutions to:
[IMAGE:0]



\textbf{A.} [IMAGE:0] \\
\textbf{B.} [IMAGE:1] \\
\textbf{C.} [IMAGE:2] \\
\textbf{D.} [IMAGE:3] \\

\textbf{Answer:} B \\
\textbf{Explanation:} [IMAGE:0]
[IMAGE:1]
[IMAGE:2]
Thus,
[IMAGE:3]
and
[IMAGE:4]
.
[IMAGE:5]
and
[IMAGE:6]
[IMAGE:7]

\hrule
\vspace{1em}


\noindent
\textbf{Q454.} Find the sum of the real number solutions of:
[IMAGE:0]



\textbf{A.} [IMAGE:0] \\
\textbf{B.} [IMAGE:1] \\
\textbf{C.} [IMAGE:2] \\
\textbf{D.} [IMAGE:3] \\

\textbf{Answer:} H \\
\textbf{Explanation:} [IMAGE:0]
is a hidden condition
Though the
[IMAGE:1]
, it can not be possible to find the sum of the real number solutions.
[IMAGE:2]
,
[IMAGE:3]
.

\hrule
\vspace{1em}


\noindent
\textbf{Q455.} A circuit contains two fixed resistors, X and Y, and a variable resistor W. The variable power supply(PS) has no internal resistance.
How can the most probable changes in resistance and power supply make it: the power dissipated in X decreases and the power dissipated in Y decreases?



\textbf{A.} [IMAGE:0] \\
\textbf{B.} [IMAGE:1] \\
\textbf{C.} [IMAGE:2] \\
\textbf{D.} [IMAGE:3] \\

\textbf{Answer:} D \\
\textbf{Explanation:} When the resistance of W is increased, The W and Y in total has a larger resistance; X owns less voltage; W and Y owns more voltage.
When the resistance of W is decreased, The W and Y in total has a smaller resistance; X owns more voltage; W and Y owns less voltage.
When the power supply is increased, X owns more voltage; W and Y owns more voltage.
When the power supply is decreased, X owns more voltage; W and Y owns more voltage.
Thus, "PS: increases, W: impossible", "PS: stays constant, W: impossible", "PS: decreases, W: any".

\hrule
\vspace{1em}


\noindent
\textbf{Q456.} Find the sum of the solutions of:
[IMAGE:0]



\textbf{A.} [IMAGE:0] \\
\textbf{B.} [IMAGE:1] \\
\textbf{C.} [IMAGE:2] \\
\textbf{D.} [IMAGE:3] \\

\textbf{Answer:} B \\
\textbf{Explanation:} [IMAGE:0]
[IMAGE:1]
[IMAGE:2]

\hrule
\vspace{1em}


\noindent
\textbf{Q457.} Find the sum of the solutions of:
[IMAGE:0]



\textbf{A.} [IMAGE:0] \\
\textbf{B.} [IMAGE:1] \\
\textbf{C.} [IMAGE:2] \\
\textbf{D.} [IMAGE:3] \\

\textbf{Answer:} A \\
\textbf{Explanation:} [IMAGE:0]
[IMAGE:1]
[IMAGE:2]

\hrule
\vspace{1em}


\noindent
\textbf{Q458.} Find the sum of the solutions of:
[IMAGE:0]



\textbf{A.} [IMAGE:0] \\
\textbf{B.} [IMAGE:1] \\
\textbf{C.} [IMAGE:2] \\
\textbf{D.} [IMAGE:3] \\

\textbf{Answer:} E \\
\textbf{Explanation:} [IMAGE:0]
[IMAGE:1]
[IMAGE:2]

\hrule
\vspace{1em}


\noindent
\textbf{Q459.} A circuit contains two fixed resistors, X and Y, and a variable resistor W. The variable power supply(PS) has no internal resistance.
How can the most probable changes in resistance and power supply make it: the power dissipated in X increases and the power dissipated in Y decreases?



\textbf{A.} [IMAGE:0] \\
\textbf{B.} [IMAGE:1] \\
\textbf{C.} [IMAGE:2] \\
\textbf{D.} [IMAGE:3] \\

\textbf{Answer:} A \\
\textbf{Explanation:} When the resistance of W is increased, The W and Y in total has a larger resistance; X owns less voltage; W and Y owns more voltage.
When the resistance of W is decreased, The W and Y in total has a smaller resistance; X owns more voltage; W and Y owns less voltage.
When the power supply is increased, X owns more voltage; W and Y owns more voltage.
When the power supply is decreased, X owns more voltage; W and Y owns more voltage.
Thus, "PS: impossible, W: increases", "PS: impossible, W: stays constant", "PS: any, W: decreases".

\hrule
\vspace{1em}


\noindent
\textbf{Q460.} Find the sum of the solutions of:
[IMAGE:0]



\textbf{A.} [IMAGE:0] \\
\textbf{B.} [IMAGE:1] \\
\textbf{C.} [IMAGE:2] \\
\textbf{D.} [IMAGE:3] \\

\textbf{Answer:} E \\
\textbf{Explanation:} [IMAGE:0]
[IMAGE:1]
[IMAGE:2]

\hrule
\vspace{1em}


\noindent
\textbf{Q461.} Consider the four lines with the following equations.
Which two lines have no real intersection?
[IMAGE:0]
[IMAGE:1]
[IMAGE:2]
[IMAGE:3]



\textbf{A.} [IMAGE:0] \\
\textbf{B.} [IMAGE:1] \\
\textbf{C.} [IMAGE:2] \\
\textbf{D.} [IMAGE:3] \\

\textbf{Answer:} B \\
\textbf{Explanation:} If two lines have no real intersection, the two lines are parallel and distinct.

\hrule
\vspace{1em}


\noindent
\textbf{Q462.} Consider the four lines with the following equations.
Which two lines are coincident (identical)?
[IMAGE:0]
[IMAGE:1]
[IMAGE:2]
[IMAGE:3]



\textbf{A.} [IMAGE:0] \\
\textbf{B.} [IMAGE:1] \\
\textbf{C.} [IMAGE:2] \\
\textbf{D.} [IMAGE:3] \\

\textbf{Answer:} F \\
\textbf{Explanation:} Divide one equation by a constant to check if it matches another.

\hrule
\vspace{1em}


\noindent
\textbf{Q463.} Consider the four lines with the following equations.
Which two lines are horizontal or vertical?
[IMAGE:0]
[IMAGE:1]
[IMAGE:2]
[IMAGE:3]



\textbf{A.} [IMAGE:0] \\
\textbf{B.} [IMAGE:1] \\
\textbf{C.} [IMAGE:2] \\
\textbf{D.} [IMAGE:3] \\

\textbf{Answer:} B \\
\textbf{Explanation:} Horizontal lines have the form
[IMAGE:0]
, vertical lines have the form
[IMAGE:1]
.

\hrule
\vspace{1em}


\noindent
\textbf{Q464.} Consider the four lines with the following equations.
Which two lines have the same
[IMAGE:0]
-intercept?
[IMAGE:1]
[IMAGE:2]
[IMAGE:3]
[IMAGE:4]



\textbf{A.} [IMAGE:0] \\
\textbf{B.} [IMAGE:1] \\
\textbf{C.} [IMAGE:2] \\
\textbf{D.} [IMAGE:3] \\

\textbf{Answer:} F \\
\textbf{Explanation:} Find the
[IMAGE:0]
-intercept by setting
[IMAGE:1]
.

\hrule
\vspace{1em}


\noindent
\textbf{Q465.} A circuit contains two fixed resistors, X and Y, and a variable voltage source V. The internal resistance of the voltage source is negligible.
The voltage of V increases 10V then decreases 9.9V. What happens to the power dissipated in X and in Y after the whole process?



\textbf{A.} [IMAGE:0] \\
\textbf{B.} [IMAGE:1] \\
\textbf{C.} [IMAGE:2] \\
\textbf{D.} [IMAGE:3] \\

\textbf{Answer:} F \\
\textbf{Explanation:} When the voltage of V increases (10-9.9>0), the power dissipated in both X and Y increases because power is proportional to the square of the voltage.

\hrule
\vspace{1em}


\noindent
\textbf{Q466.} Consider the four lines with the following equations.
Which two lines intersect at the origin perpendicularly?
[IMAGE:0]
[IMAGE:1]
[IMAGE:2]
[IMAGE:3]



\textbf{A.} [IMAGE:0] \\
\textbf{B.} [IMAGE:1] \\
\textbf{C.} [IMAGE:2] \\
\textbf{D.} [IMAGE:3] \\

\textbf{Answer:} D \\
\textbf{Explanation:} Substitute
[IMAGE:0]
into the equations. If satisfied, the line passes through the origin.
And the gradient is found respectively, if two lines are perpendicular to each other, the product of the gradient is -1.

\hrule
\vspace{1em}


\noindent
\textbf{Q467.} A circuit contains two fixed resistors, X and Y, and a variable resistor W. The power supply has no internal resistance.
The resistance of W decreases 10 Ω and then increases 9Ω. What happens to the current passing over X and Y after the whole process?



\textbf{A.} X: decreases,Y: decreases \\
\textbf{B.} X: decreases,Y: stays constant \\
\textbf{C.} X: decreases,Y: increases \\
\textbf{D.} X: increases,Y: decreases \\

\textbf{Answer:} D \\
\textbf{Explanation:} When the resistance of W is decreased (-10+9=-1<0), The W and Y in total has a smaller resistance; X owns more voltage; Y owns less voltage. Because of the constant value of resistance of X and Y, the current passing over X increases and the current passing over Y decreases.

\hrule
\vspace{1em}


\noindent
\textbf{Q468.} Consider the four lines with the following equations.
Which two lines are parallel?
[IMAGE:0]
[IMAGE:1]
[IMAGE:2]
[IMAGE:3]



\textbf{A.} [IMAGE:0] \\
\textbf{B.} [IMAGE:1] \\
\textbf{C.} [IMAGE:2] \\
\textbf{D.} [IMAGE:3] \\

\textbf{Answer:} A \\
\textbf{Explanation:} Find the slopes of each line. If two lines are parallel, their slopes are equal.

\hrule
\vspace{1em}


\noindent
\textbf{Q469.} A circuit contains two fixed resistors, X and Y, and a variable resistor W. The power supply has no internal resistance.
The resistance of W decreases 10 Ω and then increases 12Ω. What happens to the current passing over X and Y after the whole process?



\textbf{A.} X: decreases,Y: decreases \\
\textbf{B.} X: decreases,Y: stays constant \\
\textbf{C.} X: decreases,Y: increases \\
\textbf{D.} X: increases,Y: decreases \\

\textbf{Answer:} C \\
\textbf{Explanation:} When the resistance of W is increased (-10+12=2>0), The W and Y in total has a larger resistance; X owns less voltage; Y owns more voltage. Because of the constant value of resistance of X and Y, the current passing over X decreases and the current passing over Y increases.

\hrule
\vspace{1em}


\noindent
\textbf{Q470.} Consider the four lines with the following equations.
Which two lines are perpendicular?
[IMAGE:0]
[IMAGE:1]
[IMAGE:2]
[IMAGE:3]



\textbf{A.} [IMAGE:0] \\
\textbf{B.} [IMAGE:1] \\
\textbf{C.} [IMAGE:2] \\
\textbf{D.} [IMAGE:3] \\

\textbf{Answer:} A \\
\textbf{Explanation:} The gradient is found respectively, if two lines are perpendicular to each other, the product of the gradient is -1.

\hrule
\vspace{1em}


\noindent
\textbf{Q471.} A circuit contains two fixed resistors, X and Y, and a variable resistor W. The power supply has no internal resistance.
The resistance of W decreases 10 Ω and then increases 8Ω. What happens to the power dissipated in X and in Y after the whole process?



\textbf{A.} X: decreases,Y: decreases \\
\textbf{B.} X: decreases,Y: stays constant \\
\textbf{C.} X: decreases,Y: increases \\
\textbf{D.} X: increases,Y: decreases \\

\textbf{Answer:} D \\
\textbf{Explanation:} When the resistance of W is decreased totally (-10+8=-2<0), The W and Y in total has a smaller resistance; X owns more voltage; Y owns less voltage.

\hrule
\vspace{1em}


\noindent
\textbf{Q472.} Consider the four lines with the following equations.
Which two lines are perpendicular?
[IMAGE:0]
[IMAGE:1]
[IMAGE:2]
[IMAGE:3]



\textbf{A.} [IMAGE:0] \\
\textbf{B.} [IMAGE:1] \\
\textbf{C.} [IMAGE:2] \\
\textbf{D.} [IMAGE:3] \\

\textbf{Answer:} A \\
\textbf{Explanation:} The gradient is found respectively, if two lines are perpendicular to each other, the product of the gradient is -1.

\hrule
\vspace{1em}


\noindent
\textbf{Q473.} Consider the four lines with the following equations.
Which two lines are perpendicular?
[IMAGE:0]
[IMAGE:1]
[IMAGE:2]
[IMAGE:3]



\textbf{A.} [IMAGE:0] \\
\textbf{B.} [IMAGE:1] \\
\textbf{C.} [IMAGE:2] \\
\textbf{D.} [IMAGE:3] \\

\textbf{Answer:} F \\
\textbf{Explanation:} The gradient is found respectively, if two lines are perpendicular to each other, the product of the gradient is -1.

\hrule
\vspace{1em}


\noindent
\textbf{Q474.} Consider the four lines with the following equations.
Which two lines are perpendicular?
[IMAGE:0]
[IMAGE:1]
[IMAGE:2]
[IMAGE:3]



\textbf{A.} [IMAGE:0] \\
\textbf{B.} [IMAGE:1] \\
\textbf{C.} [IMAGE:2] \\
\textbf{D.} [IMAGE:3] \\

\textbf{Answer:} E \\
\textbf{Explanation:} The gradient is found respectively, if two lines are perpendicular to each other, the product of the gradient is -1.

\hrule
\vspace{1em}


\noindent
\textbf{Q475.} The densities of two metals P and Q are
[IMAGE:0]
and
[IMAGE:1]
respectively. What is the density of an alloy made from equal volumes of metals P and Q (with the total volume remaining unchanged)?



\textbf{A.} [IMAGE:0] \\
\textbf{B.} [IMAGE:1] \\
\textbf{C.} [IMAGE:2] \\
\textbf{D.} [IMAGE:3] \\

\textbf{Answer:} C \\
\textbf{Explanation:} [IMAGE:0]

\hrule
\vspace{1em}


\noindent
\textbf{Q476.} A fair spinner has eight equal sections. Each section has one number written on it, as shown.
The spinner is spun twice, and the two numbers scored are added. What is the probability that the absolute difference of the two numbers is 1?



\textbf{A.} [IMAGE:0] \\
\textbf{B.} [IMAGE:1] \\
\textbf{C.} [IMAGE:2] \\
\textbf{D.} [IMAGE:3] \\

\textbf{Answer:} C \\
\textbf{Explanation:} The possible solutions are: (1,2), (2,1), (2,3), (3,2), which is 3/64+3/64+4/64+4/64=14/64=7/32.
(pay attention to the word "absolute", and the disturbance term is option D)

\hrule
\vspace{1em}


\noindent
\textbf{Q477.} A fair spinner has eight equal sections. Each section has one number written on it, as shown.
The spinner is spun
[IMAGE:0]
times, and the N numbers scored are added. What is the probability that the sum of the N numbers is
[IMAGE:1]
?



\textbf{A.} [IMAGE:0] \\
\textbf{B.} [IMAGE:1] \\
\textbf{C.} [IMAGE:2] \\
\textbf{D.} [IMAGE:3] \\

\textbf{Answer:} C \\
\textbf{Explanation:} The possible solutions are: all numbers are 3, which is
[IMAGE:0]
.
Thus,
[IMAGE:1]
.

\hrule
\vspace{1em}


\noindent
\textbf{Q478.} The densities of two metals P and Q are
[IMAGE:0]
and
[IMAGE:1]
respectively. What is the density o
A
f an alloy made from equal masses of metals P and Q (with the total volume remaining unchanged)?



\textbf{A.} [IMAGE:0] \\
\textbf{B.} [IMAGE:1] \\
\textbf{C.} [IMAGE:2] \\
\textbf{D.} [IMAGE:3] \\

\textbf{Answer:} D \\
\textbf{Explanation:} [IMAGE:0]

\hrule
\vspace{1em}


\noindent
\textbf{Q479.} A fair spinner has eight equal sections. Each section has one number written on it, as shown.
The spinner is spun three times, and the three numbers scored are added. What is the probability that the product of the three numbers is 6?



\textbf{A.} [IMAGE:0] \\
\textbf{B.} [IMAGE:1] \\
\textbf{C.} 9/64
[IMAGE:2] \\
\textbf{D.} [IMAGE:3] \\

\textbf{Answer:} C \\
\textbf{Explanation:} The possible solutions are: (1,2,3) or (1,3,2) or (2,1,3) or (2,3,1) or (3,1,2) or (3,2,1), which is 12/512×6=72/512=9/64.

\hrule
\vspace{1em}


\noindent
\textbf{Q480.} A steel cylinder with a volume of 0.1m3 contains oxygen with a density of 8kg/m3. During welding, 4 of 1 of the oxygen is used. Then, 0.6kg of another gas with a density of 12kg/m3 is added. Assuming no leakage occurs, what is the density of the gas mixture in the cylinder?



\textbf{A.} [IMAGE:0] \\
\textbf{B.} [IMAGE:1] \\
\textbf{C.} [IMAGE:2] \\
\textbf{D.} [IMAGE:3] \\

\textbf{Answer:} D \\
\textbf{Explanation:} [IMAGE:0]

\hrule
\vspace{1em}


\noindent
\textbf{Q481.} A fair spinner has eight equal sections. Each section has one number written on it, as shown.
The spinner is spun
[IMAGE:0]
times, and the N numbers scored are added. What is the probability that the sum of the N numbers is 3N-1?



\textbf{A.} [IMAGE:0] \\
\textbf{B.} [IMAGE:1] \\
\textbf{C.} [IMAGE:2] \\
\textbf{D.} [IMAGE:3] \\

\textbf{Answer:} D \\
\textbf{Explanation:} The possible solutions are: N-1 numbers are 3 and one number is 2, which is
[IMAGE:0]
.
Thus,
[IMAGE:1]

\hrule
\vspace{1em}


\noindent
\textbf{Q482.} A container filled with water has a total mass of 450g. When a 200g small stone is placed into the container, water overflows. After removing the excess water and measuring again, the total mass becomes 550g. What is the density of the small stone?



\textbf{A.} [IMAGE:0] \\
\textbf{B.} [IMAGE:1] \\
\textbf{C.} [IMAGE:2] \\
\textbf{D.} [IMAGE:3] \\

\textbf{Answer:} C \\
\textbf{Explanation:} [IMAGE:0]

\hrule
\vspace{1em}


\noindent
\textbf{Q483.} A fair spinner has eight equal sections. Each section has one number written on it, as shown.
The spinner is spun
[IMAGE:0]
times, and the N numbers scored are added. What is the probability that the sum of the N numbers is 3N?



\textbf{A.} [IMAGE:0] \\
\textbf{B.} [IMAGE:1] \\
\textbf{C.} [IMAGE:2] \\
\textbf{D.} [IMAGE:3] \\

\textbf{Answer:} C \\
\textbf{Explanation:} The possible solutions are: all the numbers are 3, which is
[IMAGE:0]
.

\hrule
\vspace{1em}


\noindent
\textbf{Q484.} A
fair spinner has eight equal sections. Each section has one number written on it, as shown.
The spinner is spun three times, and the three numbers scored are added. What is the probability that the sum of the three numbers is 9?



\textbf{A.} [IMAGE:0] \\
\textbf{B.} [IMAGE:1] \\
\textbf{C.} [IMAGE:2] \\
\textbf{D.} [IMAGE:3] \\

\textbf{Answer:} A \\
\textbf{Explanation:} The possible solutions are: all the numbers are 3, which is 1/8.

\hrule
\vspace{1em}


\noindent
\textbf{Q485.} Two liquids P and Q can be mixed together in any proportion. The density of liquid P is
[IMAGE:0]
and the density of liquid Q is
[IMAGE:1]
. A volume
[IMAGE:2]
of liquid P and a volume
[IMAGE:3]
of liquid Q are mixed together to create a chemical reaction which makes total volume increase to
[IMAGE:4]
What is the density of the mixture?



\textbf{A.} [IMAGE:0] \\
\textbf{B.} [IMAGE:1] \\
\textbf{C.} [IMAGE:2] \\
\textbf{D.} [IMAGE:3] \\

\textbf{Answer:} E \\
\textbf{Explanation:} The density is total mass divided by total volume which is E. (pay attention to the difference of "increase by" and "increase to")

\hrule
\vspace{1em}


\noindent
\textbf{Q486.} A fair spinner has eight equal sections. Each section has one number written on it, as shown.
The spinner is spun twice, and the two numbers scored are added. What is the probability that the sum of the two numbers is 7?



\textbf{A.} [IMAGE:0] \\
\textbf{B.} [IMAGE:1] \\
\textbf{C.} [IMAGE:2] \\
\textbf{D.} [IMAGE:3] \\

\textbf{Answer:} A \\
\textbf{Explanation:} It is impossible to get the sum 7 of the two numbers, which means the probability is 0.

\hrule
\vspace{1em}


\noindent
\textbf{Q487.} A fair spinner has eight equal sections. Each section has one number written on it, as shown.
The spinner is spun twice, and the two numbers scored are added. What is the probability that the sum of the two numbers is 6?



\textbf{A.} [IMAGE:0] \\
\textbf{B.} [IMAGE:1] \\
\textbf{C.} [IMAGE:2] \\
\textbf{D.} [IMAGE:3] \\

\textbf{Answer:} B \\
\textbf{Explanation:} The possible solutions are: all the numbers are 3, which is 16/64=1/4.

\hrule
\vspace{1em}


\noindent
\textbf{Q488.} A fair spinner has eight equal sections. Each section has one number written on it, as shown.
The spinner is spun twice, and the two numbers scored are added. What is the probability that the sum of the two numbers is 4?



\textbf{A.} [IMAGE:0] \\
\textbf{B.} [IMAGE:1] \\
\textbf{C.} [IMAGE:2] \\
\textbf{D.} [IMAGE:3] \\

\textbf{Answer:} C \\
\textbf{Explanation:} The possible solutions are: 1+3 or 3+1 which is 12/64+12/64=24/64=3/8.

\hrule
\vspace{1em}


\noindent
\textbf{Q489.} Two liquids P and Q can be mixed together in any proportion. The density of liquid P is
[IMAGE:0]
and the density of liquid Q is
. A volume
[IMAGE:1]
of liquid P and a volume
[IMAGE:2]
of liquid Q are mixed together to create a chemical reaction which makes total volume increase by
[IMAGE:3]
. What is the density of the mixture?



\textbf{A.} [IMAGE:0] \\
\textbf{B.} [IMAGE:1] \\
\textbf{C.} [IMAGE:2] \\
\textbf{D.} [IMAGE:3] \\

\textbf{Answer:} F \\
\textbf{Explanation:} The density is total mass divided by total volume which is F. (pay attention to the difference of "increase by" and "increase to")

\hrule
\vspace{1em}


\noindent
\textbf{Q490.} A fair spinner has eight equal sections. Each section has one number written on it, as shown.
The spinner is spun twice, and the two numbers scored are added. What is the probability that the sum of the two numbers is 3?



\textbf{A.} [IMAGE:0] \\
\textbf{B.} [IMAGE:1] \\
\textbf{C.} [IMAGE:2] \\
\textbf{D.} [IMAGE:3] \\

\textbf{Answer:} B \\
\textbf{Explanation:} The possible solutions are: 2+1 or 1+2 which is 3/64+3/64=6/64=3/32.

\hrule
\vspace{1em}


\noindent
\textbf{Q491.} For two perpendicular lines, with gradients
[IMAGE:0]
and
[IMAGE:1]
, and x-intercepts
[IMAGE:2]
and
[IMAGE:3]
respectively, find the relationship between
[IMAGE:4]
and
[IMAGE:5]
if the line with gradient
[IMAGE:6]
passes through the point (0, 2) and the line with gradient
[IMAGE:7]
passes through the point (0, 1) :



\textbf{A.} [IMAGE:0] \\
\textbf{B.} [IMAGE:1] \\
\textbf{C.} [IMAGE:2] \\
\textbf{D.} [IMAGE:3] \\

\textbf{Answer:} B \\
\textbf{Explanation:} [IMAGE:0]
[IMAGE:1]
, so
[IMAGE:2]
[IMAGE:3]
, so
[IMAGE:4]
Thus,
[IMAGE:5]
[IMAGE:6]

\hrule
\vspace{1em}


\noindent
\textbf{Q492.} For two perpendicular lines, with gradients
[IMAGE:0]
and
[IMAGE:1]
, and x-intercepts
[IMAGE:2]
and
[IMAGE:3]
respectively, find the relationship between
[IMAGE:4]
and
[IMAGE:5]
if the lines intersect at the point (0,1):



\textbf{A.} [IMAGE:0] \\
\textbf{B.} [IMAGE:1] \\
\textbf{C.} [IMAGE:2] \\
\textbf{D.} [IMAGE:3] \\

\textbf{Answer:} D \\
\textbf{Explanation:} [IMAGE:0]
[IMAGE:1]
, so
[IMAGE:2]
[IMAGE:3]
, so
[IMAGE:4]
Thus,
[IMAGE:5]
[IMAGE:6]

\hrule
\vspace{1em}


\noindent
\textbf{Q493.} Two liquids P and Q can be mixed together in any proportion. The density of liquid P is
[IMAGE:0]
and the density of liquid Q is
[IMAGE:1]
. A volume
[IMAGE:2]
of liquid P and a volume
[IMAGE:3]
of liquid Q are mixed together to produce a volume that is equal to
[IMAGE:4]
. What is the density of the mixture?



\textbf{A.} [IMAGE:0] \\
\textbf{B.} [IMAGE:1] \\
\textbf{C.} [IMAGE:2] \\
\textbf{D.} [IMAGE:3] \\

\textbf{Answer:} D \\
\textbf{Explanation:} [IMAGE:0]

\hrule
\vspace{1em}


\noindent
\textbf{Q494.} If two lines are perpendicular, with gradients
[IMAGE:0]
and
[IMAGE:1]
, find
[IMAGE:2]
and
[IMAGE:3]
:



\textbf{A.} [IMAGE:0] \\
\textbf{B.} [IMAGE:1] \\
\textbf{C.} [IMAGE:2] \\
\textbf{D.} [IMAGE:3] \\

\textbf{Answer:} A \\
\textbf{Explanation:} For perpendicular lines,
[IMAGE:0]
If we simply it as
[IMAGE:1]
, it will be complex.
Actually, we can get
[IMAGE:2]
.
Thus,
[IMAGE:3]
,
[IMAGE:4]
.

\hrule
\vspace{1em}


\noindent
\textbf{Q495.} If two lines are perpendicular, with gradients
[IMAGE:0]
and
[IMAGE:1]
, find
[IMAGE:2]
and
[IMAGE:3]
:



\textbf{A.} [IMAGE:0] \\
\textbf{B.} [IMAGE:1] \\
\textbf{C.} [IMAGE:2] \\
\textbf{D.} [IMAGE:3] \\

\textbf{Answer:} D \\
\textbf{Explanation:} For perpendicular lines,
[IMAGE:0]
If we simply it as
[IMAGE:1]
, it will be complex.
Actually, we can get
[IMAGE:2]
.
Thus,
[IMAGE:3]
,
[IMAGE:4]
.

\hrule
\vspace{1em}


\noindent
\textbf{Q496.} If two lines are perpendicular, with gradients
[IMAGE:0]
and
[IMAGE:1]
, find
[IMAGE:2]
:



\textbf{A.} [IMAGE:0] \\
\textbf{B.} [IMAGE:1] \\
\textbf{C.} [IMAGE:2] \\
\textbf{D.} [IMAGE:3] \\

\textbf{Answer:} B \\
\textbf{Explanation:} For perpendicular lines,
[IMAGE:0]
, so
[IMAGE:1]
, hence
[IMAGE:2]
.

\hrule
\vspace{1em}


\noindent
\textbf{Q497.} If two lines are perpendicular, with gradients
[IMAGE:0]
and
[IMAGE:1]
, find
[IMAGE:2]
:



\textbf{A.} [IMAGE:0] \\
\textbf{B.} [IMAGE:1] \\
\textbf{C.} [IMAGE:2] \\
\textbf{D.} [IMAGE:3] \\

\textbf{Answer:} B \\
\textbf{Explanation:} For perpendicular lines,
[IMAGE:0]
, so
[IMAGE:1]
, hence
[IMAGE:2]
.

\hrule
\vspace{1em}


\noindent
\textbf{Q498.} Two liquids P and Q can be mixed together in any proportion. The density of liquid P is
[IMAGE:0]
and the density of liquid Q is
[IMAGE:1]
. A volume
[IMAGE:2]
of liquid P and a volume
[IMAGE:3]
of liquid Q are mixed together to produce a volume that is equal to
[IMAGE:4]
. What is the density of the mixture?



\textbf{A.} [IMAGE:0] \\
\textbf{B.} [IMAGE:1] \\
\textbf{C.} [IMAGE:2] \\
\textbf{D.} [IMAGE:3] \\

\textbf{Answer:} B \\
\textbf{Explanation:} The density is total mass divided by total volume which is B.

\hrule
\vspace{1em}


\noindent
\textbf{Q499.} For two lines that are perpendicular to each other, with gradient
[IMAGE:0]
and
[IMAGE:1]
.
[IMAGE:2]
and
[IMAGE:3]
are real number.
Find the correct relationship:



\textbf{A.} [IMAGE:0] \\
\textbf{B.} [IMAGE:1] \\
\textbf{C.} [IMAGE:2] \\
\textbf{D.} [IMAGE:3] \\

\textbf{Answer:} B \\
\textbf{Explanation:} When two lines are perpendicular to each other, the product of the gradient is -1.
[IMAGE:0]
But,
[IMAGE:1]
and
[IMAGE:2]
have no real number solutions for this case. (The option B is a disturbance term. )

\hrule
\vspace{1em}


\noindent
\textbf{Q500.} For two lines that are perpendicular to each other, with gradient
[IMAGE:0]
and
[IMAGE:1]
.
Find the correct relationship:



\textbf{A.} [IMAGE:0] \\
\textbf{B.} [IMAGE:1] \\
\textbf{C.} [IMAGE:2] \\
\textbf{D.} [IMAGE:3] \\

\textbf{Answer:} A \\
\textbf{Explanation:} When two lines are perpendicular to each other, the product of the gradient is -1.
[IMAGE:0]
Thus,
[IMAGE:1]
.

\hrule
\vspace{1em}


\noindent
\textbf{Q501.} For two lines that are perpendicular to each other, with gradient
[IMAGE:0]
and
[IMAGE:1]
.
Find the correct relationship:



\textbf{A.} [IMAGE:0] \\
\textbf{B.} [IMAGE:1] \\
\textbf{C.} [IMAGE:2] \\
\textbf{D.} [IMAGE:3] \\

\textbf{Answer:} A \\
\textbf{Explanation:} When two lines are perpendicular to each other, the product of the gradient is -1.
[IMAGE:0]
Thus,
[IMAGE:1]
.

\hrule
\vspace{1em}


\noindent
\textbf{Q502.} For two lines that are perpendicular to each other, with gradient
[IMAGE:0]
and
[IMAGE:1]
.
Find the correct relationship:



\textbf{A.} [IMAGE:0] \\
\textbf{B.} [IMAGE:1] \\
\textbf{C.} [IMAGE:2] \\
\textbf{D.} [IMAGE:3] \\

\textbf{Answer:} C \\
\textbf{Explanation:} When two lines are perpendicular to each other, the product of the gradient is -1.
[IMAGE:0]
[IMAGE:1]
.

\hrule
\vspace{1em}


\noindent
\textbf{Q503.} Two liquids P and Q can be mixed together in any proportion. The density of liquid P is
[IMAGE:0]
and the density of liquid Q is
[IMAGE:1]
. A volume
[IMAGE:2]
of liquid P and a volume
[IMAGE:3]
of liquid Q are mixed together to produce a volume that is equal to
[IMAGE:4]
. What is the density of the mixture?



\textbf{A.} [IMAGE:0] \\
\textbf{B.} [IMAGE:1] \\
\textbf{C.} [IMAGE:2] \\
\textbf{D.} [IMAGE:3] \\

\textbf{Answer:} B \\
\textbf{Explanation:} The density is total mass divided by total volume which is B.

\hrule
\vspace{1em}


\noindent
\textbf{Q504.} For two lines that are perpendicular to each other, with gradient
[IMAGE:0]
and
[IMAGE:1]
.
Find the correct relationship:



\textbf{A.} [IMAGE:0] \\
\textbf{B.} [IMAGE:1] \\
\textbf{C.} [IMAGE:2] \\
\textbf{D.} [IMAGE:3] \\

\textbf{Answer:} B \\
\textbf{Explanation:} When two lines are perpendicular to each other, the product of the gradient is -1.

\hrule
\vspace{1em}


\noindent
\textbf{Q505.} Two liquids P and Q can be mixed together in any proportion. The density of liquid P is
[IMAGE:0]
and the density of liquid Q is
[IMAGE:1]
. A volume
[IMAGE:2]
of liquid P and a volume
[IMAGE:3]
of liquid Q are mixed together to produce a volume that is equal to
[IMAGE:4]
. What is the density of the mixture?



\textbf{A.} [IMAGE:0] \\
\textbf{B.} [IMAGE:1] \\
\textbf{C.} [IMAGE:2] \\
\textbf{D.} [IMAGE:3] \\

\textbf{Answer:} C \\
\textbf{Explanation:} The density is total mass divided by total volume which is C.

\hrule
\vspace{1em}


\noindent
\textbf{Q506.} A ball decelerates uniformly from +28.0m/s
to +14.0m/s
in 0.001s
, then accelerates back to +20.0m/s
in 0.006s
.
What is the total displacement during contact?



\textbf{A.} 0.100m \\
\textbf{B.} 0.123m \\
\textbf{C.} 0.132m \\
\textbf{D.} 0.231m \\

\textbf{Answer:} B \\
\textbf{Explanation:} Stage 1:
[IMAGE:0]
. Stage 2:
[IMAGE:1]
. Total
[IMAGE:2]
.

\hrule
\vspace{1em}


\noindent
\textbf{Q507.} A ball (v=12.0m/s)
compresses a racket string by 0.04m
before rebounding at 8.0m/s.
What is the peak deceleration?



\textbf{A.} [IMAGE:0] \\
\textbf{B.} [IMAGE:1] \\
\textbf{C.} [IMAGE:2] \\
\textbf{D.} [IMAGE:3] \\

\textbf{Answer:} A \\
\textbf{Explanation:} Energy loss implies non-constant force, but assuming average deceleration:
[IMAGE:0]
.

\hrule
\vspace{1em}


\noindent
\textbf{Q508.} A topspin ball slows horizontally from 30.0m/s
to 24.0m/s
while gaining 6.0m/s
downward due to spin. Contact time is 0.00425s.
What is the net acceleration?



\textbf{A.} [IMAGE:0] \\
\textbf{B.} [IMAGE:1] \\
\textbf{C.} [IMAGE:2] \\
\textbf{D.} [IMAGE:3] \\

\textbf{Answer:} C \\
\textbf{Explanation:} [IMAGE:0]
,
[IMAGE:1]
. Net
[IMAGE:2]
.
[IMAGE:3]
.

\hrule
\vspace{1em}


\noindent
\textbf{Q509.} A tennis ball (
[IMAGE:0]
) penetrates a net, slowing to
[IMAGE:1]
over
[IMAGE:2]
.
What is the deceleration magnitude?



\textbf{A.} [IMAGE:0] \\
\textbf{B.} [IMAGE:1] \\
\textbf{C.} [IMAGE:2] \\
\textbf{D.} [IMAGE:3] \\

\textbf{Answer:} E \\
\textbf{Explanation:} [IMAGE:0]
.

\hrule
\vspace{1em}


\noindent
\textbf{Q510.} A tennis ball approaches at 29.0m/s
horizontally. The racket applies a constant upward force, adding a vertical velocity of 5.0m/s
while reducing horizontal speed to 17.0m/s
. The contact lasts 0.005s
.
What is the magnitude of the net acceleration?



\textbf{A.} [IMAGE:0] \\
\textbf{B.} [IMAGE:1] \\
\textbf{C.} [IMAGE:2] \\
\textbf{D.} [IMAGE:3] \\

\textbf{Answer:} A \\
\textbf{Explanation:} [IMAGE:0]
,
[IMAGE:1]
. Net
[IMAGE:2]
. Acceleration
[IMAGE:3]
.

\hrule
\vspace{1em}


\noindent
\textbf{Q511.} A tennis ball strikes the court surface at 18.0m/s
at a
[IMAGE:0]
angle. The bounce reverses the vertical velocity component and reduces the horizontal speed by 20%
. The contact time is 0.02s
.
What is the magnitude of the average acceleration during impact? (
[IMAGE:1]
)



\textbf{A.} [IMAGE:0] \\
\textbf{B.} [IMAGE:1] \\
\textbf{C.} [IMAGE:2] \\
\textbf{D.} [IMAGE:3] \\

\textbf{Answer:} B \\
\textbf{Explanation:} Vertical velocity changes from
[IMAGE:0]
to
[IMAGE:1]
(
[IMAGE:2]
), while horizontal velocity changes from
[IMAGE:3]
to
[IMAGE:4]
(
[IMAGE:5]
). Total
[IMAGE:6]
. Acceleration
[IMAGE:7]
.

\hrule
\vspace{1em}


\noindent
\textbf{Q512.} A tennis ball travelling at 24.0m/s
is hit by a racket. As a result of the impact, the ball returns back along its original path having undergone a change in velocity of 36.0m/s
. The acceleration of the ball whilst in contact with the racket is constant with magnitude
[IMAGE:0]
.
What is the total distance travelled by the ball whilst in contact with the racket?



\textbf{A.} 0.00cm \\
\textbf{B.} 4.80cm \\
\textbf{C.} 7.50cm \\
\textbf{D.} 15.0cm \\

\textbf{Answer:} C \\
\textbf{Explanation:} the initial velocity is
[IMAGE:0]
; the terminal one is
[IMAGE:1]
, the process is not symmetrical in time;
the first half is
[IMAGE:2]
.
the second half is
[IMAGE:3]
.
Thus, the answer is 7.5cm.

\hrule
\vspace{1em}


\noindent
\textbf{Q513.} A tennis ball travelling at
[IMAGE:0]
is hit by a racket. As a result of the impact, the ball returns back along its original path having undergone a change in velocity of
[IMAGE:1]
. The acceleration of the ball whilst in contact with the racket is constant with magnitude
[IMAGE:2]
.
What is the total distance travelled by the ball whilst in contact with the racket?



\textbf{A.} [IMAGE:0] \\
\textbf{B.} [IMAGE:1] \\
\textbf{C.} [IMAGE:2] \\
\textbf{D.} [IMAGE:3] \\

\textbf{Answer:} D \\
\textbf{Explanation:} the initial velocity is
[IMAGE:0]
; the terminal one is
[IMAGE:1]
, the process is not symmetrical in time; the first half is
[IMAGE:2]
; the second half is
[IMAGE:3]
. Thus, the answer is
[IMAGE:4]
.

\hrule
\vspace{1em}


\noindent
\textbf{Q514.} A tennis ball travelling at 10.0m/s
is hit by a racket. As a result of the impact, the ball returns back along its original path having undergone a change in velocity of 20.0m/s
. The acceleration of the ball whilst in contact with the racket is constant with magnitude
[IMAGE:0]
.
What is the total distance travelled by the ball whilst in contact with the racket?



\textbf{A.} 2.00cm \\
\textbf{B.} 2.50cm \\
\textbf{C.} 3.00cm \\
\textbf{D.} 14.4cm \\

\textbf{Answer:} B \\
\textbf{Explanation:} the initial velocity is 10.0m/s
; the terminal one is -10.0m/s
, the process is symmetrical in time; the first half is
[IMAGE:0]
. Thus, the answer is 2.5cm
.

\hrule
\vspace{1em}


\noindent
\textbf{Q515.} A tennis ball travelling at 11.0m/s
is hit by a racket. As a result of the impact, the ball returns back along its original path having undergone a change in velocity of 22.0m/s
. The acceleration of the ball whilst in contact with the racket is constant with magnitude
[IMAGE:0]
.
What is the total distance travelled by the ball whilst in contact with the racket?



\textbf{A.} 2.00cm \\
\textbf{B.} 4.00cm \\
\textbf{C.} 9.00cm \\
\textbf{D.} 14.4cm \\

\textbf{Answer:} A \\
\textbf{Explanation:} the initial velocity is 11.0m/s
; the terminal one is -11.0m/s
, the process is symmetrical in time; the first half is 11.0*11.0/6000/2=0.01m=1.00cm
. Thus, the answer is 2.00cm
.

\hrule
\vspace{1em}


\noindent
\textbf{Q516.} Liquid fuel causes effective mass to oscillate as m(t) = m₀ - kt + 0.001εcos(ωt) with constant thrust.
Acceleration behavior?



\textbf{A.} [IMAGE:0] \\
\textbf{B.} [IMAGE:1] \\
\textbf{C.} [IMAGE:2] \\
\textbf{D.} [IMAGE:3] \\

\textbf{Answer:} G \\
\textbf{Explanation:} a(t) = F/[m(t)] shows oscillation can affect the acceleration which depends on the parameter settings of the mass equation, even the 0.001 is a small number.

\hrule
\vspace{1em}


\noindent
\textbf{Q517.} Rocket's exhaust velocity decreases linearly with time (
[IMAGE:0]
,
[IMAGE:1]
) while mass flow rate is constant as
[IMAGE:2]
. The initial mass of rocket is
[IMAGE:3]
.
[IMAGE:4]
is a constant.
Check the acceleration behavior?



\textbf{A.} [IMAGE:0] \\
\textbf{B.} [IMAGE:1] \\
\textbf{C.} [IMAGE:2] \\
\textbf{D.} [IMAGE:3] \\

\textbf{Answer:} F \\
\textbf{Explanation:} Thrust formula:
[IMAGE:0]
(momentum theorem)
Given conditions:
[IMAGE:1]
(constant mass flow rate)
[IMAGE:2]
(exhaust velocity increases linearly with time)
Therefore, thrust:
[IMAGE:3]
(linear increase)
Rocket mass:
[IMAGE:4]
(linear decrease)
Acceleration:
[IMAGE:5]
When
[IMAGE:6]
, the acceleration exhibits precisely linear growth.

\hrule
\vspace{1em}


\noindent
\textbf{Q518.} A rokect with mass
[IMAGE:0]
. For first half of fuel (total mass is
[IMAGE:1]
): burns at rate R with thrust F. For second half: burns at 2R with thrust 1.8F.
What describes acceleration?



\textbf{A.} [IMAGE:0] \\
\textbf{B.} [IMAGE:1] \\
\textbf{C.} [IMAGE:2] \\
\textbf{D.} [IMAGE:3] \\

\textbf{Answer:} B \\
\textbf{Explanation:} First phase:
[IMAGE:0]
with final value
[IMAGE:1]
Second phase:
[IMAGE:2]
, whose numerator is suddently increasing and denominator is decreasing faster.

\hrule
\vspace{1em}


\noindent
\textbf{Q519.} A rocket (mass is
[IMAGE:0]
) adjusts its thrust to always equal 10% of its instantaneous fuel mass (F = 0.2M). Fuel burns at constant rate(
[IMAGE:1]
, where
[IMAGE:2]
is a constant, and the inital mass of fuel is
[IMAGE:3]
).
How does acceleration behave?



\textbf{A.} Constant at 0.1
[IMAGE:0] \\
\textbf{B.} Increases \\
\textbf{C.} Decreases \\
\textbf{D.} Proportional to 1/m \\

\textbf{Answer:} C \\
\textbf{Explanation:} Based on acceleration-force equation, it has
initial state:
[IMAGE:0]
operating state:
[IMAGE:1]
it can be proved that
[IMAGE:2]

\hrule
\vspace{1em}


\noindent
\textbf{Q520.} A rocket adjusts its thrust to always equal 20% of its instantaneous total mass (F = 0.2m). Fuel burns at constant rate.
How does acceleration behave?



\textbf{A.} Constant at 0.1
[IMAGE:0] \\
\textbf{B.} lncreases linearly \\
\textbf{C.} Decreases exponentially \\
\textbf{D.} Constant at 0.2/m \\

\textbf{Answer:} D \\
\textbf{Explanation:} a = F/m = 0.2m/m = 0.2 m/s² always.

\hrule
\vspace{1em}


\noindent
\textbf{Q521.} A rocket travelling in space is burning its fuel at a increasing rate
[IMAGE:0]
, where
[IMAGE:1]
is inital rate of burning fuel,
[IMAGE:2]
denotes time and
[IMAGE:3]
is a constant. By expelling the burnt fuel through a nozzle, the engine is applying a constant force to the rocket.
What is happening to the magnitude of the velocity of the rocket?



\textbf{A.} [IMAGE:0] \\
\textbf{B.} [IMAGE:1] \\
\textbf{C.} [IMAGE:2] \\
\textbf{D.} [IMAGE:3] \\

\textbf{Answer:} B,E \\
\textbf{Explanation:} The purposive force is a constant; the mass is decreasing.
Thus, the acceleration is therefore increasing;
the jerk(rate of change of acceleration) is
[IMAGE:0]
, which means the acceleration is increasing at an increasing rate in time.

\hrule
\vspace{1em}


\noindent
\textbf{Q522.} A rocket travelling in space is burning its fuel at a decreasing rate
[IMAGE:0]
, where
[IMAGE:1]
is inital rate of burning fuel,
[IMAGE:2]
denotes time and
[IMAGE:3]
is a constant. By expelling the burnt fuel through a nozzle, the engine is applying a constant force to the rocket.
What is happening to the magnitude of the velocity of the rocket?



\textbf{A.} [IMAGE:0] \\
\textbf{B.} [IMAGE:1] \\
\textbf{C.} [IMAGE:2] \\
\textbf{D.} [IMAGE:3] \\

\textbf{Answer:} B \\
\textbf{Explanation:} The purposive force is a constant; the mass is decreasing.
PS: Though the fuel consumption is at a decreasing rate, the mass is still decreasing.
Thus, the acceleration is therefore increasing;
the jerk(rate of change of acceleration) is
[IMAGE:0]
, which means the acceleration increasing is at a contant rate in time.

\hrule
\vspace{1em}


\noindent
\textbf{Q523.} A rocket travelling in space is burning its fuel at a constant rate. By expelling the burnt fuel through a nozzle, the engine is applying a constant force to the rocket.
What is happening to the magnitude of the distance of the rocket?



\textbf{A.} It is dncreasing at an increasing rate. \\
\textbf{B.} It is dncreasing at a constant rate. \\
\textbf{C.} It is dncreasing at a decreasing rate. \\
\textbf{D.} It is not changing. \\

\textbf{Answer:} E \\
\textbf{Explanation:} The purposive force is a constant; the mass is decreasing; the acceleration is therefore increasing; so the velocity is creasing at an increasing rate and the distance is creasing at an increasing rate.

\hrule
\vspace{1em}


\noindent
\textbf{Q524.} A 1500 kg car is driving on a horizontal curve with a radius of 50 m. The coefficient of kinetic friction between the tires and the dry road is 0.
15
. When the road is wet, the coefficient of kinetic friction decreases to 0.
1
. What is the maximum safe speed for the car to navigate the curve on a dry road? (Gravitational field strength g=10N kg−1)



\textbf{A.} 27.8
km/h \\
\textbf{B.} 2
5
.
45
km/h \\
\textbf{C.} 3
1
.
18
km/h \\
\textbf{D.} 40
.78
km/h \\

\textbf{Answer:} C \\
\textbf{Explanation:} [IMAGE:0]

\hrule
\vspace{1em}


\noindent
\textbf{Q525.} A rocket travelling in space is burning its fuel at a constant rate. By expelling the burnt fuel through a nozzle, the engine is applying a constant force to the rocket.
What is happening to the magnitude of the velocity of the rocket?



\textbf{A.} [IMAGE:0] \\
\textbf{B.} [IMAGE:1] \\
\textbf{C.} [IMAGE:2] \\
\textbf{D.} [IMAGE:3] \\

\textbf{Answer:} E \\
\textbf{Explanation:} The purposive force is a constant; the mass is decreasing; the acceleration is therefore increasing; so the velocity is creasing at an increasing rate.

\hrule
\vspace{1em}


\noindent
\textbf{Q526.} A 2.0 kg object is at rest on a horizontal surface with a coefficient of friction 0.
15
. A force of 20 N is applied at an angle of 37° above the horizontal (sin37°=0.6, cos37°=0.8). What is the magnitude of the object’s acceleration as it starts moving? (Take g=10 m/s2g=10m/s2.)



\textbf{A.} 5.0 m/s² \\
\textbf{B.} 6.0 m/s² \\
\textbf{C.} 7.
4
m/s² \\
\textbf{D.} 8.0 m/s² \\

\textbf{Answer:} C \\
\textbf{Explanation:} [IMAGE:0]

\hrule
\vspace{1em}


\noindent
\textbf{Q527.} A 5.0 kg object is placed on a conveyor belt inclined at 30° to the horizontal. The conveyor belt accelerates upward at 2.0 m/s². The coefficient of kinetic friction between the object and the belt is 0.
15
. What is the magnitude of the object's acceleration relative to the conveyor belt? (Gravitational field strength g=10N kg−1, sin30°=0.5, cos30°=3​/2≈0.866)



\textbf{A.} 0.5
6
m/s² \\
\textbf{B.} 1.0
2
m/s² \\
\textbf{C.} 4.44
m/s² \\
\textbf{D.} 2.0
5
m/s² \\

\textbf{Answer:} C \\
\textbf{Explanation:} [IMAGE:0]

\hrule
\vspace{1em}


\noindent
\textbf{Q528.} A future vehicle of mass 500 kg travels in a straight line along a horizontal road, as shown in the acceleration/deceleration–time graph.
What is the average resultant force acting on the vehicle over the time for which it is accelerating / decelerating?



\textbf{A.} [IMAGE:0] \\
\textbf{B.} [IMAGE:1] \\
\textbf{C.} [IMAGE:2] \\
\textbf{D.} [IMAGE:3] \\

\textbf{Answer:} E \\
\textbf{Explanation:} What is the average resultant force acting on the vehicle over the time for which it is accelerating?
The vehicle is accelerating from 0s to 10s (all the time because acceleration is bigger than zero, which may be a trap).
0s~5s:
[IMAGE:0]
5s~10s:
[IMAGE:1]
Thus, the acceleration in average is therefore
[IMAGE:2]
;
[IMAGE:3]
. (D option is a disturbance term)

\hrule
\vspace{1em}


\noindent
\textbf{Q529.} A
8
.0 kg object is placed on a horizontal rotating disk. The maximum static frictional force between the object and the disk is
8
.0 N. The disk starts rotating around its central axis with gradually increasing angular velocity. What is the angular velocity of the disk when the object is about to slip? (Gravitational field strength g=10N kg−1)



\textbf{A.} 2.
45
rad/s \\
\textbf{B.} 3.0
5
rad/s \\
\textbf{C.} 2
.
0
0 rad/s \\
\textbf{D.} 5.
3
0 rad/s \\

\textbf{Answer:} F \\
\textbf{Explanation:} Maximum Static Friction Provides Centripetal Force:
When the object is about to slip, the maximum static friction provides the necessary centripetal force. Centripetal force equation:
Given
[IMAGE:0]
,
[IMAGE:1]
, and assume r=1.0m.
[IMAGE:2]

\hrule
\vspace{1em}


\noindent
\textbf{Q530.} A car of mass 700 kg travels in a straight line along a horizontal road, as shown in the deceleration–time graph.
What is the average resultant force acting on the car over the time for which it is decelerating?
[IMAGE:0]



\textbf{A.} [IMAGE:0] \\
\textbf{B.} [IMAGE:1] \\
\textbf{C.} [IMAGE:2] \\
\textbf{D.} [IMAGE:3] \\

\textbf{Answer:} C \\
\textbf{Explanation:} By the gradient of the graph in the linear region; the velocity of 5s is
[IMAGE:0]
; the acceleration in average is therefore
[IMAGE:1]
;
[IMAGE:2]
.

\hrule
\vspace{1em}


\noindent
\textbf{Q531.} A 4.0 kg object is at rest on a horizontal surface with a coefficient of friction 0.
6
. Two forces are applied: a horizontal force of 24 N to the right and a vertical upward force of 8 N. What is the magnitude of the object’s acceleration as it starts moving? (Take g=10 m/s
2.
)



\textbf{A.} 2.0 m/s² \\
\textbf{B.} 3.0 m/s² \\
\textbf{C.} 1.2
m/s² \\
\textbf{D.} 5.0 m/s² \\

\textbf{Answer:} C \\
\textbf{Explanation:} Vertical Force Analysis: The vertical upward force reduces the normal force.
Normal force N=mg−Fvertical=4×10−8=32 .
Friction Calculation: Friction f=μN=0.
6
×32=
19.2
N.
Net horizontal force Fnet=Fhorizontal−f=24−
19.2
=
4.8
.
Acceleration Calculation: By Newton’s second law, a=Fnetm=
4.8/
4=
1.2
m/s
2
.

\hrule
\vspace{1em}


\noindent
\textbf{Q532.} A future vehicle of mass 500 kg travels in a straight line along a horizontal road, as shown in the acceleration–time graph.
What is the final kinetic energy of the vehicle?



\textbf{A.} [IMAGE:0] \\
\textbf{B.} [IMAGE:1] \\
\textbf{C.} [IMAGE:2] \\
\textbf{D.} [IMAGE:3] \\

\textbf{Answer:} F \\
\textbf{Explanation:} T
he vehicle is accelerating from 0s to 10s (all the time because acceleration is not zero).
0s~5s:
[IMAGE:0]
5s~10s:
[IMAGE:1]
Thus,
[IMAGE:2]

\hrule
\vspace{1em}


\noindent
\textbf{Q533.} A 5.0 kg object is at rest on an inclined plane with an angle of 30°. The coefficient of kinetic friction between the object and the plane is 0.
15
. A force of 10.0 N parallel to the plane upward and a force of 8.0 N perpendicular to the plane downward are applied simultaneously. What is the magnitude of the object's acceleration as it begins to move? (Gravitational field strength g=10N kg−1, sin30°=0.5, cos30°=3​/2≈0.866)



\textbf{A.} 1
0.5 m/s² \\
\textbf{B.} 5
.0
5
m/s² \\
\textbf{C.} 1.
2
5 m/s² \\
\textbf{D.} 2.0
5
m/s² \\

\textbf{Answer:} F \\
\textbf{Explanation:} [IMAGE:0]

\hrule
\vspace{1em}


\noindent
\textbf{Q534.} A 2.0 kg object rests on a horizontal surface with a coefficient of friction 0.
15
. Two forces are applied: a horizontal force of 20 N and a vertical upward force of 4 N. What is the magnitude of the acceleration? (Take g=10 m/s
2
.)



\textbf{A.} 6
.0 m/s² \\
\textbf{B.} 7.25 m/s² \\
\textbf{C.} 7.5 m/s² \\
\textbf{D.} 8.
8
m/s² \\

\textbf{Answer:} D \\
\textbf{Explanation:} Vertical Force: Reduces normal force.
N=2×10−4=16 N.
Friction: f=0.
15
×16=
2.4
N.
Horizontal Net Force: 20−
2.4
=
17.6
N.
Acceleration: a=
17.6/
2=
8.8
m/s
2
.

\hrule
\vspace{1em}


\noindent
\textbf{Q535.} A future vehicle of mass 600 kg travels in a straight line along a horizontal road, as shown in the acceleration–time graph.
What is the final velocity of the vehicle?



\textbf{A.} [IMAGE:0] \\
\textbf{B.} [IMAGE:1] \\
\textbf{C.} [IMAGE:2] \\
\textbf{D.} [IMAGE:3] \\

\textbf{Answer:} D \\
\textbf{Explanation:} The vehicle is accelerating from 0s to 10s (all the time because acceleration is not zero).
0s~5s:
[IMAGE:0]
5s~10s:
[IMAGE:1]

\hrule
\vspace{1em}


\noindent
\textbf{Q536.} A 4.0 kg object is at rest on a horizontal surface. The coefficient of kinetic friction between the object and the surface is 0.
15
. A vertical upward force of 8.0 N and three horizontal forces (6.0 N, 8.0 N, and 10.0 N, all mutually perpendicular) are applied simultaneously. What is the magnitude of the object's acceleration as it begins to move? (Gravitational field strength g=10N kg
−1
)



\textbf{A.} 1.0
0
m/s² \\
\textbf{B.} 1.5
0
m/s² \\
\textbf{C.} 2.
34
m/s² \\
\textbf{D.} 1
.
94
m/s² \\

\textbf{Answer:} C \\
\textbf{Explanation:} [IMAGE:0]

\hrule
\vspace{1em}


\noindent
\textbf{Q537.} A future vehicle of mass 400 kg travels in a straight line along a horizontal road, as shown in the acceleration–time graph.
What is the average resultant force acting on the vehicle over the time for which it is accelerating?



\textbf{A.} 380N \\
\textbf{B.} 420N \\
\textbf{C.} 550N \\
\textbf{D.} 5090N \\

\textbf{Answer:} E \\
\textbf{Explanation:} The vehicle is accelerating from 0s to 10s (all the time because acceleration is not zero).
0s~5s:
[IMAGE:0]
5s~10s:
[IMAGE:1]
Thus, the acceleration in average is therefore
[IMAGE:2]
;
[IMAGE:3]
.

\hrule
\vspace{1em}


\noindent
\textbf{Q538.} A point object of mass 2
.0 kg is at rest on a horizontal surface with a coefficient of friction 0.
1
. Two perpendicular forces are applied simultaneously: a horizontal force of 20 N and a vertical upward force of 4 N. What is the magnitude of the acceleration of the object as it begins to move? (Take g=10 m/s
2
)



\textbf{A.} 5.0 m/s² \\
\textbf{B.} 7.25 m/s² \\
\textbf{C.} 8
.5 m/s² \\
\textbf{D.} 9.7
m/s² \\

\textbf{Answer:} D \\
\textbf{Explanation:} [IMAGE:0]

\hrule
\vspace{1em}


\noindent
\textbf{Q539.} A car of mass 700 kg travels in a straight line along a horizontal road, as shown in the speed–time graph.
What is the average resultant force acting on the car over the time for which it is accelerating?



\textbf{A.} 380N \\
\textbf{B.} 420N \\
\textbf{C.} 550N \\
\textbf{D.} 1290N \\

\textbf{Answer:} E \\
\textbf{Explanation:} The terminal velocity is
[IMAGE:0]
; the acceleration in average is therefore
[IMAGE:1]
;
[IMAGE:2]
.

\hrule
\vspace{1em}


\noindent
\textbf{Q540.} .
A 3.0 kg object is at rest on a horizontal surface. The coefficient of kinetic friction between the object and the surface is 0.
005
. A vertical upward force of 10.0 N and two perpendicular horizontal forces (5.0 N and 12.0 N) are applied simultaneously. What is the magnitude of the object's acceleration as it begins to move? (Gravitational field strength g=10N kg−1)



\textbf{A.} 2.0 m/s² \\
\textbf{B.} 2.5 m/s² \\
\textbf{C.} 3.0 m/s² \\
\textbf{D.} 3.5 m/s² \\

\textbf{Answer:} F \\
\textbf{Explanation:} 1.
Vertical Force Analysis:
The upward vertical force reduces the normal reaction force.
Normal force N=mg−Fvertical​=3×10−10=20N.
2.
Frictional Force Calculation:
Kinetic friction f=μN=0.
005
×20=
0.1
N.
3.
Resultant Horizontal Force:
The vector sum of the two perpendicular horizontal forces is:
[IMAGE:0]

\hrule
\vspace{1em}


\noindent
\textbf{Q541.} A car of mass 820 kg travels in a straight line along a horizontal road, as shown in the distance–time graph.
What is the average resultant force acting on the car over the time for which it is accelerating?



\textbf{A.} 380N \\
\textbf{B.} 1148N \\
\textbf{C.} 1150N \\
\textbf{D.} 1190N \\

\textbf{Answer:} B \\
\textbf{Explanation:} By the gradient of the graph in the linear region; the terminal velocity is
[IMAGE:0]
; the acceleration in average is therefore
[IMAGE:1]
;
[IMAGE:2]
.

\hrule
\vspace{1em}


\noindent
\textbf{Q542.} An object of weight
6
0 N hangs from the end of a light inextensible string of length 0.5 m, which is attached to the ceiling. A force of
6
0 N is applied to the object at an angle of 45° above the horizontal, causing it to move to a new equilibrium position. By how much has the gravitational potential energy of the object increased?



\textbf{A.} 2.40J \\
\textbf{B.} 1.80J \\
\textbf{C.} 3.60J \\
\textbf{D.} 4.20J \\

\textbf{Answer:} C \\
\textbf{Explanation:} [IMAGE:0]

\hrule
\vspace{1em}


\noindent
\textbf{Q543.} A car of mass 380 kg travels in a straight line along a horizontal road.
The car accelerates non-uniformly from rest for 5.0 seconds and then moves at constant speed, as shown in the distance–time graph.
What is the average resultant force acting on the car over the time for which it is accelerating
?



\textbf{A.} 380N \\
\textbf{B.} 420N \\
\textbf{C.} 500N \\
\textbf{D.} 1200N \\

\textbf{Answer:} A \\
\textbf{Explanation:} By the gradient of the graph in the linear region; the terminal velocity is
[IMAGE:0]
; the acceleration in average is therefore
[IMAGE:1]
;
[IMAGE:2]
.

\hrule
\vspace{1em}


\noindent
\textbf{Q544.} A
1
0
0
N object slides down from a smooth inclined plane of height 0.4 m and then travels 1 m on a stationary conveyor belt. The coefficient of kinetic friction between the object and the belt is 0.1. Finally, the object hits a spring with a spring constant of 200 N/m. Ignoring other frictional forces, what is the kinetic energy of the object when it hits the spring?



\textbf{A.} 30.8J \\
\textbf{B.} 4.32J \\
\textbf{C.} 6.75J \\
\textbf{D.} 9.25J \\

\textbf{Answer:} A \\
\textbf{Explanation:} [IMAGE:0]

\hrule
\vspace{1em}


\noindent
\textbf{Q545.} A 20 N object is placed on a horizontal conveyor belt moving at a constant speed of
6
m/s to the right. The coefficient of kinetic friction between the object and the belt is 0.
1
. When the object accelerates until its speed matches the belt's speed, by how much has the kinetic energy of the object increased?



\textbf{A.} 2.0J \\
\textbf{B.} 4.0J \\
\textbf{C.} 9.2J \\
\textbf{D.} 8.0J \\

\textbf{Answer:} E \\
\textbf{Explanation:} [IMAGE:0]

\hrule
\vspace{1em}


\noindent
\textbf{Q546.} A car of mass 710 kg travels in a straight line along a horizontal road.
The car accelerates non-uniformly from rest for 5.0 seconds and then moves at constant speed, as shown in the distance–time graph.
What is the average resultant force acting on the car over the time for which it is accelerating?



\textbf{A.} 380N \\
\textbf{B.} 420N \\
\textbf{C.} 426N \\
\textbf{D.} 1200N \\

\textbf{Answer:} C \\
\textbf{Explanation:} By the gradient of the graph in the linear region; the terminal velocity is
[IMAGE:0]
; the acceleration in average is therefore
[IMAGE:1]
;
[IMAGE:2]
.

\hrule
\vspace{1em}


\noindent
\textbf{Q547.} A
12
0 N object is thrown vertically upward with an initial velocity of 2 m/s. Ignoring air resistance, by how much has the gravitational potential energy of the object increased during its ascent?



\textbf{A.} 2.4J \\
\textbf{B.} 2.8J \\
\textbf{C.} 3.5J \\
\textbf{D.} 4.2J \\

\textbf{Answer:} A \\
\textbf{Explanation:} [IMAGE:0]

\hrule
\vspace{1em}


\noindent
\textbf{Q548.} A
3
0 N object is placed on a smooth inclined plane with an angle of 30°, connected via a light inextensible string over a pulley to a
2
0 N hanging object. The system starts from rest and moves until the hanging object
d
escend
s 1
m. Ignoring pulley friction and air resistance, by how much has the gravitational potential energy of the hanging object increased?



\textbf{A.} 40J \\
\textbf{B.} 15J \\
\textbf{C.} 20J \\
\textbf{D.} 25J \\

\textbf{Answer:} C \\
\textbf{Explanation:} The system consists of an object on an inclined plane and a hanging object connected by a string over a pulley. As the system moves, the hanging object rises while the object on the plane slides down. By analyzing the energy changes, the increase in gravitational potential energy of the hanging object can be determined.
[IMAGE:0]

\hrule
\vspace{1em}


\noindent
\textbf{Q549.} A car of mass 500 kg travels in a straight line along a horizontal road.
The car accelerates non-uniformly from rest for 5.0 seconds and then moves at constant speed, as shown in the distance–time graph.
What is the average resultant force acting on the car over the time for which it is accelerating?



\textbf{A.} 320N \\
\textbf{B.} 480N \\
\textbf{C.} 1000N \\
\textbf{D.} 1200N \\

\textbf{Answer:} C \\
\textbf{Explanation:} By the gradient of the graph in the linear region; the terminal velocity is
[IMAGE:0]
; the acceleration in average is therefore
[IMAGE:1]
;
[IMAGE:2]
.

\hrule
\vspace{1em}


\noindent
\textbf{Q550.} During the soldering process, precise control of the soldering iron tip temperature is crucial. To maintain a constant tip temperature, a soldering iron is equipped with a temperature sensor and feedback control system. Assuming the tip (mass 2.0 g, copper) needs to be kept at 250°C while the ambient temperature is 20°C, and the heat exchange rate between the tip and the environment is 0.2 W/°C (i.e., the tip loses 0.5 W of heat for every 1°C above the ambient temperature)
Calculate the thermal power required from the soldering iron to maintain the tip at 250°C.



\textbf{A.} 10W \\
\textbf{B.} 11.5W \\
\textbf{C.} 23W \\
\textbf{D.} 46W \\

\textbf{Answer:} D \\
\textbf{Explanation:} Temperature difference:
[IMAGE:0]
Heat loss power:
[IMAGE:1]
Thermal power required is equal to the hear loss power, which is
[IMAGE:2]
.

\hrule
\vspace{1em}


\noindent
\textbf{Q551.} An object of weight
100
N hangs from the end of a light inextensible string of length
1
m, which is attached to the ceiling. A constant horizontal wind force of
80
N blows on the object, causing it to move to a new equilibrium position. By how much has the gravitational potential energy of the object increased as a result of its change of position?



\textbf{A.} 21
.5
J \\
\textbf{B.} 2.8J \\
\textbf{C.} 3.5
0
J \\
\textbf{D.} 4.2J \\

\textbf{Answer:} A \\
\textbf{Explanation:} [IMAGE:0]

\hrule
\vspace{1em}


\noindent
\textbf{Q552.} A soldering iron needs to adjust the tip temperature for welding different materials. To quickly reach the required welding temperature, a designer develops a new tip material whose specific heat capacity varies with temperature (relationship:
[IMAGE:0]
, where t is the temperature in °C). When the soldering iron heats this new tip (mass 1.0 g) with a thermal power of 30 W, the tip's temperature rises from 20°C to 200°C in 30 s.
Calculate the heat transferred to the surrounding environment during this process.



\textbf{A.} [IMAGE:0] \\
\textbf{B.} [IMAGE:1] \\
\textbf{C.} [IMAGE:2] \\
\textbf{D.} [IMAGE:3] \\

\textbf{Answer:} C \\
\textbf{Explanation:} Heat provided by the soldering iron:
[IMAGE:0]
Heat absorbed by the tip (using average specific heat capacity because of the linearity of
[IMAGE:1]
equation):
Average specific heat capacity:
[IMAGE:2]
Heat absorbed:
[IMAGE:3]
Heat transferred to the environment:
[IMAGE:4]

\hrule
\vspace{1em}


\noindent
\textbf{Q553.} An object weighing
10
0 N hangs from a light inextensible string of length 0.
8
m attached to the ceiling. A constant force of
70
N is applied to the object at an angle of 37° above the horizontal, moving it to a new equilibrium position. By how much has the gravitational potential energy of the object increased?



\textbf{A.} 2.8 J \\
\textbf{B.} 3.5 J \\
\textbf{C.} 4.2 J \\
\textbf{D.} 5.6 J
A \\

\textbf{Answer:} F \\
\textbf{Explanation:} [IMAGE:0]

\hrule
\vspace{1em}


\noindent
\textbf{Q554.} During continuous operation, the copper tip of a soldering iron (mass 1.8 g, specific heat capacity
[IMAGE:0]
) experiences process of repeated heating and cooling. In a specific soldering task, the tip is first heated to 300°C from the room temperature(20°C), then rapidly cooled to 50°C, then heated again to 250°C, and finally cooled to room temperature (20°C). Assuming each heating and cooling process is linear and takes 10 s. The thermal power is 50 W when it comes to rising up the temperature of the copper tip.
Calculate the total heat transferred to the surrounding environment during the entire process.



\textbf{A.} 320J \\
\textbf{B.} 480J \\
\textbf{C.} 640J \\
\textbf{D.} 1200J \\

\textbf{Answer:} F \\
\textbf{Explanation:} Heat absorbed during first heating:
[IMAGE:0]
Heat lost during first cooling:
[IMAGE:1]
Heat absorbed during second heating:
[IMAGE:2]
Heat lost during second cooling:
[IMAGE:3]
Total heat transferred to the environment:
[IMAGE:4]
.
PS: If write down the
[IMAGE:5]
equation at the very beginning, the calculation process will become easier.

\hrule
\vspace{1em}


\noindent
\textbf{Q555.} An object weighing
15
0 N hangs from the end of a light inextensible string of length 0.5 m, which is attached to the ceiling. A constant horizontal force is applied to the object, causing it to move to a new equilibrium position where the string makes an angle of 30° with the vertical. By how much has the gravitational potential energy of the object increased?



\textbf{A.} 3.0 J \\
\textbf{B.} 4.
02
J \\
\textbf{C.} 6.0
1
J \\
\textbf{D.} 7.5
3
J \\

\textbf{Answer:} F \\
\textbf{Explanation:} [IMAGE:0]

\hrule
\vspace{1em}


\noindent
\textbf{Q556.} To improve soldering efficiency, a designer coats the surface of a copper tip (mass 2.0 g, specific heat capacity
[IMAGE:0]
) with a special material that significantly enhances the tip's thermal conductivity but also increases its heat capacity by 20%. When the soldering iron heats the tip with a thermal power of 50 W, the tip's temperature rises by 220°C in 8 s.
Calculate the heat lost to the environment. after coating.



\textbf{A.} [IMAGE:0] \\
\textbf{B.} [IMAGE:1] \\
\textbf{C.} [IMAGE:2] \\
\textbf{D.} [IMAGE:3] \\

\textbf{Answer:} A \\
\textbf{Explanation:} Heat provided by the soldering iron:
[IMAGE:0]
Heat absorbed by the tip:
[IMAGE:1]
Heat lost to the environment:
[IMAGE:2]

\hrule
\vspace{1em}


\noindent
\textbf{Q557.} A
n object weighing
120
N hangs from a light inextensible string of length 0.35 m attached to the ceiling. A constant horizontal force of
90
N is applied to the object, moving it to a new equilibrium position where the string is no longer vertical. By how much has the gravitational potential energy of the object increased?



\textbf{A.} 8.4
J \\
\textbf{B.} 2.8 J \\
\textbf{C.} 3.5 J \\
\textbf{D.} 4.2 J \\

\textbf{Answer:} A \\
\textbf{Explanation:} [IMAGE:0]

\hrule
\vspace{1em}


\noindent
\textbf{Q558.} During the soldering process, the copper tip of a soldering iron (mass
[IMAGE:0]
, specific heat capacity
[IMAGE:1]
) accumulates heat due to prolonged use. When the soldering iron stops heating, the tip begins to cool and its temperature drops by
[IMAGE:2]
in
[IMAGE:3]
. Assuming all the heat lost by the tip is absorbed by the surrounding environment and the cooling rate is constant
Calculate the heat lost by the tip per second during cooling.



\textbf{A.} [IMAGE:0] \\
\textbf{B.} [IMAGE:1] \\
\textbf{C.} [IMAGE:2] \\
\textbf{D.} [IMAGE:3] \\

\textbf{Answer:} A \\
\textbf{Explanation:} Total heat lost by the tip:
[IMAGE:0]
Heat lost per second:
[IMAGE:1]
.

\hrule
\vspace{1em}


\noindent
\textbf{Q559.} An object weighing
60
N hangs from the end of a light inextensible string of length 0.45 m, which is attached to the ceiling. A constant horizontal force of
36
N is applied to the object, causing it to move to a new equilibrium position where the string is no longer vertical. By how much has the gravitational potential energy of the object increased?



\textbf{A.} 3.0 J \\
\textbf{B.} 4.0 J \\
\textbf{C.} 2.7
J \\
\textbf{D.} 6.0 J \\

\textbf{Answer:} E \\
\textbf{Explanation:} [IMAGE:0]

\hrule
\vspace{1em}


\noindent
\textbf{Q560.} A soldering iron is equipped with an aluminum tip of mass 3.0 g (specific heat capacity of aluminum =
[IMAGE:0]
). When the soldering iron heats the tip with a thermal power of 40 W, the tip's temperature rises to a certain temperature in 30 s. However, due to a heat sink on the tip's surface, some heat is lost to the environment. If only 85% of the heat provided by the heating power is actually absorbed by the tip.
Calculate the heat lost to the environment.



\textbf{A.} 120J \\
\textbf{B.} 160J \\
\textbf{C.} 180J \\
\textbf{D.} 1200J \\

\textbf{Answer:} C \\
\textbf{Explanation:} Heat provided by the soldering iron:
[IMAGE:0]
Heat absorbed by the tip:
[IMAGE:1]
Heat lost to the environment:
[IMAGE:2]

\hrule
\vspace{1em}


\noindent
\textbf{Q561.} When a Ra-226 nucleus (with velocity u) undergoes α-decay, it forms a Rn-222 nucleus and an α-particle. In the laboratory reference frame, the total kinetic energy of the Ra-226 nucleus after decay is
10
E. The ratio of the kinetic energy of the Rn-222 nucleus to that of the α-particle is 1:4. What is the kinetic energy of the α-particle?



\textbf{A.} E​
/
5 \\
\textbf{B.} 2
E​
/
5 \\
\textbf{C.} E​
/
4 \\
\textbf{D.} 4E​ \\

\textbf{Answer:} E \\
\textbf{Explanation:} Ea=4/(4+1)*10E=8E

\hrule
\vspace{1em}


\noindent
\textbf{Q562.} A soldering iron has a copper tip of mass 2.5g.
The tip is heated with 30W
of thermal power. In 60s
, the temperature of the tip increases by
[IMAGE:0]
.
How much energy is transferred from the tip to the surroundings in this time? (specific heat capacity of copper =
[IMAGE:1]
).



\textbf{A.} 320J \\
\textbf{B.} 480J \\
\textbf{C.} 640J \\
\textbf{D.} 1200J \\

\textbf{Answer:} E \\
\textbf{Explanation:} By the conservation of energy; during this time; the energy to heat the tip minus the energy dissipated into the air equals the energy to raise the temperature of the tip: therefore
[IMAGE:0]

\hrule
\vspace{1em}


\noindent
\textbf{Q563.} A stationary actinium-227 (Ac-227) nucleus undergoes alpha decay to form francium-223 (Fr-223) and an alpha particle. After decay, the two particles leave tracks in a cloud chamber. If the track length of the alpha particle is
233/4
times that of Fr-223 (assuming equal motion time and constant resistance), and the total kinetic energy released is
0.5
E, what is the kinetic energy of the alpha particle?



\textbf{A.} 3E/227 \\
\textbf{B.} 446
E/227 \\
\textbf{C.} E/2 \\
\textbf{D.} 111.5
E/(223+4) \\

\textbf{Answer:} D \\
\textbf{Explanation:} [IMAGE:0]

\hrule
\vspace{1em}


\noindent
\textbf{Q564.} A soldering iron has a copper tip of mass 1.0g.
The tip is heated with 20W
of thermal power. In 45s
, the temperature of the tip increases by
[IMAGE:0]
.
How much energy is transferred from the tip to the surroundings in this time? (specific heat capacity of copper =
[IMAGE:1]
).



\textbf{A.} 320J \\
\textbf{B.} 480J \\
\textbf{C.} 640J \\
\textbf{D.} 700J \\

\textbf{Answer:} E \\
\textbf{Explanation:} By the conservation of energy; during this time; the energy to heat the tip minus the energy dissipated into the air equals the energy to raise the temperature of the tip: therefore
[IMAGE:0]
.

\hrule
\vspace{1em}


\noindent
\textbf{Q565.} A stationary radium-226 nucleus (Ra-226) undergoes alpha decay to form a radon-222 nucleus (Rn-222) and an alpha particle. The relative velocity between the radon-222 nucleus and the alpha particle after decay is
3
v. What is the kinetic energy of the alpha particle?



\textbf{A.} [IMAGE:0] \\
\textbf{B.} [IMAGE:1] \\
\textbf{C.} [IMAGE:2] \\
\textbf{D.} [IMAGE:3] \\

\textbf{Answer:} D \\
\textbf{Explanation:} [IMAGE:0]

\hrule
\vspace{1em}


\noindent
\textbf{Q566.} A soldering iron has a copper tip of mass 2.0g.
The tip is heated with 27W
of thermal power. In 40s
, the temperature of the tip increases by
[IMAGE:0]
.
How much energy is transferred from the tip to the surroundings in this time? (specific heat capacity of copper =
[IMAGE:1]
).



\textbf{A.} 320J \\
\textbf{B.} 760J \\
\textbf{C.} 780J \\
\textbf{D.} 880J \\

\textbf{Answer:} B \\
\textbf{Explanation:} By the conservation of energy; during this time; the energy to heat the tip minus the energy dissipated into the air equals the energy to raise the temperature of the tip: therefore
[IMAGE:0]
.

\hrule
\vspace{1em}


\noindent
\textbf{Q567.} A stationary uranium-238 nucleus (U-238) undergoes alpha decay to form a thorium-234 nucleus (Th-234). Subsequently, the thorium-234 nucleus undergoes beta decay to form a protactinium-234 nucleus (Pa-234) and an electron. The total kinetic energy produced by the two decays is
0.5
E. What is the kinetic energy of the alpha particle released in the first decay?



\textbf{A.} 117
E
​
/
238 \\
\textbf{B.} 4
E
​
/
234 \\
\textbf{C.} 234
E
​
/
238 \\
\textbf{D.} 234
E
​
/(
2
37 \\

\textbf{Answer:} A \\
\textbf{Explanation:} [IMAGE:0]
[IMAGE:1]
[IMAGE:2]
[IMAGE:3]

\hrule
\vspace{1em}


\noindent
\textbf{Q568.} A soldering iron has a copper tip of mass 2.0g.
The tip is heated with 20W
of thermal power. In 33s
, the temperature of the tip increases by
[IMAGE:0]
.
How much energy is transferred from the tip to the surroundings in this time? (specific heat capacity of copper =
[IMAGE:1]
).



\textbf{A.} 320J \\
\textbf{B.} 500J \\
\textbf{C.} 640J \\
\textbf{D.} 1200J \\

\textbf{Answer:} B \\
\textbf{Explanation:} By the conservation of energy; during this time; the energy to heat the tip minus the energy dissipated into the air equals the energy to raise the temperature of the tip: therefore
[IMAGE:0]
.

\hrule
\vspace{1em}


\noindent
\textbf{Q569.} A stationary americium-243 (Am-243) nucleus undergoes alpha decay to form neptunium-239 (Np-239) and an alpha particle. The total kinetic energy released in the decay is
0.5
E, and the two particles move in opposite directions. What is the kinetic energy of the alpha particle?



\textbf{A.} 4E/243 \\
\textbf{B.} 478
E/243 \\
\textbf{C.} E/2 \\
\textbf{D.} 119.5
E/(239+4) \\

\textbf{Answer:} D \\
\textbf{Explanation:} [IMAGE:0]

\hrule
\vspace{1em}


\noindent
\textbf{Q570.} A stationary radium-226 nucleus (Ra-226) undergoes alpha decay to form a radon-222 nucleus (Rn-222). Subsequently, the radon-222 nucleus undergoes another alpha decay to form a polonium-218 nucleus (Po-218). The total kinetic energy produced by the two decays is
0.5
E. What is the kinetic energy of the alpha particle released in the second decay?



\textbf{A.} 109
E​
/
226 \\
\textbf{B.} 4E​
/
218 \\
\textbf{C.} 436
E​
/
226 \\
\textbf{D.} 218E​
/
222 \\

\textbf{Answer:} A \\
\textbf{Explanation:} [IMAGE:0]

\hrule
\vspace{1em}


\noindent
\textbf{Q571.} Mike heats ice cubes and observes the physical changes of ice. During this process, he measures and draws a graph of temperature versus time, as shown in the figure below. Based on the figure, which of the following analysis is correct?



\textbf{A.} The BC segment in the figure represents the melting process of ice. \\
\textbf{B.} Ice has an uncertain melting point,indicating that ice is not a crystal. \\
\textbf{C.} The water temperature rises slowly,indicating that the specific heat capacity of water is smaller than that of ice. \\
\textbf{D.} The temperature of boiling water remains constant,indicating that boiling does not require heat absorption. \\

\textbf{Answer:} A \\
\textbf{Explanation:} A. The AB segment in the figure represents the temperature rise of ice, while the BC segment represents the melting process of ice; hence, A is correct.
B. From the figure, it can be seen that the temperature of ice remains constant at 0\circ C, indicating that ice is a crystal; hence, B is incorrect.
C. Since the mass of ice and water is the same and they are heated by the same alcohol lamp, the slow rise in water temperature indicates that the specific heat capacity of water is larger than that of ice; hence, C is incorrect.
D. The temperature of boiling water remains constant, but it requires continuous heat absorption. If heating is stopped, boiling will also stop; hence, D is incorrect.

\hrule
\vspace{1em}


\noindent
\textbf{Q572.} A stationary californium-252 (Cf-252) nucleus undergoes alpha decay to form curium-248 (Cm-248) and an alpha particle. The total kinetic energy released in the decay is
0.5
E. If the alpha particle and Cm-248 move in opposite directions, and their kinetic energies are determined by momentum conservation, what is the kinetic energy of the alpha particle?



\textbf{A.} 4E/252 \\
\textbf{B.} 124
E/252 \\
\textbf{C.} E/2 \\
\textbf{D.} 248E/(248+4) \\

\textbf{Answer:} B \\
\textbf{Explanation:} [IMAGE:0]

\hrule
\vspace{1em}


\noindent
\textbf{Q573.} A glass of water is placed in a refrigerator; with a mass of 0.2 kg and initial temperature of 10 degrees; Find the equilibrium temperature: The latent heat of ice is 300kJ/kg; the specific heat capacity of water is 2.09 kJ/(kg\cdot degree); the specific heat capacity of water is 4200 kJ/(kg\cdot degree); Under standard atmospheric pressure , water can completely freeze and the freezing temperature (i.e., the ice point) of the water is -20 degrees Celsius.
Calculate the amount of heat that needs to be released for the water in this cup to turn into ice at 0 degrees Celsius.



\textbf{A.} [IMAGE:0] \\
\textbf{B.} [IMAGE:1] \\
\textbf{C.} [IMAGE:2] \\
\textbf{D.} [IMAGE:3] \\

\textbf{Answer:} C \\
\textbf{Explanation:} Assume that the equilibrium temperature is x; the heat released by the water equals the heat absorbs the ice:
[IMAGE:0]
PS: some formulas are listed as follows
[IMAGE:1]
[IMAGE:2]
[IMAGE:3]

\hrule
\vspace{1em}


\noindent
\textbf{Q574.} A stationary nitrogen molecule (N₂) decomposes into two nitrogen atoms (N) at high temperatures. The total kinetic energy of the two nitrogen atoms after decomposition is
0.5
E, and the mass of each nitrogen atom is 1
6
u (atomic mass units). What is the kinetic energy of one of the nitrogen atoms?



\textbf{A.} E
/
4 \\
\textbf{B.} E
​
/
1
6 \\
\textbf{C.} 14
E
​
/
28 \\
\textbf{D.} 4
E
​
/
141
A \\

\textbf{Answer:} A \\
\textbf{Explanation:} [IMAGE:0]

\hrule
\vspace{1em}


\noindent
\textbf{Q575.} A glass of water is placed in a refrigerator; with a mass of 0.4 kg and initial temperature of 5 degrees; Find the equilibrium temperature: The latent heat of ice is 300kJ/kg; the specific heat capacity of ice is 2.09 kJ/(kg\cdot degree); the specific heat capacity of water is 4200 kJ/(kg\cdot degree); Under standard atmospheric pressure , water can completely freeze and the freezing temperature (i.e., the ice point) of the water is 0 degrees Celsius.
Calculate the amount of heat that needs to be released for the water in this cup to turn into ice at 0 degrees Celsius.



\textbf{A.} [IMAGE:0] \\
\textbf{B.} [IMAGE:1] \\
\textbf{C.} [IMAGE:2] \\
\textbf{D.} [IMAGE:3] \\

\textbf{Answer:} D \\
\textbf{Explanation:} Assume that the equilibrium temperature is x; the heat released by the water equals the heat absorbs the ice:
[IMAGE:0]
PS: some formulas are listed as follows
[IMAGE:1]
[IMAGE:2]
[IMAGE:3]

\hrule
\vspace{1em}


\noindent
\textbf{Q576.} When a stationary radium-226 nucleus decays by alpha emission to form a radon-222 nucleus, the total kinetic energy produced by the decay is
0.5
E. What is the kinetic energy of the alpha particle?



\textbf{A.} 4E​
/
226 \\
\textbf{B.} 4
44
E​
/452 \\
\textbf{C.} 2E​ \\
\textbf{D.} 111E
/452 \\

\textbf{Answer:} D \\
\textbf{Explanation:} [IMAGE:0]

\hrule
\vspace{1em}


\noindent
\textbf{Q577.} A piece of ice undergoes the following three processes:
Ice at -20°C is heated to 0°C, absorbing heat
[IMAGE:0]
;
Ice at 0°C melts into water at 0°C, absorbing heat
[IMAGE:1]
;
Water at 20°C is heated to 40°C, absorbing heat
[IMAGE:2]
.
It is known that the specific heat capacity of ice is less than that of water. The mass remains constant throughout the entire process. Which of the following is true?



\textbf{A.} [IMAGE:0] \\
\textbf{B.} [IMAGE:1] \\
\textbf{C.} [IMAGE:2] \\
\textbf{D.} [IMAGE:3] \\

\textbf{Answer:} C \\
\textbf{Explanation:} To compare the heat absorbed in each process, we need to use the formulas for heat absorption:
Heat absorbed by ice from -20°C to 0°C:
[IMAGE:0]
Heat absorbed by ice melting at 0°C:
[IMAGE:1]
where
[IMAGE:2]
is the latent heat of fusion for ice.
Heat absorbed by water from 20°C to 40°C:
[IMAGE:3]
Given that the specific heat capacity of ice
[IMAGE:4]
is less than that of water
[IMAGE:5]
, we have:
[IMAGE:6]
Thus,
[IMAGE:7]
.
The latent heat of fusion
[IMAGE:8]
for ice is generally much larger than the specific heat capacities
[IMAGE:9]
and
[IMAGE:10]
multiplied by the temperature change. Therefore:
[IMAGE:11]
Combining these results, we get:
[IMAGE:12]

\hrule
\vspace{1em}


\noindent
\textbf{Q578.} A stationary plutonium-242 (Pu-242) nucleus decays by alpha emission to form a uranium-238 (U-238) nucleus and an alpha particle. The total kinetic energy produced by the decay is
0.5
E.
What is the kinetic energy of the alpha particle?



\textbf{A.} 119
E/242 \\
\textbf{B.} 4E/238 \\
\textbf{C.} E/2 \\
\textbf{D.} 476
E/242 \\

\textbf{Answer:} A \\
\textbf{Explanation:} [IMAGE:0]

\hrule
\vspace{1em}


\noindent
\textbf{Q579.} In a sealed container, the temperature of a gas remains constant. When the volume is 4 m
3
, the pressure is 6000 Pa. If the volume expands to
16
m
3
, what is the new pressure?



\textbf{A.} 4000 Pa \\
\textbf{B.} 3000 Pa \\
\textbf{C.} 2000 Pa \\
\textbf{D.} 1000 Pa \\

\textbf{Answer:} F \\
\textbf{Explanation:} [IMAGE:0]

\hrule
\vspace{1em}


\noindent
\textbf{Q580.} The ice is submerged into a glass of water; the 1.0kg ice is at -14 degrees; 0.2 kg water is at 5 degrees; Find the equilibrium temperature: The latent heat of ice is 300kJ/kg; the specific heat capacity of ice is 2.09 kJ/(kg\cdot degree); the specific heat capacity of water is 4200 kJ/(kg\cdot degree); The ice can completely melt when the temperature between 0 degrees and 5 degrees; the melting point of ice under standard atmospheric pressure is 0 degrees:



\textbf{A.} 5.23 \\
\textbf{B.} 8.7 \\
\textbf{C.} 7.7 \\
\textbf{D.} 0.87 \\

\textbf{Answer:} E \\
\textbf{Explanation:} Assume that the equilibrium temperature is x; the heat released by the water equals the heat absorbs the ice:
[IMAGE:0]
.
[IMAGE:1]
[IMAGE:2]
[IMAGE:3]
PS: some formulas are listed as follows
[IMAGE:4]
[IMAGE:5]
[IMAGE:6]

\hrule
\vspace{1em}


\noindent
\textbf{Q581.} A circuit contains a variable resistor whose resistance R can be adjusted. When the resistance is 10Ω, the current flowing through the circuit is 2A. Assuming the voltage of the power supply remains constant, what is the current when the resistance is increased to
5
0Ω?



\textbf{A.} 0.5A \\
\textbf{B.} 1.0A \\
\textbf{C.} 1.5A \\
\textbf{D.} 2.0A \\

\textbf{Answer:} F \\
\textbf{Explanation:} [IMAGE:0]

\hrule
\vspace{1em}


\noindent
\textbf{Q582.} The ice is submerged into a glass of water; the 1.0kg ice is at -60 degrees; 1 kg water is at 5 degrees; Find the equilibrium temperature: The latent heat of ice is 300kJ/kg; the specific heat capacity of ice is 2.09 kJ/(kg\cdot degree); the specific heat capacity of water is 4200 kJ/(kg\cdot degree); The ice can completely melt when the temperature between 0 degrees and 5 degrees; the melting point of ice under standard atmospheric pressure is 0 degrees:



\textbf{A.} 5.23 \\
\textbf{B.} 10.00 \\
\textbf{C.} 3.12 \\
\textbf{D.} 2.45 \\

\textbf{Answer:} D \\
\textbf{Explanation:} Assume that the equilibrium temperature is x; the heat released by the water equals the heat absorbs the ice:
[IMAGE:0]
.
[IMAGE:1]
[IMAGE:2]
PS: some formulas are listed as follows
[IMAGE:3]
[IMAGE:4]
[IMAGE:5]

\hrule
\vspace{1em}


\noindent
\textbf{Q583.} A cylinder contains a fixed amount of ideal gas. When the piston is at the midpoint, the volume of the gas is 500cm
3
and the pressure is 1.5atm. If the piston is slowly compressed, reducing the gas volume to
1
0cm
3,
what is the new pressure of the gas, assuming the temperature remains constant?



\textbf{A.} 10atm \\
\textbf{B.} 75atm \\
\textbf{C.} 20atm \\
\textbf{D.} 25atm \\

\textbf{Answer:} B \\
\textbf{Explanation:} [IMAGE:0]

\hrule
\vspace{1em}


\noindent
\textbf{Q584.} The ice is submerged into a glass of water; the 1.0kg ice is at 0 degrees; 1 kg water is at 6 degrees; Find the equilibrium temperature: The latent heat of ice is 300kJ/kg; the specific heat capacity of water is 4200 kJ/Kg*degree; The ice can completely melt when the temperature between 0 degrees and 6 degrees; the melting point of ice under standard atmospheric pressure is 0 degrees:



\textbf{A.} 5.2 \\
\textbf{B.} 2.5 \\
\textbf{C.} 3.0 \\
\textbf{D.} 10.0 \\

\textbf{Answer:} C \\
\textbf{Explanation:} Assume that the equilibrium temperature is x; the heat released by the water equals the heat absorbs the ice:
[IMAGE:0]
.
[IMAGE:1]
[IMAGE:2]
where
[IMAGE:3]
corresponds to the process of ice at 0 degrees turning into water at 0 degrees,
[IMAGE:4]
corresponds to the process of the water formed from the ice absorbing heat, and
[IMAGE:5]
corresponds to the process of the original water in the glass releasing heat.

\hrule
\vspace{1em}


\noindent
\textbf{Q585.} In a hydraulic system, the force F remains constant. Piston A has an area of 0.
1
m
2
and generates a pressure of 5000 Pa. If replaced by Piston B with an area of
0.4
m
2,
what is the new pressure?



\textbf{A.} 2000 Pa \\
\textbf{B.} 2500 Pa \\
\textbf{C.} 4000 Pa \\
\textbf{D.} 5000 Pa \\

\textbf{Answer:} F \\
\textbf{Explanation:} [IMAGE:0]

\hrule
\vspace{1em}


\noindent
\textbf{Q586.} The ice is submerged into a glass of water; the 2.0kg ice is at 0 degrees; 1 kg water is at 25 degrees; Find the equilibrium temperature: The latent heat of ice is 300kJ/kg; the specific heat capacity of water is 4200 kJ/Kg*degree; The ice can completely melt when the temperature between 0 degrees and 25 degrees; the melting point of ice under standard atmospheric pressure is 0 degrees:



\textbf{A.} 5.6 \\
\textbf{B.} 7.4 \\
\textbf{C.} 6.6 \\
\textbf{D.} 8.3 \\

\textbf{Answer:} D \\
\textbf{Explanation:} Assume that the equilibrium temperature is x; the heat released by the water equals the heat absorbs the ice:
[IMAGE:0]
.
[IMAGE:1]
[IMAGE:2]
[IMAGE:3]
where
[IMAGE:4]
corresponds to the process of ice at 0 degrees turning into water at 0 degrees,
[IMAGE:5]
corresponds to the process of the water formed from the ice absorbing heat, and
[IMAGE:6]
corresponds to the process of the original water in the glass releasing heat.

\hrule
\vspace{1em}


\noindent
\textbf{Q587.} The gravitational force F between the Earth and the Moon is inversely proportional to the square of the distance r between them. When the Earth and Moon are 3.84×10
5
km apart, the force is 2×10
20
N. What is the distance between them when the gravitational force decreases to
1.25
×10
19
N?



\textbf{A.} 1.92×10
5
km
a \\
\textbf{B.} 3.84×10
5
km \\
\textbf{C.} 5.76×10
5
km \\
\textbf{D.} 7.68×10
5
km \\

\textbf{Answer:} F \\
\textbf{Explanation:} [IMAGE:0]

\hrule
\vspace{1em}


\noindent
\textbf{Q588.} The ice is submerged into a glass of water; the 0.5kg ice is at 0 degrees; 0.5 kg water is at 18 degrees; Find the equilibrium temperature: The latent heat of ice is 300kJ/kg; the specific heat capacity of water is 4200 kJ/Kg*degree; The ice can completely melt when the temperature between 0 degrees and 18 degrees; the melting point of ice under standard atmospheric pressure is 0 degrees:



\textbf{A.} 5.6 \\
\textbf{B.} 7.4 \\
\textbf{C.} 3.6 \\
\textbf{D.} 9.0 \\

\textbf{Answer:} D \\
\textbf{Explanation:} Assume that the equilibrium temperature is x; the heat released by the water equals the heat absorbs the ice:
[IMAGE:0]
.
[IMAGE:1]
[IMAGE:2]
where
[IMAGE:3]
corresponds to the process of ice at 0 degrees turning into water at 0 degrees,
[IMAGE:4]
corresponds to the process of the water formed from the ice absorbing heat, and
[IMAGE:5]
corresponds to the process of the original water in the glass releasing heat.

\hrule
\vspace{1em}


\noindent
\textbf{Q589.} A car starts from a stationary state and undergoes uniform acceleration with an acceleration of a = 2 meters/second² for 6 seconds. Then it moves at a constant speed until the total time reaches 15 seconds. If the total distance covered is 1
44
meters, what is the constant speed after the uniform acceleration stage?



\textbf{A.} 5 m/s \\
\textbf{B.} 8 m/s \\
\textbf{C.} 10 m/s \\
\textbf{D.} 12 m/s \\

\textbf{Answer:} D \\
\textbf{Explanation:} Constant speed stage speed: 2
*
6 = 12 m/s

\hrule
\vspace{1em}


\noindent
\textbf{Q590.} The ice is submerged into a glass of water; the 0.5kg ice is at 0 degrees; 1 kg water is at 16 degrees; Find the equilibrium temperature: The latent heat of ice is 300kJ/kg; the specific heat capacity of water is 4200 kJ/Kg*degree; The ice can completely melt when the temperature between 0 degrees and 16 degrees; the melting point of ice under standard atmospheric pressure is 0 degrees:



\textbf{A.} 5.6 \\
\textbf{B.} 7.4 \\
\textbf{C.} 10 \\
\textbf{D.} 10.6 \\

\textbf{Answer:} D \\
\textbf{Explanation:} Assume that the equilibrium temperature is x; the heat released by the water equals the heat absorbs the ice:
[IMAGE:0]
.
[IMAGE:1]
[IMAGE:2]
[IMAGE:3]
where
[IMAGE:4]
corresponds to the process of ice at 0 degrees turning into water at 0 degrees,
[IMAGE:5]
corresponds to the process of the water formed from the ice absorbing heat, and
[IMAGE:6]
corresponds to the process of the original water in the glass releasing heat.

\hrule
\vspace{1em}


\noindent
\textbf{Q591.} The electrostatic force F between two point charges is inversely proportional to the square of the distance r between them. When the charges are 3m apart, the force is 12N. What is the distance between the charges when the force becomes
6.75
N?



\textbf{A.} 0.5m \\
\textbf{B.} 1m \\
\textbf{C.} 2m \\
\textbf{D.} 2.5m \\

\textbf{Answer:} F \\
\textbf{Explanation:} [IMAGE:0]

\hrule
\vspace{1em}


\noindent
\textbf{Q592.} The velocity v of an object is inversely proportional to the square root of time t. When v=12m/s, t=4s. What is the value of t when v=
3
m/s?
Options:



\textbf{A.} 16/9s \\
\textbf{B.} 4s \\
\textbf{C.} 9/16s \\
\textbf{D.} 27/14s \\

\textbf{Answer:} F \\
\textbf{Explanation:} [IMAGE:0]

\hrule
\vspace{1em}


\noindent
\textbf{Q593.} The quantities a and b
are positive. aa is inversely proportional to the square root of b. When a=
4
, b=16. What is the value of b when a=
0.64
?



\textbf{A.} 16 \\
\textbf{B.} 49 \\
\textbf{C.} 81 \\
\textbf{D.} 625 \\

\textbf{Answer:} D \\
\textbf{Explanation:} [IMAGE:0]

\hrule
\vspace{1em}


\noindent
\textbf{Q594.} The quantities
x
and y
are positive. x is inversely proportional to the square of
y
. When x=3, y=4. What is the value of
y
when x=
1.92
?



\textbf{A.} 1 \\
\textbf{B.} 2 \\
\textbf{C.} 5 \\
\textbf{D.} 16 \\

\textbf{Answer:} C \\
\textbf{Explanation:} [IMAGE:0]

\hrule
\vspace{1em}


\noindent
\textbf{Q595.} A
5
kg object is initially at rest on a smooth horizontal surface. Starting at t=0, the object is subjected to a horizontal force that varies with time as follows:
\cdot 
From 0 to 0.05 s, the force increases linearly from 0 to 10 N;
\cdot 
From 0.05 to 0.10 s, the force remains constant at 10 N;
\cdot 
From 0.10 to 0.15 s, the force decreases linearly back to 0.
What is the kinetic energy of the object at t=0.
0.075
s?



\textbf{A.} 0J \\
\textbf{B.} 0.50J \\
\textbf{C.} 1.25J \\
\textbf{D.} 0.025J \\

\textbf{Answer:} D \\
\textbf{Explanation:} The kinetic energy is equal to the work done by the force, which is the integral of the force over time.
From 0 to 0.05 s: Force increases linearly from 0 to 10 N (triangle).
From 0.05 to 0.10 s: Force remains constant at 10 N (rectangle).
From 0.10 to 0.15 s: Force decreases linearly from 10 N to 0 (triangle).

\hrule
\vspace{1em}


\noindent
\textbf{Q596.} An object of mass
5
kg is at rest at time t=0. A resultant force acts on the object in a constant direction. The magnitude of the resultant force acting on the object varies with time as follows:
\cdot 
The force starts at 0 N, increases to 10 N at t=0.1 s and remains constant until t=0.2 s, then increases linearly to 20 N, and finally decreases linearly to 0 N at t=0.3 s.
What is the kinetic energy of the object at time t=0.
1
s?



\textbf{A.} 0.025J \\
\textbf{B.} 2.34J \\
\textbf{C.} 4.02J \\
\textbf{D.} 3.06J \\

\textbf{Answer:} A \\
\textbf{Explanation:} The kinetic energy equals the work done by the force; the integration of force over time equals the change in momentum. The graph shows that the force increases from 0 N to 10 N at t=0.1 s and remains constant until t=0.2 s, then increases linearly to 20 N, and finally decreases linearly to 0 N at t=0.3 s.

\hrule
\vspace{1em}


\noindent
\textbf{Q597.} An object of mass 5 kg is at rest at time t=0. A resultant force acts on the object in a constant direction. The magnitude of the resultant force acting on the object varies with time as follows:
\cdot 
The force starts at 0 N, increases to 100 N at t=0.1 s, and then decreases linearly to 0 N at t=0.2 s .
What is the kinetic energy of the object at time t=0.
1
s?



\textbf{A.} 0J \\
\textbf{B.} 5J \\
\textbf{C.} 2.5J \\
\textbf{D.} 2J \\

\textbf{Answer:} C \\
\textbf{Explanation:} [IMAGE:0]

\hrule
\vspace{1em}


\noindent
\textbf{Q598.} A rocket of mass 10 kg is launched with a varying thrust. The thrust varies with time as follows:
The thrust starts at 0 N, increases to 200 N at t=0.1 s, and then decreases linearly to 0 N at t=0.2 s.
Ignoring air resistance and other forces, what is the kinetic energy of the rocket at t=0.
1
s?



\textbf{A.} 0J \\
\textbf{B.} 1J \\
\textbf{C.} 2J \\
\textbf{D.} 3J \\

\textbf{Answer:} F \\
\textbf{Explanation:} [IMAGE:0]

\hrule
\vspace{1em}


\noindent
\textbf{Q599.} A
2
kg object is initially at rest on a smooth horizontal surface. Starting at t=0, the object is subjected to a horizontal force that varies with time as follows:
\cdot 
From 0 to 0.6 s, the force increases linearly from 0 to
10
N;
\cdot 
From 0.6 to 1
.
2 s, the force remains constant at
10
N;
\cdot 
From
1.2
to
1.6
s, the force decreases linearly back to 0.
What is the kinetic energy of the object at t=0.
8
s?



\textbf{A.} 36
J \\
\textbf{B.} 72 J \\
\textbf{C.} 14 J \\
\textbf{D.} 26 J \\

\textbf{Answer:} A \\
\textbf{Explanation:} The kinetic energy is equal to the work done by the force, which is the integral of the force over time (the area under the force-time graph).

\hrule
\vspace{1em}


\noindent
\textbf{Q600.} A ball A with a mass of 2 kilograms is moving at a speed of 8 meters/second and collides elastically with a stationary ball B whose mass is 4 kilograms. After the collision, the speed of ball A becomes -2 meters/second (in the opposite direction). What is the speed of ball B after the collision? (The result should be rounded to two decimal places.)



\textbf{A.} 1.0 m/s \\
\textbf{B.} 2.0 m/s \\
\textbf{C.} 3.0 m/s \\
\textbf{D.} 6
.0 m/s \\

\textbf{Answer:} E \\
\textbf{Explanation:} In an elastic collision, both momentum and kinetic energy are conserved. Using the conservation of momentum:
[IMAGE:0]
Since kinetic energy is conserved, the answer is correct. The velocity of ball B after the collision is 4 m/s, corresponding to option
E
.

\hrule
\vspace{1em}


\noindent
\textbf{Q601.} An object of mass
6
kg is at rest at time = 0 s. A resultant force acts on the object in a constant direction.The magnitude of the resultant force acting on the object varies with time as shown by the graph.
What is the kinetic energy of the object at time =
6
s?



\textbf{A.} 4J \\
\textbf{B.} 11J \\
\textbf{C.} 16J \\
\textbf{D.} 12J \\

\textbf{Answer:} G \\
\textbf{Explanation:} 

\hrule
\vspace{1em}


\noindent
\textbf{Q602.} An object of mass
3
kg is at rest at time = 0 s. A resultant force acts on the object in a constant direction.The magnitude of the resultant force acting on the object varies with time as shown by the graph.
What is the kinetic energy of the object at time =
2
s?



\textbf{A.} 96J \\
\textbf{B.} 81J \\
\textbf{C.} 12J \\
\textbf{D.} 125J \\

\textbf{Answer:} F \\
\textbf{Explanation:} The kinetic energy equals the work done by the force; the integration of force on time equals the change in momentum; mv; therefore the terminal K.E. can be found:1/2 mv
2
.

\hrule
\vspace{1em}


\noindent
\textbf{Q603.} An object with a mass of 1 kilogram has a velocity of 1 m/s at time t = 0 seconds. A constant force in a certain direction acts on this object. The variation of the magnitude
of the force acting on this object with time is shown in the figure.
What is the
kin
etic energy of t
he obje
ct at
time =
0.
1
0 s
?



\textbf{A.} 1.2
J \\
\textbf{B.} 0.8
1
J \\
\textbf{C.} 3
.
6
J \\
\textbf{D.} 1.
25
J \\

\textbf{Answer:} F \\
\textbf{Explanation:} The kinetic energy equals the work done by the force; the integration of force on time equals the change in momentum;
mv=1*1+
mv
1
; therefore the terminal K.E. can be found:1/2 mv
2
.

\hrule
\vspace{1em}


\noindent
\textbf{Q604.} .An object of mass
4
kg is at rest at time = 0 s. A resultant force acts on the object in a constant direction.The magnitude of the resultant force acting on the object varies with time as shown by
the graph.
What is the kinetic energy of the object at time = 0.
1
0 s?



\textbf{A.} 0J \\
\textbf{B.} 0.313
J \\
\textbf{C.} 1J \\
\textbf{D.} 0.
1
25
J \\

\textbf{Answer:} D \\
\textbf{Explanation:} The kinetic energy equals the work done by the force; the integration of force on time equals the change in momentum; mv; therefore the terminal K.E. can be found:1/2 mv2.

\hrule
\vspace{1em}


\noindent
\textbf{Q605.} A 1500 kg car is driving on a horizontal curve with a radius of 50 m. The coefficient of kinetic friction between the tires and the dry road is 0.4. When the road is wet, the coefficient of kinetic friction decreases to 0.3. What is the maximum safe speed for the car to navigate the curve on a wet road? (Gravitational field strength g=10N kg−1)



\textbf{A.} 27.8 km/h \\
\textbf{B.} 21.6 km/h \\
\textbf{C.} 30.0 km/h \\
\textbf{D.} 40.78 km/h \\

\textbf{Answer:} F \\
\textbf{Explanation:} The kinetic friction between the tires and the road provides the necessary centripetal force for the car to navigate the curve.
Centripetal force equation:
[IMAGE:0]
[IMAGE:1]
[IMAGE:2]
[IMAGE:3]

\hrule
\vspace{1em}


\noindent
\textbf{Q606.} A 2.0 kg object is at rest on a horizontal surface with a coefficient of friction 0.25. A force of 20 N is applied at an angle of 37° above the horizontal (sin37°=0.6, cos37°=0.8). What is the magnitude of the object’s acceleration as it starts moving? (Take g=10 m/s2g=10m/s2.)



\textbf{A.} 5.0 m/s² \\
\textbf{B.} 6.0 m/s² \\
\textbf{C.} 7.0 m/s² \\
\textbf{D.} 8.0 m/s² \\

\textbf{Answer:} C \\
\textbf{Explanation:} [IMAGE:0]
[IMAGE:1]

\hrule
\vspace{1em}


\noindent
\textbf{Q607.} A 5.0 kg object is placed on a conveyor belt inclined at 30° to the horizontal. The conveyor belt accelerates upward at 2.0 m/s². The coefficient of kinetic friction between the object and the belt is 0.4. What is the magnitude of the object's acceleration relative to the conveyor belt? (Gravitational field strength g=10N kg−1, sin30°=0.5, cos30°=3​/2≈0.866)



\textbf{A.} 0.56 m/s² \\
\textbf{B.} 1.02 m/s² \\
\textbf{C.} 1.56 m/s² \\
\textbf{D.} 2.05 m/s² \\

\textbf{Answer:} C \\
\textbf{Explanation:} The normal force is influenced by both the vertical component of gravity and the vertical component of the inertial force due to the belt's acceleration.
[IMAGE:0]
[IMAGE:1]
[IMAGE:2]
[IMAGE:3]
[IMAGE:4]

\hrule
\vspace{1em}


\noindent
\textbf{Q608.} A 2.0 kg object is placed on a horizontal rotating disk. The maximum static frictional force between the object and the disk is 12.0 N. The disk starts rotating around its central axis with gradually increasing angular velocity. What is the angular velocity of the disk when the object is about to slip? (Gravitational field strength g=10N kg−1)



\textbf{A.} 2.45 rad/s \\
\textbf{B.} 3.05 rad/s \\
\textbf{C.} 4.40 rad/s \\
\textbf{D.} 5.30 rad/s \\

\textbf{Answer:} A \\
\textbf{Explanation:} Maximum Static Friction Provides Centripetal Force:
When the object is about to slip, the maximum static friction provides the necessary centripetal force. Centripetal force equation:
[IMAGE:0]
Given fmax​=12.0N, m=2.0kg, and assuming r=1.0m.
[IMAGE:1]
[IMAGE:2]

\hrule
\vspace{1em}


\noindent
\textbf{Q609.} A 4.0 kg object is at rest on a horizontal surface with a coefficient of friction 0.25. Two forces are applied: a horizontal force of 24 N to the right and a vertical upward force of 8 N. What is the magnitude of the object’s acceleration as it starts moving? (Take g=10 m/s
2.
)



\textbf{A.} 2.0 m/s² \\
\textbf{B.} 3.0 m/s² \\
\textbf{C.} 4.0 m/s² \\
\textbf{D.} 5.0 m/s² \\

\textbf{Answer:} C \\
\textbf{Explanation:} Vertical Force Analysis: The vertical upward force reduces the normal force.
Normal force N=mg−Fvertical=4×10−8=32 .
Friction Calculation: Friction f=μN=0.25×32=8 N.
Net horizontal force Fnet=Fhorizontal−f=24−8=16 .
Acceleration Calculation: By Newton’s second law, a=Fnetm=164=4 m/s
2
.

\hrule
\vspace{1em}


\noindent
\textbf{Q610.} A 5.0 kg object is at rest on an inclined plane with an angle of 30°. The coefficient of kinetic friction between the object and the plane is 0.2. A force of 10.0 N parallel to the plane upward and a force of 8.0 N perpendicular to the plane downward are applied simultaneously. What is the magnitude of the object's acceleration as it begins to move? (Gravitational field strength g=10N kg−1, sin30°=0.5, cos30°=3​/2≈0.866)



\textbf{A.} 10.5 m/s² \\
\textbf{B.} 5.05 m/s² \\
\textbf{C.} 1.25 m/s² \\
\textbf{D.} 2.05 m/s² \\

\textbf{Answer:} B \\
\textbf{Explanation:} The normal force is affected by both the vertical component of gravity and the external perpendicular force.
[IMAGE:0]
[IMAGE:1]
[IMAGE:2]
[IMAGE:3]

\hrule
\vspace{1em}


\noindent
\textbf{Q611.} A 2.0 kg object rests on a horizontal surface with a coefficient of friction 0.25. Two forces are applied: a horizontal force of 20 N and a vertical upward force of 4 N. What is the magnitude of the acceleration? (Take g=10 m/s
2
.)



\textbf{A.} 5.0 m/s² \\
\textbf{B.} 7.25 m/s² \\
\textbf{C.} 7.5 m/s² \\
\textbf{D.} 8.0 m/s² \\

\textbf{Answer:} D \\
\textbf{Explanation:} Vertical Force: Reduces normal force.
N=2×10−4=16 N.
Friction: f=0.25×16=4 N.
Horizontal Net Force: 20−4=16 N.
Acceleration: a=162=8 m/s
2
.

\hrule
\vspace{1em}


\noindent
\textbf{Q612.} Which of the following quantities is a vector for an object sliding on an inclined plane?



\textbf{A.} Gravitational potential energy \\
\textbf{B.} Kinetic energy \\
\textbf{C.} Acceleration \\
\textbf{D.} Mass \\

\textbf{Answer:} C \\
\textbf{Explanation:} Acceleration is a vector because it has both magnitude and direction (down the incline). Gravitational potential energy, kinetic energy, and mass are scalars.

\hrule
\vspace{1em}


\noindent
\textbf{Q613.} A 4.0 kg object is at rest on a horizontal surface. The coefficient of kinetic friction between the object and the surface is 0.3. A vertical upward force of 8.0 N and three horizontal forces (6.0 N, 8.0 N, and 10.0 N, all mutually perpendicular) are applied simultaneously. What is the magnitude of the object's acceleration as it begins to move? (Gravitational field strength g=10N kg
−1
)



\textbf{A.} 1.00 m/s² \\
\textbf{B.} 1.50 m/s² \\
\textbf{C.} 2.50 m/s² \\
\textbf{D.} 1.14 m/s² \\

\textbf{Answer:} D \\
\textbf{Explanation:} Vertical Force Analysis:
The upward vertical force reduces the normal reaction force.
[IMAGE:0]
[IMAGE:1]
[IMAGE:2]
[IMAGE:3]
[IMAGE:4]

\hrule
\vspace{1em}


\noindent
\textbf{Q614.} Which of the following quantities is a vector?



\textbf{A.} An object travels 10 km on a straight road, its displacement is 10 km. \\
\textbf{B.} An object travels 10 km on a straight road, its distance is 10 km. \\
\textbf{C.} An object travels 10 km on a straight road, its speed is 10 km/h. \\
\textbf{D.} An object travels 10 km on a straight road, its acceleration is 0. \\

\textbf{Answer:} A \\
\textbf{Explanation:} Displacement is a vector because it has both magnitude and direction. Distance is a scalar, and speed and acceleration are vectors, but the question only provides magnitude information.

\hrule
\vspace{1em}


\noindent
\textbf{Q615.} If the resultant of two forces is zero, they must satisfy:



\textbf{A.} Equal magnitude, same direction \\
\textbf{B.} Equal magnitude, opposite directions \\
\textbf{C.} Acting at the same point \\
\textbf{D.} Acting along the same line \\

\textbf{Answer:} B \\
\textbf{Explanation:} For the resultant to be zero, the forces must be equal in magnitude, opposite in direction, and colinear (opposite directions imply colinearity). Option B is correct; others are incomplete or incorrect.

\hrule
\vspace{1em}


\noindent
\textbf{Q616.} A point object of mass 2.0 kg is at rest on a horizontal surface with a coefficient of friction 0.25. Two perpendicular forces are applied simultaneously: a horizontal force of 20 N and a vertical upward force of 4 N. What is the magnitude of the acceleration of the object as it begins to move? (Take g=10 m/s
2
)



\textbf{A.} 5.0 m/s² \\
\textbf{B.} 7.25 m/s² \\
\textbf{C.} 7.5 m/s² \\
\textbf{D.} 8.0 m/s² \\

\textbf{Answer:} D \\
\textbf{Explanation:} Vertical Force Analysis: The vertical upward force reduces the normal force.
Horizontal Net Force: Net horizontal force
[IMAGE:0]
Acceleration Calculation: By Newton’s second law,
[IMAGE:1]

\hrule
\vspace{1em}


\noindent
\textbf{Q617.} Which pair of quantities will always result in a vector when operated?



\textbf{A.} Mass × Acceleration \\
\textbf{B.} Force ÷ Time \\
\textbf{C.} Velocity × Time \\
\textbf{D.} Work ÷ Displacement \\

\textbf{Answer:} A \\
\textbf{Explanation:} Mass (scalar) × Acceleration (vector) results in force (vector). Other options:
\cdot 
B: Force (vector) ÷ Time (scalar) gives impulse (vector), but the operation is ambiguous.
\cdot 
C: Velocity (vector) × Time (scalar) gives displacement (vector), but not directly listed.
\cdot 
D: Work (scalar) ÷ Displacement (vector) is meaningless.

\hrule
\vspace{1em}


\noindent
\textbf{Q618.} A 3.0 kg object is at rest on a horizontal surface. The coefficient of kinetic friction between the object and the surface is 0.2. A vertical upward force of 10.0 N and two perpendicular horizontal forces (5.0 N and 12.0 N) are applied simultaneously. What is the magnitude of the object's acceleration as it begins to move? (Gravitational field strength g=10N kg−1)



\textbf{A.} 2.0 m/s² \\
\textbf{B.} 2.5 m/s² \\
\textbf{C.} 3.0 m/s² \\
\textbf{D.} 3.5 m/s² \\

\textbf{Answer:} C \\
\textbf{Explanation:} Vertical Force Analysis:
The upward vertical force reduces the normal reaction force.
Normal force N=mg−Fvertical​=3×10−10=20N.
Frictional Force Calculation:
Kinetic friction f=μN=0.2×20=4.0N.
Resultant Horizontal Force:
The vector sum of the two perpendicular horizontal forces is:
[IMAGE:0]
[IMAGE:1]

\hrule
\vspace{1em}


\noindent
\textbf{Q619.} A car travels 5 km east, then 12 km north. What is the magnitude of its displacement?



\textbf{A.} 7 km \\
\textbf{B.} 13 km \\
\textbf{C.} 17 km \\
\textbf{D.} 60 km \\

\textbf{Answer:} B \\
\textbf{Explanation:} Displacement is a vector representing the straight-line distance from start to end. Using the Pythagorean theorem:
[IMAGE:0]
Other options are total distance or incorrect calculations.

\hrule
\vspace{1em}


\noindent
\textbf{Q620.} An object of weight 60 N hangs from the end of a light inextensible string of length 0.5 m, which is attached to the ceiling. A force of 30 N is applied to the object at an angle of 45° above the horizontal, causing it to move to a new equilibrium position. By how much has the gravitational potential energy of the object increased?



\textbf{A.} 2.40 J \\
\textbf{B.} 3.00 J \\
\textbf{C.} 18.5 J \\
\textbf{D.} 4.20 J \\

\textbf{Answer:} C \\
\textbf{Explanation:} [IMAGE:0]

\hrule
\vspace{1em}


\noindent
\textbf{Q621.} Which of the following is a vector quantity?



\textbf{A.} Velocity \\
\textbf{B.} Kinetic Energy \\
\textbf{C.} Mass \\
\textbf{D.} Time \\

\textbf{Answer:} A \\
\textbf{Explanation:} Velocity is a vector, requiring both magnitude and direction (e.g., 20 m/s north). The other options (kinetic energy, mass, time, density) are scalars, as they are fully described by magnitude alone.

\hrule
\vspace{1em}


\noindent
\textbf{Q622.} A 30 N object slides down from a smooth inclined plane of height 0.4 m and then travels 1 m on a stationary conveyor belt. The coefficient of kinetic friction between the object and the belt is 0.1. Finally, the object hits a spring with a spring constant of 200 N/m. Ignoring other frictional forces, what is the kinetic energy of the object when it hits the spring?



\textbf{A.} 2.10 J \\
\textbf{B.} 4.32 J \\
\textbf{C.} 6.75 J \\
\textbf{D.} 9.25 J \\

\textbf{Answer:} D \\
\textbf{Explanation:} Height of the incline: 0.4 m
Gravitational acceleration: 9.8 m/s²
Mass of the object: m = 30 / 9.8 ≈ 3.06 kg
Using conservation of mechanical energy: mgh = 0.5mv²
Substituting values: v = sqrt(2gh) = sqrt(2×9.8×0.4) ≈ 2.8 m/s
Kinetic energy: KE₁ = 0.5 × 3.06 × (2.8)² ≈ 12.25 J
[IMAGE:0]
[IMAGE:1]
[IMAGE:2]
[IMAGE:3]
[IMAGE:4]
Final kinetic energy: 12.25 - 3 ≈ 9.25 J

\hrule
\vspace{1em}


\noindent
\textbf{Q623.} Which of the following is a vector quantity?



\textbf{A.} Temperature \\
\textbf{B.} Power \\
\textbf{C.} Electric Current \\
\textbf{D.} Work \\

\textbf{Answer:} E \\
\textbf{Explanation:} Electric field strength is a vector, as its direction is defined by field lines. The other options (temperature, power, electric current, work) are scalars. Although electric current has direction, it is treated as a scalar because its superposition follows algebraic rules, not vector rules.

\hrule
\vspace{1em}


\noindent
\textbf{Q624.} Which of the following is a vector quantity?



\textbf{A.} Mass \\
\textbf{B.} Speed \\
\textbf{C.} Time \\
\textbf{D.} Momentum \\

\textbf{Answer:} D \\
\textbf{Explanation:} Momentum is a vector, defined as the product of mass and velocity, and it follows the direction of velocity. The other options (mass, speed, time, energy) are scalars, as they lack directional dependence.

\hrule
\vspace{1em}


\noindent
\textbf{Q625.} Which of the following is a vector quantity?



\textbf{A.} Power \\
\textbf{B.} Distance \\
\textbf{C.} Force \\
\textbf{D.} Work \\

\textbf{Answer:} C \\
\textbf{Explanation:} Force is a vector because its effect depends on direction (e.g., 10 N upward). The other options (power, distance, work, density) are scalars, as they only involve magnitude.

\hrule
\vspace{1em}


\noindent
\textbf{Q626.} A 20 N object is placed on a horizontal conveyor belt moving at a constant speed of 3 m/s to the right. The coefficient of kinetic friction between the object and the belt is 0.2. When the object accelerates until its speed matches the belt's speed, by how much has the kinetic energy of the object increased?



\textbf{A.} 2.0 J \\
\textbf{B.} 4.0 J \\
\textbf{C.} 9.2 J \\
\textbf{D.} 8.0 J \\

\textbf{Answer:} C \\
\textbf{Explanation:} The object on the conveyor belt experiences a frictional force that accelerates it until its speed matches the belt's speed. By calculating the work done by friction, the increase in kinetic energy can be determined.
Initial velocity of the object: 0 m/s
Acceleration: a = f/m = 4 / (20/9.8) ≈ 1.96 m/s²
Time to reach belt speed: t = v/a = 3 / 1.96 ≈ 1.53 s
Distance moved during acceleration: s = 0.5 × a × t² ≈ 0.5 × 1.96 × (1.53)² ≈ 2.3 m
[IMAGE:0]

\hrule
\vspace{1em}


\noindent
\textbf{Q627.} Which of the following is a vector quantity?



\textbf{A.} Time \\
\textbf{B.} Acceleration \\
\textbf{C.} Mass \\
\textbf{D.} Energy \\

\textbf{Answer:} B \\
\textbf{Explanation:} A vector has both magnitude and direction. Acceleration is a vector because it requires both magnitude and direction (e.g., 5 m/s² east). The other options (time, mass, temperature, energy) are scalars, as they only have magnitude.

\hrule
\vspace{1em}


\noindent
\textbf{Q628.} A mass
m
is connected to a fixed point on a smooth horizontal surface via a light spring. The mass executes simple harmonic motion on the surface. At a certain moment, the mass starts moving from position
A
, passes through position
B
after some time, and then reaches position
C
after more time. Positions
A
and
C
are the points of maximum displacement, and position
B
is the equilibrium position.
Question
: Which of the following statements is correct?



\textbf{A.} The kinetic energy of the mass increases and the elastic potential energy decreases as the mass moves from
A
to
B
. \\
\textbf{B.} All the mechanical energy is converted into kinetic energy when the mass is at position
B
. \\
\textbf{C.} Mechanical energy is
not
conserved as the mass moves from
B
to
C
. \\
\textbf{D.} The elastic potential energy is zero when the mass is at position
C
. \\

\textbf{Answer:} A \\
\textbf{Explanation:} A is correct: As the mass moves from
A
to
B
, its velocity increases, so kinetic energy increases; meanwhile, the deformation of the spring decreases, so elastic potential energy decreases.
B is incorrect: At position
B
, kinetic energy is maximum, but there is still elastic potential energy (the spring may be stretched or compressed), so not all mechanical energy is converted into kinetic energy.
C is
in
correct: On a smooth surface, only the spring force does work, so mechanical energy is conserved.
D is incorrect: At position
C
, the spring is stretched to the maximum, so elastic potential energy is maximum.

\hrule
\vspace{1em}


\noindent
\textbf{Q629.} A 20 N object is thrown vertically upward with an initial velocity of 2 m/s. Ignoring air resistance, by how much has the gravitational potential energy of the object increased during its ascent?



\textbf{A.} 2.1 J \\
\textbf{B.} 2.8 J \\
\textbf{C.} 3.5 J \\
\textbf{D.} 4.2 J \\

\textbf{Answer:} D \\
\textbf{Explanation:} The object is thrown vertically upward, and air resistance is ignored. By calculating the height the object rises, the increase in gravitational potential energy can be determined.
[IMAGE:0]
[IMAGE:1]

\hrule
\vspace{1em}


\noindent
\textbf{Q630.} A mass
m
is connected to a fixed point on a smooth horizontal surface via a light spring and executes simple harmonic motion. At a certain moment, the mass has velocity
v
and the spring is deformed by
x
.
Question
: Which of the following statements is correct?



\textbf{A.} [IMAGE:0] \\
\textbf{B.} The kinetic energy of the mass is zero when the spring deformation is zero. \\
\textbf{C.} The sum of the mass's kinetic energy and elastic potential energy decreases
during the motion. \\
\textbf{D.} The elastic potential energy is zero when the mass's velocity is zero. \\

\textbf{Answer:} A \\
\textbf{Explanation:} [IMAGE:0]

\hrule
\vspace{1em}


\noindent
\textbf{Q631.} A 60 N object is placed on a smooth inclined plane with an angle of 30°, connected via a light inextensible string over a pulley to a 40 N hanging object. The system starts from rest and moves until the hanging object descends 0.5 m. Ignoring pulley friction and air resistance, by how much has the gravitational potential energy of the hanging object increased?



\textbf{A.} 10 J \\
\textbf{B.} 15 J \\
\textbf{C.} 20 J \\
\textbf{D.} 25 J \\

\textbf{Answer:} C \\
\textbf{Explanation:} The system consists of an object on an inclined plane and a hanging object connected by a string over a pulley. As the system moves, the hanging object rises while the object on the plane slides down. By analyzing the energy changes, the increase in gravitational potential energy of the hanging object can be determined.
[IMAGE:0]

\hrule
\vspace{1em}


\noindent
\textbf{Q632.} An object of weight 50 N hangs from the end of a light inextensible string of length 0.5 m, which is attached to the ceiling. A constant horizontal wind force of 40 N blows on the object, causing it to move to a new equilibrium position. By how much has the gravitational potential energy of the object increased as a result of its change of position?



\textbf{A.} 2.1J \\
\textbf{B.} 2.8J \\
\textbf{C.} 3.50J \\
\textbf{D.} 4.2J \\

\textbf{Answer:} D \\
\textbf{Explanation:} The object moves to a new equilibrium position under the horizontal wind force, forming an angle θ with the vertical. By analyzing the forces and calculating the height increase, the change in gravitational potential energy can be determined.
Force Analysis
Weight of the object: 50 N downward.
Tension in the string: T along the string.
Horizontal wind force: 40 N.
Equilibrium Conditions
Horizontal: T*sinθ = 40 N
Vertical: T*cosθ = 50 N
Angle Calculation
From the equations:
tanθ = 40 / 50 = 0.8 \to  θ ≈ 38.66°
Height Increase
The height increase h = L(1 - cosθ) ≈ 0.5(1 - cos38.66°) ≈ 0.5 × 0.215 ≈ 0.1075 m
Potential Energy Change
ΔU = mgh = 50 × 0.1075 ≈ 5.375 J

\hrule
\vspace{1em}


\noindent
\textbf{Q633.} Scenario
: A mass
m
is connected to a ceiling via a light spring and executes simple harmonic motion in the vertical direction.
Question
: Which of the following statements is correct?



\textbf{A.} The elastic potential energy is maximum when the mass is at the highest point. \\
\textbf{B.} The elastic potential energy is minimum when the mass is at the lowest point. \\
\textbf{C.} The sum of the mass's kinetic energy, gravitational potential energy, and elastic potential energy remains constant during the motion. \\
\textbf{D.} The kinetic energy is zero when the mass passes through the equilibrium position. \\

\textbf{Answer:} C \\
\textbf{Explanation:} \cdot 
A is incorrect: At the highest point, the spring may be compressed, so elastic potential energy is not necessarily maximum.
\cdot 
B is incorrect: At the lowest point, the spring is stretched the most, so elastic potential energy is maximum.
\cdot 
C is correct: In vertical simple harmonic motion without air resistance, mechanical energy (kinetic, gravitational potential, and elastic potential energy) is conserved.
\cdot 
D is incorrect: At the equilibrium position, velocity is maximum, so kinetic energy is maximum.

\hrule
\vspace{1em}


\noindent
\textbf{Q634.} Scenario
: A mass
m
is connected to a fixed point on a smooth horizontal surface via a light spring. The mass executes simple harmonic motion on the surface.
Question
: Which of the following statements is correct?



\textbf{A.} The kinetic energy is minimum and the elastic potential energy is minimum when the mass passes through the equilibrium position. \\
\textbf{B.} The elastic potential energy is zero when the mass is at the maximum displacement. \\
\textbf{C.} The sum of the mass's kinetic energy and elastic potential energy remains constant during the motion. \\
\textbf{D.} The elastic potential energy is negative when the spring is compressed. \\

\textbf{Answer:} C \\
\textbf{Explanation:} \cdot 
A is
in
correct: At the equilibrium position, velocity is maximum, so kinetic energy is maximum and elastic potential energy is minimum.
\cdot 
B is incorrect: At maximum displacement, elastic potential energy is maximum.
\cdot 
C is correct: On a smooth surface, only the spring force does work, so the sum of kinetic and elastic potential energy is conserved.
\cdot 
D is incorrect: Elastic potential energy is always positive.

\hrule
\vspace{1em}


\noindent
\textbf{Q635.} An object weighing 50 N hangs from a light inextensible string of length 0.4 m attached to the ceiling. A constant force of 35 N is applied to the object at an angle of 37° above the horizontal, moving it to a new equilibrium position. By how much has the gravitational potential energy of the object increased?



\textbf{A.} 2.8 J \\
\textbf{B.} 3.5 J \\
\textbf{C.} 4.2 J \\
\textbf{D.} 5.6 J \\

\textbf{Answer:} C \\
\textbf{Explanation:} The object is in equilibrium under three forces: gravity (50 N downward), the applied force (35 N at 37°), and tension TT. Resolving forces:
[IMAGE:0]
[IMAGE:1]
[IMAGE:2]
[IMAGE:3]

\hrule
\vspace{1em}


\noindent
\textbf{Q636.} Scenario
: A mass
m
is hung from a vertical light spring with spring constant
k
. The mass is pulled down and released, executing simple harmonic motion.
Question
: Which of the following statements is correct?



\textbf{A.} The elastic potential energy is maximum when the mass is at the lowest point. \\
\textbf{B.} The kinetic energy is maximum and the elastic potential energy is minimum when the mass passes through the equilibrium position. \\
\textbf{C.} The sum of the mass's kinetic energy and elastic potential energy remains constant during the motion. \\
\textbf{D.} The acceleration of the mass is zero when the spring is stretched to its maximum length. \\

\textbf{Answer:} A \\
\textbf{Explanation:} \cdot 
A is correct: At the lowest point, the spring is stretched the most, so elastic potential energy is maximum.
\cdot 
B is incorrect: At the equilibrium position, kinetic energy is maximum, but elastic potential energy is not necessarily minimum.
\cdot 
C is incorrect: The sum of kinetic and elastic potential energy is only conserved if no other forces do work, but here gravity is also doing work.
\cdot 
D is incorrect: At maximum spring extension, the acceleration is not zero because the spring force exceeds gravity.

\hrule
\vspace{1em}


\noindent
\textbf{Q637.} An object weighing 60 N hangs from the end of a light inextensible string of length 0.5 m, which is attached to the ceiling. A constant horizontal force is applied to the object, causing it to move to a new equilibrium position where the string makes an angle of 30° with the vertical. By how much has the gravitational potential energy of the object increased?



\textbf{A.} 3.0 J \\
\textbf{B.} 4.02 J \\
\textbf{C.} 6.01 J \\
\textbf{D.} 7.53 J \\

\textbf{Answer:} B \\
\textbf{Explanation:} (English Version)
Determine the equilibrium position: The object reaches a new equilibrium position where the string makes a 30° angle with the vertical.
Force analysis: The object experiences a weight mg=60N, tension T in the string, and a horizontal force F. The equilibrium conditions are:
[IMAGE:0]
[IMAGE:1]
[IMAGE:2]
[IMAGE:3]
[IMAGE:4]
[IMAGE:5]
[IMAGE:6]

\hrule
\vspace{1em}


\noindent
\textbf{Q638.} What is the correct statement below:
1.
The elastic potential energy of a spring is directly proportional to its deformation.
2.
When an object undergoes simple harmonic motion
only
under the action of a spring, mechanical energy is conserved.
3.
In the presence of friction, the mechanical energy of the spring and object system is conserved.



\textbf{A.} 1 \\
\textbf{B.} 2 \\
\textbf{C.} 1 and 3 \\
\textbf{D.} 2 and 3 \\

\textbf{Answer:} B \\
\textbf{Explanation:} Option 1 is incorrect, elastic potential energy is proportional to the square of deformation; Option 2 is correct, mechanical energy is conserved in simple harmonic motion; Option 3 is incorrect, friction does negative work, so mechanical energy is not conserved.

\hrule
\vspace{1em}


\noindent
\textbf{Q639.} What is the correct statement below:
1.
The change in elastic potential energy is equal to the work done by the elastic force.
2.
When a spring oscillator vibrates, kinetic and elastic potential energy are converted into each other.
3.
In inelastic collisions, mechanical energy is conserved.



\textbf{A.} 1 \\
\textbf{B.} 2 \\
\textbf{C.} 1 and 2 \\
\textbf{D.} 2 and 3 \\

\textbf{Answer:} F \\
\textbf{Explanation:} Option 1 is correct, the change in elastic potential energy equals the work done by the elastic force; Option 2 is correct, kinetic and elastic potential energy are converted during vibration; Option 3 is incorrect, mechanical energy is not conserved in inelastic collisions.

\hrule
\vspace{1em}


\noindent
\textbf{Q640.} What is the correct statement below:
1.
The greater the spring constant, the greater the elastic potential energy for the same deformation.
2.
When an object moves under the action of a spring, the sum of kinetic and elastic potential energy remains constant.
3.
In the presence of air resistance, the mechanical energy of the spring and object system is conserved.



\textbf{A.} 1 \\
\textbf{B.} 2 \\
\textbf{C.} 1 and 3 \\
\textbf{D.} 2 and 3 \\

\textbf{Answer:} A \\
\textbf{Explanation:} Option 1 is correct, as shown by the elastic potential energy formula
[IMAGE:0]
Option 2 is incorrect, the sum of kinetic and elastic potential energy is only conserved in the absence of other forces doing work; Option 3 is incorrect, air resistance does negative work, so mechanical energy is not conserved.

\hrule
\vspace{1em}


\noindent
\textbf{Q641.} An object weighing 60 N hangs from a light inextensible string of length 0.35 m attached to the ceiling. A constant horizontal force of 45 N is applied to the object, moving it to a new equilibrium position where the string is no longer vertical. By how much has the gravitational potential energy of the object increased?



\textbf{A.} 2.1 J \\
\textbf{B.} 2.8 J \\
\textbf{C.} 3.5 J \\
\textbf{D.} 4.2 J \\

\textbf{Answer:} D \\
\textbf{Explanation:} The object is in equilibrium under three forces: gravity (60 N downward), horizontal force (45 N rightward), and tension TT along the string. Resolving forces:
[IMAGE:0]
[IMAGE:1]
[IMAGE:2]
Since cos36.87
\circ 
=0.8:
[IMAGE:3]
[IMAGE:4]

\hrule
\vspace{1em}


\noindent
\textbf{Q642.} What is the correct statement below:
1.
Elastic potential energy is a scalar quantity and its magnitude is independent of the path taken by the elastic force.
2.
When a spring is compressed, the elastic potential energy is negative.
3.
In elastic collisions, mechanical energy is conserved.



\textbf{A.} 1 \\
\textbf{B.} 2 \\
\textbf{C.} 1 and 3 \\
\textbf{D.} 2 and 3 \\

\textbf{Answer:} C \\
\textbf{Explanation:} Option 1 is correct, elastic potential energy is path-independent and depends only on initial and final positions; Option 2 is incorrect, elastic potential energy is always positive; Option 3 is correct, mechanical energy is conserved in elastic collisions.

\hrule
\vspace{1em}


\noindent
\textbf{Q643.} An object weighing 60 N hangs from the end of a light inextensible string of length 0.45 m, which is attached to the ceiling. A constant horizontal force of 36 N is applied to the object, causing it to move to a new equilibrium position where the string is no longer vertical. By how much has the gravitational potential energy of the object increased?



\textbf{A.} 3.0 J \\
\textbf{B.} 4.0 J \\
\textbf{C.} 5.4 J \\
\textbf{D.} 6.0 J \\

\textbf{Answer:} C \\
\textbf{Explanation:} Determine the equilibrium position: The object reaches a new equilibrium position where the string makes an angle θ with the vertical due to the horizontal force.
Force analysis: The horizontal force F=36N and weight mg=60N. The tension T in the string satisfies:
[IMAGE:0]
[IMAGE:1]
[IMAGE:2]
[IMAGE:3]
[IMAGE:4]

\hrule
\vspace{1em}


\noindent
\textbf{Q644.} What is the correct statement below:
1.
Within the elastic limit, the force exerted by a spring is proportional to its deformation.
2.
The elastic potential energy is directly proportional to the square of the spring's deformation.
3.
When a spring oscillator vibrates on a smooth horizontal surface, the mechanical energy of the system is conserved.



\textbf{A.} 1 \\
\textbf{B.} 2 \\
\textbf{C.} 1 and 3 \\
\textbf{D.} 2 and 3 \\

\textbf{Answer:} F \\
\textbf{Explanation:} Option 1 is correct, in accordance with Hooke's Law; Option 2 is correct, as the formula for elastic potential energy is
[IMAGE:0]
Option 3 is ambiguous because the question does not specify whether there are other forces doing work, but on a smooth surface, if there are no other forces, mechanical energy is conserved.

\hrule
\vspace{1em}


\noindent
\textbf{Q645.} Which statements are correct?
1.
Electromagnetic forces can transmit without a medium.
2.
The unit of force is the same as the unit of power.
3.
An accelerating object must have a non-zero net force.



\textbf{A.} 1 \\
\textbf{B.} 3 \\
\textbf{C.} 1 and 3 \\
\textbf{D.} 2 \\

\textbf{Answer:} C \\
\textbf{Explanation:} 1.
Correct. Electromagnetic forces act in a vacuum.
2.
Incorrect. Force is in Newtons (N), power in Watts (W).
3.
Correct. Acceleration requires a net force (Newton’s second law).

\hrule
\vspace{1em}


\noindent
\textbf{Q646.} A radium-226 nucleus (Ra-226) moving at velocity u undergoes alpha decay to form a radon-222 nucleus (Rn-222) and an alpha particle. In the laboratory frame, the total energy of the radium-226 nucleus is E. The kinetic energy ratio of the radon-222 nucleus to the alpha particle is 1:4. What is the kinetic energy of the alpha particle?



\textbf{A.} E​/5 \\
\textbf{B.} 4E​/5 \\
\textbf{C.} E​/4 \\
\textbf{D.} 4E​/226 \\

\textbf{Answer:} B \\
\textbf{Explanation:} [IMAGE:0]
Since the radium-226 is moving, relativistic effects are considered. The total energy E of radium-226 is:
[IMAGE:1]
The total kinetic energy K of the decay products is:
[IMAGE:2]
Using the kinetic energy ratio 1:4, the kinetic energy of radon-222 is K​/5 and that of the alpha particle is 4K​/5.

\hrule
\vspace{1em}


\noindent
\textbf{Q647.} Which statements are incorrect?
1.
Elastic forces are contact forces.
2.
Action and reaction forces can act on the same object.
3.
The unit of voltage is Newton (N).



\textbf{A.} 1 \\
\textbf{B.} 2 \\
\textbf{C.} 3 \\
\textbf{D.} 2 and 3 \\

\textbf{Answer:} D \\
\textbf{Explanation:} 1.
Correct. Elastic forces require contact.
2.
Incorrect. Action-reaction forces act on different objects.
3.
Incorrect. Voltage is measured in Volts (V).

\hrule
\vspace{1em}


\noindent
\textbf{Q648.} Which statements about force are correct?
1.
Frictional forces are always contact forces.
2.
The unit "Newton" can also describe electromotive force.
3.
A stationary object has a net force of zero.



\textbf{A.} 1 \\
\textbf{B.} 3 \\
\textbf{C.} 1 and 3 \\
\textbf{D.} 2 \\

\textbf{Answer:} C \\
\textbf{Explanation:} 1.
Correct. Friction requires contact between surfaces.
2.
Incorrect. Electromotive force is measured in Volts (V).
3.
Correct. A stationary object is in equilibrium with zero net force.

\hrule
\vspace{1em}


\noindent
\textbf{Q649.} A stationary actinium-227 (Ac-227) nucleus undergoes alpha decay to form francium-223 (Fr-223) and an alpha particle. After decay, the two particles leave tracks in a cloud chamber. If the track length of the alpha particle is 233/4 times that of Fr-223 (assuming equal motion time and constant resistance), and the total kinetic energy released is E, what is the kinetic energy of the alpha particle?



\textbf{A.} 3E/227 \\
\textbf{B.} 223E/227 \\
\textbf{C.} E/2 \\
\textbf{D.} 223E/(223+4) \\

\textbf{Answer:} B \\
\textbf{Explanation:} Track Length vs. Velocity: Under constant resistance and equal motion time, track length is proportional to velocity, i.e.,
[IMAGE:0]
Momentum Conservation: Total momentum after decay is zero, so m
α
v
α
=m
Fr
v
Fr
​. Substituting masses m
α
=4u, m
Fr
=223u:
[IMAGE:1]
[IMAGE:2]

\hrule
\vspace{1em}


\noindent
\textbf{Q650.} Regarding Earth's gravitational pull on the Moon, which statement is correct?
1.
Earth's gravitational pull on the Moon is greater than the Moon's pull on Earth.
2.
Earth's gravitational pull on the Moon and the Moon's pull on Earth act on the same object.
3.
Earth's gravitational pull on the Moon and the Moon's pull on Earth are an action-reaction pair.



\textbf{A.} 1 \\
\textbf{B.} 2 \\
\textbf{C.} 3 \\
\textbf{D.} None of the above \\

\textbf{Answer:} C \\
\textbf{Explanation:} 1.
According to Newton's third law, Earth's gravitational pull on the Moon and the Moon's pull on Earth are equal in magnitude.
2.
Earth's gravitational pull acts on the Moon, while the Moon's pull acts on Earth, so they do not act on the same object.
3.
Earth's gravitational pull on the Moon and the Moon's pull on Earth are an action-reaction pair.

\hrule
\vspace{1em}


\noindent
\textbf{Q651.} When a magnet attracts a nail, which statement is correct?
1.
The magnet's attractive force on the nail requires contact.
2.
The nail's attractive force on the magnet is less than the magnet's force on the nail.
3.
The magnet's force on the nail and the nail's force on the magnet are an action-reaction pair.



\textbf{A.} 1 \\
\textbf{B.} 2 \\
\textbf{C.} 3 \\
\textbf{D.} None of the above \\

\textbf{Answer:} C \\
\textbf{Explanation:} 1.
The magnet's attractive force on the nail does not require contact.
2.
According to Newton's third law, action and reaction forces are equal in magnitude.
3.
The magnet's force on the nail and the nail's force on the magnet are an action-reaction pair.

\hrule
\vspace{1em}


\noindent
\textbf{Q652.} A stationary radium-226 nucleus (Ra-226) undergoes alpha decay to form a radon-222 nucleus (Rn-222) and an alpha particle. The relative velocity between the radon-222 nucleus and the alpha particle after decay is v. What is the kinetic energy of the alpha particle?



\textbf{A.} [IMAGE:0] \\
\textbf{B.} [IMAGE:1] \\
\textbf{C.} [IMAGE:2] \\
\textbf{D.} [IMAGE:3] \\

\textbf{Answer:} D \\
\textbf{Explanation:} Conservation of Momentum
Before decay, the momentum of the radium-226 nucleus is zero. After decay, the momentum of the radon-222 nucleus and the alpha particle are equal in magnitude and opposite in direction. Let the mass of the alpha particle be 4 and the mass of radon-222 be 222. The conservation of momentum can be expressed as:
[IMAGE:0]
[IMAGE:1]
[IMAGE:2]

\hrule
\vspace{1em}


\noindent
\textbf{Q653.} Which statement about the nature of force is correct?
1.
Force always requires contact to produce an effect.
2.
The unit of force is the same as the unit of pressure.
3.
Force can change the state of motion of an object.



\textbf{A.} 1 \\
\textbf{B.} 2 \\
\textbf{C.} 3 \\
\textbf{D.} None of the above \\

\textbf{Answer:} C \\
\textbf{Explanation:} 1.
Force does not always require contact, such as gravitational and electromagnetic forces.
2.
The unit of force is the newton (N), while the unit of pressure is the pascal (Pa), which are different.
3.
Force can indeed change the state of motion of an object, as stated in Newton's first law.

\hrule
\vspace{1em}


\noindent
\textbf{Q654.} Which statement about action and reaction forces is correct?
1.
Action and reaction forces always act on the same object.
2.
Action and reaction forces can cancel each other out.
3.
Action and reaction forces always arise and disappear simultaneously.



\textbf{A.} 1 \\
\textbf{B.} 2 \\
\textbf{C.} 3 \\
\textbf{D.} None of the above \\

\textbf{Answer:} C \\
\textbf{Explanation:} 1.
Action and reaction forces act on different objects.
2.
Since they act on different objects, action and reaction forces cannot cancel each other out.
3.
Action and reaction forces always arise and disappear simultaneously.

\hrule
\vspace{1em}


\noindent
\textbf{Q655.} A stationary uranium-238 nucleus (U-238) undergoes alpha decay to form a thorium-234 nucleus (Th-234). Subsequently, the thorium-234 nucleus undergoes beta decay to form a protactinium-234 nucleus (Pa-234) and an electron. The total kinetic energy produced by the two decays is E. What is the kinetic energy of the alpha particle released in the first decay?



\textbf{A.} 4
E
​/238 \\
\textbf{B.} 4
E
​/234 \\
\textbf{C.} 234
E
​/238 \\
\textbf{D.} 234
E
​/(237 \\

\textbf{Answer:} C \\
\textbf{Explanation:} First Decay (U-238 \to  Th-234 + α particle); Second Decay (Th-234 \to  Pa-234 + electron); Total Kinetic Energy E = E₁ + E₂
[IMAGE:0]

\hrule
\vspace{1em}


\noindent
\textbf{Q656.} Which statement about the unit of force is correct?
1.
The international unit of force is kilogram-force (kgf).
2.
1 newton equals 1 kilogram\cdot meter/second².
3.
The unit of force is the same as the unit of pressure.



\textbf{A.} 1 \\
\textbf{B.} 2 \\
\textbf{C.} 3 \\
\textbf{D.} None of the above \\

\textbf{Answer:} B \\
\textbf{Explanation:} 1.
The international unit of force is the newton (N), while kilogram-force (kgf) is a common unit but not an international one.
2.
1 newton is indeed equal to 1 kilogram\cdot meter/second².
3.
The unit of force is the newton (N), while the unit of pressure is the pascal (Pa), which are different.

\hrule
\vspace{1em}


\noindent
\textbf{Q657.} When one object exerts a force on another object, what happens?
1.
The other object exerts an equal and oppositely directed force on the first object.
2.
The other object exerts an equal and similarly directed force on the first object.
3.
The other object exerts an unequal and oppositely directed force on the first object.



\textbf{A.} 1 \\
\textbf{B.} 2 \\
\textbf{C.} 3 \\
\textbf{D.} None of the above \\

\textbf{Answer:} A \\
\textbf{Explanation:} According to Newton's third law, action and reaction forces are always equal in magnitude, opposite in direction, and act on different objects.

\hrule
\vspace{1em}


\noindent
\textbf{Q658.} Which statement about force is correct?
1.
Gravitational and electromagnetic forces require contact to produce an effect.
2.
Action and reaction forces are always equal in magnitude, opposite in direction, and act on the same object.
3.
The international unit of force is kilogram-force (kgf).



\textbf{A.} 1 \\
\textbf{B.} 2 \\
\textbf{C.} 3 \\
\textbf{D.} None of the above \\

\textbf{Answer:} D \\
\textbf{Explanation:} 1.
Gravitational and electromagnetic forces do not require contact, such as Earth's gravity on objects and magnetic interactions.
2.
Action and reaction forces act on different objects, not the same one.
3.
The international unit of force is the newton (N), while kilogram-force (kgf) is a common unit but not an international one.

\hrule
\vspace{1em}


\noindent
\textbf{Q659.} A stationary americium-243 (Am-243) nucleus undergoes alpha decay to form neptunium-239 (Np-239) and an alpha particle. The total kinetic energy released in the decay is E, and the two particles move in opposite directions. What is the kinetic energy of the alpha particle?



\textbf{A.} 4E/243 \\
\textbf{B.} 239E/243 \\
\textbf{C.} E/2 \\
\textbf{D.} E/(239+4) \\

\textbf{Answer:} B \\
\textbf{Explanation:} By conservation of momentum, the magnitudes of the momenta of the alpha particle and Np-239 are equal. Let the mass of the alpha particle be 4u and that of Np-239 be 239u.
[IMAGE:0]

\hrule
\vspace{1em}


\noindent
\textbf{Q660.} If the net work done on an object is zero:
1.
Its kinetic energy remains constant;
2.
Its acceleration is zero;
3.
Its velocity direction may change.



\textbf{A.} 1 and 3 \\
\textbf{B.} Only 2 \\
\textbf{C.} Only 1 \\
\textbf{D.} All \\

\textbf{Answer:} A \\
\textbf{Explanation:} 1 is correct (work-energy theorem); 2 is wrong (net work zero does not imply zero net force, e.g., uniform circular motion); 3 is correct (direction can change). Thus, A is correct.

\hrule
\vspace{1em}


\noindent
\textbf{Q661.} A stationary radium-226 nucleus (Ra-226) undergoes alpha decay to form a radon-222 nucleus (Rn-222). Subsequently, the radon-222 nucleus undergoes another alpha decay to form a polonium-218 nucleus (Po-218). The total kinetic energy produced by the two decays is E. What is the kinetic energy of the alpha particle released in the second decay?



\textbf{A.} 4E​/226 \\
\textbf{B.} 4E​/218 \\
\textbf{C.} 218E​/226 \\
\textbf{D.} 218E​/222 \\

\textbf{Answer:} C \\
\textbf{Explanation:} (1)First Decay (Ra-226 \to  Rn-222 + α particle)
Initial momentum is zero. After decay, the momentum of Rn-222 and the α particle are equal in magnitude and opposite in direction. Mass of α particle = 4, mass of Rn-222 = 222.
[IMAGE:0]
(2)Second Decay (Rn-222 \to  Po-218 + α particle)
Initial momentum is that of Rn-222. After decay, the momentum of Po-218 and the α particle are equal in magnitude and opposite in direction.Mass of α particle = 4, mass of Po-218 = 218.
[IMAGE:1]
[IMAGE:2]

\hrule
\vspace{1em}


\noindent
\textbf{Q662.} Which statement about inertia is correct?
1.
Mass is the measure of inertia;
2.
Faster-moving objects have greater inertia;
3.
In zero gravity, inertia disappears.



\textbf{A.} Only 1 \\
\textbf{B.} 1 and 3 \\
\textbf{C.} Only 3 \\
\textbf{D.} None \\

\textbf{Answer:} A \\
\textbf{Explanation:} 1 is correct (inertia depends on mass); 2 is wrong (inertia is independent of velocity); 3 is wrong (inertia exists regardless of gravity). Thus, A is correct.

\hrule
\vspace{1em}


\noindent
\textbf{Q663.} For an object in uniformly accelerated linear motion:
1.
Acceleration magnitude increases with time;
2.
Velocity change is proportional to time;
3.
Displacement differences in equal time intervals are equal.



\textbf{A.} Only 2 \\
\textbf{B.} 2 and 3 \\
\textbf{C.} Only 3 \\
\textbf{D.} All \\

\textbf{Answer:} A \\
\textbf{Explanation:} 1 is wrong (acceleration is constant); 2 is correct (Δv = a\cdot Δt); 3 is wrong (displacement differences depend on time squared). Thus, A is correct.

\hrule
\vspace{1em}


\noindent
\textbf{Q664.} A stationary californium-252 (Cf-252) nucleus undergoes alpha decay to form curium-248 (Cm-248) and an alpha particle. The total kinetic energy released in the decay is E. If the alpha particle and Cm-248 move in opposite directions, and their kinetic energies are determined by momentum conservation, what is the kinetic energy of the alpha particle?



\textbf{A.} 4E/252 \\
\textbf{B.} 248E/252 \\
\textbf{C.} E/2 \\
\textbf{D.} 248E/(248+4) \\

\textbf{Answer:} B \\
\textbf{Explanation:} By conservation of momentum, the magnitudes of the momenta of the alpha particle and Cm-248 are equal. Let the mass of the alpha particle be 4u and that of Cm-248 be 248u. From momentum conservation: 4vα=248vCm  leading to the velocity ratio vα:vCm=248:4 Kinetic energy is proportional to mass and the square of velocity. Thus, the ratio of kinetic energies is:
[IMAGE:0]

\hrule
\vspace{1em}


\noindent
\textbf{Q665.} Which statement about acceleration and velocity is correct?
1.
Acceleration direction is always the same as velocity direction;
2.
Negative acceleration means the object must be slowing down;
3.
Greater rate of velocity change implies greater acceleration.



\textbf{A.} 1 and 3 \\
\textbf{B.} Only 3 \\
\textbf{C.} 2 and 3 \\
\textbf{D.} All \\

\textbf{Answer:} B \\
\textbf{Explanation:} 1 is wrong (acceleration direction matches velocity change direction, not velocity); 2 is wrong (negative acceleration may not imply deceleration if velocity direction is opposite); 3 is correct (acceleration is defined as rate of velocity change). Thus, B is correct.

\hrule
\vspace{1em}


\noindent
\textbf{Q666.} Which statement about Newton’s second law is correct?
1.
An object’s acceleration is inversely proportional to its mass;
2.
Under the same force, an object with larger mass has greater acceleration;
3.
If an object’s acceleration is zero, the net force on it must be zero.



\textbf{A.} 1 and 2 \\
\textbf{B.} 1 and 3 \\
\textbf{C.} 2 and 3 \\
\textbf{D.} Only 3 \\

\textbf{Answer:} B \\
\textbf{Explanation:} 1 is correct (from
a=F/m
, larger mass leads to smaller acceleration); 2 is wrong (larger mass reduces acceleration); 3 is correct (zero acceleration implies zero net force). Thus, B is correct.

\hrule
\vspace{1em}


\noindent
\textbf{Q667.} A stationary nitrogen molecule (N₂) decomposes into two nitrogen atoms (N) at high temperatures. The total kinetic energy of the two nitrogen atoms after decomposition is E, and the mass of each nitrogen atom is 14u (atomic mass units). What is the kinetic energy of one of the nitrogen atoms?



\textbf{A.} E/
14​ \\
\textbf{B.} E
​/14 \\
\textbf{C.} 14
E
​/28 \\
\textbf{D.} 4
E
​/141 \\

\textbf{Answer:} A \\
\textbf{Explanation:} Before decomposition, the nitrogen molecule is stationary, so its momentum is zero. After decomposition, the two nitrogen atoms have equal magnitude but opposite direction momenta. Let the mass of each nitrogen atom be 14u, and their velocities be v₁ and v₂. By the conservation of momentum:
[IMAGE:0]
Since m₁ = m₂ = 14u, this simplifies to:
[IMAGE:1]
he total kinetic energy is E, so:
[IMAGE:2]
Substituting m₁ = m₂ = 14u and v₁ = -v₂, we get:
[IMAGE:3]
Therefore, the kinetic energy of one nitrogen atom is:
[IMAGE:4]

\hrule
\vspace{1em}


\noindent
\textbf{Q668.} Which of the following statements about acceleration and velocity is correct:
1.
The greater the acceleration, the faster the velocity changes;
2.
The direction of acceleration is the same as the direction of the change in velocity;
3.
When acceleration is zero, velocity must be zero;



\textbf{A.} 1and2 \\
\textbf{B.} 2 \\
\textbf{C.} 3 \\
\textbf{D.} 1and3 \\

\textbf{Answer:} A \\
\textbf{Explanation:} Acceleration being greater does mean velocity changes faster (definition), and the direction of acceleration is indeed the same as the direction of the change in velocity, so options 1 and 2 are correct. Option 3 is incorrect because acceleration being zero does not mean velocity is zero; for example, in uniform linear motion.

\hrule
\vspace{1em}


\noindent
\textbf{Q669.} Which of the following statements about mass is correct:
1.
Mass is a measure of an object's inertia;
2.
The greater the mass, the greater the object's inertia;
3.
Mass is directly proportional to an object's acceleration;



\textbf{A.} 1and2 \\
\textbf{B.} 2 \\
\textbf{C.} 3 \\
\textbf{D.} 1and3 \\

\textbf{Answer:} A \\
\textbf{Explanation:} Mass is indeed a measure of an object's inertia, and the greater the mass, the greater the inertia, so options 1 and 2 are correct. Option 3 is incorrect because mass is inversely proportional to acceleration (Newton's second law).

\hrule
\vspace{1em}


\noindent
\textbf{Q670.} Which of the following statements about force and motion is correct:
1.
Force is the cause of changing an object's state of motion;
2.
Force is the cause of maintaining an object's motion;
3.
The greater the net force acting on an object, the greater its acceleration;



\textbf{A.} 1and2 \\
\textbf{B.} 2 \\
\textbf{C.} 3 \\
\textbf{D.} 1and3 \\

\textbf{Answer:} D \\
\textbf{Explanation:} Force is indeed the cause of changing an object's state of motion (Newton's first law), and the greater the net force, the greater the acceleration (Newton's second law), so options 1 and 3 are correct. Option 2 is incorrect because force is not the cause of maintaining motion but rather the cause of changing it.

\hrule
\vspace{1em}


\noindent
\textbf{Q671.} When a stationary radium-226 nucleus decays by alpha emission to form a radon-222 nucleus, the total kinetic energy produced by the decay is E. What is the kinetic energy of the alpha particle?



\textbf{A.} 4E​/226 \\
\textbf{B.} 4E​/222 \\
\textbf{C.} 2E​ \\
\textbf{D.} 226/222E​ \\

\textbf{Answer:} D \\
\textbf{Explanation:} By the conservation of momentum, the momentum of the radium-226 nucleus before decay is zero. After decay, the momentum of the radon-222 nucleus and the alpha particle are equal in magnitude and opposite in direction. Let the mass of the alpha particle be 4 and the mass of radon-222 be 222. The conservation of momentum can be expressed as:
[IMAGE:0]
[IMAGE:1]
Simplifying this equation results in:
[IMAGE:2]
Therefore, the kinetic energy of the alpha particle is:
[IMAGE:3]

\hrule
\vspace{1em}


\noindent
\textbf{Q672.} Which of the following statements about acceleration is correct:
1.
Acceleration is a measure of the change in velocity;
2.
Acceleration is the result of a force acting on an object;
3.
The greater the acceleration, the greater the object's velocity must be;



\textbf{A.} 1and2 \\
\textbf{B.} 2 \\
\textbf{C.} 3 \\
\textbf{D.} 1and3 \\

\textbf{Answer:} A \\
\textbf{Explanation:} Acceleration is indeed a measure of the change in velocity (definition) and is the result of a force acting on an object (Newton's second law), so options 1 and 2 are correct. Option 3 is incorrect because greater acceleration means velocity changes faster, but it does not necessarily mean the velocity is greater.

\hrule
\vspace{1em}


\noindent
\textbf{Q673.} Which of the following statements about acceleration is correct:
1.
Acceleration is directly proportional to an object's mass;
2.
Acceleration is directly proportional to the net force acting on the object;
3.
The direction of acceleration is the same as the direction of velocity;



\textbf{A.} 1 and 2 \\
\textbf{B.} 2 \\
\textbf{C.} 3 \\
\textbf{D.} 1 and 3 \\

\textbf{Answer:} B \\
\textbf{Explanation:} Acceleration is inversely proportional to an object's mass (Newton's second law), so option 1 is incorrect. The direction of acceleration is the same as the direction of the net force, not necessarily the same as the direction of velocity, so option 3 is incorrect. Option 2 is correct and aligns with Newton's second law.

\hrule
\vspace{1em}


\noindent
\textbf{Q674.} A stationary plutonium-242 (Pu-242) nucleus decays by alpha emission to form a uranium-238 (U-238) nucleus and an alpha particle. The total kinetic energy produced by the decay is E.
What is the kinetic energy of the alpha particle?



\textbf{A.} 4E/242 \\
\textbf{B.} 4E/238 \\
\textbf{C.} E/2 \\
\textbf{D.} 238E/242 \\

\textbf{Answer:} D \\
\textbf{Explanation:} By conservation of momentum, the magnitudes of the momenta of U-238 and the alpha particle are equal. Let the mass of the alpha particle be 4u and that of U-238 be 238u.
From momentum conservation: 4vα=238vU4
vα
​=238
v
U​, leading to the velocity ratio v
α
:v
U
=238:4. Kinetic energy is proportional to mass and the square of velocity. Thus, the ratio of kinetic energies is:
[IMAGE:0]
[IMAGE:1]

\hrule
\vspace{1em}


\noindent
\textbf{Q675.} A line with gradient 2 is first reflected in y=
x
, then reflected in the x-axis. What is the final gradient?



\textbf{A.} m \\
\textbf{B.} -m \\
\textbf{C.} 1/m \\
\textbf{D.} -1/m \\

\textbf{Answer:} B \\
\textbf{Explanation:} [IMAGE:0]

\hrule
\vspace{1em}


\noindent
\textbf{Q676.} A line with a non-zero gradient
m
is reflected in the line
y
=
x
and makes an angle of
30
\circ 
with the original line.
And |m|<1;
What is the value of
m
?



\textbf{A.} [IMAGE:0] \\
\textbf{B.} [IMAGE:1] \\
\textbf{C.} [IMAGE:2] \\
\textbf{D.} [IMAGE:3] \\

\textbf{Answer:} D \\
\textbf{Explanation:} [IMAGE:0]

\hrule
\vspace{1em}


\noindent
\textbf{Q677.} In a sealed container, the temperature of a gas remains constant. When the volume is 4 m34m3, the pressure is 6000 Pa6000Pa. If the volume expands to 6 m36m3, what is the new pressure?



\textbf{A.} 4000 Pa \\
\textbf{B.} 3000 Pa \\
\textbf{C.} 2000 Pa \\
\textbf{D.} 1000 Pa \\

\textbf{Answer:} A \\
\textbf{Explanation:} By Boyle’s law, at constant temperature, the pressure PP of a gas is inversely proportional to its volume
[IMAGE:0]
[IMAGE:1]

\hrule
\vspace{1em}


\noindent
\textbf{Q678.} A line with a non-zero gradient
m
is reflected in the line
y
=
x
and is parallel to the line
y
=3
x
−
4. What is the value of
m
?



\textbf{A.} 3 \\
\textbf{B.} 1/3 \\
\textbf{C.} -3 \\
\textbf{D.} -1/3 \\

\textbf{Answer:} B \\
\textbf{Explanation:} The gradient of the reflected line is
[IMAGE:0]
If it is parallel to
y
=3
x
−
4, their gradients must be equ
al. Thus,
[IMAGE:1]

\hrule
\vspace{1em}


\noindent
\textbf{Q679.} A circuit contains a variable resistor whose resistance R can be adjusted. When the resistance is 10Ω, the current flowing through the circuit is 2A. Assuming the voltage of the power supply remains constant, what is the current when the resistance is increased to 20Ω?



\textbf{A.} 0.5A \\
\textbf{B.} 1.0A \\
\textbf{C.} 1.5A \\
\textbf{D.} 2.0A \\

\textbf{Answer:} B \\
\textbf{Explanation:} According to Ohm's Law, the current I is inversely proportional to the resistance R:
[IMAGE:0]
where V is the constant voltage of the power supply. Given R=10Ω and I=2A, the voltage is:
[IMAGE:1]
When the resistance is increased to 20Ω, the current becomes:
[IMAGE:2]

\hrule
\vspace{1em}


\noindent
\textbf{Q680.} A line with a non-zero gradient
m
is reflected in the line
y
=
x
and is perpendicular to the line
[IMAGE:0]
What is the value of
m
?



\textbf{A.} 2 \\
\textbf{B.} -2 \\
\textbf{C.} 0.5 \\
\textbf{D.} -0.5 \\

\textbf{Answer:} C \\
\textbf{Explanation:} [IMAGE:0]

\hrule
\vspace{1em}


\noindent
\textbf{Q681.} A cylinder contains a fixed amount of ideal gas. When the piston is at the midpoint, the volume of the gas is 500cm
3
and the pressure is 1.5atm. If the piston is slowly compressed, reducing the gas volume to 300cm
3,
what is the new pressure of the gas, assuming the temperature remains constant?



\textbf{A.} 1.0atm \\
\textbf{B.} 1.5atm \\
\textbf{C.} 2.0atm \\
\textbf{D.} 2.5atm \\

\textbf{Answer:} D \\
\textbf{Explanation:} According to Boyle's Law, at constant temperature, the pressure of an ideal gas is inversely proportional to its volume:
[IMAGE:0]
[IMAGE:1]
[IMAGE:2]
[IMAGE:3]

\hrule
\vspace{1em}


\noindent
\textbf{Q682.} A line with a non-zero gradient
m
is reflected in the line
y
=
x
and coincides with the original line. What is the value of
m
?



\textbf{A.} 1 \\
\textbf{B.} -1 \\
\textbf{C.} 2 \\
\textbf{D.} -2 \\

\textbf{Answer:} B \\
\textbf{Explanation:} If the reflected line coincides with the original line, the original line must be symmetric about
y
=
-
x
+b
.
m=1/m=-1

\hrule
\vspace{1em}


\noindent
\textbf{Q683.} A line with a non-zero gradient
m
is reflected in the line
y
=
x
and is parallel to the line
y
=2
x
+3. What is the value of
m
?



\textbf{A.} 2 \\
\textbf{B.} 0.5 \\
\textbf{C.} -2 \\
\textbf{D.} -0.5 \\

\textbf{Answer:} B \\
\textbf{Explanation:} [IMAGE:0]

\hrule
\vspace{1em}


\noindent
\textbf{Q684.} In a hydraulic system, the force F remains constant. Piston A has an area of 0.2 m
2
and generates a pressure of 5000 Pa. If replaced by Piston B with an area of 0.5 m
2,
what is the new pressure?



\textbf{A.} 2000 Pa \\
\textbf{B.} 2500 Pa \\
\textbf{C.} 4000 Pa \\
\textbf{D.} 5000 Pa \\

\textbf{Answer:} A \\
\textbf{Explanation:} Inverse Relationship
:
Pressure PP is inversely proportional to the area A
[IMAGE:0]
Substituting P1=5000 Pa,
A
1​=0.2m
2
, A2=0.5 m
2
:
[IMAGE:1]

\hrule
\vspace{1em}


\noindent
\textbf{Q685.} A line with a gradient
m
is reflected in the line
y
=
x
and is perpendicular to the original line. What is the value of
m
?



\textbf{A.} 1 \\
\textbf{B.} -1 \\
\textbf{C.} 0.5 \\
\textbf{D.} 0 \\

\textbf{Answer:} D \\
\textbf{Explanation:} By using the graphical method, it can be known that this straight line is parallel to the x-axis; that is, its slope is 0. If it is parallel to the y-axis, then its slope does not exist.

\hrule
\vspace{1em}


\noindent
\textbf{Q686.} A line with a non-zero gradient
m
is reflected in the line
y
=
−
x
. What is the gradient of the reflected line?



\textbf{A.} m \\
\textbf{B.} -m \\
\textbf{C.} 1/m \\
\textbf{D.} −
1
/m \\

\textbf{Answer:} D \\
\textbf{Explanation:} [IMAGE:0]

\hrule
\vspace{1em}


\noindent
\textbf{Q687.} The gravitational force F between the Earth and the Moon is inversely proportional to the square of the distance r between them. When the Earth and Moon are 3.84×10
5
km apart, the force is 2×10
20
N. What is the distance between them when the gravitational force decreases to 5×10
19
N?



\textbf{A.} 1.92×10
5
km \\
\textbf{B.} 3.84×10
5
km \\
\textbf{C.} 5.76×10
5
km \\
\textbf{D.} 7.68×10
5
km \\

\textbf{Answer:} D \\
\textbf{Explanation:} By Newton's Law of Gravitation, the gravitational force is inversely proportional to the square of the distance:
[IMAGE:0]
where k is a constant.
Substituting
k
:
[IMAGE:1]
Simplifying:
[IMAGE:2]

\hrule
\vspace{1em}


\noindent
\textbf{Q688.} A line with a non-zero gradient
m
is reflected in the line
y
=
x
. What is the gradient of the reflected line?



\textbf{A.} m \\
\textbf{B.} −
m \\
\textbf{C.} 1
/m
​ \\
\textbf{D.} −
1
/m
​ \\

\textbf{Answer:} C \\
\textbf{Explanation:} [IMAGE:0]

\hrule
\vspace{1em}


\noindent
\textbf{Q689.} A car starts from rest and accelerates uniformly at a=2 m/s
2
for 5 seconds. Then it moves at a constant speed until the total time reaches 15 seconds. If the total distance traveled is 125 meters, what is the constant speed during the uniform motion phase?



\textbf{A.} 5 m/s \\
\textbf{B.} 8 m/s \\
\textbf{C.} 10 m/s \\
\textbf{D.} 12 m/s \\

\textbf{Answer:} C \\
\textbf{Explanation:} Acceleration Phase (0-5 seconds):
Distance s1=0.5
\cdot 
2
\cdot 
52=25 m, final velocity v=2
\cdot 
5=10 m/s.
Uniform Motion Phase (5-15 seconds):
Time t
2
=10 s, distance s
2
=10
\cdot 
v.
Total Distance:
25+10v=125
\implies 
10v=100
\implies 
v=10 m/s.
Thus, the correct answer is C.

\hrule
\vspace{1em}


\noindent
\textbf{Q690.} The electrostatic force F between two point charges is inversely proportional to the square of the distance r between them. When the charges are 3m apart, the force is 12N. What is the distance between the charges when the force becomes 27N?



\textbf{A.} 0.5m \\
\textbf{B.} 1m \\
\textbf{C.} 2m \\
\textbf{D.} 2.5m \\

\textbf{Answer:} C \\
\textbf{Explanation:} By Coulomb's Law, the electrostatic force is inversely proportional to the square of the distance:
[IMAGE:0]
where k is a constant. When F=12N and r=3m: k=F
\cdot 
r
2
=12×32 =108
For
F
=27N:
[IMAGE:1]

\hrule
\vspace{1em}


\noindent
\textbf{Q691.} The velocity v of an object is inversely proportional to the square root of time t. When v=12m/s, t=4s. What is the value of t when v=6m/s?



\textbf{A.} 16/9s \\
\textbf{B.} 4s \\
\textbf{C.} 9/16s \\
\textbf{D.} 27/14s \\

\textbf{Answer:} F \\
\textbf{Explanation:} Velocity v is inversely proportional to the square root of time t, which can be expressed as
[IMAGE:0]
where k is a constant. When v=12m/s and t=4s, substituting into the equation gives:
[IMAGE:1]
When v=6m/s, substituting into the equation gives:
[IMAGE:2]

\hrule
\vspace{1em}


\noindent
\textbf{Q692.} Which statement about gravity and mass is correct?



\textbf{A.} Objects of the same mass have the same weight everywhere on Earth. \\
\textbf{B.} gravity
is a vector quantity directed vertically downward. \\
\textbf{C.} Objects in space are weightless and have zero mass. \\
\textbf{D.} Gravity is the same as Earth's gravitational pull. \\

\textbf{Answer:} B \\
\textbf{Explanation:} Option A is incorrect. Weight
W
=
mg
varies with location due to varying
g
.
Option B is correct.
Gravity
is a vector that points toward the Earth's center.
Option C is wrong. Mass remains constant. Weightlessness refers to apparent weight, not actual mass.
Option D is wrong. Gravity is the force we experience, which is slightly different from Earth's gravitational pull due to Earth's rotation.

\hrule
\vspace{1em}


\noindent
\textbf{Q693.} The quantities a and b are positive. aa is inversely proportional to the square root of b. When a=4, b=16. What is the value of b when a=8?



\textbf{A.} 1 \\
\textbf{B.} 4 \\
\textbf{C.} 8 \\
\textbf{D.} 16 \\

\textbf{Answer:} C \\
\textbf{Explanation:} Since
$𝑎$
is inversely proportional to the square root of b, the relationship is:
[IMAGE:0]
[IMAGE:1]
[IMAGE:2]

\hrule
\vspace{1em}


\noindent
\textbf{Q694.} Which statement about falling objects is correct?



\textbf{A.} All objects fall with an acceleration of 9.8m/s². \\
\textbf{B.} Objects of different masses fall with the same acceleration in a vacuum. \\
\textbf{C.} A feather falls slower than an iron ball in air because of different accelerations. \\
\textbf{D.} Gravitational acceleration changes with the shape of an object. \\

\textbf{Answer:} B \\
\textbf{Explanation:} Option A is incorrect. In air, objects experience air resistance, so their acceleration is less than 9.8m/s².
Option B is correct. In a vacuum, all objects fall with the same acceleration due to gravity.
Option C is wrong. The acceleration is the same, but air resistance affects the feather more, resulting in lower net acceleration.
Option D is wrong. Gravitational acceleration is unrelated to object shape but depends on altitude and latitude.

\hrule
\vspace{1em}


\noindent
\textbf{Q695.} The quantities
x
and y are positive. x is inversely proportional to the square of
y
. When x=3, y=4. What is the value of
y
when x=12
x
=12?



\textbf{A.} 1 \\
\textbf{B.} 2 \\
\textbf{C.} 8 \\
\textbf{D.} 16 \\

\textbf{Answer:} B \\
\textbf{Explanation:} Since x is inversely proportional to the square of y, the relationship is:
[IMAGE:0]
[IMAGE:1]
[IMAGE:2]

\hrule
\vspace{1em}


\noindent
\textbf{Q696.} Which statement about gravitational acceleration is correct?



\textbf{A.} Gravitational acceleration is 9.8m/s² everywhere on Earth. \\
\textbf{B.} The higher the altitude, the smaller the gravitational acceleration. \\
\textbf{C.} Gravitational acceleration changes with the mass of an object. \\
\textbf{D.} Gravitational acceleration is independent of latitude. \\

\textbf{Answer:} B \\
\textbf{Explanation:} Option A is incorrect. Gravitational acceleration varies with location. It's about 9.83m/s² at the poles and 9.78m/s² at the equator.
Option B is correct. Gravitational acceleration decreases with increasing altitude because you're farther from the Earth's center.
Option C is wrong. Gravitational acceleration depends on Earth's mass and radius, not the object's mass.
Option D is wrong. Latitude affects gravitational acceleration due to Earth's rotation and shape.

\hrule
\vspace{1em}


\noindent
\textbf{Q697.} Which statement about gravity on Earth is correct?



\textbf{A.} The magnitude of gravity is the same everywhere on Earth. \\
\textbf{B.} Gravity decreases with increasing altitude. \\
\textbf{C.} The gravity at the Earth's poles is less than at the equator. \\
\textbf{D.} Gravity is inversely proportional to the mass of an object. \\

\textbf{Answer:} B \\
\textbf{Explanation:} Option A is incorrect because gravity varies slightly depending on location. It's slightly stronger at the poles and weaker at the equator due to Earth's oblate shape.
Option B is correct. As altitude increases, distance from the Earth's center increases, so gravity decreases according to the inverse square law.
Option C is wrong. The poles are closer to the Earth's center due to the oblate shape, so gravity is stronger at the poles.
Option D is wrong. Gravity is directly proportional to mass.
F
=
mg
shows that force increases with mass.

\hrule
\vspace{1em}


\noindent
\textbf{Q698.} A 2 kg object is initially at rest on a smooth horizontal surface. Starting at t=0, the object is subjected to a horizontal force that varies with time as follows:
From 0 to 0.05 s, the force increases linearly from 0 to 10 N;
From 0.05 to 0.10 s, the force remains constant at 10 N;
From 0.10 to 0.15 s, the force decreases linearly back to 0.
What is the kinetic energy of the object at t=0.15 s?



\textbf{A.} 0 J \\
\textbf{B.} 0.50 J \\
\textbf{C.} 1.25 J \\
\textbf{D.} 2.50 J \\

\textbf{Answer:} D \\
\textbf{Explanation:} The kinetic energy is equal to the work done by the force, which is the integral of the
force over time. From 0 to 0.05 s: Force increases linearly from 0 to 10 N (triangle).
From 0.05 to 0.10 s: Force remains constant at 10 N (rectangle).
From 0.10 to 0.15 s: Force decreases linearly from 10 N to 0 (triangle).

\hrule
\vspace{1em}


\noindent
\textbf{Q699.} Which statement about the center of gravity is correct?



\textbf{A.} The center of gravity must be at the geometric center of the object. \\
\textbf{B.} Only regular-shaped objects have centers of gravity. \\
\textbf{C.} The center of gravity can be outside the object. \\
\textbf{D.} The position of the center of gravity is independent of the object's shape. \\

\textbf{Answer:} C \\
\textbf{Explanation:} Option A is incorrect because the center of gravity may not be at the geometric center of the object, as in cases of irregular or asymmetric shapes.
Option B states only regular-shaped objects have centers of gravity, which is wrong. All objects have a center of gravity.
Option C is correct. For example, when you bend a paperclip, the center of gravity is not on the paperclip itself.
Option D is wrong because the position of the center of gravity relates to the shape of the object. When you peel an orange, the center of gravity of the orange shifts.

\hrule
\vspace{1em}


\noindent
\textbf{Q700.} An object of mass 2 kg is at rest at time t=0. A resultant force acts on the object in a constant direction. The magnitude of the resultant force acting on the object varies with time as follows:
The force starts at 0 N, increases to 10 N at t=0.1 s and remains constant until t=0.2 s, then increases linearly to 20 N, and finally decreases linearly to 0 N at t=0.3 s.
What is the kinetic energy of the object at time t=0.3 s?



\textbf{A.} 0 J \\
\textbf{B.} 2.34 J \\
\textbf{C.} 4.02J \\
\textbf{D.} 3.06 J \\

\textbf{Answer:} C \\
\textbf{Explanation:} The kinetic energy equals the work done by the force; the integration of force over time equals the change in momentum. The graph shows that the force increases from 0 N to 10 N at t=0.1 s and remains constant until t=0.2 s, then increases linearly to 20 N, and finally decreases linearly to 0 N at t=0.3 s.

\hrule
\vspace{1em}


\noindent
\textbf{Q701.} Which statements about weight are correct?
1.
Weight is a vector.
2.
Weight equals mass multiplied by 9.8.
3.
An object in free fall has zero weight.



\textbf{A.} 1 \\
\textbf{B.} 2 \\
\textbf{C.} 3 \\
\textbf{D.} 1 and 3 \\

\textbf{Answer:} D \\
\textbf{Explanation:} 1.
Correct. Weight is a vector aligned with gravitational acceleration.
2.
Incorrect. Weight is mass multiplied by actual gravitational acceleration (not necessarily 9.8).
3.
Correct. An object in free fall experiences apparent weightlessness.

\hrule
\vspace{1em}


\noindent
\textbf{Q702.} An object of mass 50 kg is at rest at time t=0. A resultant force acts on the object in a constant direction. The magnitude of the resultant force acting on the object varies with time as follows:
The force starts at 0 N, increases to 100 N at t=0.1 s, and then decreases linearly to 0 N at t=0.2 s.
What is the kinetic energy of the object at time t=0.2 s?



\textbf{A.} 0 J \\
\textbf{B.} 5 J \\
\textbf{C.} 1 J \\
\textbf{D.} 2 J \\

\textbf{Answer:} C \\
\textbf{Explanation:} The kinetic energy equals the work done by the force; the integration of force over time equals the change in momentum. The graph shows that the force increases from 0 N to 100 N at t=0.1 s and then decreases linearly to 0 N at t=0.2 s.
[IMAGE:0]
[IMAGE:1]

\hrule
\vspace{1em}


\noindent
\textbf{Q703.} Which statement is incorrect?
1.
The position of the center of gravity depends on the object’s shape.
2.
Gravity is perfectly uniform on Earth’s surface.
3.
The value of gravitational acceleration is independent of an object’s mass.



\textbf{A.} 1 \\
\textbf{B.} 2 \\
\textbf{C.} 3 \\
\textbf{D.} 2 and 3 \\

\textbf{Answer:} B \\
\textbf{Explanation:} 1.
Correct. The center of gravity varies with shape.
2.
Incorrect. Gravity varies due to Earth’s topography and density.
3.
Correct. Gravitational acceleration is mass-independent.

\hrule
\vspace{1em}


\noindent
\textbf{Q704.} Which statement about gravitational acceleration is correct?
1.
In a vacuum, light and heavy objects have the same gravitational acceleration.
2.
Gravitational acceleration is independent of altitude.
3.
Gravitational acceleration is smaller at the poles than at the equator.



\textbf{A.} 1 \\
\textbf{B.} 2 \\
\textbf{C.} 3 \\
\textbf{D.} 1 and 3 \\

\textbf{Answer:} A \\
\textbf{Explanation:} 1.
Correct. In a vacuum, all objects experience the same gravitational acceleration (neglecting air resistance).
2.
Incorrect. Gravitational acceleration decreases with altitude.
3.
Incorrect. Gravitational acceleration is greater at the poles due to reduced centrifugal force.

\hrule
\vspace{1em}


\noindent
\textbf{Q705.} A rocket of mass 100 kg is launched with a varying thrust. The thrust varies with time as follows:
The thrust starts at 0 N, increases to 200 N at t=0.1 s, and then decreases linearly to 0 N at t=0.2 s.
Ignoring air resistance and other forces, what is the kinetic energy of the rocket at t=0.2 s?



\textbf{A.} 0 J \\
\textbf{B.} 1J \\
\textbf{C.} 2J \\
\textbf{D.} 3J \\

\textbf{Answer:} C \\
\textbf{Explanation:} The kinetic energy equals the work done by the force; the integration of force over time equals the change in momentum. The graph shows that the thrust increases from 0 N to 200 N at t=0.1 s and then decreases linearly to 0 N at t=0.2 s.
[IMAGE:0]
[IMAGE:1]
[IMAGE:2]

\hrule
\vspace{1em}


\noindent
\textbf{Q706.} Which statement is correct?
The position of the center of gravity depends on the object’s shape.
Gravity is perfectly uniform on Earth’s surface.
The value of gravitational acceleration is independent of an object’s mass.



\textbf{A.} 1 \\
\textbf{B.} 2 \\
\textbf{C.} 3 \\
\textbf{D.} 2 and 3 \\

\textbf{Answer:} B \\
\textbf{Explanation:} Correct. The center of gravity varies with shape.
Incorrect. Gravity varies due to Earth’s topography and density.
Correct. Gravitational acceleration is mass-independent.

\hrule
\vspace{1em}


\noindent
\textbf{Q707.} Which statement about the center of gravity is correct?
1.
The center of gravity must be inside the object.
2.
The center of gravity is the equivalent point where gravity acts.
3.
Gravitational acceleration is always 9.8 m/s² on Earth.



\textbf{A.} 1 \\
\textbf{B.} 2 \\
\textbf{C.} 3 \\
\textbf{D.} 1 and 2 \\

\textbf{Answer:} B \\
\textbf{Explanation:} 1.
Incorrect. The center of gravity can be outside the object (e.g., a ring’s center of gravity is at its center).
2.
Correct. The center of gravity is the equivalent point of gravitational action.
3.
Incorrect. Gravitational acceleration varies with latitude and altitude

\hrule
\vspace{1em}


\noindent
\textbf{Q708.} The light intensity is reduced by 50% after passing through a medium. A student incorrectly calculates the original intensity by increasing the reduced value by 60%, resulting in an error of
23
lux. What is the correct original intensity?



\textbf{A.} 115
lux \\
\textbf{B.} 120 lux \\
\textbf{C.} 150 lux \\
\textbf{D.} 180 lux \\

\textbf{Answer:} A \\
\textbf{Explanation:} [IMAGE:0]

\hrule
\vspace{1em}


\noindent
\textbf{Q709.} A 3 kg object is initially at rest on a smooth horizontal surface. Starting at t=0, the object is subjected to a horizontal force that varies with time as follows:
From 0 to 0.06 s, the force increases linearly from 0 to 6 N;
From 0.06 to 0.12 s, the force remains constant at 6 N;
From 0.12 to 0.18 s, the force decreases linearly back to 0.
What is the kinetic energy of the object at t=0.18 s?



\textbf{A.} 0 J \\
\textbf{B.} 0.72 J \\
\textbf{C.} 1.44 J \\
\textbf{D.} 2.16 J \\

\textbf{Answer:} C \\
\textbf{Explanation:} The kinetic energy is equal to the work done by the force, which is the integral of the force over time (the area under the force-time graph).

\hrule
\vspace{1em}


\noindent
\textbf{Q710.} The power of a machine is reduced by 35% due to friction. A technician incorrectly restores the original power by increasing the reduced value by 30%, resulting in an error of
15.5
W. What is the correct original power?



\textbf{A.} 140W \\
\textbf{B.} 163W \\
\textbf{C.} 182W \\
\textbf{D.} 100W \\

\textbf{Answer:} D \\
\textbf{Explanation:} [IMAGE:0]

\hrule
\vspace{1em}


\noindent
\textbf{Q711.} The intensity of a sound wave is reduced by 40% due to obstacles. A student incorrectly calculates the original intensity by increasing the reduced value by 50%, resulting in an error of 2
8
W/m². What is the correct original intensity?



\textbf{A.} 120 W/m² \\
\textbf{B.} 150 W/m² \\
\textbf{C.} 2
80 W/m² \\
\textbf{D.} 200 W/m² \\

\textbf{Answer:} C \\
\textbf{Explanation:} [IMAGE:0]

\hrule
\vspace{1em}


\noindent
\textbf{Q712.} A ball A with a mass of 2 kg moves at 5 m/s and collides elastically with a stationary ball B of mass 3 kg. After the collision, ball A's velocity becomes -1 m/s (opposite direction). What is the velocity of ball B after the collision in meters per second?



\textbf{A.} 1.0 m/s \\
\textbf{B.} 2.0 m/s \\
\textbf{C.} 3.0 m/s \\
\textbf{D.} 4.0 m/s \\

\textbf{Answer:} D \\
\textbf{Explanation:} In an elastic collision, both momentum and kinetic energy are conserved. Using the conservation of momentum:
[IMAGE:0]
Since kinetic energy is conserved, the answer is correct. The velocity of ball B after the collision is 4 m/s, corresponding to option D.

\hrule
\vspace{1em}


\noindent
\textbf{Q713.} In a circuit, the voltage is reduced by 20% due to increased load. An engineer incorrectly restores the original voltage by increasing the reduced value by
50
%, resulting in an error of
21
V. What is the correct original voltage?



\textbf{A.} 160V \\
\textbf{B.} 80V \\
\textbf{C.} 100V \\
\textbf{D.} 60V \\

\textbf{Answer:} E \\
\textbf{Explanation:} Analysis:
Let the original voltage be V. After a 20% reduction, it becomes 0.8V. The engineer’s calculation 0.8V×1.
5
=
1.2
V
1.2V-V=21      V=105V

\hrule
\vspace{1em}


\noindent
\textbf{Q714.} An object of mass 8 kg is at rest at time = 0 s. A resultant force acts on the object in a constant direction.The magnitude of the resultant force acting on the object varies with time as shown by the graph.
What is the kinetic energy of the object at time = 0.20 s?



\textbf{A.} 26J \\
\textbf{B.} 11J \\
\textbf{C.} 16J \\
\textbf{D.} 12J \\

\textbf{Answer:} C \\
\textbf{Explanation:} The kinetic energy equals the work done by the force; the integration of force on time equals the change in momentum; mv; therefore the terminal K.E. can be found:1/2 mv2.

\hrule
\vspace{1em}


\noindent
\textbf{Q715.} The elastic potential energy of a spring is reduced by 30% due to temperature drop. A student incorrectly calculates the original energy by increasing the reduced value by 40%, resulting in an error of
3.6
J. What is the correct original elastic potential energy?



\textbf{A.} 150J \\
\textbf{B.} 180J \\
\textbf{C.} 200J \\
\textbf{D.} 240J \\

\textbf{Answer:} B \\
\textbf{Explanation:} [IMAGE:0]

\hrule
\vspace{1em}


\noindent
\textbf{Q716.} A voltage is reduced by 25% due to increased resistance. A student incorrectly restores the original voltage by increasing the reduced value by 20%, resulting in an error of
12
V. What is the correct original voltage?



\textbf{A.} 36V \\
\textbf{B.} 90V \\
\textbf{C.} 60V \\
\textbf{D.} 120V \\

\textbf{Answer:} D \\
\textbf{Explanation:} [IMAGE:0]

\hrule
\vspace{1em}


\noindent
\textbf{Q717.} An object of mass 3 kg is at rest at time = 0 s. A resultant force acts on the object in a constant direction.The magnitude of the resultant force acting on the object varies with time as shown by the graph.
What is the kinetic energy of the object at time = 0.20 s?



\textbf{A.} 96J \\
\textbf{B.} 81J \\
\textbf{C.} 16J \\
\textbf{D.} 125J \\

\textbf{Answer:} A \\
\textbf{Explanation:} The kinetic energy equals the work done by the force; the integration of force on time equals the change in momentum; mv; therefore the terminal K.E. can be found:1/2 mv2.

\hrule
\vspace{1em}


\noindent
\textbf{Q718.} The power of a machine is reduced by 40% due to losses. A technician incorrectly restores the original power by increasing the reduced value by 50%, resulting in an error of 2
7
W. What is the correct original power?



\textbf{A.} 120 W \\
\textbf{B.} 150W \\
\textbf{C.} 180W \\
\textbf{D.} 270W \\

\textbf{Answer:} D \\
\textbf{Explanation:} Let the original power be P
P
. After a 40% reduction, it becomes 0.6P0.6
P
. The technician incorrectly calculates 0.6P×1.5=0.9P. The error is P
−
0.9P=0.1P=2
7
, solving P=2
7
0.

\hrule
\vspace{1em}


\noindent
\textbf{Q719.} The velocity of an object is reduced by 25% due to friction. A student incorrectly calculates the original velocity by increasing the reduced value by 30%, resulting in an error of
20
m/s. What is the correct original velocity?



\textbf{A.} 8
0
0 m/s \\
\textbf{B.} 192 m/s \\
\textbf{C.} 210 m/s \\
\textbf{D.} 225 m/s \\

\textbf{Answer:} A \\
\textbf{Explanation:} [IMAGE:0]

\hrule
\vspace{1em}


\noindent
\textbf{Q720.} An object of mass 1 kg is at rest at time = 0 s. A resultant force acts on the object in a constant direction.
The magnitude of the resultant force acting on the object varies with time as shown by
the graph.
What is the kinetic energy of the object at time = 0.20 s?



\textbf{A.} 1.2J \\
\textbf{B.} 0.81J \\
\textbf{C.} 3.6J \\
\textbf{D.} 1.25J \\

\textbf{Answer:} F \\
\textbf{Explanation:} The kinetic energy equals the work done by the force; the integration of force on time equals the change in momentum; mv; therefore the terminal K.E. can be found:1/2 mv2.

\hrule
\vspace{1em}


\noindent
\textbf{Q721.} In a circuit, the current is reduced by 30% due to increased resistance. An engineer incorrectly restores the original current by increasing the reduced value by 40%, resulting in an error of 1
6
A. What is the correct original current?



\textbf{A.} 8
00 A \\
\textbf{B.} 120 A \\
\textbf{C.} 150 A \\
\textbf{D.} 600A \\

\textbf{Answer:} A \\
\textbf{Explanation:} [IMAGE:0]

\hrule
\vspace{1em}


\noindent
\textbf{Q722.} In an experiment, the spring constant of a spring is reduced by 20% due to temperature increase. A student incorrectly calculates the original constant by increasing the reduced value by
50
%, resulting in an error of 1
5
N/m. What is the correct original spring constant?



\textbf{A.} 6
0 N/m \\
\textbf{B.} 95 N/m \\
\textbf{C.} 10 N/m \\
\textbf{D.} 7
5 N/m \\

\textbf{Answer:} D \\
\textbf{Explanation:} Let the original spring constant be k. After a 20% reduction, it becomes 0.8k. The student incorrectly calculates it as 0.8k×1.
5
=
1.2
k,
1.2k-k=15N/m   k=75N/m

\hrule
\vspace{1em}


\noindent
\textbf{Q723.} An object of mass 2 kg is at rest at time = 0 s. A resultant force acts on the object in a constant direction.
The magnitude of the resultant force acting on the object varies with time as shown by
the graph.
What is the kinetic energy of the object at time = 0.20 s?



\textbf{A.} 0J \\
\textbf{B.} 0.81J \\
\textbf{C.} 1J \\
\textbf{D.} 1.25J \\

\textbf{Answer:} D \\
\textbf{Explanation:} The kinetic energy equals the work done by the force; the integration of force on time equals the change in momentum; mv; therefore the terminal K.E. can be found:1/2 mv2.

\hrule
\vspace{1em}


\noindent
\textbf{Q724.} A submarine reduces its own weight by discharging water. At first, the speed of discharging water increases with time, and then gradually decreases. If the thrust remains constant, how will the magnitude of acceleration change?



\textbf{A.} First increases at a constant rate, then at a decreasing rate \\
\textbf{B.} Always increases at a constant rate \\
\textbf{C.} First increases at an increasing rate, then at a decreasing rate \\
\textbf{D.} Not changing \\

\textbf{Answer:} C \\
\textbf{Explanation:} [IMAGE:0]

\hrule
\vspace{1em}


\noindent
\textbf{Q725.} A truck continues to load goods, and its rate of mass increase decreases over time. If the engine provides a constant thrust, how will the magnitude of the truck's acceleration change?



\textbf{A.} Increasing at an increasing rate \\
\textbf{B.} Increasing at a constant rate \\
\textbf{C.} Increasing at a decreasing rate \\
\textbf{D.} Not changing \\

\textbf{Answer:} G \\
\textbf{Explanation:} [IMAGE:0]

\hrule
\vspace{1em}


\noindent
\textbf{Q726.} A deflated balloon will continue to be subjected to a thrust force during the process of gas release, and the magnitude of the thrust force is proportional to the speed of gas release. If the speed of gas release slows down, how will the magnitude of the acceleration of the balloon change?



\textbf{A.} Increasing at an increasing rate \\
\textbf{B.} Increasing at a constant rate \\
\textbf{C.} Increasing at a decreasing rate \\
\textbf{D.} Not changing \\

\textbf{Answer:} G \\
\textbf{Explanation:} [IMAGE:0]

\hrule
\vspace{1em}


\noindent
\textbf{Q727.} The spacecraft adjusts its orbit in space by constantly injecting fuel at a constant rate. If the engine applies a constant force, how will the magnitude of the acceleration in space change?



\textbf{A.} Increasing at an increasing rate \\
\textbf{B.} Increasing at a constant rate \\
\textbf{C.} Increasing at a decreasing rate \\
\textbf{D.} Not changing \\

\textbf{Answer:} A \\
\textbf{Explanation:} [IMAGE:0]

\hrule
\vspace{1em}


\noindent
\textbf{Q728.} The skier continued to glide on the snow and constantly maintained a constant speed to clear the snow from the skis until it was completely removed; in this way, the overall weight of the skis was reduced. If the engine continues to provide a constant thrust, how will the acceleration of the skis change?



\textbf{A.} Increasing at an increasing rate
until reaching a certain value. \\
\textbf{B.} Increasing at a constant rate \\
\textbf{C.} Increasing at a decreasing rate \\
\textbf{D.} Not changing \\

\textbf{Answer:} A \\
\textbf{Explanation:} Analysis: By Newton's second law,
[IMAGE:0]
The sled's mass m
de
creases over time
[IMAGE:1]
until reaching a certain value,
while the thrust F is constant, so acceleration
in
creases
until reaching a certain value. Since mass
de
creases at a constant rate (m=m
-
k
t
), the derivative
[IMAGE:2]
indicates that the increasing speed increases over time.
Thus, the rate of increase in acceleration shows an increasing trend until reaching a certain value.

\hrule
\vspace{1em}


\noindent
\textbf{Q729.} An electrically charged sphere moves on a smooth ice surface. The magnetic force it experiences is proportional to its speed and is in the same direction (the direction of the magnetic force is perpendicular to the ice surface). Then, how will the magnitude of the sphere's acceleration change?



\textbf{A.} It is increasing at an increasing rate. \\
\textbf{B.} It is increasing at a constant rate. \\
\textbf{C.} It is increasing at a decreasing rate. \\
\textbf{D.} It is not changing. \\

\textbf{Answer:} A \\
\textbf{Explanation:} The magnetic force F = -kv (where k is a constant), so the acceleration a = F/m = -kv/m. The magnitude of the acceleration is |a| = kv/m. Since the sphere is accelerated by a forward force, the velocity v increases with time. This causes |a| to also increase accordingly. The rate of increase in velocity v depends on a, thus resulting in a scenario similar to exponential growth. The rate at which |a| increases also increases over time, meaning the rate of acceleration's increase is gradually increasing. Therefore, option A is correct.

\hrule
\vspace{1em}


\noindent
\textbf{Q730.} A balloon rises at a constant speed and absorbs gas, causing its mass to increase linearly. Assuming that the buoyancy and resistance are both constant and independent of the speed, how will the magnitude of the balloon's acceleration change?



\textbf{A.} It is increasing at an increasing rate. \\
\textbf{B.} It is increasing at a constant rate. \\
\textbf{C.} It is increasing at a decreasing rate. \\
\textbf{D.} It is not changing. \\

\textbf{Answer:} F \\
\textbf{Explanation:} Analysis: In a state of uniform motion, the resultant force is zero: the buoyancy B is equal to the sum of the weight mg and the resistance f. As the mass m increases linearly, the weight mg also increases, thereby resulting in the resultant force Fnet = B - mg - f. Since B and f are constant, Fnet decreases linearly with m. According to Newton's second law, a = Fnet / m, Fnet and m both change linearly. Substituting Fnet = B - mg - f into a, we get a = (B - mg - f) / m = (B - f) / m - g. As m increases linearly, (B - f) / m also decreases linearly, causing a to decrease linearly as well. Therefore, the acceleration decreases at a constant rate, so option F is correct.

\hrule
\vspace{1em}


\noindent
\textbf{Q731.} A car is traveling on a straight road. Its traction force will increase linearly over time until it reaches a constant value. Assuming the resistance remains constant, how will the magnitude of the car's acceleration change?



\textbf{A.} It is increasing at an increasing rate. \\
\textbf{B.} It is growing at a constant rate until it reaches a fixed value.
. \\
\textbf{C.} It is increasing at a decreasing rate. \\
\textbf{D.} It is not changing. \\

\textbf{Answer:} B \\
\textbf{Explanation:} Analysis: The traction force F(t) increases linearly, while the resistance f remains constant. The net force Fnet = F(t) - f. When F(t) > f, the net force is positive, and the acceleration a = Fnet / m increases linearly with the increase of F(t) until it reaches a constant value. This means that the magnitude (positive value) of the acceleration increases at a constant rate until it reaches the constant value.

\hrule
\vspace{1em}


\noindent
\textbf{Q732.} Which statement about gravity and mass is correct?



\textbf{A.} Objects of the same mass have the same weight everywhere on Earth. \\
\textbf{B.} gravity is a vector quantity directed vertically downward. \\
\textbf{C.} Objects in space are weightless and have zero mass. \\
\textbf{D.} Gravity is the same as Earth's gravitational pull. \\

\textbf{Answer:} B \\
\textbf{Explanation:} Option A is incorrect. Weight
W
=
mg
varies with location due to varying
g
.
Option B is correct. Gravity is a vector that points toward the Earth's center.
Option C is wrong. Mass remains constant. Weightlessness refers to apparent weight, not actual mass.
Option D is wrong. Gravity is the force we experience, which is slightly different from Earth's gravitational pull due to Earth's rotation

\hrule
\vspace{1em}


\noindent
\textbf{Q733.} Which statement about falling objects is correct?



\textbf{A.} All objects fall with an acceleration of 9.8m/s² \\
\textbf{B.} Objects of different masses fall with the same acceleration in a vacuum. \\
\textbf{C.} A feather falls slower than an iron ball in air because of different accelerations. \\
\textbf{D.} Gravitational acceleration changes with the shape of an object. \\

\textbf{Answer:} B \\
\textbf{Explanation:} Option A is incorrect. In air, objects experience air resistance, so their acceleration is less than 9.8m/s².
Option B is correct. In a vacuum, all objects fall with the same acceleration due to gravity.
Option C is wrong. The acceleration is the same, but air resistance affects the feather more, resulting in lower net acceleration.
Option D is wrong. Gravitational acceleration is unrelated to object shape but depends on altitude and latitude.

\hrule
\vspace{1em}


\noindent
\textbf{Q734.} Which statement about gravitational acceleration is correct?



\textbf{A.} Gravitational acceleration is 9.8m/s² everywhere on Earth. \\
\textbf{B.} The higher the altitude, the smaller the gravitational acceleration. \\
\textbf{C.} Gravitational acceleration changes with the mass of an object. \\
\textbf{D.} Gravitational acceleration is independent of latitude. \\

\textbf{Answer:} B \\
\textbf{Explanation:} Option A is incorrect. Gravitational acceleration varies with location. It's about 9.83m/s² at the poles and 9.78m/s² at the equator.
Option B is correct. Gravitational acceleration decreases with increasing altitude because you're farther from the Earth's center.
Option C is wrong. Gravitational acceleration depends on Earth's mass and radius, not the object's mass.
Option D is wrong. Latitude affects gravitational acceleration due to Earth's rotation and shape.

\hrule
\vspace{1em}


\noindent
\textbf{Q735.} A crane is lifting a load at a constant power. As the speed of the load decreases, how will the magnitude of its acceleration change?



\textbf{A.} It is increasing at an increasing rate. \\
\textbf{B.} It is increasing at a constant rate. \\
\textbf{C.} It is increasing at a decreasing rate. \\
\textbf{D.} It is not changing. \\

\textbf{Answer:} A \\
\textbf{Explanation:} Power P = Fv, so when the power remains constant, as the speed v decreases, the pulling force F = P/v will increase. According to Newton's second law F - mg = ma, when F increases, the resultant force F - mg will also increase, thereby causing the acceleration a = (F - mg) / m to also increase. However, since F = P/v, when F increases, the rate at which F increases will increase with v's decrease (because v is in the denominator), which leads to an increasing rate of acceleration increase, thus forming the phenomenon of the rate of acceleration increase gradually increasing. Therefore, the rate of increase of acceleration is gradually increasing, so option A is correct.

\hrule
\vspace{1em}


\noindent
\textbf{Q736.} Which statement about gravity on Earth is correct?



\textbf{A.} The magnitude of gravity is the same everywhere on Earth. \\
\textbf{B.} Gravity decreases with increasing altitude. \\
\textbf{C.} The gravity at the Earth's poles is less than at the equator. \\
\textbf{D.} Gravity is inversely proportional to the mass of an object. \\

\textbf{Answer:} B \\
\textbf{Explanation:} Option A is incorrect because gravity varies slightly depending on location. It's slightly stronger at the poles and weaker at the equator due to Earth's oblate shape.
Option B is correct. As altitude increases, distance from the Earth's center increases, so gravity decreases according to the inverse square law.
Option C is wrong. The poles are closer to the Earth's center due to the oblate shape, so gravity is stronger at the poles.
Option D is wrong. Gravity is directly proportional to mass.
F
=
mg
shows that force increases with mass.

\hrule
\vspace{1em}


\noindent
\textbf{Q737.} Which statement about the center of gravity is correct?



\textbf{A.} The center of gravity must be at the geometric center of the object. \\
\textbf{B.} Only regular-shaped objects have centers of gravity. \\
\textbf{C.} The center of gravity can be outside the object. \\
\textbf{D.} The position of the center of gravity is independent of the object's shape. \\

\textbf{Answer:} C \\
\textbf{Explanation:} Option A is incorrect because the center of gravity may not be at the geometric center of the object, as in cases of irregular or asymmetric shapes.
Option B states only regular-shaped objects have centers of gravity, which is wrong. All objects have a center of gravity.
Option C is correct. For example, when you bend a paperclip, the center of gravity is not on the paperclip itself.
Option D is wrong because the position of the center of gravity relates to the shape of the object. When you peel an orange, the center of gravity of the orange shifts.

\hrule
\vspace{1em}


\noindent
\textbf{Q738.} When a rocket ascends, it releases sandbags at a constant rate, which causes its total mass to decrease uniformly. Given that the lift force remains constant and the resultant force acting on the rocket also remains constant, how will the magnitude of the rocket's acceleration change?



\textbf{A.} It is increasing at an increasing rate. \\
\textbf{B.} It is increasing at a constant rate. \\
\textbf{C.} It is increasing at a decreasing rate. \\
\textbf{D.} It is not changing. \\

\textbf{Answer:} B \\
\textbf{Explanation:} Analysis: By Newton’s second law
F
=
ma
, with constant net force
F
and mass
m
decreasing linearly over time, acceleration
a
=
F
/
m
increases linearly as mass decreases. Thus, acceleration increases at a constant rate, making option B correct.

\hrule
\vspace{1em}


\noindent
\textbf{Q739.} Which statements about weight are correct?
Weight is a vector.
Weight equals mass multiplied by 9.8.
An object in free fall has zero weight.



\textbf{A.} 1 \\
\textbf{B.} 2 \\
\textbf{C.} 3 \\
\textbf{D.} 1 and 3 \\

\textbf{Answer:} D \\
\textbf{Explanation:} Correct. Weight is a vector aligned with gravitational acceleration.
Incorrect. Weight is mass multiplied by actual gravitational acceleration (not necessarily 9.8).
Correct. An object in free fall experiences apparent weightlessness.

\hrule
\vspace{1em}


\noindent
\textbf{Q740.} Which statement is incorrect?
The position of the center of gravity depends on the object’s shape.
Gravity is perfectly uniform on Earth’s surface.
The value of gravitational acceleration is independent of an object’s mass.



\textbf{A.} 1 \\
\textbf{B.} 2 \\
\textbf{C.} 3 \\
\textbf{D.} 2 and 3 \\

\textbf{Answer:} B \\
\textbf{Explanation:} Correct. The center of gravity varies with shape.
Incorrect. Gravity varies due to Earth’s topography and density.
Correct. Gravitational acceleration is mass-independent.

\hrule
\vspace{1em}


\noindent
\textbf{Q741.} The kinematic equation of an object is given by:
[IMAGE:0]
Find the maximum speed of the object during its motion.



\textbf{A.} 3
m
/
s \\
\textbf{B.} 4
m
/
s \\
\textbf{C.} 5
m
/
s \\
\textbf{D.} 10
m
/
s \\

\textbf{Answer:} D \\
\textbf{Explanation:} [IMAGE:0]

\hrule
\vspace{1em}


\noindent
\textbf{Q742.} Which statement about gravitational acceleration is correct?
In a vacuum, light and heavy objects have the same gravitational acceleration.
Gravitational acceleration is independent of altitude.
Gravitational acceleration is smaller at the poles than at the equator.



\textbf{A.} 1 \\
\textbf{B.} 2 \\
\textbf{C.} 3 \\
\textbf{D.} 1 and 3 \\

\textbf{Answer:} A \\
\textbf{Explanation:} Correct. In a vacuum, all objects experience the same gravitational acceleration (neglecting air resistance).
Incorrect. Gravitational acceleration decreases with altitude.
Incorrect. Gravitational acceleration is greater at the poles due to reduced centrifugal force.

\hrule
\vspace{1em}


\noindent
\textbf{Q743.} The distance traveled by a car varies with time according to the equation:
[IMAGE:0]
Find
of the car during its journey.
There was no deceleration process.



\textbf{A.} −
6
m
/
s
2 \\
\textbf{B.} −
3
m
/
s
2 \\
\textbf{C.} 0
m
/
s
2 \\
\textbf{D.} 3
m
/
s
2 \\

\textbf{Answer:} C \\
\textbf{Explanation:} [IMAGE:0]

\hrule
\vspace{1em}


\noindent
\textbf{Q744.} Which statement is correct?
The direction of gravity on Earth’s surface is always vertically downward.
An object’s weight is the same at different latitudes.
Gravitational acceleration points toward Earth’s center.



\textbf{A.} 1 \\
\textbf{B.} 2 \\
\textbf{C.} 3 \\
\textbf{D.} 2 and 3 \\

\textbf{Answer:} E \\
\textbf{Explanation:} Incorrect. Gravity points toward Earth’s center, but "vertically downward" may not align exactly with the center at all locations.
Incorrect. Weight varies with gravitational acceleration at different latitudes.
Incorrect. Gravitational acceleration approximates toward Earth’s center but is slightly affected by Earth’s rotation.

\hrule
\vspace{1em}


\noindent
\textbf{Q745.} Which statement about the center of gravity is correct?
The center of gravity must be inside the object.
The center of gravity is the equivalent point where gravity acts.
Gravitational acceleration is always 9.8 m/s² on Earth.



\textbf{A.} 1 \\
\textbf{B.} 2 \\
\textbf{C.} 3 \\
\textbf{D.} 1 and 2 \\

\textbf{Answer:} B \\
\textbf{Explanation:} Incorrect. The center of gravity can be outside the object (e.g., a ring’s center of gravity is at its center).
Correct. The center of gravity is the equivalent point of gravitational action.
Incorrect. Gravitational acceleration varies with latitude and altitude

\hrule
\vspace{1em}


\noindent
\textbf{Q746.} The kinematic equation of a particle is given by:
[IMAGE:0]
Find the minimum speed of the particle during its motion.



\textbf{A.} −
3
.56
m/s \\
\textbf{B.} 0m/s \\
\textbf{C.} 1.23m/s \\
\textbf{D.} 2.48m/s \\

\textbf{Answer:} B \\
\textbf{Explanation:} [IMAGE:0]

\hrule
\vspace{1em}


\noindent
\textbf{Q747.} The light intensity is reduced by 50% after passing through a medium. A student incorrectly calculates the original intensity by increasing the reduced value by 60%, resulting in an error of 45 lux. What is the correct original intensity?



\textbf{A.} 225 lux \\
\textbf{B.} 120 lux \\
\textbf{C.} 150 lux \\
\textbf{D.} 180 lux \\

\textbf{Answer:} A \\
\textbf{Explanation:} Let the original intensity be L.
[IMAGE:0]

\hrule
\vspace{1em}


\noindent
\textbf{Q748.} The displacement of a spring oscillator varies with time according to the equation:
[IMAGE:0]
The displacement of a spring oscillator varies with time according to the equation:
[IMAGE:1]
Find the maximum acceleration of the oscillator during its.
Find the maximum acceleration of the oscillator during its.



\textbf{A.} 2
m
/
s
2 \\
\textbf{B.} 4
m
/
s
2 \\
\textbf{C.} 5
m
/
s
2 \\
\textbf{D.} 8
m
/
s
2 \\

\textbf{Answer:} C \\
\textbf{Explanation:} [IMAGE:0]

\hrule
\vspace{1em}


\noindent
\textbf{Q749.} The power of a machine is reduced by 35% due to friction. A technician incorrectly restores the original power by increasing the reduced value by 30%, resulting in an error of 21 W. What is the correct original power?



\textbf{A.} 140 W \\
\textbf{B.} 163 W \\
\textbf{C.} 182 W \\
\textbf{D.} 200 W \\

\textbf{Answer:} E \\
\textbf{Explanation:} Let the original power be P.
[IMAGE:0]

\hrule
\vspace{1em}


\noindent
\textbf{Q750.} The intensity of a sound wave is reduced by 40% due to obstacles. A student incorrectly calculates the original intensity by increasing the reduced value by 50%, resulting in an error of 24 W/m². What is the correct original intensity?



\textbf{A.} 120 W/m² \\
\textbf{B.} 150 W/m² \\
\textbf{C.} 180 W/m² \\
\textbf{D.} 200 W/m² \\

\textbf{Answer:} E \\
\textbf{Explanation:} Let the original intensity be I.
[IMAGE:0]

\hrule
\vspace{1em}


\noindent
\textbf{Q751.} The displacement-time equation of a particle moving in a straight line is given by:
[IMAGE:0]
Find the maximum speed of the particle during its motion.



\textbf{A.} 0 \\
\textbf{B.} 3
m
/
s \\
\textbf{C.} 6
m
/
s \\
\textbf{D.} 12
m
/
s \\

\textbf{Answer:} A \\
\textbf{Explanation:} [IMAGE:0]

\hrule
\vspace{1em}


\noindent
\textbf{Q752.} In a circuit, the voltage is reduced by 20% due to increased load. An engineer incorrectly restores the original voltage by increasing the reduced value by 50%, resulting in an error of 12 V. What is the correct original voltage?



\textbf{A.} 160 V \\
\textbf{B.} 80 V \\
\textbf{C.} 100 V \\
\textbf{D.} 60 V \\

\textbf{Answer:} D \\
\textbf{Explanation:} Let the original voltage be V. After a 20% reduction, it becomes 0.8V. The engineer’s calculation 0.8V×1.5=1.2V 1.2V-V=12
V=60V

\hrule
\vspace{1em}


\noindent
\textbf{Q753.} The elastic potential energy of a spring is reduced by 30% due to temperature drop. A student incorrectly calculates the original energy by increasing the reduced value by 40%, resulting in an error of 42 J. What is the correct original elastic potential energy?



\textbf{A.} 150 J \\
\textbf{B.} 180 J \\
\textbf{C.} 200 J \\
\textbf{D.} 240 J \\

\textbf{Answer:} C \\
\textbf{Explanation:} Let the original energy be E. After a 30% reduction, it becomes 0.7E. The student calculates
[IMAGE:0]

\hrule
\vspace{1em}


\noindent
\textbf{Q754.} A voltage is reduced by 25% due to increased resistance. A student incorrectly restores the original voltage by increasing the reduced value by 20%, resulting in an error of 9 V. What is the correct original voltage?



\textbf{A.} 36 V \\
\textbf{B.} 90 V \\
\textbf{C.} 60 V \\
\textbf{D.} 72 V \\

\textbf{Answer:} B \\
\textbf{Explanation:} Let the original voltage be V
V
. After a 25% reduction, it becomes 0.75V
[IMAGE:0]
[IMAGE:1]

\hrule
\vspace{1em}


\noindent
\textbf{Q755.} The power of a machine is reduced by 40% due to losses. A technician incorrectly restores the original power by increasing the reduced value by 50%, resulting in an error of 24 W. What is the correct original power?



\textbf{A.} 120 W \\
\textbf{B.} 150 W \\
\textbf{C.} 180 W \\
\textbf{D.} 200 W \\

\textbf{Answer:} E \\
\textbf{Explanation:} Let the original power be P
P
. After a 40% reduction, it becomes 0.6P0.6
P
. The technician incorrectly calculates 0.6P×1.5=0.9P. The error is P
−
0.9P=0.1P=24, solving P=240.

\hrule
\vspace{1em}


\noindent
\textbf{Q756.} The velocity of an object is reduced by 25% due to friction. A student incorrectly calculates the original velocity by increasing the reduced value by 30%, resulting in an error of 18 m/s. What is the correct original velocity?



\textbf{A.} 180 m/s \\
\textbf{B.} 192 m/s \\
\textbf{C.} 210 m/s \\
\textbf{D.} 225 m/s \\

\textbf{Answer:} E \\
\textbf{Explanation:} [IMAGE:0]
v−0.975v=0.025v=18
v
=720

\hrule
\vspace{1em}


\noindent
\textbf{Q757.} In a circuit, the current is reduced by 30% due to increased resistance. An engineer incorrectly restores the original current by increasing the reduced value by 40%, resulting in an error of 12 A. What is the correct original current?



\textbf{A.} 100 A \\
\textbf{B.} 120 A \\
\textbf{C.} 150 A \\
\textbf{D.} 150 A \\

\textbf{Answer:} D \\
\textbf{Explanation:} [IMAGE:0]
[IMAGE:1]
[IMAGE:2]

\hrule
\vspace{1em}


\noindent
\textbf{Q758.} The energy of a system varies with time as E(t)=
−
t
3
+
9
t
2
(t>0
t
>0). Find the maximum power.



\textbf{A.} 6 \\
\textbf{B.} 8 \\
\textbf{C.} 27 \\
\textbf{D.} 12 \\

\textbf{Answer:} C \\
\textbf{Explanation:} [IMAGE:0]

\hrule
\vspace{1em}


\noindent
\textbf{Q759.} In an experiment, the spring constant of a spring is reduced by 20% due to temperature increase. A student incorrectly calculates the original constant by increasing the reduced value by 50%, resulting in an error of 12 N/m. What is the correct original spring constant?



\textbf{A.} 60 N/m \\
\textbf{B.} 95 N/m \\
\textbf{C.} 10 N/m \\
\textbf{D.} 25 N/m \\

\textbf{Answer:} A \\
\textbf{Explanation:} Let the original spring constant be k. After a 20% reduction, it becomes 0.8k. The student incorrectly calculates it as 0.8k×1.5=1.2k, 1.2k-k=12N/m
k=60N/m

\hrule
\vspace{1em}


\noindent
\textbf{Q760.} A force varies with position as F(x)=
12
x
−
2
x
2
(x>0
x
>0). Find the maximum force.



\textbf{A.} 16/3 \\
\textbf{B.} 4 \\
\textbf{C.} 8 \\
\textbf{D.} 6 \\

\textbf{Answer:} E \\
\textbf{Explanation:} [IMAGE:0]

\hrule
\vspace{1em}


\noindent
\textbf{Q761.} A submarine reduces its mass by ejecting water, initially at a constant rate and then slowing down. If the thrust is constant, how does the acceleration magnitude change?



\textbf{A.} First increases at a constant rate, then at a decreasing rate \\
\textbf{B.} Always increases at a constant rate \\
\textbf{C.} First increases at an increasing rate, then at a decreasing rate \\
\textbf{D.} Not changing \\

\textbf{Answer:} C \\
\textbf{Explanation:} The mass loss rate starts constant and later slows.
[IMAGE:0]
decreasing mass causes acceleration to increase. Initially, with constant mass loss,
[IMAGE:1]
grows (increasing rate). Later, as mass loss slows,
[IMAGE:2]
grows more slowly (decreasing rate). Thus, acceleration first increases at an increasing rate, then at a decreasing rate.

\hrule
\vspace{1em}


\noindent
\textbf{Q762.} The electric potential in a field is V(x)=x
3
−
6x
2
(x>0
x
>0). Find the maximum electric field strength.



\textbf{A.} 0 \\
\textbf{B.} 3 \\
\textbf{C.} 6 \\
\textbf{D.} 9 \\

\textbf{Answer:} E \\
\textbf{Explanation:} [IMAGE:0]

\hrule
\vspace{1em}


\noindent
\textbf{Q763.} The velocity of a particle is given by v(t)=
6
t
−
t
2
(t>0
t
>0). Find the maximum acceleration.



\textbf{A.} 0 \\
\textbf{B.} 6 \\
\textbf{C.} 4 \\
\textbf{D.} -2 \\

\textbf{Answer:} B \\
\textbf{Explanation:} [IMAGE:0]

\hrule
\vspace{1em}


\noindent
\textbf{Q764.} A truck continuously loads cargo, increasing its mass at a constant rate. If the engine provides constant thrust, how does the truck's acceleration magnitude change?



\textbf{A.} Increasing at an increasing rate \\
\textbf{B.} Increasing at a constant rate \\
\textbf{C.} Increasing at a decreasing rate \\
\textbf{D.} Not changing \\

\textbf{Answer:} G \\
\textbf{Explanation:} The truck's mass m
m
increases at a constant rate
[IMAGE:0]
and thrust F
F
is constant. From
[IMAGE:1]
acceleration decreases as mass increases. The derivative
[IMAGE:2]
shows that the rate of decrease slows over time because m
m
grows. Thus, acceleration decreases at a decreasing rate
.

\hrule
\vspace{1em}


\noindent
\textbf{Q765.} An object moves along a straight line with displacement function
(t>0). Find the maximum velocity.



\textbf{A.} 1 \\
\textbf{B.} 2 \\
\textbf{C.} 3 \\
\textbf{D.} 4 \\

\textbf{Answer:} C \\
\textbf{Explanation:} [IMAGE:0]

\hrule
\vspace{1em}


\noindent
\textbf{Q766.} A leaking balloon experiences a constant thrust as it releases gas. If the gas release rate accelerates, how does the balloon's acceleration magnitude change?



\textbf{A.} Increasing at an increasing rate \\
\textbf{B.} Increasing at a constant rate \\
\textbf{C.} Increasing at a decreasing rate \\
\textbf{D.} Not changing \\

\textbf{Answer:} F \\
\textbf{Explanation:} The balloon's mass m
m
decreases with an accelerating rate
[IMAGE:0]
[IMAGE:1]
decreasing mass causes acceleration to increase. However, if the mass loss rate itself increases, the second derivative of acceleration
[IMAGE:2]
becomes negative, leading to a decreasing rate of acceleration increase. This contradiction suggests a design flaw. A corrected scenario might involve increasing mass or adjusted conditions.

\hrule
\vspace{1em}


\noindent
\textbf{Q767.} A water pump's flow rate decreases in an arithmetic sequence. The total water pumped in the first 5 minutes is 2
5
0 liters, and in the next 5 minutes, it is
20
0 liters. What is the total water pumped in 30 minutes?



\textbf{A.} -450 \\
\textbf{B.} -200 \\
\textbf{C.} 50 \\
\textbf{D.} 750 \\

\textbf{Answer:} D \\
\textbf{Explanation:} [IMAGE:0]

\hrule
\vspace{1em}


\noindent
\textbf{Q768.} The amplitude of a spring oscillator decreases in an arithmetic sequence. The total amplitude of the first 10 oscillations is 120 cm, and the next 10 oscillations is 60 cm. What is the total amplitude of the first 50 oscillations?



\textbf{A.} -900 \\
\textbf{B.} -600 \\
\textbf{C.} 0 \\
\textbf{D.} 300 \\

\textbf{Answer:} D \\
\textbf{Explanation:} [IMAGE:0]

\hrule
\vspace{1em}


\noindent
\textbf{Q769.} A satellite adjusts its orbit in space by ejecting fuel at a constant rate. If the engine applies a constant force, how does the magnitude of the satellite's acceleration change?



\textbf{A.} Increasing at an increasing rate \\
\textbf{B.} Increasing at a constant rate \\
\textbf{C.} Increasing at a decreasing rate \\
\textbf{D.} Not changing \\

\textbf{Answer:} A \\
\textbf{Explanation:} The satellite's mass m
m
decreases over time
[IMAGE:0]
while the force F
F
is constant. From
[IMAGE:1]
decreasing mass causes acceleration to increase. Since the mass loss rate is constant, the derivative
[IMAGE:2]
grows as m
m
decreases. Thus, acceleration increases at an increasing rate.

\hrule
\vspace{1em}


\noindent
\textbf{Q770.} An object decelerates uniformly. The total displacement in the first 20 seconds is -
8
0 meters, and in the next 20 seconds, it is -
4
0 meters. What is the total displacement in the first 100 seconds?



\textbf{A.} -750 \\
\textbf{B.} -350 \\
\textbf{C.} -50 \\
\textbf{D.} 0 \\

\textbf{Answer:} D \\
\textbf{Explanation:} [IMAGE:0]

\hrule
\vspace{1em}


\noindent
\textbf{Q771.} A sled sliding on ice continuously collects snow, increasing its mass. If the engine applies a constant thrust, how does the magnitude of the sled's acceleration change?



\textbf{A.} Increasing at an increasing rate \\
\textbf{B.} Increasing at a constant rate \\
\textbf{C.} Increasing at a decreasing rate \\
\textbf{D.} Not changing \\

\textbf{Answer:} G \\
\textbf{Explanation:} By Newton's second law,
[IMAGE:0]
The sled's mass mm increases over time
[IMAGE:1]
while the thrust FF is constant, so acceleration decreases. Since mass increases at a constant rate (m=m0+kt), the derivative
[IMAGE:2]
shows that the rate of decrease slows over time. Thus, acceleration decreases at a decreasing rate.

\hrule
\vspace{1em}


\noindent
\textbf{Q772.} Current in a circuit changes in an arithmetic sequence. The total charge in the first 20 seconds is
6
0 Coulombs, and in the next 20 seconds, it is
4
0 Coulombs. What is the total charge in the first 100 seconds?



\textbf{A.} -750 \\
\textbf{B.} -350 \\
\textbf{C.} -50 \\
\textbf{D.} 50 \\

\textbf{Answer:} E \\
\textbf{Explanation:} [IMAGE:0]

\hrule
\vspace{1em}


\noindent
\textbf{Q773.} An object moves with uniformly variable motion. The total displacement in the first 20 seconds is
4
0 meters, and in the next 20 seconds, it is -
2
0 meters. What is the total displacement in the first 100 seconds?



\textbf{A.} -750 \\
\textbf{B.} -350 \\
\textbf{C.} -50 \\
\textbf{D.} 50 \\

\textbf{Answer:} B \\
\textbf{Explanation:} [IMAGE:0]

\hrule
\vspace{1em}


\noindent
\textbf{Q774.} A charged sphere moves on a smooth icy surface, experiencing a magnetic force proportional to its speed but in the opposite direction (with the magnetic field perpendicular to the ice). How does the magnitude of the sphere’s acceleration change?



\textbf{A.} It is increasing at an increasing rate. \\
\textbf{B.} It is increasing at a constant rate. \\
\textbf{C.} It is increasing at a decreasing rate. \\
\textbf{D.} It is not changing. \\

\textbf{Answer:} G \\
\textbf{Explanation:} The magnetic force
F
=
−
kv
(where
k
is a constant), so acceleration
a
=
F
/
m
=
−
kv
/
m
. The magnitude of acceleration is |
a
|=
kv
/
m
. As the sphere slows down due to the opposing force, speed
v
decreases over time. This causes |
a
| to decrease as well. The rate of decrease in
v
depends on
a
, leading to a scenario similar to exponential decay. The rate at which |
a
| decreases also diminishes over time, meaning acceleration decreases at a decreasing rate, making option G correct.

\hrule
\vspace{1em}


\noindent
\textbf{Q775.} A weather station records temperature changes of
32
°C in the first 6 hours and -
40
°C in the next 6 hours. Assuming temperature changes form an arithmetic sequence over time, find the total temperature change in the first 24 hours.



\textbf{A.} -
288
°C \\
\textbf{B.} -96°C \\
\textbf{C.} -72°C \\
\textbf{D.} -
312
°C \\

\textbf{Answer:} A \\
\textbf{Explanation:} [IMAGE:0]

\hrule
\vspace{1em}


\noindent
\textbf{Q776.} Uniformly accelerated circular motion
has an angular displacement of 1
2
radians in the first 3 revolutions and -
24
radians in the next 3 revolutions. Find its total angular displacement in the first 12 revolutions.
Let angular displacement per revolution form an arithmetic sequence with common difference
d
.



\textbf{A.} -
168
rad \\
\textbf{B.} -72 rad \\
\textbf{C.} 0 rad \\
\textbf{D.} 36 rad \\

\textbf{Answer:} A \\
\textbf{Explanation:} [IMAGE:0]

\hrule
\vspace{1em}


\noindent
\textbf{Q777.} A spring oscillator's total displacement in the first 4 cycles is 1
0
cm, and in the next 4 cycles, it is -
54
cm. Find its total displacement in the first 20 cycles.
Let displacement per cycle form an arithmetic sequence with common difference
d
.



\textbf{A.} -240 cm \\
\textbf{B.} -
59
0 cm \\
\textbf{C.} -
560
cm \\
\textbf{D.} 0 cm \\

\textbf{Answer:} B \\
\textbf{Explanation:} [IMAGE:0]

\hrule
\vspace{1em}


\noindent
\textbf{Q778.} An object in uniform accelerated motion has a displacement of 2
0
m in the first 5 seconds and -
80
m in the next 5 seconds. Find its total displacement in the first 20 seconds.



\textbf{A.} -300 m \\
\textbf{B.} -
5
00 m \\
\textbf{C.} -100 m \\
\textbf{D.} 0 m \\

\textbf{Answer:} E \\
\textbf{Explanation:} [IMAGE:0]

\hrule
\vspace{1em}


\noindent
\textbf{Q779.} A balloon ascends at a constant speed while releasing gas, causing its mass to decrease linearly. Assuming buoyant force and drag force are both constant and independent of speed, how does the magnitude of the balloon’s acceleration change?



\textbf{A.} It is increasing at an increasing rate. \\
\textbf{B.} It is increasing at a constant rate. \\
\textbf{C.} It is increasing at a decreasing rate. \\
\textbf{D.} It is not changing. \\

\textbf{Answer:} B \\
\textbf{Explanation:} At constant speed, net force is zero: buoyant force
B
equals the sum of weight
mg
and drag force
f
. As mass
m
decreases linearly, weight
mg
decreases, leading to a net force
Fnet
​
=
B
−
mg
−
f
. Since
B
and
f
are constant,
Fnet
​
increases linearly as
m
decreases. By Newton’s second law
a
=
Fnet
​
/
m
, both
Fnet
​
and
m
change linearly. Substituting
Fnet
​
=
B
−
mg
−
f
into
a
, we get
a
=(
B
−
mg
−
f
)/
m
=(
B
−
f
)/
m
−
g
. As
m
decreases linearly, (
B
−
f
)/
m
increases linearly, making
a
increase linearly. Thus, acceleration increases at a constant rate, making option B correct.

\hrule
\vspace{1em}


\noindent
\textbf{Q780.} In a free-fall experiment, the displacement of an object in the first 2 seconds is
5
meters, and in the next 2 seconds, it is
25
meters. Assuming displacement changes form an arithmetic sequence over time, find the total displacement in the first 8 seconds.



\textbf{A.} 80 m \\
\textbf{B.} 100 m \\
\textbf{C.} 120 m \\
\textbf{D.} 140 m \\

\textbf{Answer:} D \\
\textbf{Explanation:} [IMAGE:0]

\hrule
\vspace{1em}


\noindent
\textbf{Q781.} A car travels on a straight road with its traction force decreasing linearly over time until it reaches a constant value. Assuming constant resistance, how does the magnitude of the car’s acceleration change?



\textbf{A.} It is increasing at an increasing rate. \\
\textbf{B.} It is increasing at a constant rate. \\
\textbf{C.} It is increasing at a decreasing rate. \\
\textbf{D.} It is not changing. \\

\textbf{Answer:} F \\
\textbf{Explanation:} Traction force
F
(
t
) decreases linearly, and resistance
f
is constant. Net force
Fnet
​
=
F
(
t
)
−
f
. When
F
(
t
)>
f
, net force is positive, and acceleration
a
=
Fnet
​
/
m
decreases linearly as
F
(
t
) decreases, meaning the magnitude of acceleration (positive) is decreasing at a constant rate. When
F
(
t
)<
f
, net force becomes negative, and deceleration
a
=(
f
−
F
(
t
))/
m
increases linearly as
F
(
t
) continues to decrease. However, if the question focuses on the deceleration phase, the magnitude of acceleration (now negative, so its magnitude is |
a
|=(
f
−
F
(
t
))/
m
) increases linearly. But typically, such questions might only consider the acceleration phase before
F
(
t
) equals resistance. Thus, during the acceleration phase, the magnitude of acceleration decreases at a constant rate, making option F correct.

\hrule
\vspace{1em}


\noindent
\textbf{Q782.} A crane lifts a heavy load with constant power. As the load’s speed increases, what happens to the magnitude of its acceleration?



\textbf{A.} It is increasing at an increasing rate. \\
\textbf{B.} It is increasing at a constant rate. \\
\textbf{C.} It is increasing at a decreasing rate. \\
\textbf{D.} It is not changing. \\

\textbf{Answer:} G \\
\textbf{Explanation:} Power
P
=
Fv
, so with constant power, as speed
v
increases, the tension force
F
=
P
/
v
decreases. By Newton’s second law
F
−
mg
=
ma
, as
F
decreases, the net force
F
−
mg
decreases, causing acceleration
a
=(
F
−
mg
)/
m
to decrease. However, since
F
=
P
/
v
, the rate at which
F
decreases slows down as
v
increases (because
v
is in the denominator), leading to a deceleration in the decrease of acceleration. Thus, acceleration decreases at a decreasing rate, making option G correct.

\hrule
\vspace{1em}


\noindent
\textbf{Q783.} .
A light spring has a natural length of 0.60 m and a spring constant of 100 N/m. It is stretched by a force starting from zero and increasing at a constant rate of
1
N/s until reaching its maximum value. When the strain energy of the spring is 0.36 J, what is the average power used to stretch the spring?



\textbf{A.} 0.015 W \\
\textbf{B.} 0.020 W \\
\textbf{C.} 0.025 W \\
\textbf{D.} 0.0
42
W \\

\textbf{Answer:} D \\
\textbf{Explanation:} [IMAGE:0]

\hrule
\vspace{1em}


\noindent
\textbf{Q784.} A hot air balloon ascending releases sandbags at a constant rate, causing its total mass to decrease uniformly. Given that the buoyant force remains constant and the net force on the balloon is also constant, how does the magnitude of the balloon’s acceleration change?



\textbf{A.} It is increasing at an increasing rate. \\
\textbf{B.} It is increasing at a constant rate. \\
\textbf{C.} It is increasing at a decreasing rate. \\
\textbf{D.} It is not changing. \\

\textbf{Answer:} B \\
\textbf{Explanation:} By Newton’s second law
F
=
ma
, with constant net force
F
and mass
m
decreasing linearly over time, acceleration
a
=
F
/
m
increases linearly as mass decreases. Thus, acceleration increases at a constant rate, making option B correct.

\hrule
\vspace{1em}


\noindent
\textbf{Q785.} A light spring oscillator has a spring constant of 20 N/m. The oscillator is pulled from its equilibrium position by a tension force that starts at zero and increases at a constant rate of 0.
5
0 N/s until it reaches its maximum value. When the force reaches its maximum value, the elastic potential energy of the oscillator is 0.45 J. What is the average power of the work done by the force?



\textbf{A.} 0.1
5
W \\
\textbf{B.} 0.30 W \\
\textbf{C.} 0.45 W \\
\textbf{D.} 0.60 W \\

\textbf{Answer:} A \\
\textbf{Explanation:} [IMAGE:0]

\hrule
\vspace{1em}


\noindent
\textbf{Q786.} The kinematic equation of an object is given by:
[IMAGE:0]
Find the maximum speed of the object during its motion.



\textbf{A.} 3
m
/
s \\
\textbf{B.} 4
m
/
s \\
\textbf{C.} 5
m
/
s \\
\textbf{D.} 7
m
/
s \\

\textbf{Answer:} C \\
\textbf{Explanation:} V
elocity v(t) is the first derivative of displacement x(t) with respect to time t:
[IMAGE:0]
[IMAGE:1]
[IMAGE:2]
The maximum speed is 5
m
/
s

\hrule
\vspace{1em}


\noindent
\textbf{Q787.} A light rubber band has an unstretched length of 0.20 m and a spring constant of 30 N/m. The rubber band is stretched by a tension force that starts at zero and increases at a constant rate of 0.40 N/s until it reaches its maximum value. When the force reaches its maximum value, the elastic potential energy of the rubber band is 0.
32
J. What is the average power used to stretch the rubber band?



\textbf{A.} 0.020 W \\
\textbf{B.} 0.040 W \\
\textbf{C.} 0.060 W \\
\textbf{D.} 0.0
57
W \\

\textbf{Answer:} D \\
\textbf{Explanation:} [IMAGE:0]

\hrule
\vspace{1em}


\noindent
\textbf{Q788.} The distance traveled by a car varies with time according to the equation:
[IMAGE:0]
Find of the car during its journey.
There was no deceleration process.



\textbf{A.} −
6
m
/
s
2 \\
\textbf{B.} −
3
m
/
s
2 \\
\textbf{C.} 0
m
/
s
2 \\
\textbf{D.} 3
m
/
s
2 \\

\textbf{Answer:} C \\
\textbf{Explanation:} Acceleration a(t) is the second derivative of displacement s(t) with respect to time t:
[IMAGE:0]
This is a linear function with no extremum. There was no deceleration process.
So the minimum acceleration is 0

\hrule
\vspace{1em}


\noindent
\textbf{Q789.} A light balloon is filled with gas, with an initial volume of 0.05 m³. The gas pressure relates to volume as
P
=
kV
, where
k
=200Pa/m3. The gas pressure starts at zero and increases at a constant rate of
1
Pa/s until it reaches its maximum value. When the pressure reaches its maximum value, the elastic potential energy stored in the balloon is 10.0 J. What is the average power of the work done by the gas during inflation?



\textbf{A.} 0.50 W \\
\textbf{B.} 1.00 W \\
\textbf{C.} 0
.
316
W \\
\textbf{D.} 5.00 W \\

\textbf{Answer:} C \\
\textbf{Explanation:} [IMAGE:0]

\hrule
\vspace{1em}


\noindent
\textbf{Q790.} The kinematic equation of a particle is given by:
[IMAGE:0]
Find the minimum speed of the particle during its motion.



\textbf{A.} −
3.56m/s \\
\textbf{B.} -0.167m/s \\
\textbf{C.} 1.23m/s \\
\textbf{D.} 2.48m/s \\

\textbf{Answer:} B \\
\textbf{Explanation:} [IMAGE:0]
[IMAGE:1]
To find the minimum speed, take the derivative of v(t) and set it to zero:
[IMAGE:2]
[IMAGE:3]
[IMAGE:4]

\hrule
\vspace{1em}


\noindent
\textbf{Q791.} A light spring has an unstretched length of 0.50 m and a spring constant of 40 N/m. The spring is stretched by a tension force that starts at zero and increases at a constant rate of 0.
2
0 N/s until it reaches its maximum value. When the force reaches its maximum value, the elastic potential energy of the spring is 0.25 J. What is the average power used to stretch the spring?



\textbf{A.} 0.02
2
W \\
\textbf{B.} 0.040 W \\
\textbf{C.} 0.0
11
W \\
\textbf{D.} 0.10
2
W \\

\textbf{Answer:} A \\
\textbf{Explanation:} [IMAGE:0]

\hrule
\vspace{1em}


\noindent
\textbf{Q792.} The displacement of a spring oscillator varies with time according to the equation:
[IMAGE:0]
Find the maximum acceleration of the oscillator during its.



\textbf{A.} 2
m
/
s
2 \\
\textbf{B.} 4
m
/
s
2 \\
\textbf{C.} 5
m
/
s
2 \\
\textbf{D.} 8
m
/
s
2 \\

\textbf{Answer:} C \\
\textbf{Explanation:} Acceleratio
n
a
(
t
) is the second derivative of displacement
x
(
t
) with respect to time
t
:
[IMAGE:0]
To find the maximum acceleration, we can express
a
(
t
) in the form
[IMAGE:1]
[IMAGE:2]
The maximum value is 5

\hrule
\vspace{1em}


\noindent
\textbf{Q793.} A spring with a spring constant of 40 N/m is stretched by a force increasing at 1N/s. When the strain energy is 0.08 J, the average power is:



\textbf{A.} 0.004 W \\
\textbf{B.} 0.00
3
W \\
\textbf{C.} 0.01
3
2 W \\
\textbf{D.} 0.0
3
16 W \\

\textbf{Answer:} D \\
\textbf{Explanation:} [IMAGE:0]

\hrule
\vspace{1em}


\noindent
\textbf{Q794.} The displacement-time equation of a particle moving in a straight line is given by:
[IMAGE:0]
Find the maximum speed of the particle during its motion.



\textbf{A.} 0 \\
\textbf{B.} 3
m
/
s \\
\textbf{C.} 6
m
/
s \\
\textbf{D.} 9
m
/
s \\

\textbf{Answer:} D \\
\textbf{Explanation:} [IMAGE:0]
[IMAGE:1]
To find the maximum speed, take the derivative of
v
(
t
) and set it to zero:
[IMAGE:2]
Solving gives
t
=2
s
. Substituting into
v
(
t
):
[IMAGE:3]
Since the question asks for the maximum speed magnitude, but the absolute value of
−
3
m
/
s
is 3
m
/
s
. However, at
t
=0, the velocity is
v
(0)=9
m
/
s
, so the maximum speed is 9
m
/
s
, option D.

\hrule
\vspace{1em}


\noindent
\textbf{Q795.} A spring with a natural length of 0.25 m and spring constant
1
00 N/m is stretched by a force increasing at 0.80 N/s. When the strain energy is 0.50 J, the average power is:



\textbf{A.} 0.02
1
W \\
\textbf{B.} 0.050 W \\
\textbf{C.} 0.
057
W \\
\textbf{D.} 0.
376
W \\

\textbf{Answer:} C \\
\textbf{Explanation:} [IMAGE:0]

\hrule
\vspace{1em}


\noindent
\textbf{Q796.} The energy of a system varies with time as E(t)=
−
t3+6t2
E
(
t
)=
−
t
3+6
t
2 (t>0
t
>0). Find the maximum power.



\textbf{A.} 6 \\
\textbf{B.} 8 \\
\textbf{C.} 10 \\
\textbf{D.} 12 \\

\textbf{Answer:} D \\
\textbf{Explanation:} Power is the first derivative of energy:
[IMAGE:0]
Take the derivative of power:
[IMAGE:1]
Set to zero, solve t=2. Substitute into power:
[IMAGE:2]

\hrule
\vspace{1em}


\noindent
\textbf{Q797.} A spring with a spring constant of
6
0 N/m is stretched by a force increasing at 0.
5
0 N/s. When the strain energy is 0.18 J, the average power is:



\textbf{A.} 0.0
2
0 W \\
\textbf{B.} 0.0
19
W \\
\textbf{C.} 0.03
2
W \\
\textbf{D.} 0.04
2
W \\

\textbf{Answer:} B \\
\textbf{Explanation:} [IMAGE:0]

\hrule
\vspace{1em}


\noindent
\textbf{Q798.} A force varies with position as F(x)=8x
−
3x2
F
(
x
)=8
x
−
3
x
2 (x>0
x
>0). Find the maximum force.



\textbf{A.} 16/3 \\
\textbf{B.} 4 \\
\textbf{C.} 8 \\
\textbf{D.} 6 \\

\textbf{Answer:} A \\
\textbf{Explanation:} Take the derivative of the force function:
[IMAGE:0]
Set the derivative to zero, solve
[IMAGE:1]
Substitute back:
[IMAGE:2]

\hrule
\vspace{1em}


\noindent
\textbf{Q799.} A spring with a spring constant of 1
5
0 N/m and natural length 0.30 m is stretched by a force increasing at 0.60 N/s. When the strain energy stored is 0.54 J, what is the average power?



\textbf{A.} 0.015 W \\
\textbf{B.} 0.03
2
W \\
\textbf{C.} 0.0
25
W \\
\textbf{D.} 0.06
2
W \\

\textbf{Answer:} C \\
\textbf{Explanation:} [IMAGE:0]

\hrule
\vspace{1em}


\noindent
\textbf{Q800.} The electric potential in a field is V(x)=x3
−
6x2+9x
V
(
x
)=
x
3
−
6
x
2+9
x
(x>0
x
>0). Find the maximum electric field strength.



\textbf{A.} 0 \\
\textbf{B.} 3 \\
\textbf{C.} 6 \\
\textbf{D.} 9 \\

\textbf{Answer:} B \\
\textbf{Explanation:} Electric field strength is the negative derivative of potential:
[IMAGE:0]
Find the critical point:
[IMAGE:1]
[IMAGE:2]

\hrule
\vspace{1em}


\noindent
\textbf{Q801.} A light spring has a natural length of 0.50 m and a spring constant of 80 N/m. It is stretched by a force that starts at zero and increases at a constant rate of 0.40 N/s until reaching its maximum value. When the strain energy of the spring is
6.4
J, what is the average power used to stretch the spring?



\textbf{A.} 0.016 W \\
\textbf{B.} 0.0
80
W \\
\textbf{C.} 0.064 W \\
\textbf{D.} 0.0
3
0 W \\

\textbf{Answer:} B \\
\textbf{Explanation:} [IMAGE:0]

\hrule
\vspace{1em}


\noindent
\textbf{Q802.} The velocity of a particle is given by v(t)=4t
−
t2
v
(
t
)=4
t
−
t
2 (t>0
t
>0). Find the maximum acceleration.



\textbf{A.} 0 \\
\textbf{B.} 2 \\
\textbf{C.} 4 \\
\textbf{D.} -2 \\

\textbf{Answer:} C \\
\textbf{Explanation:} Acceleration is the first derivative of velocity:
[IMAGE:0]
The rate of change of acceleration is:
[IMAGE:1]
Since the rate is always negative, acceleration decreases monotonically. The maximum occurs at t=0:
[IMAGE:2]

\hrule
\vspace{1em}


\noindent
\textbf{Q803.} A solid pyramid has a height of 160 m and a square base. The material density is 2600 kg/m³. If atmospheric pressure increases by
3
0 kPa, by how much will the average pressure increase?



\textbf{A.} 10 kPa \\
\textbf{B.} 20 kPa \\
\textbf{C.} 30 kPa \\
\textbf{D.} 40 kPa \\

\textbf{Answer:} C \\
\textbf{Explanation:} [IMAGE:0]

\hrule
\vspace{1em}


\noindent
\textbf{Q804.} An object moves along a straight line with displacement function
[IMAGE:0]
(t>0). Find the maximum velocity.



\textbf{A.} 1 \\
\textbf{B.} 2 \\
\textbf{C.} 3 \\
\textbf{D.} 4 \\

\textbf{Answer:} C \\
\textbf{Explanation:} Velocity is the first derivative of displacement:
[IMAGE:0]
Acceleration is the first derivative of velocity:
[IMAGE:1]
Set a(t)=0, solve t=1. Substitute t=1 into the velocity equation:
[IMAGE:2]
The second derivative (rate of change of acceleration) is negative, indicating a maximum velocity.

\hrule
\vspace{1em}


\noindent
\textbf{Q805.} A solid cone has a height of 240 m and a circular base. The material density is 2200 kg/m³, and atmospheric pressure is 1
1
0 kPa. If the base area is doubled, how will the average pressure change?



\textbf{A.} Increase \\
\textbf{B.} Decrease \\
\textbf{C.} Remain the same \\
\textbf{D.} Cannot be determined \\

\textbf{Answer:} C \\
\textbf{Explanation:} [IMAGE:0]

\hrule
\vspace{1em}


\noindent
\textbf{Q806.} A water pump's flow rate decreases in an arithmetic sequence. The total water pumped in the first 5 minutes is 200 liters, and in the next 5 minutes, it is 150 liters. What is the total water pumped in 30 minutes?



\textbf{A.} -450 \\
\textbf{B.} -200 \\
\textbf{C.} 50 \\
\textbf{D.} 750 \\

\textbf{Answer:} E \\
\textbf{Explanation:} [IMAGE:0]
[IMAGE:1]
[IMAGE:2]
[IMAGE:3]

\hrule
\vspace{1em}


\noindent
\textbf{Q807.} A solid rectangular prism and a solid cylinder have the same base area of 100 m². The prism has a height of 1
0
0 m and a density of 2400 kg/m³; the cylinder has a height of 180 m and a density of 2000 kg/m³. Atmospheric pressure is 100 kPa. What is the sum of their average pressures?



\textbf{A.} 8
2
00 kPa \\
\textbf{B.} 62
00 kPa \\
\textbf{C.} 668
0 kPa \\
\textbf{D.} 7
100 kPa \\

\textbf{Answer:} B \\
\textbf{Explanation:} [IMAGE:0]

\hrule
\vspace{1em}


\noindent
\textbf{Q808.} The amplitude of a spring oscillator decreases in an arithmetic sequence. The total amplitude of the first 10 oscillations is 120 cm, and the next 10 oscillations is 60 cm. What is the total amplitude of the first 50 oscillations?



\textbf{A.} -900 \\
\textbf{B.} -600 \\
\textbf{C.} 0 \\
\textbf{D.} 300 \\

\textbf{Answer:} D \\
\textbf{Explanation:} [IMAGE:0]
[IMAGE:1]
[IMAGE:2]
[IMAGE:3]

\hrule
\vspace{1em}


\noindent
\textbf{Q809.} A solid pyramid has a height of
21
0 m and a square base. The average pressure on the ground is 450 kPa, and atmospheric pressure is 100 kPa. What is the density of the pyramid material?



\textbf{A.} 500 kg/m³ \\
\textbf{B.} 600 kg/m³ \\
\textbf{C.} 700 kg/m³ \\
\textbf{D.} 800 kg/m³ \\

\textbf{Answer:} A \\
\textbf{Explanation:} [IMAGE:0]

\hrule
\vspace{1em}


\noindent
\textbf{Q810.} An object decelerates uniformly. The total displacement in the first 20 seconds is -70 meters, and in the next 20 seconds, it is -30 meters. What is the total displacement in the first 100 seconds?



\textbf{A.} -750 \\
\textbf{B.} -350 \\
\textbf{C.} -50 \\
\textbf{D.} 50 \\

\textbf{Answer:} D \\
\textbf{Explanation:} Sum of the first 20 terms
[IMAGE:0]
[IMAGE:1]
[IMAGE:2]
[IMAGE:3]

\hrule
\vspace{1em}


\noindent
\textbf{Q811.} A solid cylinder and a solid cone have the same height of 200 m and the same base area. The density of the cylinder is 2
0
0 kg/m³, and the density of the cone is 300 kg/m³. Atmospheric pressure is 100 kPa. What is the difference in average pressure between the two?



\textbf{A.} 2
00 kPa \\
\textbf{B.} 2
2
0 kPa \\
\textbf{C.} 3
2
0 kPa \\
\textbf{D.} 4
1
0 kPa \\

\textbf{Answer:} A \\
\textbf{Explanation:} [IMAGE:0]

\hrule
\vspace{1em}


\noindent
\textbf{Q812.} Current in a circuit changes in an arithmetic sequence. The total charge in the first 20 seconds is 30 Coulombs, and in the next 20 seconds, it is 10 Coulombs. What is the total charge in the first 100 seconds?



\textbf{A.} -750 \\
\textbf{B.} -350 \\
\textbf{C.} -50 \\
\textbf{D.} 50 \\

\textbf{Answer:} C \\
\textbf{Explanation:} Sum of the first 20 terms S1=30; next 20 terms S2=10 Common difference
[IMAGE:0]
[IMAGE:1]
[IMAGE:2]

\hrule
\vspace{1em}


\noindent
\textbf{Q813.} A solid prism of height 200 m has a triangular base. The density of the material is 200 kg/m³. Atmospheric pressure is 100 kPa. What is the average pressure on the ground under the prism?



\textbf{A.} 4
5
0 kPa \\
\textbf{B.} 5
00
kPa \\
\textbf{C.} 640 kPa \\
\textbf{D.} 740 kPa \\

\textbf{Answer:} B \\
\textbf{Explanation:} [IMAGE:0]

\hrule
\vspace{1em}


\noindent
\textbf{Q814.} An object moves with uniformly variable motion. The total displacement in the first 20 seconds is 50 meters, and in the next 20 seconds, it is -10 meters. What is the total displacement in the first 100 seconds?



\textbf{A.} -750 \\
\textbf{B.} -350 \\
\textbf{C.} -50 \\
\textbf{D.} 50 \\

\textbf{Answer:} B \\
\textbf{Explanation:} Sum of the first 20 terms S1=50,
next 20 terms S2=
−
10.
Common difference
[IMAGE:0]
First term a1
a
1
​
satisfies
[IMAGE:1]
Sum of first 100 terms
[IMAGE:2]

\hrule
\vspace{1em}


\noindent
\textbf{Q815.} A weather station records temperature changes of 30°C in the first 6 hours and -42°C in the next 6 hours. Assuming temperature changes form an arithmetic sequence over time, find the total temperature change in the first 24 hours.



\textbf{A.} -120°C \\
\textbf{B.} -96°C \\
\textbf{C.} -72°C \\
\textbf{D.} -312°C \\

\textbf{Answer:} C \\
\textbf{Explanation:} Let temperature change per hour form an arithmetic sequence with common difference
d
. For the first 6 hours:
[IMAGE:0]
For hours 7-12:
[IMAGE:1]
[IMAGE:2]
[IMAGE:3]

\hrule
\vspace{1em}


\noindent
\textbf{Q816.} A solid cylinder of height
1
00 m has a circular base. The density of the material is 2800 kg/m³. Atmospheric pressure is 100 kPa. What is the average pressure on the ground under the cylinder?



\textbf{A.} 8
50
0 kPa \\
\textbf{B.} 290
0 kPa \\
\textbf{C.} 1040 kPa \\
\textbf{D.} 1140 kPa \\

\textbf{Answer:} B \\
\textbf{Explanation:} [IMAGE:0]

\hrule
\vspace{1em}


\noindent
\textbf{Q817.} A solid rectangular prism of height 1
0
0 m has a rectangular base. The density of the material is 250 kg/m³. Atmospheric pressure is 100 kPa. What is the average pressure on the ground under the prism?



\textbf{A.} 80 kPa \\
\textbf{B.} 180 kPa \\
\textbf{C.} 20 kPa \\
\textbf{D.} 3
5
0 kPa \\

\textbf{Answer:} D \\
\textbf{Explanation:} [IMAGE:0]

\hrule
\vspace{1em}


\noindent
\textbf{Q818.} Uniformly accelerated circular motion has an angular displacement of 18 radians in the first 3 revolutions and -18 radians in the next 3 revolutions. Find its total angular displacement in the first 12 revolutions.
Let angular displacement per revolution form an arithmetic sequence with common difference
d
.



\textbf{A.} -36 rad \\
\textbf{B.} -72 rad \\
\textbf{C.} 0 rad \\
\textbf{D.} 36 rad \\

\textbf{Answer:} E \\
\textbf{Explanation:} Let angular displacement per revolution form an arithmetic sequence with common difference
d
. For the first 3 revolutions:
[IMAGE:0]
For revolutions 4-6:
[IMAGE:1]
[IMAGE:2]
[IMAGE:3]

\hrule
\vspace{1em}


\noindent
\textbf{Q819.} A solid cone of height 420 m has a circular base. The density of the material is 2100 kg/m³. Atmospheric pressure is 1
1
0 kPa. What is the average pressure on the ground under the cone?



\textbf{A.} 98 kPa \\
\textbf{B.} 108 kPa \\
\textbf{C.} 198 kPa \\
\textbf{D.} 980 kPa \\

\textbf{Answer:} F \\
\textbf{Explanation:} [IMAGE:0]

\hrule
\vspace{1em}


\noindent
\textbf{Q820.} A spring oscillator's total displacement in the first 4 cycles is 16 cm, and in the next 4 cycles, it is -48 cm. Find its total displacement in the first 20 cycles.
Let displacement per cycle form an arithmetic sequence with common difference
d
.



\textbf{A.} -240 cm \\
\textbf{B.} -160 cm \\
\textbf{C.} -560 cm \\
\textbf{D.} 0 cm \\

\textbf{Answer:} C \\
\textbf{Explanation:} Let displacement per cycle form an arithmetic sequence with common difference
d
.
For the first 4 cycles:
[IMAGE:0]
For cycles 5-8:
[IMAGE:1]
[IMAGE:2]
[IMAGE:3]

\hrule
\vspace{1em}


\noindent
\textbf{Q821.} A solid pyramid with a height of 20 m has a square base. The density of the material is
20
00 kg/m³. Atmospheric pressure is 100 kPa. What is the average pressure on the ground under the pyramid?



\textbf{A.} 100 kPa \\
\textbf{B.} 260
kPa \\
\textbf{C.} 300 kPa \\
\textbf{D.} 400 kPa \\

\textbf{Answer:} B \\
\textbf{Explanation:} [IMAGE:0]

\hrule
\vspace{1em}


\noindent
\textbf{Q822.} Three identical diodes are labeled A, B, and C, each with different polarity connections. Each diode is connected, in turn, across the same battery, which has a voltage of 6V and negligible internal resistance. The diodes are tested for conduction. What is the correct order of the diodes when arranged from "non-conducting" to "conducting"?
[IMAGE:0]



\textbf{A.} A
>
B
>
C \\
\textbf{B.} A
>
C
>
B \\
\textbf{C.} B
>
A
>
C \\
\textbf{D.} B
>
C
>
A \\

\textbf{Answer:} B \\
\textbf{Explanation:} Diode conduction depends on the polarity of the connection.
\cdot 
A: Diode is forward-biased (anode connected to positive terminal), current flows.
\cdot 
B: Diode is reverse-biased (cathode connected to positive terminal), current does not flow.
\cdot 
C: Diode is forward-biased but parallel to a reverse-biased diode, current flows through the forward-biased diode.
Order from non-conducting to conducting: B < C < A.

\hrule
\vspace{1em}


\noindent
\textbf{Q823.} An object in uniform accelerated motion has a displacement of 25 m in the first 5 seconds and -75 m in the next 5 seconds. Find its total displacement in the first 20 seconds.



\textbf{A.} -300 m \\
\textbf{B.} -500 m \\
\textbf{C.} -100 m \\
\textbf{D.} 0 m \\

\textbf{Answer:} B \\
\textbf{Explanation:} Let displacement per second form an arithmetic sequence with common difference
d
.
For the first 5 seconds:
[IMAGE:0]
For seconds 6-10:
[IMAGE:1]
[IMAGE:2]
[IMAGE:3]

\hrule
\vspace{1em}


\noindent
\textbf{Q824.} Three conductors of the same material with different shapes are connected to the same battery (negligible internal resistance):
\cdot 
A: Length L, diameter d
\cdot 
B: Length 2L, diameter d
\cdot 
C: Length L, diameter 2d
If arranged in order of
decreasing
total power, which option is correct?



\textbf{A.} B,A,C \\
\textbf{B.} C,A,B \\
\textbf{C.} A,B,C \\
\textbf{D.} B,C,A \\

\textbf{Answer:} B \\
\textbf{Explanation:} [IMAGE:0]

\hrule
\vspace{1em}


\noindent
\textbf{Q825.} In a free-fall experiment, the displacement of an object in the first 2 seconds is 10 meters, and in the next 2 seconds, it is 30 meters. Assuming displacement changes form an arithmetic sequence over time, find the total displacement in the first 8 seconds.



\textbf{A.} 80 m \\
\textbf{B.} 100 m \\
\textbf{C.} 120 m \\
\textbf{D.} 140 m \\

\textbf{Answer:} E \\
\textbf{Explanation:} Explanation
Let the displacement per second form an arithmetic sequence with common difference
d
. For the first 2 seconds:
[IMAGE:0]
For the next 2 seconds (terms 3 and 4):
[IMAGE:1]
Solving gives
[IMAGE:2]
Total displacement for 8 seconds:
[IMAGE:3]

\hrule
\vspace{1em}


\noindent
\textbf{Q826.} A battery (voltage 12V, internal resistance 1Ω) is connected to three different circuits, each containing a variable resistor. The total power is measured when the variable resistor is set to
R
1
​
=
2.5
Ω,
R
2
​
=2Ω, and
R
3
​
=3Ω. What is the correct order of the circuits when arranged in order of increasing power?
[IMAGE:0]



\textbf{A.} R
1
​
<
R
2
​
<
R
3
​ \\
\textbf{B.} R
1
​
<
R
3
​
<
R
2
​ \\
\textbf{C.} R
2
​
<
R
1
​
<
R
3
​ \\
\textbf{D.} R
2
​
<
R
3
​
<
R
1 \\

\textbf{Answer:} E \\
\textbf{Explanation:} [IMAGE:0]

\hrule
\vspace{1em}


\noindent
\textbf{Q827.} Three different resistors R1
​
=1Ω, R2
​
=2Ω, and R3
​
=3Ω are used to form three different circuits, labeled X, Y, and Z. Each circuit is connected, in turn, across the same battery, which has a voltage of 6V and negligible internal resistance. The voltage across each resistor is measured. What is the correct order of the circuits when arranged in order of
decreasing
voltage?
[IMAGE:0]



\textbf{A.} X
>
Y
>
Z \\
\textbf{B.} X
>
Z
>
Y \\
\textbf{C.} Y
>
X
>
Z \\
\textbf{D.} Y
>
Z
>
X \\

\textbf{Answer:} D \\
\textbf{Explanation:} [IMAGE:0]

\hrule
\vspace{1em}


\noindent
\textbf{Q828.} Three identical resistors are connected to the same battery (negligible internal resistance) as follows:
\cdot 
A: Two in series then parallel with the third
\cdot 
B: All three in parallel
\cdot 
C: All three in series
If arranged in order of
decreasing
total power, which option is correct?
[IMAGE:0]



\textbf{A.} C, A, B \\
\textbf{B.} B,A,C \\
\textbf{C.} A,C,B \\
\textbf{D.} C,B,A \\

\textbf{Answer:} B \\
\textbf{Explanation:} [IMAGE:0]

\hrule
\vspace{1em}


\noindent
\textbf{Q829.} A 1500 kg car is driving on a horizontal curve with a radius of 50 m. The coefficient of kinetic friction between the tires and the dry road is 0.4. When the road is wet, the coefficient of kinetic friction decreases to 0.
1
. What is the maximum safe speed for the car to navigate the curve on a wet road? (Gravitational field strength g=10N kg−1)



\textbf{A.} 27.8 km/h \\
\textbf{B.} 2
5
.
45
km/h \\
\textbf{C.} 30.0 km/h \\
\textbf{D.} 40.78 km/h \\

\textbf{Answer:} B \\
\textbf{Explanation:} The kinetic friction between the tires and the road provides the necessary centripetal force for the car to navigate the curve.
Centripetal force equation:
[IMAGE:0]
[IMAGE:1]
[IMAGE:2]
[IMAGE:3]

\hrule
\vspace{1em}


\noindent
\textbf{Q830.} A 2.0 kg object is at rest on a horizontal surface with a coefficient of friction 0.
5
. A force of 20 N is applied at an angle of 37° above the horizontal (sin37°=0.6, cos37°=0.8). What is the magnitude of the object’s acceleration as it starts moving? (Take g=10 m/s2g=10m/s2.)



\textbf{A.} 5.0 m/s² \\
\textbf{B.} 6.0 m/s² \\
\textbf{C.} 7.0 m/s² \\
\textbf{D.} 8.0 m/s² \\

\textbf{Answer:} B \\
\textbf{Explanation:} [IMAGE:0]
[IMAGE:1]

\hrule
\vspace{1em}


\noindent
\textbf{Q831.} A 5.0 kg object is placed on a conveyor belt inclined at 30° to the horizontal. The conveyor belt accelerates upward at 2.0 m/s². The coefficient of kinetic friction between the object and the belt is 0.
1
. What is the magnitude of the object's acceleration relative to the conveyor belt? (Gravitational field strength g=10N kg−1, sin30°=0.5, cos30°=3​/2≈0.866)



\textbf{A.} 0.56 m/s² \\
\textbf{B.} 1.02 m/s² \\
\textbf{C.} 1.56 m/s² \\
\textbf{D.} 2.05 m/s² \\

\textbf{Answer:} C \\
\textbf{Explanation:} The normal force is influenced by both the vertical component of gravity and the vertical component of the inertial force due to the belt's acceleration.
[IMAGE:0]
[IMAGE:1]
[IMAGE:2]
[IMAGE:3]
[IMAGE:4]

\hrule
\vspace{1em}


\noindent
\textbf{Q832.} Three identical springs are used to form three different mechanical systems, labeled D, E, and F. Each system is connected, in turn, to the same mass block, and the total spring constant of each system is measured. What is the correct order of the systems when arranged in order of
decreasing
spring constant?
[IMAGE:0]



\textbf{A.} D
>
E
>
F \\
\textbf{B.} D
>
F
>
E \\
\textbf{C.} E
>
D
>
F \\
\textbf{D.} E
>
F
>
D \\

\textbf{Answer:} D \\
\textbf{Explanation:} The total spring constant depends on the connection method.
\cdot 
D is two springs in series, with a total spring constant of
k
/2
​
.
\cdot 
E is two springs in parallel, with a total spring constant of 2
k
.
\cdot 
F is a single spring, with a total spring constant of
k
.
Total spring constant order:
k
/2
​
<
k
<2
k
, so spring constant order:
E > F > D
.

\hrule
\vspace{1em}


\noindent
\textbf{Q833.} A 2.0 kg object is placed on a horizontal rotating disk. The maximum static frictional force between the object and the disk is
8
.0 N. The disk starts rotating around its central axis with gradually increasing angular velocity. What is the angular velocity of the disk when the object is about to slip? (Gravitational field strength g=10N kg−1)



\textbf{A.} 2.45 rad/s \\
\textbf{B.} 3.05 rad/s \\
\textbf{C.} 2
.
0
0 rad/s \\
\textbf{D.} 5.30 rad/s \\

\textbf{Answer:} C \\
\textbf{Explanation:} Maximum Static Friction Provides Centripetal Force:
When the object is about to slip, the maximum static friction provides the necessary centripetal force. Centripetal force equation:
[IMAGE:0]
Given fmax​=12.0N, m=2.0kg, and assuming r=1.0m.
[IMAGE:1]
[IMAGE:2]

\hrule
\vspace{1em}


\noindent
\textbf{Q834.} Three identical inductors are used to form three different circuits, labeled M, N, and O. Each circuit is connected, in turn, across the same AC power supply, which has a constant voltage and negligible internal resistance. The total inductance of each circuit is measured. What is the correct order of the circuits when arranged in order of
decreasing
inductance?
[IMAGE:0]



\textbf{A.} M
>
N
>
O \\
\textbf{B.} M
>
O
>
N \\
\textbf{C.} N
>
M
>
O \\
\textbf{D.} N
>
O
>
M \\

\textbf{Answer:} B \\
\textbf{Explanation:} The total inductance depends on the connection method.
\cdot 
M is two inductors in series, with a total inductance of 2
L
.
\cdot 
N is two inductors in parallel, with a total inductance of
L
/2
​
.
\cdot 
O is a single inductor, with a total inductance of
L
.
Total inductance order:
L
/2
​
<
L
<2
L
, so inductance order:
M > O > N
.

\hrule
\vspace{1em}


\noindent
\textbf{Q835.} A 4.0 kg object is at rest on a horizontal surface with a coefficient of friction 0.
5
. Two forces are applied: a horizontal force of 24 N to the right and a vertical upward force of 8 N. What is the magnitude of the object’s acceleration as it starts moving? (Take g=10 m/s
2.
)



\textbf{A.} 2.0 m/s² \\
\textbf{B.} 3.0 m/s² \\
\textbf{C.} 4.0 m/s² \\
\textbf{D.} 5.0 m/s² \\

\textbf{Answer:} A \\
\textbf{Explanation:} Vertical Force Analysis: The vertical upward force reduces the normal force.
Normal force N=mg−Fvertical=4×10−8=32 .
Friction Calculation: Friction f=μN=0.
5
×32=
16
N.
Net horizontal force Fnet=Fhorizontal−f=24−
16
=
8
.
Acceleration Calculation: By Newton’s second law, a=Fnetm=
8/
4=
2
m/s
2
.

\hrule
\vspace{1em}


\noindent
\textbf{Q836.} 3.
Three identical resistors are connected in the following configurations to the same battery (negligible internal resistance):
\cdot 
A: All three in series
\cdot 
B: Two in parallel combined with the third in series
\cdot 
C: All three in parallel
[IMAGE:0]
If arranged in order of
decreasing
total power, which option is correct?



\textbf{A.} A, B, C \\
\textbf{B.} C,B,A \\
\textbf{C.} B,A,C \\
\textbf{D.} A,C,B \\

\textbf{Answer:} B \\
\textbf{Explanation:} [IMAGE:0]

\hrule
\vspace{1em}


\noindent
\textbf{Q837.} A 5.0 kg object is at rest on an inclined plane with an angle of 30°. The coefficient of kinetic friction between the object and the plane is 0.
1
. A force of 10.0 N parallel to the plane upward and a force of 8.0 N perpendicular to the plane downward are applied simultaneously. What is the magnitude of the object's acceleration as it begins to move? (Gravitational field strength g=10N kg−1, sin30°=0.5, cos30°=3​/2≈0.866)



\textbf{A.} 10.5 m/s² \\
\textbf{B.} 5.05 m/s² \\
\textbf{C.} 1.25 m/s² \\
\textbf{D.} 2.05 m/s² \\

\textbf{Answer:} F \\
\textbf{Explanation:} The normal force is affected by both the vertical component of gravity and the external perpendicular force.
[IMAGE:0]
[IMAGE:1]
[IMAGE:2]
[IMAGE:3]

\hrule
\vspace{1em}


\noindent
\textbf{Q838.} Three identical capacitors are used to form three different circuits, labeled as P, Q and R respectively. Each circuit is successively connected to the same power supply, which has a constant voltage and negligible internal resistance. The total charge stored in each circuit is measured. When these circuits are arranged in the order of decreasing charge quantity, what is the correct sequence?
[IMAGE:0]



\textbf{A.} P
>
Q
>
R \\
\textbf{B.} P
>
R
>
Q \\
\textbf{C.} Q
>
P
>
R \\
\textbf{D.} Q
>
R
>
P \\

\textbf{Answer:} D \\
\textbf{Explanation:} The total charge stored in a capacitor is proportional to the total capacitance. A larger total capacitance results in a higher charge.
\cdot 
P
is two capacitors in series, with a total capacitance of
C
/2
​
.
\cdot 
Q
is two capacitors in parallel, with a total capacitance of 2
C
.
\cdot 
R
is a single capacitor, with a total capacitance of
C
.
Total capacitance order:
C
/2
​
<
C
<2
C
, so charge order: P < R < Q.

\hrule
\vspace{1em}


\noindent
\textbf{Q839.} A 2.0 kg object rests on a horizontal surface with a coefficient of friction 0.
5
. Two forces are applied: a horizontal force of 20 N and a vertical upward force of 4 N. What is the magnitude of the acceleration? (Take g=10 m/s
2
.)



\textbf{A.} 6
.0 m/s² \\
\textbf{B.} 7.25 m/s² \\
\textbf{C.} 7.5 m/s² \\
\textbf{D.} 8.0 m/s² \\

\textbf{Answer:} A \\
\textbf{Explanation:} Vertical Force: Reduces normal force.
N=2×10−4=16 N.
Friction: f=0.
5
×16=
8
N.
Horizontal Net Force: 20−
8
=1
2
N.
Acceleration: a=1
2/
2=
6
m/s
2
.

\hrule
\vspace{1em}


\noindent
\textbf{Q840.} A 4.0 kg object is at rest on a horizontal surface. The coefficient of kinetic friction between the object and the surface is 0.
2
. A vertical upward force of 8.0 N and three horizontal forces (6.0 N, 8.0 N, and 10.0 N, all mutually perpendicular) are applied simultaneously. What is the magnitude of the object's acceleration as it begins to move? (Gravitational field strength g=10N kg
−1
)



\textbf{A.} 1.00 m/s² \\
\textbf{B.} 1.50 m/s² \\
\textbf{C.} 2.50 m/s² \\
\textbf{D.} 1.
94
m/s² \\

\textbf{Answer:} D \\
\textbf{Explanation:} Vertical Force Analysis:
The upward vertical force reduces the normal reaction force.
[IMAGE:0]
[IMAGE:1]
[IMAGE:2]
[IMAGE:3]
[IMAGE:4]

\hrule
\vspace{1em}


\noindent
\textbf{Q841.} A point object of mass 2
.0 kg is at rest on a horizontal surface with a coefficient of friction 0.
5
. Two perpendicular forces are applied simultaneously: a horizontal force of 20 N and a vertical upward force of 4 N. What is the magnitude of the acceleration of the object as it begins to move? (Take g=10 m/s
2
)



\textbf{A.} 5.0 m/s² \\
\textbf{B.} 7.25 m/s² \\
\textbf{C.} 8
.5 m/s² \\
\textbf{D.} 8.0 m/s² \\

\textbf{Answer:} D \\
\textbf{Explanation:} Vertical Force Analysis: The vertical upward force reduces the normal force.
Horizontal Net Force: Net horizontal force
[IMAGE:0]
Acceleration Calculation: By Newton’s second law,
[IMAGE:1]

\hrule
\vspace{1em}


\noindent
\textbf{Q842.} Use three identical resistors to form three different circuits, labeled as A, B and C respectively. Each circuit is successively connected to V, 3V and v batteries, and the internal resistance of the batteries can be ignored. Measure the total power generated by each circuit. When these circuits are arranged in order of power from largest to smallest, what is the correct sequence?
[IMAGE:0]



\textbf{A.} A < B < C \\
\textbf{B.} A < C < B \\
\textbf{C.} B < A < C \\
\textbf{D.} B < C < A \\

\textbf{Answer:} B \\
\textbf{Explanation:} [IMAGE:0]

\hrule
\vspace{1em}


\noindent
\textbf{Q843.} A 3.0 kg object is at rest on a horizontal surface. The coefficient of kinetic friction between the object and the surface is 0.
05
. A vertical upward force of 10.0 N and two perpendicular horizontal forces (5.0 N and 12.0 N) are applied simultaneously. What is the magnitude of the object's acceleration as it begins to move? (Gravitational field strength g=10N kg−1)



\textbf{A.} 2.0 m/s² \\
\textbf{B.} 2.5 m/s² \\
\textbf{C.} 3.0 m/s² \\
\textbf{D.} 3.5 m/s² \\

\textbf{Answer:} E \\
\textbf{Explanation:} Vertical Force Analysis:
The upward vertical force reduces the normal reaction force.
Normal force N=mg−Fvertical​=3×10−10=20N.
Frictional Force Calculation:
Kinetic friction f=μN=0.
05
×20=
1
.0N.
Resultant Horizontal Force:
The vector sum of the two perpendicular horizontal forces is:
[IMAGE:0]
[IMAGE:1]

\hrule
\vspace{1em}


\noindent
\textbf{Q844.} An object of weight
3
0 N hangs from the end of a light inextensible string of length 0.5 m, which is attached to the ceiling. A force of 30 N is applied to the object at an angle of 45° above the horizontal, causing it to move to a new equilibrium position. By how much has the gravitational potential energy of the object increased?



\textbf{A.} 2.40 J \\
\textbf{B.} 1
.
8
0 J \\
\textbf{C.} 3.60 J \\
\textbf{D.} 4.20 J \\

\textbf{Answer:} B \\
\textbf{Explanation:} Force Resolution:
[IMAGE:0]
Height Change: The vertical rise is Δh=0.5×(1−cosα)=0.06 m
Gravitational Potential Energy: ΔPE=
3
0×0.06=
1
.
8
J

\hrule
\vspace{1em}


\noindent
\textbf{Q845.} A beam of light enters the center of a semi-circular glass block from the air.
The path of the light in the glass is as follows: it first enters the glass from air, reflects off the coated surface, and finally exits back into the air. The refractive index of the glass is 1.5, and the angle of incidence is
60
\circ 
. What is the angle of refraction when the light exits the glass back into the air?
[IMAGE:0]
[IMAGE:1]



\textbf{A.} 30
\circ  \\
\textbf{B.} 45
\circ  \\
\textbf{C.} 60
\circ  \\
\textbf{D.} 90
\circ  \\

\textbf{Answer:} C \\
\textbf{Explanation:} As can be seen from the figure, the incident angle is also 60°.

\hrule
\vspace{1em}


\noindent
\textbf{Q846.} A
1
0 N object slides down from a smooth inclined plane of height 0.4 m and then travels 1 m on a stationary conveyor belt. The coefficient of kinetic friction between the object and the belt is 0.1. Finally, the object hits a spring with a spring constant of 200 N/m. Ignoring other frictional forces, what is the kinetic energy of the object when it hits the spring?
[IMAGE:0]



\textbf{A.} 3
.
08
J \\
\textbf{B.} 4.32 J \\
\textbf{C.} 6.75 J \\
\textbf{D.} 9.25 J \\

\textbf{Answer:} A \\
\textbf{Explanation:} Height of the incline: 0.4 m
Gravitational acceleration: 9.8 m/s²
Mass of the object: m =
1
0 / 9.8 ≈
1
.0
2
kg
Using conservation of mechanical energy: mgh = 0.5mv²
Substituting values: v = sqrt(2gh) = sqrt(2×9.8×0.4) ≈ 2.8 m/s
Kinetic energy: KE₁ = 0.5 ×
1
.0
2
× (2.8)² ≈
4.08
J
[IMAGE:0]
[IMAGE:1]
[IMAGE:2]
[IMAGE:3]
[IMAGE:4]
Final kinetic energy:
4.08
-
1
≈
3
.
08
J

\hrule
\vspace{1em}


\noindent
\textbf{Q847.} Light travels from glass (n1=1.5) into air (n2=1.0). When the angle of incidence is
60
°, what phenomenon occurs? If refraction happens, find the sine of the refracted angle.
[IMAGE:0]
[IMAGE:1]



\textbf{A.} Total internal reflection occurs \\
\textbf{B.} sin
θ
2
​
=0.67 \\
\textbf{C.} sin
θ
2
​
=0.89 \\
\textbf{D.} sin
θ
2
​
=1.0 \\

\textbf{Answer:} A \\
\textbf{Explanation:} [IMAGE:0]

\hrule
\vspace{1em}


\noindent
\textbf{Q848.} A 20 N object is placed on a horizontal conveyor belt moving at a constant speed of 3 m/s to the right. The coefficient of kinetic friction between the object and the belt is 0.1. When the object accelerates until its speed matches the belt's speed, by how much has the kinetic energy of the object increased?



\textbf{A.} 2.0 J \\
\textbf{B.} 4.0 J \\
\textbf{C.} 9.2 J \\
\textbf{D.} 8.0 J \\

\textbf{Answer:} C \\
\textbf{Explanation:} The object on the conveyor belt experiences a frictional force that accelerates it until its speed matches the belt's speed. By calculating the work done by friction, the increase in kinetic energy can be determined.
Initial velocity of the object: 0 m/s
Acceleration: a = f/m = 2 / (20/9.8) ≈ 0.98 m/s²
Time to reach belt speed: t = v/a = 3 / 0.98 ≈ 3.06 s
Distance moved during acceleration: s = 0.5 × a × t² ≈ 0.5 × 0.98 × (3.06)² ≈ 4.4 m
[IMAGE:0]

\hrule
\vspace{1em}


\noindent
\textbf{Q849.} A beam of light passes from a medium into a vacuum with an angle of incidence of 30
\circ 
. The refractive index of the medium is
. What is the angle of refraction in the vacuum?
[IMAGE:0]



\textbf{A.} 30
\circ  \\
\textbf{B.} 45
\circ  \\
\textbf{C.} 60
\circ  \\
\textbf{D.} 90
\circ  \\

\textbf{Answer:} B \\
\textbf{Explanation:} According to Snell's Law, the relationship between the angle of incidence and the angle of refraction when light passes from a medium into a vacuum

\hrule
\vspace{1em}


\noindent
\textbf{Q850.} A 10 N object is thrown vertically upward with an initial velocity of 2 m/s. Ignoring air resistance, by how much has the gravitational potential energy of the object increased during its ascent?



\textbf{A.} 2.1 J \\
\textbf{B.} 2.8 J \\
\textbf{C.} 3.5 J \\
\textbf{D.} 4.2 J \\

\textbf{Answer:} A \\
\textbf{Explanation:} The object is thrown vertically upward, and air resistance is ignored. By calculating the height the object rises, the increase in gravitational potential energy can be determined.
[IMAGE:0]
[IMAGE:1]

\hrule
\vspace{1em}


\noindent
\textbf{Q851.} A 60 N object is placed on a smooth inclined plane with an angle of 30°, connected via a light inextensible string over a pulley to a 40 N hanging object. The system starts from rest and moves until the hanging object descends 1 m. Ignoring pulley friction and air resistance, by how much has the gravitational potential energy of the hanging object increased?
[IMAGE:0]



\textbf{A.} 40 J \\
\textbf{B.} 15 J \\
\textbf{C.} 20 J \\
\textbf{D.} 25 J \\

\textbf{Answer:} A \\
\textbf{Explanation:} The system consists of an object on an inclined plane and a hanging object connected by a string over a pulley. As the system moves, the hanging object rises while the object on the plane slides down. By analyzing the energy changes, the increase in gravitational potential energy of the hanging object can be determined.
[IMAGE:0]

\hrule
\vspace{1em}


\noindent
\textbf{Q852.} An object of weight 50 N hangs from the end of a light inextensible string of length 1 m, which is attached to the ceiling. A constant horizontal wind force of 40 N blows on the object, causing it to move to a new equilibrium position. By how much has the gravitational potential energy of the object increased as a result of its change of position?



\textbf{A.} 2.1J \\
\textbf{B.} 2.8J \\
\textbf{C.} 3.50J \\
\textbf{D.} 4.2J \\

\textbf{Answer:} F \\
\textbf{Explanation:} The object moves to a new equilibrium position under the horizontal wind force, forming an angle θ with the vertical. By analyzing the forces and calculating the height increase, the change in gravitational potential energy can be determined.
Force Analysis
Weight of the object: 50 N downward.
Tension in the string: T along the string.
Horizontal wind force: 40 N.
Equilibrium Conditions
Horizontal: T*sinθ = 40 N
Vertical: T*cosθ = 50 N
Angle Calculation
From the equations:
tanθ = 40 / 50 = 0.8 \to  θ ≈ 38.66°
Height Increase
The height increase h = L(1 - cosθ) ≈ 1(1 - cos38.66°) ≈ 1 × 0.215 ≈ 0.215 m
Potential Energy Change
ΔU = mgh = 50 × 0.215 ≈ 10.75 J

\hrule
\vspace{1em}


\noindent
\textbf{Q853.} Light travels sequentially from air (n1=1.0) into glass (n2=1.5), and then into water (n3=1.
5
). If the initial angle of incidence is 60°, what is the sine of the final refracted angle sin
θ
3
​
in water?
[IMAGE:0]



\textbf{A.} 0.65 \\
\textbf{B.} 0.87 \\
\textbf{C.} 0.58 \\
\textbf{D.} 0.71 \\

\textbf{Answer:} C \\
\textbf{Explanation:} [IMAGE:0]

\hrule
\vspace{1em}


\noindent
\textbf{Q854.} An object weighing 50 N hangs from a light inextensible string of length 0.8 m attached to the ceiling. A constant force of 35 N is applied to the object at an angle of 37° above the horizontal, moving it to a new equilibrium position. By how much has the gravitational potential energy of the object increased?



\textbf{A.} 2.8 J \\
\textbf{B.} 3.5 J \\
\textbf{C.} 4.2 J \\
\textbf{D.} 5.6 J \\

\textbf{Answer:} F \\
\textbf{Explanation:} The object is in equilibrium under three forces: gravity (50 N downward), the applied force (35 N at 37°), and tension TT. Resolving forces:
[IMAGE:0]
[IMAGE:1]
[IMAGE:2]
[IMAGE:3]

\hrule
\vspace{1em}


\noindent
\textbf{Q855.} Light travels from medium A (refractive index
2.4
) to medium B (refractive index 1.8). If the angle of incidence is 60°, what is the value of sin
θ
2
​
(sine of the refracted angle)?



\textbf{A.} [IMAGE:0] \\
\textbf{B.} [IMAGE:1] \\
\textbf{C.} [IMAGE:2] \\
\textbf{D.} [IMAGE:3] \\

\textbf{Answer:} B \\
\textbf{Explanation:} [IMAGE:0]

\hrule
\vspace{1em}


\noindent
\textbf{Q856.} An object weighing 30 N hangs from the end of a light inextensible string of length 0.5 m, which is attached to the ceiling. A constant horizontal force is applied to the object, causing it to move to a new equilibrium position where the string makes an angle of 30° with the vertical. By how much has the gravitational potential energy of the object increased?



\textbf{A.} 3.0 J \\
\textbf{B.} 4.02 J \\
\textbf{C.} 6.01 J \\
\textbf{D.} 7.53 J \\

\textbf{Answer:} E \\
\textbf{Explanation:} Explanation (English Version)
Determine the equilibrium position: The object reaches a new equilibrium position where the string makes a 30° angle with the vertical.
Force analysis: The object experiences a weight mg=60N, tension T in the string, and a horizontal force F. The equilibrium conditions are:
[IMAGE:0]
[IMAGE:1]
[IMAGE:2]
[IMAGE:3]
[IMAGE:4]
[IMAGE:5]
[IMAGE:6]

\hrule
\vspace{1em}


\noindent
\textbf{Q857.} An object is placed
40
cm in front of a convex lens with a focal length of
20
cm. According to the lens formula, where is the image located? What are the characteristics of the image?
[IMAGE:0]



\textbf{A.} Image distance
4
0 cm, real image, same size as the object \\
\textbf{B.} Image distance
20
cm, virtual image, magnified \\
\textbf{C.} Image distance 30 cm, real image, reduced \\
\textbf{D.} Image distance
40
cm, virtual image, reduced \\

\textbf{Answer:} A \\
\textbf{Explanation:} [IMAGE:0]

\hrule
\vspace{1em}


\noindent
\textbf{Q858.} A
n object weighing 40 N hangs from a light inextensible string of length 0.35 m attached to the ceiling. A constant horizontal force of 30 N is applied to the object, moving it to a new equilibrium position where the string is no longer vertical. By how much has the gravitational potential energy of the object increased?



\textbf{A.} 2.1 J \\
\textbf{B.} 2.8 J \\
\textbf{C.} 3.5 J \\
\textbf{D.} 4.2 J \\

\textbf{Answer:} B \\
\textbf{Explanation:} The object is in equilibrium under three forces: gravity (40 N downward), horizontal force (45 N rightward), and tension TT along the string. Resolving forces:
[IMAGE:0]
[IMAGE:1]
[IMAGE:2]
Since cos36.87
\circ 
=0.8:
[IMAGE:3]
[IMAGE:4]

\hrule
\vspace{1em}


\noindent
\textbf{Q859.} A light beam travels from a transparent medium into air (refractive index 1.0). When the angle of incidence is 30°, the angle of refraction is 60°. Determine the refractive index of the medium and what phenomenon occurs when the angle of incidence increases to
9
0°?
[IMAGE:0]



\textbf{A.} [IMAGE:0] \\
\textbf{B.} [IMAGE:1] \\
\textbf{C.} [IMAGE:2] \\
\textbf{D.} [IMAGE:3] \\

\textbf{Answer:} B \\
\textbf{Explanation:} [IMAGE:0]

\hrule
\vspace{1em}


\noindent
\textbf{Q860.} An object weighing 30 N hangs from the end of a light inextensible string of length 0.45 m, which is attached to the ceiling. A constant horizontal force of 18 N is applied to the object, causing it to move to a new equilibrium position where the string is no longer vertical. By how much has the gravitational potential energy of the object increased?



\textbf{A.} 3.0 J \\
\textbf{B.} 4.0 J \\
\textbf{C.} 2.7 J \\
\textbf{D.} 6.0 J \\

\textbf{Answer:} C \\
\textbf{Explanation:} Determine the equilibrium position: The object reaches a new equilibrium position where the string makes an angle θ with the vertical due to the horizontal force.
Force analysis: The horizontal force F=18N and weight mg=30N. The tension T in the string satisfies:
[IMAGE:0]
[IMAGE:1]
[IMAGE:2]
[IMAGE:3]
[IMAGE:4]

\hrule
\vspace{1em}


\noindent
\textbf{Q861.} A radium-226 nucleus (Ra-226) moving at velocity u undergoes alpha decay to form a radon-222 nucleus (Rn-222) and an alpha particle. In the laboratory frame, the total energy of the radium-226 nucleus is 0.5E. The kinetic energy ratio of the radon-222 nucleus to the alpha particle is 1:4. What is the kinetic energy of the alpha particle?



\textbf{A.} E​/5 \\
\textbf{B.} 2E​/5 \\
\textbf{C.} E​/4 \\
\textbf{D.} 4E​/226 \\

\textbf{Answer:} B \\
\textbf{Explanation:} [IMAGE:0]
:
Since the radium-226 is moving, relativistic effects are considered. The total energy 0.5E of radium-226 is:
[IMAGE:1]
The total kinetic energy K of the decay products is:
[IMAGE:2]
Using the kinetic energy ratio 1:4, the kinetic energy of radon-222 is K​/5 and that of the alpha particle is 4K​/5.

\hrule
\vspace{1em}


\noindent
\textbf{Q862.} A light ray from a point source is reflected by two plane mirrors. The angle between the first mirror and the second mirror is 60
\circ 
, and the angle between the incident ray and the first mirror is
45
\circ 
. What is the angle between the emergent ray after two reflections and the original incident ray?
[IMAGE:0]



\textbf{A.} 30
\circ  \\
\textbf{B.} 60
\circ  \\
\textbf{C.} 90
\circ  \\
\textbf{D.} 120
\circ  \\

\textbf{Answer:} D \\
\textbf{Explanation:} According to the law of reflection, the angle of incidence is equal to the angle of reflection. After the first reflection from the first mirror, the reflected ray makes an angle of
45
\circ 
with the first mirror.
Next, the reflected ray strikes the second mirror. Since the angle between the two mirrors is 60
\circ 
, the angle between the incident ray and the second mirror can be calculated as
45
\circ 
(because the angle between the reflected ray from the first mirror and the second mirror is equal to the angle between the two mirrors minus the angle between the reflected ray and the first mirror).
According to the law of reflection, the reflected ray from the second mirror also makes an angle of
45
\circ 
with the second mirror. After two reflections, the angle between the emergent ray and the original incident ray can be calculated as 120
\circ 
(since the angle between the two mirrors is 60
\circ 
, and each reflection changes the direction of the ray, resulting in a total angle of 60
\circ 
×2=120
\circ 
).

\hrule
\vspace{1em}


\noindent
\textbf{Q863.} A stationary actinium-227 (Ac-227) nucleus undergoes alpha decay to form francium-223 (Fr-223) and an alpha particle. After decay, the two particles leave tracks in a cloud chamber. If the track length of the alpha particle is 233/4 times that of Fr-223 (assuming equal motion time and constant resistance), and the total kinetic energy released is 2E, what is the kinetic energy of the alpha particle?
[IMAGE:0]



\textbf{A.} 3E/227 \\
\textbf{B.} 446E/227 \\
\textbf{C.} E/2 \\
\textbf{D.} 223E/(223+4) \\

\textbf{Answer:} B \\
\textbf{Explanation:} Track Length vs. Velocity: Under constant resistance and equal motion time, track length is proportional to velocity, i.e.,
[IMAGE:0]
Momentum Conservation: Total momentum after decay is zero, so m
α
v
α
=m
Fr
v
Fr
​. Substituting masses m
α
=4u, m
Fr
=223u:
[IMAGE:1]
[IMAGE:2]

\hrule
\vspace{1em}


\noindent
\textbf{Q864.} The diagram shows light passing from air (refractive index 1.0) into a transparent medium. When the angle of incidence is 60°, the angle of refraction is 30°. If the angle of incidence changes to
90
°, what is the value of sin
θ
2?
[IMAGE:0]



\textbf{A.} [IMAGE:0] \\
\textbf{B.} [IMAGE:1] \\
\textbf{C.} [IMAGE:2] \\
\textbf{D.} [IMAGE:3] \\

\textbf{Answer:} D \\
\textbf{Explanation:} [IMAGE:0]

\hrule
\vspace{1em}


\noindent
\textbf{Q865.} A stationary radium-226 nucleus (Ra-226) undergoes alpha decay to form a radon-222 nucleus (Rn-222) and an alpha particle. The relative velocity between the radon-222 nucleus and the alpha particle after decay is 2v. What is the kinetic energy of the alpha particle?



\textbf{A.} [IMAGE:0] \\
\textbf{B.} [IMAGE:1] \\
\textbf{C.} [IMAGE:2] \\
\textbf{D.} [IMAGE:3] \\

\textbf{Answer:} D \\
\textbf{Explanation:} Conservation of Momentum
Before decay, the momentum of the radium-226 nucleus is zero. After decay, the momentum of the radon-222 nucleus and the alpha particle are equal in magnitude and opposite in direction. Let the mass of the alpha particle be 4 and the mass of radon-222 be 222. The conservation of momentum can be expressed as:
[IMAGE:0]
[IMAGE:1]
[IMAGE:2]

\hrule
\vspace{1em}


\noindent
\textbf{Q866.} The ray diagram shows light passing from air into a medium. Two angles,
x
and
y
, are shown on the diagram. When
x
=30
\circ 
,
y
=45
\circ 
. When
x
=
60
\circ 
, what is the value of sin
y
?
[IMAGE:0]



\textbf{A.} [IMAGE:0] \\
\textbf{B.} [IMAGE:1] \\
\textbf{C.} 1 \\
\textbf{D.} [IMAGE:2] \\

\textbf{Answer:} C \\
\textbf{Explanation:} According to Snell's Law, the relationship between the refractive index
n
and the angles of incidence and refraction is:
[IMAGE:0]
When
x
=30
\circ 
and
y
=45
\circ 
, substituting into the formula gives:
[IMAGE:1]
[IMAGE:2]
>1 Therefore, total reflection occurs. Thus, the angle of emission is 90°, and sin(90°) = 1.

\hrule
\vspace{1em}


\noindent
\textbf{Q867.} A stationary uranium-238 nucleus (U-238) undergoes alpha decay to form a thorium-234 nucleus (Th-234). Subsequently, the thorium-234 nucleus undergoes beta decay to form a protactinium-234 nucleus (Pa-234) and an electron. The total kinetic energy produced by the two decays is 3E. What is the kinetic energy of the alpha particle released in the first decay?



\textbf{A.} 4
E
​/238 \\
\textbf{B.} 4
E
​/234 \\
\textbf{C.} 234
E
​/238 \\
\textbf{D.} 234
E
​/(237 \\

\textbf{Answer:} C \\
\textbf{Explanation:} Explanation: First Decay (U-238 \to  Th-234 + α particle); Second Decay (Th-234 \to  Pa-234 + electron); Total Kinetic Energy E = E₁ + E₂
[IMAGE:0]

\hrule
\vspace{1em}


\noindent
\textbf{Q868.} A stationary americium -243 (Am-243) nucleus undergoes alpha decay to form neptunium-239 (Np-239) and an alpha particle. The total kinetic energy released in the decay is 2E, and the two particles move in opposite directions. What is the kinetic energy of the alpha particle?



\textbf{A.} 4E/243 \\
\textbf{B.} 478E/243 \\
\textbf{C.} E/2 \\
\textbf{D.} 239E/(239+4) \\

\textbf{Answer:} B \\
\textbf{Explanation:} By conservation of momentum, the magnitudes of the momenta of the alpha particle and Np-239 are equal. Let the mass of the alpha particle be 4u and that of Np-239 be 239u.
[IMAGE:0]

\hrule
\vspace{1em}


\noindent
\textbf{Q869.} A stationary radium-226 nucleus (Ra-226) undergoes alpha decay to form a radon-222 nucleus (Rn-222). Subsequently, the radon-222 nucleus undergoes another alpha decay to form a polonium-218 nucleus (Po-218). The total kinetic energy produced by the two decays is 2E. What is the kinetic energy of the alpha particle released in the second decay?



\textbf{A.} 4E​/226 \\
\textbf{B.} 4E​/218 \\
\textbf{C.} 436E​/226 \\
\textbf{D.} 218E​/222 \\

\textbf{Answer:} C \\
\textbf{Explanation:} (1)First Decay (Ra-226 \to  Rn-222 + α particle)
Initial momentum is zero. After decay, the momentum of Rn-222 and the α particle are equal in magnitude and opposite in direction. Mass of α particle = 4, mass of Rn-222 = 222.
[IMAGE:0]
(2)Second Decay (Rn-222 \to  Po-218 + α particle)
Initial momentum is that of Rn-222. After decay, the momentum of Po-218 and the α particle are equal in magnitude and opposite in direction.Mass of α particle = 4, mass of Po-218 = 218.
[IMAGE:1]
[IMAGE:2]

\hrule
\vspace{1em}


\noindent
\textbf{Q870.} In space, two spacecraft, A and B, are moving towards each other.
The mass of spacecraft A is 0.5 times that of spacecraft B
. Spacecraft A is moving to the right at speed v, while spacecraft B is moving to the left at speed v. They collide and stick together. What is the magnitude and direction of their common velocity after the collision?



\textbf{A.} Magnitude 0, direction not applicable \\
\textbf{B.} [IMAGE:0] \\
\textbf{C.} [IMAGE:1] \\
\textbf{D.} [IMAGE:2] \\

\textbf{Answer:} B \\
\textbf{Explanation:} By the conservation of momentum, the total momentum of the system remains constant in the absence of external forces. Before the collision:
\cdot 
Momentum of spacecraft A:
0.5
m
×
v
=
0.5
mv
(to the right)
\cdot 
Momentum of spacecraft B:
m
×(
−
v
)=
−
mv
(to the left)
Total momentum of the system:
0,5
mv
−
mv
=
-0.5
mv
(to the right)
After the collision, the combined mass of the spacecraft is
0.5
m
+
m
=
1.5
m
. Let the common velocity after the collision be
u
. Applying the conservation of momentum:
1.5
m
×
u
=
-0.5
mv

\hrule
\vspace{1em}


\noindent
\textbf{Q871.} A stationary californium-252 (Cf-252) nucleus undergoes alpha decay to form curium-248 (Cm-248) and an alpha particle. The total kinetic energy released in the decay is 2E. If the alpha particle and Cm-248 move in opposite directions, and their kinetic energies are determined by momentum conservation, what is the kinetic energy of the alpha particle?



\textbf{A.} 4E/252 \\
\textbf{B.} 496E/252 \\
\textbf{C.} E/2 \\
\textbf{D.} 248E/(248+4) \\

\textbf{Answer:} B \\
\textbf{Explanation:} By conservation of momentum, the magnitudes of the momenta of the alpha particle and Cm-248 are equal. Let the mass of the alpha particle be 4u and that of Cm-248 be 248u. From momentum conservation: 4vα=248vCm  leading to the velocity ratio vα:vCm=248:4 Kinetic energy is proportional to mass and the square of velocity. Thus, the ratio of kinetic energies is:
[IMAGE:0]

\hrule
\vspace{1em}


\noindent
\textbf{Q872.} A stationary nitrogen molecule (N₂) decomposes into two nitrogen atoms (N) at high temperatures. The total kinetic energy of the two nitrogen atoms after decomposition is E, and the mass of each nitrogen atom is 16u (atomic mass units). What is the kinetic energy of one of the nitrogen atoms?



\textbf{A.} E/
14​ \\
\textbf{B.} E
​/16 \\
\textbf{C.} 14
E
​/28 \\
\textbf{D.} 4
E
​/14 \\

\textbf{Answer:} B \\
\textbf{Explanation:} Before decomposition, the nitrogen molecule is stationary, so its momentum is zero. After decomposition, the two nitrogen atoms have equal magnitude but opposite direction momenta. Let the mass of each nitrogen atom be 14u, and their velocities be v₁ and v₂. By the conservation of momentum:
[IMAGE:0]
Since m₁ = m₂ = 16u, this simplifies to:
[IMAGE:1]
he total kinetic energy is E, so:
[IMAGE:2]
Substituting m₁ = m₂ = 16u and v₁ = -v₂, we get:
[IMAGE:3]
Therefore, the kinetic energy of one nitrogen atom is:
[IMAGE:4]

\hrule
\vspace{1em}


\noindent
\textbf{Q873.} When a stationary radium-226 nucleus decays by alpha emission to form a radon-222 nucleus, the total kinetic energy produced by the decay is 2E. What is the kinetic energy of the alpha particle?
[IMAGE:0]



\textbf{A.} 4E​/226 \\
\textbf{B.} 444E​/452 \\
\textbf{C.} 2E​ \\
\textbf{D.} /222E​ \\

\textbf{Answer:} B \\
\textbf{Explanation:} By the conservation of momentum, the momentum of the radium-226 nucleus before decay is zero. After decay, the momentum of the radon-222 nucleus and the alpha particle are equal in magnitude and opposite in direction. Let the mass of the alpha particle be 4 and the mass of radon-222 be 222. The conservation of momentum can be expressed as:
[IMAGE:0]
[IMAGE:1]
Simplifying this equation results in:
[IMAGE:2]
Therefore, the kinetic energy of the alpha particle is:
[IMAGE:3]

\hrule
\vspace{1em}


\noindent
\textbf{Q874.} A stationary plutonium-242 (Pu-242) nucleus decays by alpha emission to form a uranium-238 (U-238) nucleus and an alpha particle. The total kinetic energy produced by the decay is 2E.
What is the kinetic energy of the alpha particle?
[IMAGE:0]



\textbf{A.} 4E/242 \\
\textbf{B.} 4E/238 \\
\textbf{C.} E/2 \\
\textbf{D.} 476E/242 \\

\textbf{Answer:} D \\
\textbf{Explanation:} By conservation of momentum, the magnitudes of the momenta of U-238 and the alpha particle are equal. Let the mass of the alpha particle be 4u and that of U-238 be 238u.
From momentum conservation: 4vα=238vU4
vα
​=238
v
U​, leading to the velocity ratio v
α
:v
U
=238:4. Kinetic energy is proportional to mass and the square of velocity. Thus, the ratio of kinetic energies is:
[IMAGE:0]
[IMAGE:1]

\hrule
\vspace{1em}


\noindent
\textbf{Q875.} Two skaters, A (mass 60
kg) and B (mass
5
0
kg), are initially stationary on frictionless ice. Skater A pushes Skater B, causing B to move left at 6
m/
s
. Determine the magnitude and direction of Skater A’s velocity.
[IMAGE:0]



\textbf{A.} Magnitude (m/s):4  ;
Direction:
to the left \\
\textbf{B.} Magnitude (m/s):4  ;
Direction:
to the right \\
\textbf{C.} Magnitude (m/s):5  ;
Direction:
to the right \\
\textbf{D.} Magnitude (m/s):3  ;
Direction:
to the left \\

\textbf{Answer:} C \\
\textbf{Explanation:} By the law of conservation of momentum, the total momentum of the system remains zero before and after the push.
Let Skater A’s velocity be vA. Taking right as positive and left as negativ:
[IMAGE:0]

\hrule
\vspace{1em}


\noindent
\textbf{Q876.} On a smooth air track, there are two gliders, A and B. Glider A has
twice
times the mass of glider B. Glider A is moving to the right at speed v, while glider B is stationary. They undergo a perfectly inelastic collision (stick together after collision). What is the magnitude and direction of their common velocity after the collision?
[IMAGE:0]



\textbf{A.} Magnitude 0, direction not applicable \\
\textbf{B.} Magnitude 1
/4
​
v
, direction to the left \\
\textbf{C.} Magnitude 3
/4
​
v
, direction to the left \\
\textbf{D.} Magnitude 1
/2
​
v
, direction to the left \\

\textbf{Answer:} E \\
\textbf{Explanation:} By the conservation of momentum, the total momentum of the system remains constant in the absence of external forces. Before the collision:
\cdot 
Momentum of glider A:
2
m
×
v
=
2
mv
(to the right)
\cdot 
Momentum of glider B:
m
×0=0
Total momentum of the system:
2
mv
(to the right)
After the collision, the combined mass of the gliders is
2
m
+
m
=
3
m
. Let the common velocity after the collision be
u
. Applying the conservation of momentum:
3
m
×
u
=
2
mv
Solving for
u
:
u
=
2
mv
/3m
​
=
2/3
​
v
, direction to the right.

\hrule
\vspace{1em}


\noindent
\textbf{Q877.} A stationary ice boat of mass 2
5
0
kg is on frictionless ice. Two people are on board: Person A (mass 50
kg) and Person B (mass
6
0
kg). Person A jumps off the boat to the right with a velocity of 5
m/s relative to the boat. Determine the magnitude and direction of the boat and Person B’s common velocity after the jump.
[IMAGE:0]



\textbf{A.} Magnitude (m/s):1  ;
Direction:
to the left \\
\textbf{B.} Magnitude (m/s):2 ;
Direction:
to the left \\
\textbf{C.} Magnitude (m/s):0.5;
Direction:
to the left \\
\textbf{D.} Magnitude (m/s):5:
Direction:
to the right \\

\textbf{Answer:} A \\
\textbf{Explanation:} By conservation of momentum, the total momentum before and after the jump remains zero.
Let the velocity of the boat and Person B be v (left as negative, right as positive).
Total mass of boat + Person B: 2
5
0+50=
30
0
kg.
Momentum equation:
6
0×5+
30
0×(
−
v
)=0
\implies 
30
0
−
30
0
v
=0
\implies 
v
=1m/s

\hrule
\vspace{1em}


\noindent
\textbf{Q878.} In a sealed container, the temperature of a gas remains constant. When the volume is 4 m3, the pressure is 6000 Pa. If the volume expands to 4 m3, what is the new pressure?



\textbf{A.} 4000 Pa \\
\textbf{B.} 3000 Pa \\
\textbf{C.} 2000 Pa \\
\textbf{D.} 1000 Pa \\

\textbf{Answer:} B \\
\textbf{Explanation:} By Boyle’s law, at constant temperature, the pressure PP of a gas is inversely proportional to its volume
[IMAGE:0]
[IMAGE:1]

\hrule
\vspace{1em}


\noindent
\textbf{Q879.} A circuit contains a variable resistor whose resistance R can be adjusted. When the resistance is 10Ω, the current flowing through the circuit is 2A. Assuming the voltage of the power supply remains constant, what is the current when the resistance is increased to 40Ω?



\textbf{A.} 0.5A \\
\textbf{B.} 1.0A \\
\textbf{C.} 1.5A \\
\textbf{D.} 2.0A \\

\textbf{Answer:} A \\
\textbf{Explanation:} According to Ohm's Law, the current I is inversely proportional to the resistance R:
[IMAGE:0]
where V is the constant voltage of the power supply. Given R=10Ω and I=2A, the voltage is:
[IMAGE:1]
When the resistance is increased to 40Ω, the current becomes:
[IMAGE:2]

\hrule
\vspace{1em}


\noindent
\textbf{Q880.} A cart A slides down a smooth incline and collides with a stationary cart B at the bottom.
The quality of car A is 0.5 times that of car B.
After the collision, the two carts stick together and continue moving down the incline. The speed of cart A before the collision is v. What is the magnitude and direction of their common velocity after the collision?
[IMAGE:0]



\textbf{A.} Magnitude 0, direction not applicable \\
\textbf{B.} [IMAGE:0] \\
\textbf{C.} [IMAGE:1] \\
\textbf{D.} [IMAGE:2] \\

\textbf{Answer:} E \\
\textbf{Explanation:} By the conservation of momentum, the total momentum of the system remains constant in the absence of external forces. Before the collision:
\cdot 
Momentum of cart A:
0.5
m
×
v
=
0.5
mv
(down the incline)
\cdot 
Momentum of cart B: 0 (since it is stationary)
Total momentum of the system:
0.5
mv
(down the incline)
After the collision, the combined mass of the carts is
0.5
m
+
m
=
1.5
m
. Let the common velocity after the collision be
u
. Applying the conservation of momentum:
1.5
m
×
u
=
0.5
mv
Solving for
u
:
u
=
1/3
​
v
, direction down the incline.

\hrule
\vspace{1em}


\noindent
\textbf{Q881.} A cylinder contains a fixed amount of ideal gas. When the piston is at the midpoint, the volume of the gas is 500cm
3
and the pressure is 1.5atm. If the piston is slowly compressed, reducing the gas volume to 100cm
3,
what is the new pressure of the gas, assuming the temperature remains constant?



\textbf{A.} 1.0atm \\
\textbf{B.} 7.5atm \\
\textbf{C.} 2.0atm \\
\textbf{D.} 2.5atm \\

\textbf{Answer:} B \\
\textbf{Explanation:} According to Boyle's Law, at constant temperature, the pressure of an ideal gas is inversely proportional to its volume:
[IMAGE:0]
[IMAGE:1]
[IMAGE:2]
[IMAGE:3]

\hrule
\vspace{1em}


\noindent
\textbf{Q882.} In a hydraulic system, the force F remains constant. Piston A has an area of 0.2 m
2
and generates a pressure of 5000 Pa. If replaced by Piston B with an area of 1 m
2,
what is the new pressure?



\textbf{A.} 2000 Pa \\
\textbf{B.} 2500 Pa \\
\textbf{C.} 4000 Pa \\
\textbf{D.} 5000 Pa \\

\textbf{Answer:} E \\
\textbf{Explanation:} Inverse Relationship
:
Pressure PP is inversely proportional to the area A
[IMAGE:0]
Substituting P1=5000 Pa,
A
1​=0.2m
2
, A2=1 m
2
:
[IMAGE:1]

\hrule
\vspace{1em}


\noindent
\textbf{Q883.} The gravitational force F between the Earth and the Moon is inversely proportional to the square of the distance r between them. When the Earth and Moon are 3.84×10
5
km apart, the force is 2×10
20
N. What is the distance between them when the gravitational force decreases to 20×10
19
N?



\textbf{A.} 1.92×10
5
km \\
\textbf{B.} 3.84×10
5
km \\
\textbf{C.} 5.76×10
5
km \\
\textbf{D.} 7.68×10
5
km \\

\textbf{Answer:} B \\
\textbf{Explanation:} By Newton's Law of Gravitation, the gravitational force is inversely proportional to the square of the distance:
[IMAGE:0]
where k is a constant.
Substituting
k
:
[IMAGE:1]
Simplifying:
[IMAGE:2]

\hrule
\vspace{1em}


\noindent
\textbf{Q884.} A skater of mass
72
kg is initially stationary on frictionless ice. He throws a ball of mass
6
kg horizontally to the right with a velocity of 12
m/s. Determine the magnitude and direction of the skater’s velocity after throwing the ball.
[IMAGE:0]



\textbf{A.} Magnitude (m/s):1  ;
Direction:
to the left \\
\textbf{B.} Magnitude (m/s):1  ;
Direction:
to the right \\
\textbf{C.} Magnitude (m/s):12 ;
Direction:
to the left \\
\textbf{D.} Magnitude (m/s):5 ;
Direction:
to the rihgt \\

\textbf{Answer:} A \\
\textbf{Explanation:} By the law of conservation of momentum, the total momentum of the system remains zero before and after the throw.
Let the skater’s velocity be v
v
. Taking right as positive and left as negative:
[IMAGE:0]

\hrule
\vspace{1em}


\noindent
\textbf{Q885.} A car starts from rest and accelerates uniformly at a=2 m/s
2
for 5 seconds. Then it moves at a constant speed until the total time reaches 15 seconds. If the total distance traveled is 175 meters, what is the constant speed during the uniform motion phase?



\textbf{A.} 5 m/s \\
\textbf{B.} 8 m/s \\
\textbf{C.} 10 m/s \\
\textbf{D.} 12 m/s \\

\textbf{Answer:} E \\
\textbf{Explanation:} Acceleration Phase (0-5 seconds):
Distance s1=0.5
\cdot 
2
\cdot 
52=25 m, final velocity v=2
\cdot 
5=10 m/s.
Uniform Motion Phase (5-15 seconds):
Time t
2
=10 s, distance s
2
=10
\cdot 
v.
Total Distance:
25+10v=175
\implies 
10v=150
\implies 
v=15 m/s.
Thus, the correct answer is C.

\hrule
\vspace{1em}


\noindent
\textbf{Q886.} The electrostatic force F between two point charges is inversely proportional to the square of the distance r between them. When the charges are 3m apart, the force is 12N. What is the distance between the charges when the force becomes 108N?



\textbf{A.} 0.5m \\
\textbf{B.} 1m \\
\textbf{C.} 2m \\
\textbf{D.} 2.5m \\

\textbf{Answer:} B \\
\textbf{Explanation:} By Coulomb's Law, the electrostatic force is inversely proportional to the square of the distance: F=k/$r^2$  where k is a constant. When F=12N and r=3m: k=F\cdot r 2=12×32 =108 For F=108N: 108=108/$r^2$ \implies $r^2$=108/108=1\implies r=\sqrt{}1=1m

\hrule
\vspace{1em}


\noindent
\textbf{Q887.} The velocity v of an object is inversely proportional to the square root of time t. When v=12m/s, t=4s. What is the value of t when v=24m/s?



\textbf{A.} 16/9s \\
\textbf{B.} 4s \\
\textbf{C.} 9/16s \\
\textbf{D.} 27/14s \\

\textbf{Answer:} F \\
\textbf{Explanation:} Explanation: Velocity v is inversely proportional to the square root of time t, which can be expressed as
[IMAGE:0]
where k is a constant. When v=12m/s and t=4s, substituting into the equation gives:
[IMAGE:1]
When v=24m/s, substituting into the equation gives:
[IMAGE:2]

\hrule
\vspace{1em}


\noindent
\textbf{Q888.} The quantities a and b are positive. aa is inversely proportional to the square root of b. When a=4, b=16. What is the value of b when a=16?



\textbf{A.} 1 \\
\textbf{B.} 4 \\
\textbf{C.} 8 \\
\textbf{D.} 16 \\

\textbf{Answer:} A \\
\textbf{Explanation:} Since
$𝑎$
is inversely proportional to the square root of b, the relationship is:
[IMAGE:0]
[IMAGE:1]
[IMAGE:2]

\hrule
\vspace{1em}


\noindent
\textbf{Q889.} The quantities
x
and y
are positive. x is inversely proportional to the square of
y
. When x=3, y=4. What is the value of
y
when x=
3
?



\textbf{A.} 1 \\
\textbf{B.} 2 \\
\textbf{C.} 8 \\
\textbf{D.} 16 \\

\textbf{Answer:} F \\
\textbf{Explanation:} Explanation: Since x is inversely proportional to the square of y, the relationship is:
[IMAGE:0]
[IMAGE:1]
[IMAGE:2]

\hrule
\vspace{1em}


\noindent
\textbf{Q890.} A 2 kg object is initially at rest on a smooth horizontal surface. Starting at t=0, the object is subjected to a horizontal force that varies with time as follows:
From 0 to 0.05 s, the force increases linearly from 0 to 10 N;
From 0.05 to 0.10 s, the force remains constant at 10 N;
From 0.10 to 0.15 s, the force decreases linearly back to 0.
What is the kinetic energy of the object at t=0.0.075 s?



\textbf{A.} 0J \\
\textbf{B.} 0.05J \\
\textbf{C.} 1.25J \\
\textbf{D.} 0.625J \\

\textbf{Answer:} E \\
\textbf{Explanation:} The kinetic energy is equal to the work done by the force, which is the integral of the force over time.
From 0 to 0.05 s: Force increases linearly from 0 to 10 N (triangle).
From 0.05 to 0.10 s: Force remains constant at 10 N (rectangle).
From 0.10 to 0.15 s: Force decreases linearly from 10 N to 0 (triangle).

\hrule
\vspace{1em}


\noindent
\textbf{Q891.} An object of mass 2 kg is at rest at time t=0. A resultant force acts on the object in a constant direction. The magnitude of the resultant force acting on the object varies with time as follows:
The force starts at 0 N, increases to 10 N at t=0.1 s and remains constant until t=0.2 s, then increases linearly to 20 N, and finally decreases linearly to 0 N at t=0.3 s.
What is the kinetic energy of the object at time t=0.1 s?



\textbf{A.} 0J \\
\textbf{B.} 2.34J \\
\textbf{C.} 4.02J \\
\textbf{D.} 3.06J \\

\textbf{Answer:} F \\
\textbf{Explanation:} The kinetic energy equals the work done by the force; the integration of force over time equals the change in momentum. The graph shows that the force increases from 0 N to 10 N at t=0.1 s and remains constant until t=0.2 s, then increases linearly to 20 N, and finally decreases linearly to 0 N at t=0.3 s.

\hrule
\vspace{1em}


\noindent
\textbf{Q892.} A stationary grenade of mass 5
kg on a smooth horizontal surface explodes into two fragments. One fragment (mass 3
kg) moves to the left with 4
m/s. Determine the magnitude and direction of the velocity of the other fragment.
[IMAGE:0]



\textbf{A.} Magnitude (m/s):16  ;
Direction:
to the right \\
\textbf{B.} Magnitude (m/s):14  ;
Direction:
to the left \\
\textbf{C.} Magnitude (m/s):31;
Direction:
to the right \\
\textbf{D.} Magnitude (m/s):22;
Direction:
to the left \\

\textbf{Answer:} A \\
\textbf{Explanation:} By the law of conservation of momentum, the total momentum before the explosion is zero, and it remains zero after the explosion.
Let the velocity of the other fragment
[IMAGE:0]
Taking left as negative and right as positive:
[IMAGE:1]

\hrule
\vspace{1em}


\noindent
\textbf{Q893.} An object of mass 50 kg is at rest at time t=0. A resultant force acts on the object in a constant direction. The magnitude of the resultant force acting on the object varies with time as follows:
The force starts at 0 N, increases to 100 N at t=0.1 s, and then decreases linearly to 0 N at t=0.2 s.
What is the kinetic energy of the object at time t=0.1 s?



\textbf{A.} 0J \\
\textbf{B.} 5J \\
\textbf{C.} 0.25J \\
\textbf{D.} 2J \\

\textbf{Answer:} C \\
\textbf{Explanation:} The kinetic energy equals the work done by the force; the integration of force over time equals the change in momentum. The graph shows that the force increases from 0 N to 100 N at t=0.1 s.
[IMAGE:0]
[IMAGE:1]

\hrule
\vspace{1em}


\noindent
\textbf{Q894.} A rocket of mass 100 kg is launched with a varying thrust. The thrust varies with time as follows:
The thrust starts at 0 N, increases to 200 N at t=0.1 s, and then decreases linearly to 0 N at t=0.2 s.
Ignoring air resistance and other forces, what is the kinetic energy of the rocket at t=0.1 s?



\textbf{A.} 0J \\
\textbf{B.} 1J \\
\textbf{C.} 2J \\
\textbf{D.} 3J \\

\textbf{Answer:} E \\
\textbf{Explanation:} The kinetic energy equals the work done by the force; the integration of force over time equals the change in momentum. The graph shows that the thrust increases from 0 N to 200 N at t=0.1 s and then decreases linearly to 0 N at t=0.2 s.
[IMAGE:0]
[IMAGE:1]
[IMAGE:2]

\hrule
\vspace{1em}


\noindent
\textbf{Q895.} A ball A is moving to the right at speed
2
v and collides with a stationary ball B. The mass of ball A is half that of ball B. After the collision, the two balls stick together and continue moving. What is the magnitude and direction of their common velocity after the collision?
[IMAGE:0]



\textbf{A.} Magnitude 0, direction not applicable \\
\textbf{B.} Magnitude 1
/3
​
v
, direction to the left \\
\textbf{C.} Magnitude 2
/3
​
v
, direction to the left \\
\textbf{D.} Magnitude 4
/3
​
v
, direction to the left \\

\textbf{Answer:} F \\
\textbf{Explanation:} By the conservation of momentum, the total momentum of the system remains constant in the absence of external forces. Before the collision:
\cdot 
Momentum of ball A:
[IMAGE:0]
\cdot 
Momentum of ball B: 0 (since it is stationary)
T
otal momentum of the system: 1
/2
​
mv (to the right)
After the collision, the combined mass of the balls is
[IMAGE:1]
Let the common velocity after the collision be u. Applying the conservation of momentum:
[IMAGE:2]
Solving for u: u=
2
v
/3
​
, direction to the right.

\hrule
\vspace{1em}


\noindent
\textbf{Q896.} A 3 kg object is initially at rest on a smooth horizontal surface. Starting at t=0, the object is subjected to a horizontal force that varies with time as follows:
From 0 to 0.06 s, the force increases linearly from 0 to 6 N;
From 0.06 to 0.12 s, the force remains constant at 6 N;
From 0.12 to 0.18 s, the force decreases linearly back to 0.
What is the kinetic energy of the object at t=0.09 s?



\textbf{A.} 0.36 J \\
\textbf{B.} 0.72J \\
\textbf{C.} 1.44J \\
\textbf{D.} 2.16J \\

\textbf{Answer:} A \\
\textbf{Explanation:} The kinetic energy is equal to the work done by the force, which is the integral of the force over time (the area under the force-time graph).

\hrule
\vspace{1em}


\noindent
\textbf{Q897.} A ball A with a mass of 2 kg moves at 8 m/s and collides elastically with a stationary ball B of mass 3 kg. After the collision, ball A's velocity becomes -1 m/s (opposite direction). What is the velocity of ball B after the collision in meters per second?



\textbf{A.} 1.0 m/s \\
\textbf{B.} 2.0 m/s \\
\textbf{C.} 3.0 m/s \\
\textbf{D.} 6.0 m/s \\

\textbf{Answer:} D \\
\textbf{Explanation:} In an elastic collision, both momentum and kinetic energy are conserved. Using the conservation of momentum:
[IMAGE:0]
Since kinetic energy is conserved, the answer is correct. The velocity of ball B after the collision is 4 m/s, corresponding to option D.

\hrule
\vspace{1em}


\noindent
\textbf{Q898.} An object of mass 8 kg is at rest at time = 0 s. A resultant force acts on the object in a constant direction.The magnitude of the resultant force acting on the object varies with time as shown by the graph.
What is the kinetic energy of the object at time = 4 s?



\textbf{A.} 4J \\
\textbf{B.} 11J \\
\textbf{C.} 16J \\
\textbf{D.} 12J \\

\textbf{Answer:} A \\
\textbf{Explanation:} The kinetic energy equals the work done by the force; the integration of force on time equals the change in momentum; mv; therefore the terminal K.E. can be found:1/2 mv2.

\hrule
\vspace{1em}


\noindent
\textbf{Q899.} An object of mass 3 kg is at rest at time = 0 s. A resultant force acts on the object in a constant direction.The magnitude of the resultant force acting on the object varies with time as shown by the graph.
What is the kinetic energy of the object at time = 4 s?



\textbf{A.} 96J \\
\textbf{B.} 81J \\
\textbf{C.} 16J \\
\textbf{D.} 125J \\

\textbf{Answer:} E \\
\textbf{Explanation:} The kinetic energy equals the work done by the force; the integration of force on time equals the change in momentum; mv; therefore the terminal K.E. can be found:1/2 mv2.

\hrule
\vspace{1em}


\noindent
\textbf{Q900.} Two small balls A and B, each of mass m, are initially at rest on a smooth horizontal surface. Ball A moves to the right with a velocity of
6
m/s, while ball B remains stationary. They undergo a perfectly inelastic collision and stick together. Determine the magnitude and direction of their common velocity after the collision.
[IMAGE:0]



\textbf{A.} Magnitude (m/s):2  ;
Direction:
to the right \\
\textbf{B.} Magnitude (m/s):1  ;
Direction:
to the left \\
\textbf{C.} Magnitude (m/s):2  ;Direction:
to the left \\
\textbf{D.} Magnitude (m/s):3  ;Direction:
to the right \\

\textbf{Answer:} D \\
\textbf{Explanation:} By the law of conservation of momentum, the total momentum before the collision equals the total momentum after the collision.
Initial momentum:
[IMAGE:0]
After the collision, the combined mass is 2m. Let the common velocity be v
[IMAGE:1]

\hrule
\vspace{1em}


\noindent
\textbf{Q901.} An object of mass 1 kg is at rest at time = 0 s. A resultant force acts on the object in a constant direction.
The magnitude of the resultant force acting on the object varies with time as shown by
the graph.
[IMAGE:0]
What is the kinetic energy of the object at time = 0.10 s?



\textbf{A.} 1.2J \\
\textbf{B.} 0.81J \\
\textbf{C.} 3.6J \\
\textbf{D.} 1.25J \\

\textbf{Answer:} G \\
\textbf{Explanation:} The kinetic energy equals the work done by the force; the integration of force on time equals the change in momentum; mv; therefore the terminal K.E. can be found:1/2 mv2.

\hrule
\vspace{1em}


\noindent
\textbf{Q902.} 1.
Two identical carts, A and B, are moving towards each other on a smooth horizontal track. Cart A has three times the mass of cart B. Cart A is moving to the right at speed v, and cart B is moving to the left at speed
3
v. They collide and stick together. What is the magnitude and direction of their common velocity after the coll
ision?
[IMAGE:0]



\textbf{A.} Magnitude 0, direction not applicable \\
\textbf{B.} Magnitude 1
​
/4
v
, direction to the left \\
\textbf{C.} Magnitude 1
/2
​
v
, direction to the left \\
\textbf{D.} Magnitude 3
/4
​
v
, direction to the left \\

\textbf{Answer:} A \\
\textbf{Explanation:} By the conservation of momentum, the total momentum of the system remains constant in the absence of external forces. Before the collision:
\cdot 
Momentum of cart A: 3m×v=3mv (to the right)
\cdot 
Momentum of cart B: m×(
−
3
v)=
−
3
mv (to the left)
Total momentum of the system: 3mv
−
3
mv=
0
(to the right)
After the collision, the combined mass of the carts is 3m+m=4m. Let the common velocity after the collision be u. Applying the conservation of momentum:
4m×u=
0

\hrule
\vspace{1em}


\noindent
\textbf{Q903.} An object of mass 2 kg is at rest at time = 0 s. A resultant force acts on the object in a constant direction.
The magnitude of the resultant force acting on the object varies with time as shown by
the graph.
What is the kinetic energy of the object at time = 0.10 s?



\textbf{A.} 0J \\
\textbf{B.} 0.313J \\
\textbf{C.} 1J \\
\textbf{D.} 1.25J \\

\textbf{Answer:} B \\
\textbf{Explanation:} The kinetic energy equals the work done by the force; the integration of force on time equals the change in momentum; mv; therefore the terminal K.E. can be found:1/2 mv2.

\hrule
\vspace{1em}


\noindent
\textbf{Q904.} Calculate the shaded area enclosed by the two functions in the following figure.
[IMAGE:0]



\textbf{A.} 32/3 \\
\textbf{B.} 8/3 \\
\textbf{C.} 7/6 \\
\textbf{D.} 1/3 \\

\textbf{Answer:} A \\
\textbf{Explanation:} [IMAGE:0]

\hrule
\vspace{1em}


\noindent
\textbf{Q905.} 9.
Find the area of the plane region bounded by the continuous curve
y
=ln
x
, the x-axis, and the vertical lines
x
=21
​
and
x
=2 (as shown in Figu
10.
re).
[IMAGE:0]



\textbf{A.} 2.789 \\
\textbf{B.} 17/3 \\
\textbf{C.} [IMAGE:0] \\
\textbf{D.} [IMAGE:1] \\

\textbf{Answer:} F \\
\textbf{Explanation:} [IMAGE:0]

\hrule
\vspace{1em}


\noindent
\textbf{Q906.} Calculate the shaded area enclosed by the two functions in the following figure.
[IMAGE:0]



\textbf{A.} 1/3 \\
\textbf{B.} 8/3 \\
\textbf{C.} 7/6 \\
\textbf{D.} 1/2 \\

\textbf{Answer:} B \\
\textbf{Explanation:} [IMAGE:0]
[IMAGE:1]

\hrule
\vspace{1em}


\noindent
\textbf{Q907.} Use S to denote the area of the shaded region in the figure. What is the value of S?
[IMAGE:0]



\textbf{A.} [IMAGE:0] \\
\textbf{B.} [IMAGE:1] \\
\textbf{C.} [IMAGE:2] \\
\textbf{D.} [IMAGE:3] \\

\textbf{Answer:} D \\
\textbf{Explanation:} Subtract the negative area from the positive area.

\hrule
\vspace{1em}


\noindent
\textbf{Q908.} What is the area of the plane region enclosed by the curve x
​
+y
​
=1 and the lines x=0, y=0?



\textbf{A.} 1/3 \\
\textbf{B.} 4/3 \\
\textbf{C.} 7/6 \\
\textbf{D.} 1/2 \\

\textbf{Answer:} A \\
\textbf{Explanation:} This is a basic problem where the area of the plane region can be determined using a definite integral.
[IMAGE:0]

\hrule
\vspace{1em}


\noindent
\textbf{Q909.} Calculate the shaded area enclosed by the two functions in the following figure.
[IMAGE:0]



\textbf{A.} 12/13 \\
\textbf{B.} -4/3 \\
\textbf{C.} 7/6 \\
\textbf{D.} -1/2 \\

\textbf{Answer:} E \\
\textbf{Explanation:} [IMAGE:0]

\hrule
\vspace{1em}


\noindent
\textbf{Q910.} An object is dropped freely from a platform 45 meters high. After hitting the ground, it rebounds vertically upwards with the same speed magnitude. Ignoring air resistance, what is the maximum height reached by the object after rebounding?
(The gravitational acceleration near Earth's surface is 10 m/s²)



\textbf{A.} 15m \\
\textbf{B.} 30m \\
\textbf{C.} 45m \\
\textbf{D.} 60m \\

\textbf{Answer:} C \\
\textbf{Explanation:} The object is dropped freely from a platform 45 meters high, starting with an initial velocity of 0, and gravitational acceleration g=10m/s2.
During the free fall, the velocity v when the object hits the ground can be calculated using the displacement formula: $v^2$=$u^2$+2gh  $h_max$=$v^2$/2g=$30^2$/(2×10)=900/20=45m

\hrule
\vspace{1em}


\noindent
\textbf{Q911.} A person with a mass of 50 kg stands on a scale inside an elevator. When the elevator accelerates upward at 2 m/s², what is the scale reading in newtons? When the elevator accelerates downward at 3 m/s², what is the scale reading in newtons? (Gravitational acceleration g=10m/s2)



\textbf{A.} 600 N and 350 N \\
\textbf{B.} 500 N and 500 N \\
\textbf{C.} 400 N and 600 N \\
\textbf{D.} 350 N and 600 N \\

\textbf{Answer:} A \\
\textbf{Explanation:} When the elevator accelerates upward, the person experiences an apparent weight greater than their actual weight. Using Newton's second law:
N−mg=ma
where N is the scale reading, m is the person's mass (50 kg), g is gravitational acceleration (10 m/s²), and a is the elevator's acceleration (2 m/s²).
Substituting the values:
N=m(g+a)=50×(10+2)=50×12=600N
When the elevator accelerates downward, the person experiences an apparent weight less than their actual weight. Using Newton's second law:
mg−N=ma
where a is the elevator's acceleration (3 m/s²).
Substituting the values:
N=m(g−a)=50×(10−3)=50×7=350N

\hrule
\vspace{1em}


\noindent
\textbf{Q912.} A 30 N object is placed on a smooth inclined plane with an angle of 30°, connected via a light inextensible string over a pulley to a 20 N hanging object. The system starts from rest and moves until the hanging object descends 1 m. Ignoring pulley friction and air resistance, by how much has the gravitational potential energy of the hanging object increased?
[IMAGE:0]



\textbf{A.} 40 J \\
\textbf{B.} 15 J \\
\textbf{C.} 20 J \\
\textbf{D.} 25 J \\

\textbf{Answer:} C \\
\textbf{Explanation:} The system consists of an object on an inclined plane and a hanging object connected by a string over a pulley. As the system moves, the hanging object rises while the object on the plane slides down. By analyzing the energy changes, the increase in gravitational potential energy of the hanging object can be determined.
[IMAGE:0]

\hrule
\vspace{1em}


\noindent
\textbf{Q913.} Calculate the shaded area enclosed by the two functions in the following figure.
[IMAGE:0]



\textbf{A.} 12/13 \\
\textbf{B.} -2/3 \\
\textbf{C.} 1/6 \\
\textbf{D.} -1/2 \\

\textbf{Answer:} C \\
\textbf{Explanation:} [IMAGE:0]

\hrule
\vspace{1em}


\noindent
\textbf{Q914.} A ball starts from rest at point A on an inclined plane with an angle of 45 degrees. Ignoring friction, what is the velocity of the ball in meters per second when it reaches the bottom of the incline at point B? (Gravitational acceleration g=10m/s
2, length of the incline = 10 meters)



\textbf{A.} 10 m/s \\
\textbf{B.} 14.14 m/s \\
\textbf{C.} 20 m/s \\
\textbf{D.} 22.36 m/s \\

\textbf{Answer:} B \\
\textbf{Explanation:} The acceleration along the incline is provided by the component of gravity:
[IMAGE:0]
Using the kinematic equation:
[IMAGE:1]
Substituting the values:
[IMAGE:2]

\hrule
\vspace{1em}


\noindent
\textbf{Q915.} As shown in the figure, if one wants to calculate the directed area of the shaded region
[IMAGE:0]



\textbf{A.} 2 \\
\textbf{B.} -2 \\
\textbf{C.} 1/5 \\
\textbf{D.} -1/2 \\

\textbf{Answer:} F \\
\textbf{Explanation:} [IMAGE:0]

\hrule
\vspace{1em}


\noindent
\textbf{Q916.} An object of weight 100 N hangs from the end of a light inextensible string of length 1 m, which is attached to the ceiling. A constant horizontal wind force of 80 N blows on the object, causing it to move to a new equilibrium position. By how much has the gravitational potential energy of the object increased as a result of its change of position?



\textbf{A.} 21.5J \\
\textbf{B.} 2.8J \\
\textbf{C.} 3.50J \\
\textbf{D.} 4.2J \\

\textbf{Answer:} A \\
\textbf{Explanation:} The object moves to a new equilibrium position under the horizontal wind force, forming an angle θ with the vertical. By analyzing the forces and calculating the height increase, the change in gravitational potential energy can be determined.
Force Analysis
Weight of the object: 100 N downward.
Tension in the string: T along the string.
Horizontal wind force: 80 N.
Equilibrium Conditions
Horizontal: T*sin
θ
= 80 N
Vertical: T*cos
θ
= 100 N
Angle Calculation
From the equations:
tanθ = 80 / 100 = 0.8 \to  θ ≈ 38.66°
Height Increase
The height increase h = L(1 - cosθ) ≈ 1(1 - cos38.66°) ≈ 1 × 0.215 ≈ 0.215 m
Potential Energy Change
ΔU = mgh = 100 × 0.215 ≈ 21.5J

\hrule
\vspace{1em}


\noindent
\textbf{Q917.} A rocket is launched vertically from the ground with an initial acceleration of 10m/s2for 2 seconds, after which the engine cuts off. Ignoring air resistance, what is the maximum height reached by the rocket in meters? (Gravitational acceleration g=10m/s2)



\textbf{A.} 20m \\
\textbf{B.} 30m \\
\textbf{C.} 40m \\
\textbf{D.} 60m \\

\textbf{Answer:} C \\
\textbf{Explanation:} The rocket's motion is divided into two phases:
1.Acceleration Phase (Engine On):
[IMAGE:0]
2. Free Fall Phase (Engine Off):
[IMAGE:1]

\hrule
\vspace{1em}


\noindent
\textbf{Q918.} An object weighing 100 N hangs from a light inextensible string of length 0.8 m attached to the ceiling. A constant force of 70 N is applied to the object at an angle of 37° above the horizontal, moving it to a new equilibrium position. By how much has the gravitational potential energy of the object increased?



\textbf{A.} 2.8 J \\
\textbf{B.} 3.5 J \\
\textbf{C.} 4.2 J \\
\textbf{D.} 5.6 J \\

\textbf{Answer:} F \\
\textbf{Explanation:} The object is in equilibrium under three forces: gravity (50 N downward), the applied force (35 N at 37°), and tension TT. Resolving forces:
[IMAGE:0]

\hrule
\vspace{1em}


\noindent
\textbf{Q919.} What is the area enclosed by the line x=
17
and the curve
[IMAGE:0]
?



\textbf{A.} 4 \\
\textbf{B.} 8 \\
\textbf{C.} 15 \\
\textbf{D.} 36 \\

\textbf{Answer:} C \\
\textbf{Explanation:} [IMAGE:0]
[IMAGE:1]

\hrule
\vspace{1em}


\noindent
\textbf{Q920.} A student on Earth throws a ball at an angle of 30 degrees above the horizontal with an initial velocity of 30 m/s. Ignoring air resistance, what is the maximum height reached by the ball? (The gravitational acceleration near Earth's surface is 10 m/s²)



\textbf{A.} 11.25m \\
\textbf{B.} 30.75m \\
\textbf{C.} 45m \\
\textbf{D.} 60.5m \\

\textbf{Answer:} A \\
\textbf{Explanation:} The ball is thrown at an angle of 30 degrees, with an initial velocity u=30m/s, and gravitational acceleration g=10m/s2.
The vertical component of the initial velocity is:
[IMAGE:0]
The formula for maximum height is:
[IMAGE:1]

\hrule
\vspace{1em}


\noindent
\textbf{Q921.} An object weighing 150 N hangs from the end of a light inextensible string of length 0.5 m, which is attached to the ceiling. A constant horizontal force is applied to the object, causing it to move to a new equilibrium position where the string makes an angle of 30° with the vertical. By how much has the gravitational potential energy of the object increased?



\textbf{A.} 3.0 J \\
\textbf{B.} 4.02 J \\
\textbf{C.} 6.01 J \\
\textbf{D.} 7.53 J \\

\textbf{Answer:} F \\
\textbf{Explanation:} Determine the equilibrium position: The object reaches a new equilibrium position where the string makes a 30° angle with the vertical.
Force analysis: The object experiences a weight mg=60N, tension T in the string, and a horizontal force F. The equilibrium conditions are:
[IMAGE:0]
and
[IMAGE:1]
[IMAGE:2]

\hrule
\vspace{1em}


\noindent
\textbf{Q922.} A probe on Jupiter's moon Europa launches a small rocket vertically upward. The rocket is fired with an initial velocity of 15 m/s. The gravitational acceleration on Europa's surface is 1.3 m/s². What is the maximum height reached by the rocket?



\textbf{A.} 5.9m \\
\textbf{B.} 10.1m \\
\textbf{C.} 20.2m \\
\textbf{D.} 56.3m \\

\textbf{Answer:} E \\
\textbf{Explanation:} The maximum height of a vertically launched projectile is calculated using:
[IMAGE:0]

\hrule
\vspace{1em}


\noindent
\textbf{Q923.} What is the area enclosed by the line x=
17
and the curve
[IMAGE:0]
?
[IMAGE:1]



\textbf{A.} 2 \\
\textbf{B.} 9 \\
\textbf{C.} 12 \\
\textbf{D.} 36 \\

\textbf{Answer:} D \\
\textbf{Explanation:} [IMAGE:0]

\hrule
\vspace{1em}


\noindent
\textbf{Q924.} A student inside a high-rise elevator throws a 50 g ball vertically upward with an initial velocity of 10 m/s. The elevator is moving upward with a constant acceleration of 2 m/s². What is the maximum height of the ball relative to the elevator?



\textbf{A.} 1.0m \\
\textbf{B.} 6.41m \\
\textbf{C.} 5.0m \\
\textbf{D.} 10.0m \\

\textbf{Answer:} B \\
\textbf{Explanation:} When a ball is thrown vertically upward inside an elevator accelerating upward at 2 m/s², the effective gravitational acceleration experienced by the ball is:
[IMAGE:0]
The maximum height is calculated using:
[IMAGE:1]
=6.41m

\hrule
\vspace{1em}


\noindent
\textbf{Q925.} A
n object weighing 120 N hangs from a light inextensible string of length 0.35 m attached to the ceiling. A constant horizontal force of 90 N is applied to the object, moving it to a new equilibrium position where the string is no longer vertical. By how much has the gravitational potential energy of the object increased?



\textbf{A.} 8.4 J \\
\textbf{B.} 2.8 J \\
\textbf{C.} 3.5 J \\
\textbf{D.} 4.2 J \\

\textbf{Answer:} A \\
\textbf{Explanation:} The object is in equilibrium under three forces: gravity (40 N downward), horizontal force (45 N rightward), and tension TT along the string. Resolving forces:
[IMAGE:0]

\hrule
\vspace{1em}


\noindent
\textbf{Q926.} A light spring has a natural length of 0.60 m and a spring constant of 100 N/m. It is stretched by a force starting from zero and increasing at a constant rate of 0.5 N/s until reaching its maximum value. When the strain energy of the spring is 0.36 J, what is the average power used to stretch the spring?



\textbf{A.} 0.015 W \\
\textbf{B.} 0.020 W \\
\textbf{C.} 0.025 W \\
\textbf{D.} 0.030 W \\

\textbf{Answer:} B \\
\textbf{Explanation:} Strain energy formula:
[IMAGE:0]
Substituting U=0.36
J and k=100
N/m
k
=100N/m, solve for the extension:
[IMAGE:1]
[IMAGE:2]
Time calculation: Time to reach maximum force:
[IMAGE:3]
Average power: Power is energy divided by time:
[IMAGE:4]

\hrule
\vspace{1em}


\noindent
\textbf{Q927.} An object is thrown vertically upward on the Moon with an initial velocity of 20 m/s. The velocity-time (v-t) graph shows that the object's velocity decreases uniformly until it reaches 0. The gravitational acceleration on the Moon is 2 m/s². Based on the following description, answer the question:
[IMAGE:0]
Question: What is the distance traveled by the object from the beginning until it slows down to zero?



\textbf{A.} 10.0m \\
\textbf{B.} 250.0m \\
\textbf{C.} 50.0m \\
\textbf{D.} 100.0m \\

\textbf{Answer:} D \\
\textbf{Explanation:} The initial velocity u=30m/s, gravitational acceleration g=2m/s2.
The distance is:
[IMAGE:0]

\hrule
\vspace{1em}


\noindent
\textbf{Q928.} A rhombus has diagonals of lengths 3x
cm3
x
cm and 4x
cm4
x
cm. The area of the rhombus is 96
cm296cm2. What is the length of the longer diagonal?
[IMAGE:0]



\textbf{A.} 12 cm \\
\textbf{B.} 14 cm \\
\textbf{C.} 16 cm \\
\textbf{D.} 18 cm \\

\textbf{Answer:} C \\
\textbf{Explanation:} [IMAGE:0]

\hrule
\vspace{1em}


\noindent
\textbf{Q929.} An object weighing 60 N hangs from the end of a light inextensible string of length 0.45 m, which is attached to the ceiling. A constant horizontal force of 36 N is applied to the object, causing it to move to a new equilibrium position where the string is no longer vertical. By how much has the gravitational potential energy of the object increased?



\textbf{A.} 3.0 J \\
\textbf{B.} 4.0 J \\
\textbf{C.} 2.7 J \\
\textbf{D.} 6.0 J \\

\textbf{Answer:} E \\
\textbf{Explanation:} Determine the equilibrium position: The object reaches a new equilibrium position where the string makes an angle θ with the vertical due to the horizontal force.
Force analysis: The horizontal force F=18N and weight mg=30N. The tension T in the string satisfies:
[IMAGE:0]
[IMAGE:1]

\hrule
\vspace{1em}


\noindent
\textbf{Q930.} A light spring oscillator has a spring constant of 20 N/m. The oscillator is pulled from its equilibrium position by a tension force that starts at zero and increases at a constant rate of 0.60 N/s until it reaches its maximum value. When the force reaches its maximum value, the elastic potential energy of the oscillator is 0.45 J. What is the average power of the work done by the force?



\textbf{A.} 0.18 W \\
\textbf{B.} 0.30 W \\
\textbf{C.} 0.45 W \\
\textbf{D.} 0.60 W \\

\textbf{Answer:} A \\
\textbf{Explanation:} [IMAGE:0]

\hrule
\vspace{1em}


\noindent
\textbf{Q931.} A cylindrical container has a base radius of (x + 1) centimeters and a height of (x - 2) centimeters. There is a part of liquid in the container, whose volume accounts for 1/3 of the volume of the cylinder. When a cone with the same base area as the cylindrical container is placed with its base on the bottom of the container and its vertex touching the water surface, the vertex of the cone just touches the water surface; the height of the cone is 2. What is the total capacity of this container?
[IMAGE:0]



\textbf{A.} 120π cm³ \\
\textbf{B.} 128π cm³ \\
\textbf{C.} 240π cm³ \\
\textbf{D.} 360π cm³ \\

\textbf{Answer:} E \\
\textbf{Explanation:} 2(x-2)/3=2 x=5

\hrule
\vspace{1em}


\noindent
\textbf{Q932.} A light rubber band has an unstretched length of 0.20 m and a spring constant of 30 N/m. The rubber band is stretched by a tension force that starts at zero and increases at a constant rate of 0.40 N/s until it reaches its maximum value. When the force reaches its maximum value, the elastic potential energy of the rubber band is 0.18 J. What is the average power used to stretch the rubber band?



\textbf{A.} 0.020 W \\
\textbf{B.} 0.040 W \\
\textbf{C.} 0.060 W \\
\textbf{D.} 0.080 W \\

\textbf{Answer:} B \\
\textbf{Explanation:} [IMAGE:0]

\hrule
\vspace{1em}


\noindent
\textbf{Q933.} A probe on Venus launches a projectile vertically upward. The projectile is fired with an initial velocity of 20 m/s. What is the maximum height reached by the projectile? (The gravitational field strength on Venus' surface = 8.9 m/s²)



\textbf{A.} 10m \\
\textbf{B.} 22.47m \\
\textbf{C.} 40m \\
\textbf{D.} 89m \\

\textbf{Answer:} B \\
\textbf{Explanation:} The maximum height of a vertically launched projectile can be calculated using the formula:
[IMAGE:0]
Substitute the values:
[IMAGE:1]

\hrule
\vspace{1em}


\noindent
\textbf{Q934.} When a Ra-226 nucleus (with velocity u) undergoes α-decay, it forms a Rn-222 nucleus and an α-particle. In the laboratory reference frame, the total kinetic energy of the Ra-226 nucleus after decay is 10E. The ratio of the kinetic energy of the Rn-222 nucleus to that of the α-particle is 1:4. What is the kinetic energy of the α-particle?



\textbf{A.} E​/5 \\
\textbf{B.} 2E​/5 \\
\textbf{C.} E​/4 \\
\textbf{D.} 4E \\

\textbf{Answer:} E \\
\textbf{Explanation:} Ea=4/(4+1)*10E=8E

\hrule
\vspace{1em}


\noindent
\textbf{Q935.} A light balloon is filled with gas, with an initial volume of 0.05 m³. The gas pressure relates to volume as
P
=
kV
, where
k
=200Pa/m3. The gas pressure starts at zero and increases at a constant rate of 0.50 Pa/s until it reaches its maximum value. When the pressure reaches its maximum value, the elastic potential energy stored in the balloon is 10.0 J. What is the average power of the work done by the gas during inflation?



\textbf{A.} 0.50 W \\
\textbf{B.} 1.00 W \\
\textbf{C.} 2.00 W \\
\textbf{D.} 5.00 W \\

\textbf{Answer:} C \\
\textbf{Explanation:} [IMAGE:0]

\hrule
\vspace{1em}


\noindent
\textbf{Q936.} A right triangle has legs measuring (x+3) cm and (x
−
1) cm. The area of the triangle is
32
cm
²
. What are the lengths of the two legs?
[IMAGE:0]



\textbf{A.} 5cm and 7cm \\
\textbf{B.} 8cm and 4cm \\
\textbf{C.} 9cm and 5cm \\
\textbf{D.} 10cm and 4cm \\

\textbf{Answer:} B \\
\textbf{Explanation:} (x-1)(x+3)=32

\hrule
\vspace{1em}


\noindent
\textbf{Q937.} A student on Earth throws a ball vertically upwards with an initial velocity of 30 m/s. What is the maximum height gained by the ball? (The gravitational field strength close to Earth's surface = 10 N/kg)



\textbf{A.} 15m \\
\textbf{B.} 30m \\
\textbf{C.} 45m \\
\textbf{D.} 90m \\

\textbf{Answer:} C \\
\textbf{Explanation:} The initial velocity u=30m/s, gravitational acceleration g=10m/s2.
The maximum height is:
[IMAGE:0]

\hrule
\vspace{1em}


\noindent
\textbf{Q938.} A stationary actinium-227 (Ac-227) nucleus undergoes alpha decay to form francium-223 (Fr-223) and an alpha particle. After decay, the two particles leave tracks in a cloud chamber. If the track length of the alpha particle is 233/4 times that of Fr-223 (assuming equal motion time and constant resistance), and the total kinetic energy released is 0.5E, what is the kinetic energy of the alpha particle?
[IMAGE:0]



\textbf{A.} 3E/227 \\
\textbf{B.} 446E/227 \\
\textbf{C.} E/2 \\
\textbf{D.} 111.5E/(223+4) \\

\textbf{Answer:} D \\
\textbf{Explanation:} Track Length vs. Velocity: Under constant resistance and equal motion time, track length is proportional to velocity, i.e.,
[IMAGE:0]
Momentum Conservation: Total momentum after decay is zero, so m
α
v
α
=m
Fr
v
Fr
​.
Substituting masses m
α
=4u, m
Fr
=223u:
[IMAGE:1]
[IMAGE:2]

\hrule
\vspace{1em}


\noindent
\textbf{Q939.} In a trapezium ABCD, the parallel sides AD and BC are (x+3)
cm(
x
+3)cm and (2x
−
1)
cm(2
x
−
1)cm respectively, with a vertical height of x
cm
x
cm. The area of the trapezium is 84 cm². What is the length of the longer base BC?
[IMAGE:0]



\textbf{A.} 7 cm \\
\textbf{B.} 9 cm \\
\textbf{C.} 11 cm \\
\textbf{D.} 13 cm \\

\textbf{Answer:} D \\
\textbf{Explanation:} [IMAGE:0]

\hrule
\vspace{1em}


\noindent
\textbf{Q940.} A light spring has an unstretched length of 0.50 m and a spring constant of 40 N/m. The spring is stretched by a tension force that starts at zero and increases at a constant rate of 0.10 N/s until it reaches its maximum value. When the force reaches its maximum value, the elastic potential energy of the spring is 0.25 J. What is the average power used to stretch the spring?



\textbf{A.} 0.022 W \\
\textbf{B.} 0.040 W \\
\textbf{C.} 0.011 W \\
\textbf{D.} 0.102 W \\

\textbf{Answer:} C \\
\textbf{Explanation:} he elastic potential energy formula is
U
=21
​
kx
2, where
k
is the spring constant and
x
is the displacement.
Given
U
=0.25J,
k
=40N/m, solving for
x
:
[IMAGE:0]
Maximum force
F
max
​
=
kx
=40×0.1118≈4.47N.
[IMAGE:1]
[IMAGE:2]

\hrule
\vspace{1em}


\noindent
\textbf{Q941.} A circuit contains two resistors connected in series, with resistances of 30 ohms and 60 ohms. A current of 1.5 amperes flows through the circuit. What is the total potential difference across the two resistors in volts?



\textbf{A.} 45V \\
\textbf{B.} 90V \\
\textbf{C.} 135V \\
\textbf{D.} 180V \\

\textbf{Answer:} C \\
\textbf{Explanation:} [IMAGE:0]
[IMAGE:1]

\hrule
\vspace{1em}


\noindent
\textbf{Q942.} A spring with a spring constant of 40 N/m is stretched by a force increasing at 0.10 N/s. When the strain energy is 0.08 J, the average power is:



\textbf{A.} 0.004 W \\
\textbf{B.} 0.003 W \\
\textbf{C.} 0.012 W \\
\textbf{D.} 0.016 W \\

\textbf{Answer:} B \\
\textbf{Explanation:} [IMAGE:0]

\hrule
\vspace{1em}


\noindent
\textbf{Q943.} A stationary radium-226 nucleus (Ra-226) undergoes alpha decay to form a radon-222 nucleus (Rn-222) and an alpha particle. The relative velocity between the radon-222 nucleus and the alpha particle after decay is 3v. What is the kinetic energy of the alpha particle?



\textbf{A.} [IMAGE:0] \\
\textbf{B.} [IMAGE:1] \\
\textbf{C.} [IMAGE:2] \\
\textbf{D.} [IMAGE:3] \\

\textbf{Answer:} D \\
\textbf{Explanation:} Conservation of Momentum
Before decay, the momentum of the radium-226 nucleus is zero. After decay, the momentum of the radon-222 nucleus and the alpha particle are equal in magnitude and opposite in direction. Let the mass of the alpha particle be 4 and the mass of radon-222 be 222. The conservation of momentum can be expressed as:
[IMAGE:0]
[IMAGE:1]

\hrule
\vspace{1em}


\noindent
\textbf{Q944.} A rectangular flower bed has a length of (x+2) meters and a width of (x
−
1) meters. To enhance the landscape, a 1-meter wide brick path is planned around the flower bed. The area of the brick path is 34 square meters. What are the length and width of the flower bed?
[IMAGE:0]



\textbf{A.} Length 7m, Width 3m \\
\textbf{B.} Length 8m, Width 4m \\
\textbf{C.} Length 9m, Width 5m \\
\textbf{D.} Length 10m, Width 6m \\

\textbf{Answer:} E \\
\textbf{Explanation:} 34 = 2*[(2*(x+2)+2)/2+(2*(x-1)+2)/2]

\hrule
\vspace{1em}


\noindent
\textbf{Q945.} A spring with a natural length of 0.25 m and spring constant 200 N/m is stretched by a force increasing at 0.80 N/s. When the strain energy is 0.50 J, the average power is:



\textbf{A.} 0.021 W \\
\textbf{B.} 0.050 W \\
\textbf{C.} 0.028 W \\
\textbf{D.} 0.376 W \\

\textbf{Answer:} C \\
\textbf{Explanation:} [IMAGE:0]

\hrule
\vspace{1em}


\noindent
\textbf{Q946.} An electric heater is connected to a circuit with a resistance of 50 ohms. When a current of 2.4 amperes flows through the heater, what is the potential difference across the heater in volts?



\textbf{A.} 24V \\
\textbf{B.} 48V \\
\textbf{C.} 72V \\
\textbf{D.} 120V \\

\textbf{Answer:} D \\
\textbf{Explanation:} Using Ohm's Law, the potential difference V can be calculated with the formula:
V=IR where I is the current (2.4 A) and R is the resistance (50 Ω). Substituting the values:V=2.4×50=120V

\hrule
\vspace{1em}


\noindent
\textbf{Q947.} A spring with a spring constant of 60 N/m is stretched by a force increasing at 0.30 N/s. When the strain energy is 0.18 J, the average power is:



\textbf{A.} 0.020 W \\
\textbf{B.} 0.012 W \\
\textbf{C.} 0.032 W \\
\textbf{D.} 0.042 W \\

\textbf{Answer:} B \\
\textbf{Explanation:} [IMAGE:0]

\hrule
\vspace{1em}


\noindent
\textbf{Q948.} A copper wire of length 2.00m has a uniform cross-sectional area of 2.0×10−6m2. There is a current of 3.0A in the wire. What is the potential difference across the ends of the wire? (Resistivity of copper ρ=1.7×10−8Ω\cdot m)



\textbf{A.} 0.170 V \\
\textbf{B.} 0.051 V \\
\textbf{C.} 0.510 V \\
\textbf{D.} 0.850 V \\

\textbf{Answer:} C \\
\textbf{Explanation:} [IMAGE:0]
[IMAGE:1]

\hrule
\vspace{1em}


\noindent
\textbf{Q949.} A stationary uranium-238 nucleus (U-238) undergoes alpha decay to form a thorium-234 nucleus (Th-234). Subsequently, the thorium-234 nucleus undergoes beta decay to form a protactinium-234 nucleus (Pa-234) and an electron. The total kinetic energy produced by the two decays is 0.5E. What is the kinetic energy of the alpha particle released in the first decay?



\textbf{A.} 117
E
​/238 \\
\textbf{B.} 4
E
​/234 \\
\textbf{C.} 234
E
​/238 \\
\textbf{D.} 234
E
​/237 \\

\textbf{Answer:} A \\
\textbf{Explanation:} Explanation: First Decay (U-238 \to  Th-234 + α particle); Second Decay (Th-234 \to  Pa-234 + electron); Total Kinetic Energy 0.5E = E₁ + E₂
[IMAGE:0]

\hrule
\vspace{1em}


\noindent
\textbf{Q950.} A spring with a spring constant of 120 N/m and natural length 0.30 m is stretched by a force increasing at 0.60 N/s. When the strain energy stored is 0.54 J, what is the average power?



\textbf{A.} 0.015 W \\
\textbf{B.} 0.030 W \\
\textbf{C.} 0.028 W \\
\textbf{D.} 0.060 W \\

\textbf{Answer:} C \\
\textbf{Explanation:} [IMAGE:0]

\hrule
\vspace{1em}


\noindent
\textbf{Q951.} A rectangular swimming pool has a length of (x+3) meters and a width of
1
meters. If the water is drained through an outlet at a rate of 3 cubic meters per second, it takes
8
seconds to empty the pool. What is the depth
(
depth
= x-1)
of the swimming pool?



\textbf{A.} 1m \\
\textbf{B.} 2m \\
\textbf{C.} 3m \\
\textbf{D.} 4m \\

\textbf{Answer:} D \\
\textbf{Explanation:} (x+3)(x-1)=3*8

\hrule
\vspace{1em}


\noindent
\textbf{Q952.} An LED is connected in series in a circuit with a current of 0.5 amperes. Given that the resistance of the LED is 100 ohms, what is the potential difference across the LED in volts?



\textbf{A.} 0.5V \\
\textbf{B.} 1.0V \\
\textbf{C.} 5.0V \\
\textbf{D.} 10.0V \\

\textbf{Answer:} E \\
\textbf{Explanation:} Using Ohm's Law, the potential difference V can be calculated with the formula: V=IR
where I is the current (0.5 A) and R is the resistance (100 Ω). V=0.5×100=50V
Thus, the potential difference across the LED is 50 volts, corresponding to option E.

\hrule
\vspace{1em}


\noindent
\textbf{Q953.} A light spring has a natural length of 0.50 m and a spring constant of 80 N/m. It is stretched by a force that starts at zero and increases at a constant rate of 0.40 N/s until reaching its maximum value. When the strain energy of the spring is 0.32 J, what is the average power used to stretch the spring?



\textbf{A.} 0.016 W \\
\textbf{B.} 0.018 W \\
\textbf{C.} 0.064 W \\
\textbf{D.} 0.080 W \\

\textbf{Answer:} B \\
\textbf{Explanation:} [IMAGE:0]

\hrule
\vspace{1em}


\noindent
\textbf{Q954.} A stationary americium-243 (Am-243) nucleus undergoes alpha decay to form neptunium-239 (Np-239) and an alpha particle. The total kinetic energy released in the decay is 0.5E, and the two particles move in opposite directions. What is the kinetic energy of the alpha particle?



\textbf{A.} 4E/243 \\
\textbf{B.} 478E/243 \\
\textbf{C.} E/2 \\
\textbf{D.} 119.5E/(239+4) \\

\textbf{Answer:} D \\
\textbf{Explanation:} By conservation of momentum, the magnitudes of the momenta of the alpha particle and Np-239 are equal. Let the mass of the alpha particle be 4u and that of Np-239 be 239u.
[IMAGE:0]

\hrule
\vspace{1em}


\noindent
\textbf{Q955.} An object is pushed up an inclined plane to the top. The base length of the incline is (x
−
2) meters, and the height is (x+3) meters. The pushing force is 10 N, and the work done by the force is 120 J. What is the length of the inclined plane?
[IMAGE:0]



\textbf{A.} 5m \\
\textbf{B.} 7m \\
\textbf{C.} 9m \\
\textbf{D.} 10m \\

\textbf{Answer:} E \\
\textbf{Explanation:} The formula for work done is:
W=F×d
where W is work, F is force, and d is the distance over which the force is applied (the length of the incline).

\hrule
\vspace{1em}


\noindent
\textbf{Q956.} A solid pyramid has a height of 160 m and a square base. The material density is 2600 kg/m³. If atmospheric pressure increases by 20 kPa, by how much will the average pressure increase?



\textbf{A.} 10 kPa \\
\textbf{B.} 20 kPa \\
\textbf{C.} 30 kPa \\
\textbf{D.} 40 kPa \\

\textbf{Answer:} B \\
\textbf{Explanation:} Pressure Formula:
[IMAGE:0]
The change in average pressure is equal to the change in atmospheric pressure. Therefore, if atmospheric pressure increases by 20 kPa, the average pressure will also increase by 20 kPa.

\hrule
\vspace{1em}


\noindent
\textbf{Q957.} An al
uminum wire of length 4.00 m has a uniform cross-sectional area of 2.0×10⁻⁶ m². There is a current of 2.0 A in the wire. What is the potential difference across the ends of the wire? (Resistivity of aluminum = 2.8×10⁻⁸Ω\cdot m)



\textbf{A.} 0.112 V \\
\textbf{B.} 0.224 V \\
\textbf{C.} 0.336 V \\
\textbf{D.} 0.448 V \\

\textbf{Answer:} A \\
\textbf{Explanation:} [IMAGE:0]
Thus, the potential difference across the ends of the wire is 0.112 volts, corresponding to option A.

\hrule
\vspace{1em}


\noindent
\textbf{Q958.} A stationary radium-226 nucleus (Ra-226) undergoes alpha decay to form a radon-222 nucleus (Rn-222). Subsequently, the radon-222 nucleus undergoes another alpha decay to form a polonium-218 nucleus (Po-218). The total kinetic energy produced by the two decays is 0.5E. What is the kinetic energy of the alpha particle released in the second decay?



\textbf{A.} 109E​/226 \\
\textbf{B.} 4E​/218 \\
\textbf{C.} 436E​/226 \\
\textbf{D.} 218E​/222 \\

\textbf{Answer:} A \\
\textbf{Explanation:} (1)First Decay (Ra-226 \to  Rn-222 + α particle)
Initial momentum is zero. After decay, the momentum of Rn-222 and the α particle are equal in magnitude and opposite in direction. Mass of α particle = 4, mass of Rn-222 = 222.
[IMAGE:0]
(2)Second Decay (Rn-222 \to  Po-218 + α particle)
Initial momentum is that of Rn-222. After decay, the momentum of Po-218 and the α particle are equal in magnitude and opposite in direction.Mass of α particle = 4, mass of Po-218 = 218.
[IMAGE:1]

\hrule
\vspace{1em}


\noindent
\textbf{Q959.} In a trapezium PQRS, the parallel sides are PQ and RS. PQ =(x
−
5
) cm, RS =(x+
9
) cm and the vertical height QR = x cm.
[IMAGE:0]
The area of the trapezium is
15
cm
2
. What is the length of RS?



\textbf{A.} 9cm \\
\textbf{B.} 2cm \\
\textbf{C.} 1cm \\
\textbf{D.} 3cm \\

\textbf{Answer:} D \\
\textbf{Explanation:} Area=(x-
5
+x+
9
)*x/2=
15

\hrule
\vspace{1em}


\noindent
\textbf{Q960.} A solid cone has a height of 240 m and a circular base. The material density is 2200 kg/m³, and atmospheric pressure is 100 kPa. If the base area is doubled, how will the average pressure change?



\textbf{A.} Increase \\
\textbf{B.} Decrease \\
\textbf{C.} Remain the same \\
\textbf{D.} Cannot be determined \\

\textbf{Answer:} C \\
\textbf{Explanation:} Pressure Formula:
[IMAGE:0]
The base area cancels out in the pressure formula, so changing the base area does not affect the final pressure. Therefore, doubling the base area will not change the average pressure.

\hrule
\vspace{1em}


\noindent
\textbf{Q961.} Two resistors with resistances of 10Ω and 20Ω are connected in parallel in a circuit. The total current in the circuit is 3A. What is the current through the 20Ω resistor?



\textbf{A.} 0.5A \\
\textbf{B.} 1.0A \\
\textbf{C.} 1.5A \\
\textbf{D.} 2.0A \\

\textbf{Answer:} B \\
\textbf{Explanation:} In a parallel circuit, the voltage across each branch is equal and the same as the total voltage. The current through the 20Ω resistor I2 is: $I_2$=V_"total " /$R_2$ =20V/20Ω=1.0A

\hrule
\vspace{1em}


\noindent
\textbf{Q962.} A solid rectangular prism and a solid cylinder have the same base area of 100 m². The prism has a height of 120 m and a density of 2400 kg/m³; the cylinder has a height of 180 m and a density of 2000 kg/m³. Atmospheric pressure is 100 kPa. What is the sum of their average pressures?



\textbf{A.} 8200 kPa \\
\textbf{B.} 6900 kPa \\
\textbf{C.} 6680 kPa \\
\textbf{D.} 7100 kPa \\

\textbf{Answer:} C \\
\textbf{Explanation:} Prism Pressure Calculation:
[IMAGE:0]
Cylinder Pressure Calculation:
[IMAGE:1]
[IMAGE:2]

\hrule
\vspace{1em}


\noindent
\textbf{Q963.} In a trapezium PQRS, the parallel sides are PQ and RS. PQ =(x−1) cm, RS =(x+9) cm and the vertical height QR = x cm.
[IMAGE:0]
The area of the trapezium is 252 cm
2
. What is the length of RS?



\textbf{A.} 9cm \\
\textbf{B.} 10cm \\
\textbf{C.} 11cm \\
\textbf{D.} 12cm \\

\textbf{Answer:} E \\
\textbf{Explanation:} Area=(x-1+x+9)*x/2=252

\hrule
\vspace{1em}


\noindent
\textbf{Q964.} Two resistors with resistances of 10Ω and 20Ω are connected in parallel in a circuit. The total current in the circuit is 3A. What is the voltage across the 20Ω resistor?



\textbf{A.} 5V \\
\textbf{B.} 10V \\
\textbf{C.} 15V \\
\textbf{D.} 20V \\

\textbf{Answer:} D \\
\textbf{Explanation:} In a parallel circuit, the voltage across each branch is equal and the same as the total voltage.The total resistance is calculated using the formula:
[IMAGE:0]
Substituting the values:
[IMAGE:1]
Using Ohm's law, the total voltage
V
total​ is:
[IMAGE:2]

\hrule
\vspace{1em}


\noindent
\textbf{Q965.} A solid pyramid has a height of 150 m and a square base. The average pressure on the ground is 450 kPa, and atmospheric pressure is 100 kPa. What is the density of the pyramid material?



\textbf{A.} 500 kg/m³ \\
\textbf{B.} 600 kg/m³ \\
\textbf{C.} 700 kg/m³ \\
\textbf{D.} 800 kg/m³ \\

\textbf{Answer:} C \\
\textbf{Explanation:} Pressure Formula:
[IMAGE:0]
Substitute Known Values:
[IMAGE:1]
Solve for Density:
[IMAGE:2]

\hrule
\vspace{1em}


\noindent
\textbf{Q966.} A stationary californium-252 (Cf-252) nucleus undergoes alpha decay to form curium-248 (Cm-248) and an alpha particle. The total kinetic energy released in the decay is 0.5E. If the alpha particle and Cm-248 move in opposite directions, and their kinetic energies are determined by momentum conservation, what is the kinetic energy of the alpha particle?
[IMAGE:0]



\textbf{A.} 4E/252 \\
\textbf{B.} 124E/252 \\
\textbf{C.} E/2 \\
\textbf{D.} 248E/(248+4) \\

\textbf{Answer:} B \\
\textbf{Explanation:} By conservation of momentum, the magnitudes of the momenta of the alpha particle and Cm-248 are equal. Let the mass of the alpha particle be 4u and that of Cm-248 be 248u. From momentum conservation: 4vα=248vCm  leading to the velocity ratio vα:vCm=248:4 Kinetic energy is proportional to mass and the square of velocity. Thus, the ratio of kinetic energies is:
[IMAGE:0]

\hrule
\vspace{1em}


\noindent
\textbf{Q967.} Two resistors with resistances of 5Ω and 15Ω are connected in series in a circuit. The total voltage across the circuit is 10V. What is the voltage across the 15Ω resistor?



\textbf{A.} 2.5 V \\
\textbf{B.} 5.0 V \\
\textbf{C.} 7.5 V \\
\textbf{D.} 10.0 V \\

\textbf{Answer:} C \\
\textbf{Explanation:} In a series circuit, the total resistance is the sum of the individual resistances:
[IMAGE:0]
The current
I
in the circuit is calculated using Ohm's law:
[IMAGE:1]
The voltage across the 15Ω resistor V2​ is:
[IMAGE:2]

\hrule
\vspace{1em}


\noindent
\textbf{Q968.} In a trapezium PQRS, the parallel sides are PQ and RS. PQ =(x−3) cm, RS =(x+7) cm and the vertical height QR = x cm.
[IMAGE:0]
The area of the trapezium is 99 cm
2
. What is the length of RS?



\textbf{A.} 9cm \\
\textbf{B.} 10cm \\
\textbf{C.} 11cm \\
\textbf{D.} 12cm \\

\textbf{Answer:} A \\
\textbf{Explanation:} Area=(x-3+x+6)*x/2=99

\hrule
\vspace{1em}


\noindent
\textbf{Q969.} A solid cylinder and a solid cone have the same height of 200 m and the same base area. The density of the cylinder is 2500 kg/m³, and the density of the cone is 3000 kg/m³. Atmospheric pressure is 100 kPa. What is the difference in average pressure between the two?



\textbf{A.} 1200 kPa \\
\textbf{B.} 2200 kPa \\
\textbf{C.} 3200 kPa \\
\textbf{D.} 4100 kPa \\

\textbf{Answer:} E \\
\textbf{Explanation:} [IMAGE:0]

\hrule
\vspace{1em}


\noindent
\textbf{Q970.} A stationary nitrogen molecule (N₂) decomposes into two nitrogen atoms (N) at high temperatures. The total kinetic energy of the two nitrogen atoms after decomposition is 0.5E, and the mass of each nitrogen atom is 16u (atomic mass units). What is the kinetic energy of one of the nitrogen atoms?



\textbf{A.} E/
4 \\
\textbf{B.} E
​/16 \\
\textbf{C.} 14
E
​/28 \\
\textbf{D.} 4
E
​/141 \\

\textbf{Answer:} A \\
\textbf{Explanation:} Before decomposition, the nitrogen molecule is stationary, so its momentum is zero. After decomposition, the two nitrogen atoms have equal magnitude but opposite direction momenta. Let the mass of each nitrogen atom be 14u, and their velocities be v₁ and v₂. By the conservation of momentum:
[IMAGE:0]
Since m₁ = m₂ = 16u, this simplifies to:
[IMAGE:1]
he total kinetic energy is E, so:
[IMAGE:2]
Substituting m₁ = m₂ = 16u and v₁ = -v₂, we get:
[IMAGE:3]
Therefore, the kinetic energy of one nitrogen atom is:
[IMAGE:4]

\hrule
\vspace{1em}


\noindent
\textbf{Q971.} An aluminum wire of length 1.50 m has a uniform cross-sectional area of 3.0×10−7m2. There is a current of 2.5 A in the wire. What is the potential difference across the ends of the wire? (Resistivity of aluminum = 2.8×10−8Ω\cdot m
)



\textbf{A.} 0.140 V \\
\textbf{B.} 0.280 V \\
\textbf{C.} 0.350 V \\
\textbf{D.} 0.429 V \\

\textbf{Answer:} C \\
\textbf{Explanation:} The resistance is calculated using the formula:
[IMAGE:0]
Using Ohm's law, the potential difference
V
=
IR
:
[IMAGE:1]

\hrule
\vspace{1em}


\noindent
\textbf{Q972.} Which of the following is a correct unit of inductance?



\textbf{A.} Farad \\
\textbf{B.} Henry \\
\textbf{C.} Tesla \\
\textbf{D.} Pascal \\

\textbf{Answer:} B \\
\textbf{Explanation:} The unit of inductance is the henry (Henry). According to the definition of inductance:
L
=Φ/I​
where Φ is magnetic flux (unit: weber, Wb) and
I
is current (unit: ampere, A). Therefore, the unit of inductance is Wb/A, which is the henry (H).

\hrule
\vspace{1em}


\noindent
\textbf{Q973.} A solid prism of height 200 m has a triangular base. The density of the material is 2200 kg/m³. Atmospheric pressure is 100 kPa. What is the average pressure on the ground under the prism?



\textbf{A.} 4500 kPa \\
\textbf{B.} 5400 kPa \\
\textbf{C.} 6400 kPa \\
\textbf{D.} 7400 kPa \\

\textbf{Answer:} A \\
\textbf{Explanation:} Volume Calculation: Volume of a prism = base area × height.
Mass Calculation: Mass = volume × density.
Weight Calculation: Weight = mass × gravitational field strength (
g
=10N/kg).
Total Pressure Calculation: Average pressure = weight/base area
​
+atmospheric pressure.
[IMAGE:0]

\hrule
\vspace{1em}


\noindent
\textbf{Q974.} A copper wire of length 2.00 m has a uniform cross-sectional area of 5.0×10−7m2. There is a current of 3.0 A in the wire. What is the potential difference across the ends of the wire? (Resistivity of copper = 1.7×10−8Ω\cdot m)



\textbf{A.} 0.102 V \\
\textbf{B.} 0.204 V \\
\textbf{C.} 0.306 V \\
\textbf{D.} 0.408 V \\

\textbf{Answer:} B \\
\textbf{Explanation:} The resistance is calculated using the formula:
[IMAGE:0]
where
ρ
is the resistivity,
L
is the length, and
A
is the cross-sectional area.
Substituting the values:
[IMAGE:1]
Using Ohm's law, the potential difference V=IR:
[IMAGE:2]

\hrule
\vspace{1em}


\noindent
\textbf{Q975.} A solid cylinder of height 300 m has a circular base. The density of the material is 2800 kg/m³. Atmospheric pressure is 100 kPa. What is the average pressure on the ground under the cylinder?



\textbf{A.} 8500 kPa \\
\textbf{B.} 9400 kPa \\
\textbf{C.} 1040 kPa \\
\textbf{D.} 1140 kPa \\

\textbf{Answer:} A \\
\textbf{Explanation:} Explanation:
Volume Calculation: Volume of a cylinder = base area × height.
Mass Calculation: Mass = volume × density.
Weight Calculation: Weight = mass × gravitational field strength (
g
=10N/kg).
Total Pressure Calculation: Average pressure = aweight/base are
​
+atmospheric pressure.
[IMAGE:0]

\hrule
\vspace{1em}


\noindent
\textbf{Q976.} Which of the following is a correct unit of capacitance?



\textbf{A.} Farad \\
\textbf{B.} Henry \\
\textbf{C.} Tesla \\
\textbf{D.} Pascal \\

\textbf{Answer:} A \\
\textbf{Explanation:} The unit of capacitance is the farad (Farad). According to the definition of capacitance:
C
=
Q
/V
​
where
Q
is electric charge (unit: coulomb, C) and
V
is voltage (unit: volt, V). Therefore, the unit of capacitance is C/V, which is the farad (F).

\hrule
\vspace{1em}


\noindent
\textbf{Q977.} A solid rectangular prism of height 120 m has a rectangular base. The density of the material is 2500 kg/m³. Atmospheric pressure is 100 kPa. What is the average pressure on the ground under the prism?



\textbf{A.} 80 kPa \\
\textbf{B.} 180 kPa \\
\textbf{C.} 20 kPa \\
\textbf{D.} 380 kPa \\

\textbf{Answer:} E \\
\textbf{Explanation:} Volume Calculation: Volume of a rectangular prism = base area × height.
Mass Calculation: Mass = volume × density.
Weight Calculation: Weight = mass × gravitational field strength (
g
=10N/kg).
Total Pressure Calculation: Average pressure = weight/base area
​
+atmospheric pressure.
[IMAGE:0]

\hrule
\vspace{1em}


\noindent
\textbf{Q978.} Which of the following is a correct unit of acceleration?



\textbf{A.} Meters per second \\
\textbf{B.} Meters per second squared \\
\textbf{C.} Newton \\
\textbf{D.} Joule \\

\textbf{Answer:} B \\
\textbf{Explanation:} The unit of acceleration is meters per second squared (m/s²). According to the definition of acceleration:
a
=Δ
v
​/Δ
t
where Δ
v
is change in velocity (unit: meters per second, m/s) and Δ
t
is change in time (unit: second, s). Therefore, the unit of acceleration is m/s².

\hrule
\vspace{1em}


\noindent
\textbf{Q979.} When a stationary radium-226 nucleus decays by alpha emission to form a radon-222 nucleus, the total kinetic energy produced by the decay is 0.5E. What is the kinetic energy of the alpha particle?
[IMAGE:0]



\textbf{A.} 4E​/226 \\
\textbf{B.} 444E​/452 \\
\textbf{C.} 2E \\
\textbf{D.} 111E /452 \\

\textbf{Answer:} D \\
\textbf{Explanation:} By the conservation of momentum, the momentum of the radium-226 nucleus before decay is zero. After decay, the momentum of the radon-222 nucleus and the alpha particle are equal in magnitude and opposite in direction. Let the mass of the alpha particle be 4 and the mass of radon-222 be 222. The conservation of momentum can be expressed as:
[IMAGE:0]
Simplifying this equation results in:
[IMAGE:1]
Therefore, the kinetic energy of the alpha particle is
[IMAGE:2]

\hrule
\vspace{1em}


\noindent
\textbf{Q980.} A solid cone of height 420 m has a circular base. The density of the material is 2100 kg/m³. Atmospheric pressure is 100 kPa. What is the average pressure on the ground under the cone?



\textbf{A.} 98 kPa \\
\textbf{B.} 108 kPa \\
\textbf{C.} 198 kPa \\
\textbf{D.} 980 kPa \\

\textbf{Answer:} G \\
\textbf{Explanation:} Volume Calculation: Volume of a cone = 1/3
​
×base area×height.
Mass Calculation: Mass = volume × density.
Weight Calculation: Weight = mass × gravitational field strength (
g
=10N/kg).
Total Pressure Calculation: Average pressure = weight/base area
​
+atmospheric pressure.
[IMAGE:0]

\hrule
\vspace{1em}


\noindent
\textbf{Q981.} Which of the following is a correct unit of velocity?



\textbf{A.} Meters per second \\
\textbf{B.} Meters per second squared \\
\textbf{C.} Newton \\
\textbf{D.} Joule \\

\textbf{Answer:} A \\
\textbf{Explanation:} The unit of velocity is meters per second (m/s). According to the definition of velocity:
v
=
d
/t
​
where
d
is distance (unit: meter, m) and
t
is time (unit: second, s). Therefore, the unit of velocity is m/s.

\hrule
\vspace{1em}


\noindent
\textbf{Q982.} A solid pyramid with a height of 20 m has a square base. The density of the material is 1500 kg/m³. Atmospheric pressure is 100 kPa. What is the average pressure on the ground under the pyramid?



\textbf{A.} 100 kPa \\
\textbf{B.} 200 kPa \\
\textbf{C.} 300 kPa \\
\textbf{D.} 400 kPa \\

\textbf{Answer:} B \\
\textbf{Explanation:} [IMAGE:0]

\hrule
\vspace{1em}


\noindent
\textbf{Q983.} Which of the following is a correct unit of power?



\textbf{A.} Joule \\
\textbf{B.} Watt \\
\textbf{C.} Newton \\
\textbf{D.} Coulomb \\

\textbf{Answer:} B \\
\textbf{Explanation:} The unit of power is the watt (Watt). According to the definition of power:
P
=
E
/t
​
where
E
is energy (unit: joule, J) and
t
is time (unit: second, s). Therefore, the unit of power is J/s, which is the watt (W).

\hrule
\vspace{1em}


\noindent
\textbf{Q984.} A stationary plutonium-242 (Pu-242) nucleus decays by alpha emission to form a uranium-238 (U-238) nucleus and an alpha particle. The total kinetic energy produced by the decay is 0.5E.
What is the kinetic energy of the alpha particle?
[IMAGE:0]



\textbf{A.} 119E/242 \\
\textbf{B.} 4E/238 \\
\textbf{C.} E/2 \\
\textbf{D.} 476E/242 \\

\textbf{Answer:} A \\
\textbf{Explanation:} By conservation of momentum, the magnitudes of the momenta of U-238 and the alpha particle are equal. Let the mass of the alpha particle be 4u and that of U-238 be 238u.
From momentum conservation: 4vα=238vU, leading to the velocity ratio v
α
:v
U
=238:4. Kinetic energy is proportional to mass and the square of velocity. Thus, the ratio of kinetic energies is:
[IMAGE:0]

\hrule
\vspace{1em}


\noindent
\textbf{Q985.} Three identical diodes are labeled A, B, and C, each with different polarity connections. Each diode is connected, in turn, across the same battery, which has a voltage of 6V and negligible internal resistance. The diodes are tested for conduction. What is the correct order of the diodes when arranged from "non-conducting" to "conducting"?
[IMAGE:0]



\textbf{A.} A < B < C \\
\textbf{B.} A < C < B \\
\textbf{C.} B < A < C \\
\textbf{D.} B < C < A \\

\textbf{Answer:} D \\
\textbf{Explanation:} Diode conduction depends on the polarity of the connection.
A: Diode is forward-biased (anode connected to positive terminal), current flows.
B: Diode is reverse-biased (cathode connected to positive terminal), current does not flow.
C: Diode is forward-biased but parallel to a reverse-biased diode, current flows through the forward-biased diode.
Order from non-conducting to conducting: B < C < A.

\hrule
\vspace{1em}


\noindent
\textbf{Q986.} Which of the following is a correct unit of energy?



\textbf{A.} Newton \\
\textbf{B.} Joule \\
\textbf{C.} Watt \\
\textbf{D.} Coulomb \\

\textbf{Answer:} B \\
\textbf{Explanation:} The unit of energy is the joule (Joule). According to the definition of energy:
E
=
F
\cdot 
d
where
F
is force (unit: newton, N) and
d
is distance (unit: meter, m). Therefore, the unit of energy is N\cdot m, which is the joule (J).

\hrule
\vspace{1em}


\noindent
\textbf{Q987.} Three conductors of the same material with different shapes are connected to the same battery (negligible internal resistance):
\cdot 
A: Length L
L
, diameter d
d
\cdot 
B: Length 2L2
L
, diameter d
d
\cdot 
C: Length L
L
, diameter 2d2
d
If arranged in order of increasing total power, which option is correct?



\textbf{A.} B, A, C \\
\textbf{B.} C, A, B \\
\textbf{C.} A, B, C \\
\textbf{D.} B, C, A \\

\textbf{Answer:} A \\
\textbf{Explanation:} Total power is inversely proportional to resistance
[IMAGE:0]
Resistance is given by
[IMAGE:1]
[IMAGE:2]
[IMAGE:3]
Resistance order: B > A > C, corresponding to power order: B < A < C.

\hrule
\vspace{1em}


\noindent
\textbf{Q988.} A pendulum bob of mass
m
swings from a height
h
to the lowest point of its arc. If the length of the pendulum is
L
, what is the speed of the bob at the lowest point?



\textbf{A.} [IMAGE:0] \\
\textbf{B.} [IMAGE:1] \\
\textbf{C.} [IMAGE:2] \\
\textbf{D.} [IMAGE:3] \\

\textbf{Answer:} A \\
\textbf{Explanation:} Using energy conservation is easier than using kinematics equations :
[IMAGE:0]

\hrule
\vspace{1em}


\noindent
\textbf{Q989.} A radioactive sample initially contains 64 atoms of isotope E, which has a half-life of 2000 years. What is the ratio of the remaining E atoms to the decayed E atoms in the sample after 6000 years?



\textbf{A.} 1:3 \\
\textbf{B.} 1:7 \\
\textbf{C.} 1:15 \\
\textbf{D.} 1:1 \\

\textbf{Answer:} B \\
\textbf{Explanation:} after 6000 years, the ratio of remaining E atoms to decayed E atoms is 1:7.

\hrule
\vspace{1em}


\noindent
\textbf{Q990.} Which of the following is a correct unit of force?



\textbf{A.} Newton \\
\textbf{B.} Joule \\
\textbf{C.} Watt \\
\textbf{D.} Coulomb \\

\textbf{Answer:} A \\
\textbf{Explanation:} The unit of force is the newton (Newton). According to Newton's second law:
F
=
ma
where
m
is mass (unit: kilogram, kg) and
a
is acceleration (unit: meters per second squared, m/s²). Therefore, the unit of force is kg\cdot m/s², which is the newton (N).

\hrule
\vspace{1em}


\noindent
\textbf{Q991.} A battery (voltage 12V, internal resistance 1Ω) is connected to three different circuits, each containing a variable resistor. The total power is measured when the variable resistor is set to
R
1
​
=1Ω,
R
2
​
=2Ω, and
R
3
​
=3Ω. What is the correct order of the circuits when arranged in order of increasing power?
[IMAGE:0]



\textbf{A.} R1<R2<R3 \\
\textbf{B.} R1<R3<R2 \\
\textbf{C.} R2<R1<R3 \\
\textbf{D.} R2<R3<R1 \\

\textbf{Answer:} F \\
\textbf{Explanation:} Explanation
:
Power is calculated using
[IMAGE:0]
where
V
is the battery voltage,
R
is the variable resistor, and
r
is the internal resistance.
[IMAGE:1]
Power order:
R
3
​
<
R
2
​
<
R
1
​
.

\hrule
\vspace{1em}


\noindent
\textbf{Q992.} Which of the following is a correct unit of potential difference (voltage)?



\textbf{A.} joule per ampere \\
\textbf{B.} coulomb per ohm \\
\textbf{C.} newton per coulomb \\
\textbf{D.} watt per coulomb \\

\textbf{Answer:} E \\
\textbf{Explanation:} From the formula P=VI, voltage V=P/I​. The unit of power P is watt (W), and the unit of current I is ampere (A). Therefore, the unit of voltage is watt per ampere (W/A), corresponding to option E.

\hrule
\vspace{1em}


\noindent
\textbf{Q993.} Three different resistors R1
​
=1Ω, R2
​
=2Ω, and R3
​
=3Ω are used to form three different circuits, labeled X, Y, and Z. Each circuit is connected, in turn, across the same battery, which has a voltage of 6V and negligible internal resistance. The voltage across each resistor is measured. What is the correct order of the circuits when arranged in order of increasing voltage?
[IMAGE:0]



\textbf{A.} X < Y < Z \\
\textbf{B.} X < Z < Y \\
\textbf{C.} Y < X < Z \\
\textbf{D.} Y < Z < X \\

\textbf{Answer:} B \\
\textbf{Explanation:} X
: Resistors in series, total resistance
R
total
​
=6Ω.
[IMAGE:0]
Y
: Resistors in parallel, voltage across each resistor is 6V.
Z
:
R
1
​
and
R
2
​
in series, then parallel with
R
3
​
.
Series part voltage:
[IMAGE:1]

\hrule
\vspace{1em}


\noindent
\textbf{Q994.} A hovercraft of mass
m
moves at constant speed
v
on a horizontal surface. If the lift fan provides a force equal to the weight of the hovercraft which consumes power
P
0
, what is the power required to maintain this speed if friction is
f
?



\textbf{A.} fv \\
\textbf{B.} [IMAGE:0] \\
\textbf{C.} [IMAGE:1] \\
\textbf{D.} [IMAGE:2] \\

\textbf{Answer:} E \\
\textbf{Explanation:} Power is force times velocity:P=fv.
P is not vector but scalar.

\hrule
\vspace{1em}


\noindent
\textbf{Q995.} A sample of ancient wood contains carbon-14 (a radioactive isotope) and carbon-12 (a stable isotope) in a ratio of 1:1. The half-life of carbon-14 is 5730 years. What is the ratio of carbon-14 to carbon-12 atoms in the sample after 11460 years?



\textbf{A.} 1:3 \\
\textbf{B.} 1:4 \\
\textbf{C.} 1:15 \\
\textbf{D.} 1:16 \\

\textbf{Answer:} B \\
\textbf{Explanation:} The decay of carbon-14 follows the radioactive decay formula:
[IMAGE:0]
where NC-14,0​ is the initial number of carbon-14 atoms, t is the time elapsed, and T1/2​ is the half-life of carbon-14 (5730 years). The ratio of carbon-14 to carbon-12 atoms is:
[IMAGE:1]

\hrule
\vspace{1em}


\noindent
\textbf{Q996.} Three identical resistors are connected to the same battery (negligible internal resistance) as follows:
A: Two in series then parallel with the third
B: All three in parallel
C: All three in series
If arranged in order of increasing total power, which option is correct?
[IMAGE:0]



\textbf{A.} C, A, B \\
\textbf{B.} B, A, C \\
\textbf{C.} A, C, B \\
\textbf{D.} C, B, A \\

\textbf{Answer:} A \\
\textbf{Explanation:} Total power is inversely proportional to total resistance
[IMAGE:0]
A: Total resistance =
[IMAGE:1]
B: Total resistance =
[IMAGE:2]
C: Total resistance =3R
Resistance order: C > A > B, corresponding to power order: C < A < B.

\hrule
\vspace{1em}


\noindent
\textbf{Q997.} Which of the following is a correct unit of electric current?



\textbf{A.} Watt per volt \\
\textbf{B.} Coulomb per second \\
\textbf{C.} Volt per meter \\
\textbf{D.} Newton per coulomb \\

\textbf{Answer:} B \\
\textbf{Explanation:} The unit of electric current is the ampere (Ampere). According to the definition of current:
[IMAGE:0]
where Q is the charge (unit: coulomb, C) and t is time (unit: second, s). Therefore, the unit of current is coulomb per second (C/s), which is the ampere (A).

\hrule
\vspace{1em}


\noindent
\textbf{Q998.} A sample initially contains equal numbers of atoms of a radioactive isotope X and a stable isotope Y. Isotope X has a half-life of 2 years and decays in a single stage to the stable isotope Y. What is the ratio of the number of atoms of X to the number of atoms of Y in the sample after 6 years?



\textbf{A.} The sample contains only isotope Y. \\
\textbf{B.} 1:3 \\
\textbf{C.} 1:4 \\
\textbf{D.} 1:7 \\

\textbf{Answer:} E \\
\textbf{Explanation:} The decay of a radioactive isotope follows the half-life rule. Isotope X has a half-life of 2 years. After 6 years, the sample undergoes 3 half-lives (2 years each).

\hrule
\vspace{1em}


\noindent
\textbf{Q999.} Three identical springs are used to form three different mechanical systems, labeled D, E, and F. Each system is connected, in turn, to the same mass block, and the total spring constant of each system is measured. What is the correct order of the systems when arranged in order of increasing spring constant?
[IMAGE:0]



\textbf{A.} D < E < F \\
\textbf{B.} D < F < E \\
\textbf{C.} E < D < F \\
\textbf{D.} E < F < D \\

\textbf{Answer:} B \\
\textbf{Explanation:} The total spring constant depends on the connection method.
D is two springs in series, with a total spring constant of
k/2
​
.
E is two springs in parallel, with a total spring constant of 2
k
.
F is a single spring, with a total spring constant of
k
.
Total spring constant order:
k/2
​
<
k
<2
k
, so spring constant order: D < F < E.

\hrule
\vspace{1em}


\noindent
\textbf{Q1000.} A ball of mass
100m
is thrown vertically upward with initial speed
v
. Ignoring air resistance, what is the maximum height reached?



\textbf{A.} [IMAGE:0] \\
\textbf{B.} [IMAGE:1] \\
\textbf{C.} [IMAGE:2] \\
\textbf{D.} [IMAGE:3] \\

\textbf{Answer:} D \\
\textbf{Explanation:} Using kinematic equation:
[IMAGE:0]

\hrule
\vspace{1em}


\noindent
\textbf{Q1001.} A balloon is released and begins to rise. During the ascent, the balloon experiences a buoyant force that decreases with time (due to decreasing air density with height) and a drag force proportional to its velocity. The balloon eventually reaches a terminal velocity. The graphs below show the variation with time of three quantities (X, Y, and Z):
[IMAGE:0]
Which line in the table correctly identifies the quantities X, Y, and Z?



\textbf{A.} [IMAGE:0] \\
\textbf{B.} [IMAGE:1] \\
\textbf{C.} [IMAGE:2] \\
\textbf{D.} [IMAGE:3] \\

\textbf{Answer:} C \\
\textbf{Explanation:} \cdot 
X is Buoyant Force: The buoyant force decreases as the balloon rises due to decreasing air density, approaching a constant value, matching the X graph.
\cdot 
Y is Drag Force: Drag force increases with velocity and approaches a constant value at terminal velocity, matching the Y graph.
\cdot 
Z is Weight: Weight is constant and independent of time, matching the horizontal Z graph.

\hrule
\vspace{1em}


\noindent
\textbf{Q1002.} Three identical inductors are used to form three different circuits, labeled M, N, and O. Each circuit is connected, in turn, across the same AC power supply, which has a constant voltage and negligible internal resistance. The total inductance of each circuit is measured. What is the correct order of the circuits when arranged in order of increasing inductance?
[IMAGE:0]



\textbf{A.} M < N < O \\
\textbf{B.} M < O < N \\
\textbf{C.} N < M < O \\
\textbf{D.} N < O < M \\

\textbf{Answer:} D \\
\textbf{Explanation:} The total inductance depends on the connection method.
M is two inductors in series, with a total inductance of 2
L
.
N is two inductors in parallel, with a total inductance of
L/2
​
.
O is a single inductor, with a total inductance of
L
.
Total inductance order:
L/2
​
<
L
<2
L
, so inductance order: N < O < M.

\hrule
\vspace{1em}


\noindent
\textbf{Q1003.} Three identical resistors are connected in the following configurations to the same battery (negligible internal resistance):
A: All three in series
B: Two in parallel combined with the third in series
C: All three in parallel
[IMAGE:0]
If arranged in order of increasing total power, which option is correct?



\textbf{A.} A, B, C \\
\textbf{B.} C, B, A \\
\textbf{C.} B, A, C \\
\textbf{D.} A, C, B \\

\textbf{Answer:} A \\
\textbf{Explanation:} Total power is inversely proportional to total resistance
[IMAGE:0]
A (Series): Total resistance = R/3
B (Parallel-Series): Total resistance = 3R/2
C (Parallel): Total resistance = R/3
Ordering by resistance: A > B > C, corresponding to power: A < B < C.

\hrule
\vspace{1em}


\noindent
\textbf{Q1004.} A rocket of mass
m
ejects mass at a rate
[IMAGE:0]
with velocity
u
relative to the rocket. If the rocket starts from rest, what is its speed after time
t
?



\textbf{A.} uln2 \\
\textbf{B.} [IMAGE:0] \\
\textbf{C.} u \\
\textbf{D.} [IMAGE:1] \\

\textbf{Answer:} A \\
\textbf{Explanation:} Using rocket equation:
[IMAGE:0]

\hrule
\vspace{1em}


\noindent
\textbf{Q1005.} An object is immersed in a fluid and begins to sink from rest. During the sinking process, the object experiences a buoyant force that varies with time and a drag force proportional to its velocity. The object eventually reaches a terminal velocity. The graphs below show the variation with time of three quantities (X, Y, and Z):
[IMAGE:0]



\textbf{A.} X:Buoyant Force  Y:Drag Force  Z:Kinetic Energy \\
\textbf{B.} X:Drag Force  Y:Buoyant Force  Z:Kinetic Energy \\
\textbf{C.} X:Buoyant Force  Y:Drag Force  Z:Weight \\
\textbf{D.} X:Kinetic Energy  Y:Buoyant Force  Z:Drag Force \\

\textbf{Answer:} C \\
\textbf{Explanation:} \cdot 
X is Buoyant Force: The buoyant force decreases as the object displaces less fluid, approaching a constant value when fully submerged, matching the X graph.
\cdot 
Y is Drag Force: Drag force increases with velocity and approaches a constant value at terminal velocity, matching the Y graph.
\cdot 
Z is Weight: Weight is constant and independent of time, matching the horizontal Z graph.

\hrule
\vspace{1em}


\noindent
\textbf{Q1006.} Uranium-238 (U-238) undergoes two alpha decays. What element does it become?



\textbf{A.} Thorium-234 (Th-234) \\
\textbf{B.} Radium-226 (Ra-226) \\
\textbf{C.} Actinium-226 (Ac-226) \\
\textbf{D.} Radon-222 (Rn-222) \\

\textbf{Answer:} B \\
\textbf{Explanation:} Alpha decay involves the emission of an alpha particle (a helium nucleus consisting of 2 protons and 2 neutrons). Each alpha decay reduces the atomic number by 2 and the mass number by 4.

\hrule
\vspace{1em}


\noindent
\textbf{Q1007.} Three identical capacitors are used to form three different circuits, labeled P, Q, and R. Each circuit is connected, in turn, across the same power supply, which has a constant voltage and negligible internal resistance. The total charge stored in each circuit is measured. What is the correct order of the circuits when arranged in order of increasing charge?
[IMAGE:0]



\textbf{A.} P < Q < R \\
\textbf{B.} P < R < Q \\
\textbf{C.} Q < P < R \\
\textbf{D.} Q < R < P \\

\textbf{Answer:} B \\
\textbf{Explanation:} The total charge stored in a capacitor is proportional to the total capacitance. A larger total capacitance results in a higher charge.
P
is two capacitors in series, with a total capacitance of
C/2
​
.
Q
is two capacitors in parallel, with a total capacitance of 2
C
.
R
is a single capacitor, with a total capacitance of
C
.
Total capacitance order:
C/2
​
<
C
<2
C
, so charge order: P < R < Q.

\hrule
\vspace{1em}


\noindent
\textbf{Q1008.} A bicycle of mass m
traveling at speed v
encounters a slope inclined at an angle
[IMAGE:0]
and coasts up it until it stops. If the coefficient of friction is
[IMAGE:1]
, what is the distance traveled up the slope?



\textbf{A.} [IMAGE:0] \\
\textbf{B.} [IMAGE:1] \\
\textbf{C.} [IMAGE:2] \\
\textbf{D.} [IMAGE:3] \\

\textbf{Answer:} A \\
\textbf{Explanation:} Deceleration is
[IMAGE:0]

\hrule
\vspace{1em}


\noindent
\textbf{Q1009.} Three identical resistors are used to form three different circuits, labeled A, B, and C. Each circuit is connected, in turn, across the same battery, which has negligible internal resistance. The total power developed in each circuit is measured. What is the correct order of the circuits when arranged in order of increasing power?
[IMAGE:0]



\textbf{A.} A < B < C \\
\textbf{B.} A < C < B \\
\textbf{C.} B < A < C \\
\textbf{D.} B < C < A \\

\textbf{Answer:} B \\
\textbf{Explanation:} Explanation:
Power is calculated using
[IMAGE:0]
When voltage is constant, the circuit with the highest resistance has the lowest power.
A
is two resistors in series, with a total resistance of 2
R
.
B
is two resistors in parallel, with a total resistance of
R/2
​
.
C
is a single resistor, with a total resistance of
R
.
[IMAGE:1]

\hrule
\vspace{1em}


\noindent
\textbf{Q1010.} A submersible starts moving from rest in water. It has a constant propelling force, and water resistance increases with its speed. Eventually, it reaches a constant speed. The graphs below show the variation with time of three quantities (E, F, G):
[IMAGE:0]



\textbf{A.} [IMAGE:0] \\
\textbf{B.} [IMAGE:1] \\
\textbf{C.} [IMAGE:2] \\
\textbf{D.} [IMAGE:3] \\

\textbf{Answer:} C \\
\textbf{Explanation:} Analysis:
Weight (G) is W=mg, with mass m and g constant. Resultant force (E): Initially Fnet
​
=Fpropel
​
−
f, as water resistance f grows with speed, Fnet
​
decreases to zero at constant speed. Velocity (F) increases from rest until terminal speed is reached.

\hrule
\vspace{1em}


\noindent
\textbf{Q1011.} A medical sample contains 1000 atoms of the radioactive isotope technetium-99m (Tc-99m), which has a half-life of 6 hours. How many atoms of Tc-99m remain in the sample after 18 hours?



\textbf{A.} 125 \\
\textbf{B.} 250 \\
\textbf{C.} 375 \\
\textbf{D.} 500 \\

\textbf{Answer:} A \\
\textbf{Explanation:} After 18 hours, the half-life of Tc-99m is 6 hours, so 3 half-lives have passed.
The number of remaining atoms halves with each half-life.
Starting with 1000 atoms, after 1 half-life (6 hours) there are 500 atoms remaining, after 2 half-lives (12 hours) there are 250 atoms remaining, and after 3 half-lives (18 hours) there are 125 atoms remaining.
Thus, after 18 hours, 125 atoms of Tc-99m remain in the sample.

\hrule
\vspace{1em}


\noindent
\textbf{Q1012.} A skateboarder of mass
m
starts from rest and reaches a speed
v
in time
t
by applying a constant force
F
. If friction is constant as
f
, what is the distance covered?



\textbf{A.} [IMAGE:0] \\
\textbf{B.} [IMAGE:1] \\
\textbf{C.} [IMAGE:2] \\
\textbf{D.} [IMAGE:3] \\

\textbf{Answer:} E \\
\textbf{Explanation:} Acceleration is
[IMAGE:0]

\hrule
\vspace{1em}


\noindent
\textbf{Q1013.} A rocket is launched vertically from the ground. During the launch, the rocket engine provides a constant thrust, while the rocket's mass decreases due to fuel combustion. The rocket eventually reaches a terminal velocity. The graphs below show the variation with time of three quantities (X, Y, and Z):
[IMAGE:0]



\textbf{A.} X:Velocity  Y:Kinetic Energy  Z:Weight \\
\textbf{B.} X:Thrust  Y:Gravitational Potential Energy  Z:Kinetic Energy \\
\textbf{C.} X:Acceleration  Y:Velocity  Z:Gravitational Potential Energy \\
\textbf{D.} X:Kinetic Energy  Y:Gravitational Potential Energy  Z:Thrust \\

\textbf{Answer:} F \\
\textbf{Explanation:} \cdot 
X is Mass: The rocket's mass decreases over time due to fuel combustion, approaching a constant value once the fuel is depleted, matching the X graph.
\cdot 
Y is Acceleration: According to Newton's second law F=ma, with constant thrust F and decreasing mass m, acceleration a increases and approaches a constant value when the mass stabilizes, matching the Y graph.
\cdot 
Z is Kinetic Energy: Kinetic energy increases with the square of velocity. As velocity approaches terminal velocity, the rate of increase in kinetic energy slows down, matching the Z graph.

\hrule
\vspace{1em}


\noindent
\textbf{Q1014.} A sample contains 1000 atoms of a radioactive isotope C, which has a half-life of 5 years. How many atoms of isotope C remain in the sample after 10 years?



\textbf{A.} 125 \\
\textbf{B.} 250 \\
\textbf{C.} 500 \\
\textbf{D.} 750 \\

\textbf{Answer:} B \\
\textbf{Explanation:} After 10 years, the half-life of isotope C is 5 years, so 2 half-lives have passed.
The number of remaining atoms halves with each half-life.
Starting with 1000 atoms, after 1 half-life (5 years) there are 500 atoms remaining, and after another half-life (another 5 years) there are 250 atoms remaining.
Thus, after 10 years, 250 atoms of isotope C remain in the sample.

\hrule
\vspace{1em}


\noindent
\textbf{Q1015.} A sample initially contains equal numbers of atoms of a radioactive isotope A and a stable isotope B. Isotope A has a half-life of 4 years and decays in a single stage to the stable isotope B. What is the ratio of the number of atoms of A to the number of atoms of B in the sample after 12 years?



\textbf{A.} The sample contains only isotope B. \\
\textbf{B.} 1:3 \\
\textbf{C.} 1:2 \\
\textbf{D.} 1:15 \\

\textbf{Answer:} D \\
\textbf{Explanation:} After 12 years; X becomes 1/8 of before; Y therefore 15/8

\hrule
\vspace{1em}


\noindent
\textbf{Q1016.} A beam of light enters the center of a semi-circular glass block from the air. The path of the light in the glass is as follows: it first enters the glass from air, reflects off the coated surface, and finally exits back into the air. The refractive index of the glass is 1.5, and the angle of incidence is 45
\circ 
. What is the angle of refraction when the light exits the glass back into the air?
[IMAGE:0]
[IMAGE:1]



\textbf{A.} 30° \\
\textbf{B.} 45° \\
\textbf{C.} 60° \\
\textbf{D.} 90° \\

\textbf{Answer:} B \\
\textbf{Explanation:} Light entering the glass from air:
[IMAGE:0]
[IMAGE:1]
Reflection at the coated surface:
The angle of reflection equals the angle of incidence, so the reflected ray in the glass remains at 28\circ .
Light exiting the glass back into air:
[IMAGE:2]
[IMAGE:3]

\hrule
\vspace{1em}


\noindent
\textbf{Q1017.} A motorbike starts from rest on a horizontal road. It is driven by a constant driving force, and air resistance increases with speed. The graphs below show the variation with time of three quantities (
X,Y,Z
) for the motorbike:
[IMAGE:0]



\textbf{A.} X:acceleration  Y:air resistance  Z:kinetic energy \\
\textbf{B.} X:resultant force  Y:velocity  Z:mass \\
\textbf{C.} X:velocity  Y:driving force  Z:weight \\
\textbf{D.} X:acceleration  Y:velocity  Z:weight \\

\textbf{Answer:} D \\
\textbf{Explanation:} The weight of the motorbike remains constant (mass and gravitational acceleration don’t change, so
Z
is weight). At the start, acceleration (
X
) is maximum, decreases as air resistance increases, and becomes zero at terminal velocity. Velocity (
Y
) increases and approaches terminal velocity.

\hrule
\vspace{1em}


\noindent
\textbf{Q1018.} A train of mass
3m
moving at speed
v/2
applies its brakes and stops in time
3t
. If the braking force is
F/2
, what distance does the train travel before stopping?



\textbf{A.} [IMAGE:0] \\
\textbf{B.} [IMAGE:1] \\
\textbf{C.} [IMAGE:2] \\
\textbf{D.} [IMAGE:3] \\

\textbf{Answer:} B \\
\textbf{Explanation:} Deceleration is
[IMAGE:0]

\hrule
\vspace{1em}


\noindent
\textbf{Q1019.} Light travels from glass (n1=1.5) into air (n2=1.0). When the angle of incidence is 42°, what phenomenon occurs? If refraction happens, find the sine of the refracted angle.
[IMAGE:0]
[IMAGE:1]



\textbf{A.} Total internal reflection occurs \\
\textbf{B.} sinθ2=0.67 \\
\textbf{C.} sinθ2=0.89 \\
\textbf{D.} sinθ2=1.0 \\

\textbf{Answer:} A \\
\textbf{Explanation:} [IMAGE:0]
The incident angle is 42°, which exceeds the critical angle (42° > 41.8°). Therefore, total internal reflection occurs, and no refraction is observed.

\hrule
\vspace{1em}


\noindent
\textbf{Q1020.} A beam of light passes from a medium into a vacuum with an angle of incidence of 30
\circ 
. The refractive index of the medium is 3
​
. What is the angle of refraction in the vacuum?
[IMAGE:0]



\textbf{A.} 30° \\
\textbf{B.} 60° \\
\textbf{C.} 90° \\
\textbf{D.} 120° \\

\textbf{Answer:} B \\
\textbf{Explanation:} According to Snell's Law, the relationship between the angle of incidence and the angle of refraction when light passes from a medium into a vacuum is:

\hrule
\vspace{1em}


\noindent
\textbf{Q1021.} A sled of mass
m
slides down a frictionless incline which has angle
[IMAGE:0]
with the ground and reaches the bottom with speed
v
. If the vertical height of the incline is
h
, and the sled started from rest, what is the length of the incline?



\textbf{A.} [IMAGE:0] \\
\textbf{B.} [IMAGE:1] \\
\textbf{C.} [IMAGE:2] \\
\textbf{D.} [IMAGE:3] \\

\textbf{Answer:} D \\
\textbf{Explanation:} Using energy conservation:
[IMAGE:0]
; solve for
[IMAGE:1]
and relate to incline length
L
via
[IMAGE:2]

\hrule
\vspace{1em}


\noindent
\textbf{Q1022.} An object starts falling freely from rest and experiences air resistance proportional to its velocity. It eventually reaches terminal velocity. Which line in the table correctly identifies the quantities X, Y, and Z shown in the graphs below?
[IMAGE:0]



\textbf{A.} X:acceleration  Y:velocity  Z:kinetic energy \\
\textbf{B.} X:acceleration  Y:air resistance  Z:weight \\
\textbf{C.} X:kinetic energy  Y:velocity  Z:resultant force \\
\textbf{D.} X:resultant force  Y:velocity  Z:weight \\

\textbf{Answer:} D \\
\textbf{Explanation:} When the object falls, the initial resultant force is equal to its weight (downward). As velocity increases, air resistance grows until it balances the weight, causing the resultant force to decrease to zero (at terminal velocity). Thus:
X (Resultant force): Decreases from maximum to zero;
Y (Velocity): Increases from zero and asymptotes to terminal velocity;
Z (Weight): Remains constant because mass is unchanged.
Option B incorrectly assigns air resistance to Y instead of velocity. Option D correctly matches the behavior of each quantity.

\hrule
\vspace{1em}


\noindent
\textbf{Q1023.} Light travels sequentially from air (n1=1.0) into glass (n2=1.5), and then into water (n3=1.33). If the initial angle of incidence is 60°, what is the sine of the final refracted angle sin
θ
3
​
in water?
[IMAGE:0]



\textbf{A.} 0.65 \\
\textbf{B.} 0.87 \\
\textbf{C.} 0.50 \\
\textbf{D.} 0.71 \\

\textbf{Answer:} A \\
\textbf{Explanation:} First Refraction (Air \to  Glass):
[IMAGE:0]
[IMAGE:1]
Second Refraction (Glass \to  Water):
[IMAGE:2]
[IMAGE:3]

\hrule
\vspace{1em}


\noindent
\textbf{Q1024.} Light travels from medium A (refractive index 1.2) to medium B (refractive index 1.8). If the angle of incidence is 60°, what is the value of sin
θ
2
​
(sine of the refracted angle)?



\textbf{A.} [IMAGE:0] \\
\textbf{B.} [IMAGE:1] \\
\textbf{C.} [IMAGE:2] \\
\textbf{D.} [IMAGE:3] \\

\textbf{Answer:} A \\
\textbf{Explanation:} [IMAGE:0]
[IMAGE:1]

\hrule
\vspace{1em}


\noindent
\textbf{Q1025.} An object is released from rest on an inclined plane and begins to slide down. During the sliding process, the frictional force acting on the object is constant. The graphs below show the variation with time of three quantities (X, Y, and Z)
(
Take into account air resistance, etc.)
[IMAGE:0]
Which line in the table correctly identifies the quantities X, Y, and Z?



\textbf{A.} X:Velocity  Y:Kinetic Energy  Z:Gravitational Potential Energy \\
\textbf{B.} X:Resultant Force  Y:Gravitational Potential Energy  Z:Kinetic Energy \\
\textbf{C.} X:Acceleration  Y:Velocity  Z:Gravitational Potential Energy \\
\textbf{D.} X:Kinetic Energy  Y:Gravitational Potential Energy  Z:Resultant Force \\

\textbf{Answer:} F \\
\textbf{Explanation:} X is Acceleration: The acceleration decreases over time as the object slides down the incline. Although friction is constant, other resistive forces (like air resistance) may increase with velocity, causing acceleration to approach zero, matching the X graph.
Y is Kinetic Energy: Kinetic energy increases with the square of velocity. As velocity approaches a terminal value (when resultant force is zero), kinetic energy approaches a constant value, matching the Y graph.
Z is Gravitational Potential Energy: As the object slides down, its height decreases, causing gravitational potential energy to decrease and approach zero (assuming the ground as the zero potential energy level), matching the Z graph.

\hrule
\vspace{1em}


\noindent
\textbf{Q1026.} A wooden block has a mass of 8g and a density of 0.8g/cm3. When it floats on water with a density of 1.0g/cm3, what percentage of the block's volume is submerged in the water?



\textbf{A.} 60% \\
\textbf{B.} 70% \\
\textbf{C.} 80% \\
\textbf{D.} 90% \\

\textbf{Answer:} C \\
\textbf{Explanation:} When an object floats, the buoyant force equals the weight of the object. The buoyant force is given by:
[IMAGE:0]
The percentage of the block submerged is:
[IMAGE:1]

\hrule
\vspace{1em}


\noindent
\textbf{Q1027.} An object is placed 30 cm in front of a convex lens with a focal length of 15 cm. According to the lens formula, where is the image located? What are the characteristics of the image?
[IMAGE:0]



\textbf{A.} Image distance 30 cm, real image, same size as the object \\
\textbf{B.} Image distance 15 cm, virtual image, magnified \\
\textbf{C.} Image distance 30 cm, real image, reduced \\
\textbf{D.} Image distance 15 cm, virtual image, reduced \\

\textbf{Answer:} A \\
\textbf{Explanation:} According to the lens formula for a convex lens:
[IMAGE:0]
where
f
is the focal length,
u
is the object distance, and
v
is the image distance. Given
f
=15cm and
u
=30cm, substituting into the formula gives:
[IMAGE:1]
[IMAGE:2]

\hrule
\vspace{1em}


\noindent
\textbf{Q1028.} A cubic wooden block is floating on water, experiencing a buoyant force of 10 Newtons. If the length of each edge of the wooden block is doubled while keeping its density constant, what is the new buoyant force acting on the block?



\textbf{A.} 10N \\
\textbf{B.} 20N \\
\textbf{C.} 40N \\
\textbf{D.} 80N \\

\textbf{Answer:} D \\
\textbf{Explanation:} The formula for buoyant force is: $F_b$=ρgV_"displaced "  where Fb is the buoyant force, ρ is the density of the fluid, g is the acceleration due to gravity, and Vdisplaced is the volume of displaced fluid. The original buoyant force is 10 Newtons, so the new buoyant force is: 8×10=80N

\hrule
\vspace{1em}


\noindent
\textbf{Q1029.} A light beam travels from a transparent medium into air (refractive index 1.0). When the angle of incidence is 30°, the angle of refraction is 60°. Determine the refractive index of the medium and what phenomenon occurs when the angle of incidence increases to 60°?
[IMAGE:0]



\textbf{A.} Refractive index
[IMAGE:0]
, angle of refraction 90° \\
\textbf{B.} Refractive index
[IMAGE:1]
, total internal reflection occurs \\
\textbf{C.} Refractive index 2, angle of refraction 30° \\
\textbf{D.} Refractive index
[IMAGE:2]
, no total internal reflection \\

\textbf{Answer:} B \\
\textbf{Explanation:} [IMAGE:0]
[IMAGE:1]
When the incident angle is 60°, it exceeds the critical angle (60° > 35.26°), so total internal reflection occurs.

\hrule
\vspace{1em}


\noindent
\textbf{Q1030.} A parachute is released from rest at a high altitude and begins to descend. During the descent, the air resistance acting on the parachute is proportional to its velocity. The graphs below show the variation with time of three quantities (X, Y, and Z):
[IMAGE:0]
Which line in the table correctly identifies the quantities X, Y, and Z?



\textbf{A.} X:Velocity  Y:Kinetic Energy  Z:Weight \\
\textbf{B.} X:Resultant Force  Y:Gravitational Potential Energy  Z:Drag Force \\
\textbf{C.} X:Acceleration  Y:Velocity  Z:Drag Force \\
\textbf{D.} X:Kinetic Energy  Y:Gravitational Potential Energy  Z:Weight \\

\textbf{Answer:} F \\
\textbf{Explanation:} \cdot 
Z is Weight: Weight is constant and independent of time, matching the horizontal Z graph.
\cdot 
Y is Velocity: Velocity increases during descent but approaches terminal velocity when drag equals weight, matching the Y graph's asymptotic behavior.
\cdot 
X is Resultant Force: The resultant force (weight minus drag) decreases as velocity increases and approaches zero when drag equals weight, matching the decaying X graph.

\hrule
\vspace{1em}


\noindent
\textbf{Q1031.} A square metal block is placed flat on a table, exerting a pressure of 4 Pascals (Pa) on the table. If the length of each side of the metal block is doubled while keeping its thickness constant, what is the new pressure exerted by the metal block on the table?



\textbf{A.} 12g \\
\textbf{B.} 24g \\
\textbf{C.} 36g \\
\textbf{D.} 48g \\

\textbf{Answer:} C \\
\textbf{Explanation:} The density of an object is given by: ρ=m/V Since the density remains constant, the new mass m′ can be calculated using: m^'=ρ\cdot V^'=2×18=36g

\hrule
\vspace{1em}


\noindent
\textbf{Q1032.} A light ray from a point source is reflected by two plane mirrors. The angle between the first mirror and the second mirror is 60
\circ 
, and the angle between the incident ray and the first mirror is 30
\circ 
. What is the angle between the emergent ray after two reflections and the original incident ray?
[IMAGE:0]



\textbf{A.} 30° \\
\textbf{B.} 60° \\
\textbf{C.} 90° \\
\textbf{D.} 120° \\

\textbf{Answer:} D \\
\textbf{Explanation:} According to the law of reflection, the angle of incidence is equal to the angle of reflection. After the first reflection from the first mirror, the reflected ray makes an angle of 30
\circ 
with the first mirror.
Next, the reflected ray strikes the second mirror. Since the angle between the two mirrors is 60
\circ 
, the angle between the incident ray and the second mirror can be calculated as 30
\circ 
(because the angle between the reflected ray from the first mirror and the second mirror is equal to the angle between the two mirrors minus the angle between the reflected ray and the first mirror).
According to the law of reflection, the reflected ray from the second mirror also makes an angle of 30
\circ 
with the second mirror. After two reflections, the angle between the emergent ray and the original incident ray can be calculated as 120
\circ 
(since the angle between the two mirrors is 60
\circ 
, and each reflection changes the direction of the ray, resulting in a total angle of 60
\circ 
×2=120
\circ 
).

\hrule
\vspace{1em}


\noindent
\textbf{Q1033.} A square metal block is placed flat on a table, exerting a pressure of 4 Pascals (Pa) on the table. If the length of each side of the metal block is doubled while keeping its thickness constant, what is the new pressure exerted by the metal block on the table?



\textbf{A.} 1Pa \\
\textbf{B.} 2Pa \\
\textbf{C.} 4Pa \\
\textbf{D.} 8Pa \\

\textbf{Answer:} A \\
\textbf{Explanation:} The formula for pressure is: P=F/A    where P is pressure, F is the force (here, the weight of the metal block), and A is the area over which the force is applied. Since the thickness of the metal block remains constant, its weight F also remains unchanged. Therefore, the new pressure P′ is: P^'=F/A^' =F/4A=1/4 P The original pressure is 4 Pascals, so the new pressure is: 1/4×4=1Pa

\hrule
\vspace{1em}


\noindent
\textbf{Q1034.} An object starts from rest and is accelerated by a constant thrust force in a fluid, eventually reaching terminal velocity. The graphs below show the variation with time of three quantities (X, Y, Z):
[IMAGE:0]
Which line in the table correctly identifies X, Y, and Z?



\textbf{A.} X:acceleration  Y:fluidresistance  Z:kineticenergy \\
\textbf{B.} X:acceleration  Y:massofobject  Z:weight \\
\textbf{C.} X:kineticenergy  Y:velocity  Z:potentialenergy \\
\textbf{D.} X:resultant force  Y:fluidresistance  Z:momentum \\

\textbf{Answer:} E \\
\textbf{Explanation:} The object experiences a constant thrust, but fluid resistance increases with velocity until it balances the thrust. The resultant force (X) decreases to zero (curve declines to zero). Velocity (Y) increases asymptotically toward terminal velocity (curve rises and stabilizes). The object’s mass remains constant, so its weight (Z) is unchanged (horizontal line). Option E is correct. Other options fail because: A’s Z (kinetic energy) should keep increasing; B’s Y (mass) and Z (weight) are constants, but X (acceleration) should decrease; C and D have mismatched physical quantities; F’s X (momentum) should increase, not decrease.

\hrule
\vspace{1em}


\noindent
\textbf{Q1035.} The diagram shows light passing from air (refractive index 1.0) into a transparent medium. When the angle of incidence is 60°, the angle of refraction is 30°. If the angle of incidence changes to 30°, what is the value of sin
θ
2?
[IMAGE:0]



\textbf{A.} [IMAGE:0] \\
\textbf{B.} [IMAGE:1] \\
\textbf{C.} [IMAGE:2] \\
\textbf{D.} [IMAGE:3] \\

\textbf{Answer:} A \\
\textbf{Explanation:} [IMAGE:0]
[IMAGE:1]

\hrule
\vspace{1em}


\noindent
\textbf{Q1036.} A metal cube has a mass of 5 grams. If the length of each edge of the cube is doubled, what is the new mass of the cube?



\textbf{A.} 5g \\
\textbf{B.} 10g \\
\textbf{C.} 20g \\
\textbf{D.} 40g \\

\textbf{Answer:} D \\
\textbf{Explanation:} The mass of an object is related to its volume and density by the formula: m=ρV
For a cube, the volume is given by: V=$a^3$ The original mass is 5 grams, so the new mass is: 8×5=40g

\hrule
\vspace{1em}


\noindent
\textbf{Q1037.} An ice cream cone has a volume of 16 cm³. If both the height and the radius of the cone are increased to twice their original sizes, what is the new volume of the ice cream cone?



\textbf{A.} 32 \\
\textbf{B.} 64 \\
\textbf{C.} 96 \\
\textbf{D.} 128 \\

\textbf{Answer:} D \\
\textbf{Explanation:} The volume of a cone is given by:
[IMAGE:0]
where
r
is the radius of the base and
h
is the height.
When both the height
h
and the radius
r
are doubled, the new volume
V
′ becomes:
[IMAGE:1]
Thus, the volume becomes 8 times the original.The original volume is 16 cm³, so the new volume is 16×8=128 cm³.

\hrule
\vspace{1em}


\noindent
\textbf{Q1038.} An object is dropped from rest into a fluid and is subject to gravity, buoyancy, and drag. The object eventually reaches a terminal velocity. The graphs below show the variation with time of three quantities (X, Y, and Z) for the object:
[IMAGE:0]



\textbf{A.} X:
Resultant Force
Y:
Buoyant Force
Z:
Kinetic Energy \\
\textbf{B.} X:
Resultant Force
Y:
Velocity
Z:
Weight of Object \\
\textbf{C.} X:
Buoyant Force
Y:
Velocity
Z:
Kinetic Energy \\
\textbf{D.} X:
Buoyant Force
Y:
Resultant Force
Z:
Weight of Object \\

\textbf{Answer:} B \\
\textbf{Explanation:} When an object falls through a fluid, it experiences:
Weight (downward, constant).
Buoyant Force (upward, constant if fully submerged).
Drag Force (upward, increases with velocity).
The Resultant Force is calculated as Weight−(Buoyant Force+Drag Force). As velocity increases, drag increases, causing the resultant force to decrease until it reaches zero at terminal velocity.
X (Resultant Force): Decreases over time, matching the X graph.
Y (Velocity): Starts at zero and asymptotically approaches terminal velocity, matching the Y graph.
Z (Weight of Object): Remains constant, matching the horizontal Z graph.

\hrule
\vspace{1em}


\noindent
\textbf{Q1039.} The ray diagram shows light passing from air into a medium. Two angles,
x
and
y
, are shown on the diagram. When
x
=30
\circ 
,
y
=45
\circ 
. When
x
=45
\circ 
, what is the value of sin
y
?
[IMAGE:0]



\textbf{A.} [IMAGE:0] \\
\textbf{B.} [IMAGE:1] \\
\textbf{C.} [IMAGE:2] \\
\textbf{D.} [IMAGE:3] \\

\textbf{Answer:} B \\
\textbf{Explanation:} According to Snell's Law, the relationship between the refractive index
n
and the angles of incidence and refraction is:
[IMAGE:0]
When
x
=30
\circ 
and
y
=45
\circ 
, substituting into the formula gives:
[IMAGE:1]
[IMAGE:2]

\hrule
\vspace{1em}


\noindent
\textbf{Q1040.} A decorative truncated cone-shaped lampshade has a volume of 144 cm³. If the height and the radius of the top base of the lampshade are both increased to twice their original sizes, while keeping the radius of the bottom base unchanged, what is the new volume of the lampshade?



\textbf{A.} 288 \\
\textbf{B.} 432 \\
\textbf{C.} 576 \\
\textbf{D.} 720 \\

\textbf{Answer:} E \\
\textbf{Explanation:} The volume of a truncated cone (frustum) is given by: V=1/3 πh($R^2$+Rr+$r^2$ )
where h is the height, R is the radius of the bottom base, and r is the radius of the top base.When the height h and the radius r of the top base are both doubled, the new volume V′ becomes: V^'=1/3 π(2h)($R^2$+R(2r)+(2r)^2 )=2/3 πh($R^2$+2Rr+$4r^2$ )
Compared to the original volume, the new volume is (2)3=8 times the original volume (since volume is proportional to the cube of the height and radius).
The original volume is 144 cm³, so the new volume is 144×8=1152 cm³.

\hrule
\vspace{1em}


\noindent
\textbf{Q1041.} In space, two spacecraft, A and B, are moving towards each other. Spacecraft A has twice the mass of spacecraft B. Spacecraft A is moving to the right at speed v, while spacecraft B is moving to the left at speed v. They collide and stick together. What is the magnitude and direction of their common velocity after the collision?



\textbf{A.} Magnitude 0, direction not applicable \\
\textbf{B.} Magnitude v/3, direction to the left \\
\textbf{C.} Magnitude 2v/3, direction to the left \\
\textbf{D.} Magnitude 4v/3, direction to the left \\

\textbf{Answer:} E \\
\textbf{Explanation:} By the conservation of momentum, the total momentum of the system remains constant in the absence of external forces. Before the collision:
Momentum of spacecraft A: 2
m
×
v
=2
mv
(to the right)
Momentum of spacecraft B:
m
×(
−
v
)=
−
mv
(to the left)
Total momentum of the system: 2
mv
−
mv
=
mv
(to the right)
After the collision, the combined mass of the spacecraft is 2
m
+
m
=3
m
. Let the common velocity after the collision be
u
. Applying the conservation of momentum:
3
m
×
u
=
mv

\hrule
\vspace{1em}


\noindent
\textbf{Q1042.} A car of mass
m
starts from rest and accelerates uniformly to a speed
2v
in a time
2t
on a horizontal road. If the average horizontal force applied is
2F
, what is the distance covered by the car during this acceleration?



\textbf{A.} [IMAGE:0] \\
\textbf{B.} [IMAGE:1] \\
\textbf{C.} [IMAGE:2] \\
\textbf{D.} [IMAGE:3] \\

\textbf{Answer:} F \\
\textbf{Explanation:} The acceleration is
[IMAGE:0]
; by formula:
[IMAGE:1]

\hrule
\vspace{1em}


\noindent
\textbf{Q1043.} A metal sphere has a volume of 8 cm³. If the radius of the metal sphere is increased to twice its original size, what is the new volume of the metal sphere?



\textbf{A.} 16 \\
\textbf{B.} 32 \\
\textbf{C.} 48 \\
\textbf{D.} 64 \\

\textbf{Answer:} D \\
\textbf{Explanation:} The radius is increased to twice the original; the volume is proportional to the cube of the radius, so the new volume is (2)3=8 times the original.
Original volume is 8 cm³, so the new volume is 8×8=64 cm³.

\hrule
\vspace{1em}


\noindent
\textbf{Q1044.} Two skaters, A (mass 60
kg) and B (mass 40
kg), are initially stationary on frictionless ice. Skater A pushes Skater B, causing B to move left at 6
m/s6m/s. Determine the magnitude and direction of Skater A’s velocity.
[IMAGE:0]



\textbf{A.} Magnitude (m/s): 4; Direction: to the left \\
\textbf{B.} Magnitude (m/s): 4; Direction: to the right \\
\textbf{C.} Magnitude (m/s): 6; Direction: to the right \\
\textbf{D.} Magnitude (m/s): 3; Direction: to the left \\

\textbf{Answer:} B \\
\textbf{Explanation:} By the law of conservation of momentum, the total momentum of the system remains zero before and after the push.
Let Skater A’s velocity be vA. Taking right as positive and left as negative:
[IMAGE:0]
Simplifying:
[IMAGE:1]

\hrule
\vspace{1em}


\noindent
\textbf{Q1045.} A model of a planet has a volume of 1728 cm³. If the radius of the model is reduced to one-fourth of its original size, what is the new volume of the model?



\textbf{A.} 54 \\
\textbf{B.} 27 \\
\textbf{C.} 81 \\
\textbf{D.} 108 \\

\textbf{Answer:} A \\
\textbf{Explanation:} The radius is reduced to one-fourth of the original; the volume is proportional to the cube of the radius, so the new volume is (41)3=641 of the original.
Original volume is 1728 cm³, so the new volume is 1728÷64=27 cm³.

\hrule
\vspace{1em}


\noindent
\textbf{Q1046.} A balloon has a volume of 216 cm³. If the radius of the balloon is reduced to one-third of its original size, what is the new volume of the balloon?



\textbf{A.} 8 \\
\textbf{B.} 24 \\
\textbf{C.} 72 \\
\textbf{D.} 108 \\

\textbf{Answer:} A \\
\textbf{Explanation:} The radius is reduced to one-third of the original; the volume is proportional to the cube of the radius, so the new volume is (31)3=271 of the original.
Original volume is 216 cm³, the so new volume is 216÷27=8 cm³.

\hrule
\vspace{1em}


\noindent
\textbf{Q1047.} On a smooth air track, there are two gliders, A and B. Glider A has three times the mass of glider B. Glider A is moving to the right at speed v, while glider B is stationary. They undergo a perfectly inelastic collision (stick together after collision). What is the magnitude and direction of their common velocity after the collision?
[IMAGE:0]



\textbf{A.} Magnitude 0, direction not applicable \\
\textbf{B.} Magnitude 1/4v, direction to the left \\
\textbf{C.} Magnitude 3/4v, direction to the left \\
\textbf{D.} Magnitude 1/2v, direction to the left \\

\textbf{Answer:} F \\
\textbf{Explanation:} By the conservation of momentum, the total momentum of the system remains constant in the absence of external forces. Before the collision:
Momentum of glider A: 3
m
×
v
=3
mv
(to the right)
Momentum of glider B:
m
×0=0
Total momentum of the system: 3
mv
(to the right)
After the collision, the combined mass of the gliders is 3
m
+
m
=4
m
. Let the common velocity after the collision be
u
. Applying the conservation of momentum:
4
m
×
u
=3
mv
Solving for
u
:
u
=3
mv/4m
​
=3/4
​
v
, direction to the right.

\hrule
\vspace{1em}


\noindent
\textbf{Q1048.} A weather balloon rises at terminal velocity while carrying instruments. Later, the instruments are released, reducing the balloon’s mass to 1/3 of its original and expanding its cross-sectional area to 2 times. Which description shows how the downward air resistance (drag) force varies with time after releasing the instruments? (Assume gravitational acceleration and air density remain constant.)



\textbf{A.} Resistance remains constant initially, then gradually decreases after release \\
\textbf{B.} Resistance remains constant initially, then sharply rises and stabilizes at a new value after release \\
\textbf{C.} Resistance increases linearly with time \\
\textbf{D.} Resistance remains constant initially, then sharply drops and stabilizes at a new value after release \\

\textbf{Answer:} D \\
\textbf{Explanation:} Initially, the downward air resistance balances the net upward force (buoyancy minus weight) at terminal velocity. After releasing the instruments, the mass reduces to m/3
and the cross-sectional area increases to 2A. Using the terminal velocity formula
[IMAGE:0]
the reduced mass significantly decreases the net downward force (potentially dominated by buoyancy). The downward resistance must now balance a smaller net force. Additionally, the increased A and reduced vt cause the drag force
[IMAGE:1]
to drop abruptly to a new equilibrium. The correct graph is option D.

\hrule
\vspace{1em}


\noindent
\textbf{Q1049.} A lorry of mass m, and travelling initially at speed 2v along a horizontal road, is brought to rest by an average horizontal braking force F in time t. Ignoring any other resistive forces, what distance is travelled by the lorry during this time? (gravitational field strength = 10 N kg–1)



\textbf{A.} [IMAGE:0] \\
\textbf{B.} [IMAGE:1] \\
\textbf{C.} [IMAGE:2] \\
\textbf{D.} [IMAGE:3] \\

\textbf{Answer:} E \\
\textbf{Explanation:} The deceleration is
[IMAGE:0]
; by formula:
[IMAGE:1]

\hrule
\vspace{1em}


\noindent
\textbf{Q1050.} A stationary ice boat of mass 200
kg is on frictionless ice. Two people are on board: Person A (mass 50
kg) and Person B (mass 50
kg). Person A jumps off the boat to the right with a velocity of 5
m/s relative to the boat. Determine the magnitude and direction of the boat and Person B’s common velocity after the jump.
[IMAGE:0]



\textbf{A.} Magnitude (m/s): 1; Direction: to the left \\
\textbf{B.} Magnitude (m/s): 2; Direction: to the right \\
\textbf{C.} Magnitude (m/s): 0.5; Direction: to the left \\
\textbf{D.} Magnitude (m/s): 5; Direction: to the right \\

\textbf{Answer:} A \\
\textbf{Explanation:} By conservation of momentum, the total momentum before and after the jump remains zero. Let the velocity of the boat and Person B be v (left as negative, right as positive).
Total mass of boat + Person B: 200+50=250
kg.
Momentum equation:50×5+250×(
−
v
)=0
\implies 
250
−
250
v
=0
\implies 
v
=1m/s

\hrule
\vspace{1em}


\noindent
\textbf{Q1051.} A skater of mass 60
kg is initially stationary on frictionless ice. He throws a ball of mass 5
kg horizontally to the right with a velocity of 12
m/s. Determine the magnitude and direction of the skater’s velocity after throwing the ball.
[IMAGE:0]



\textbf{A.} Magnitude (m/s): 1; Direction: to the left \\
\textbf{B.} Magnitude (m/s): 1; Direction: to the right \\
\textbf{C.} Magnitude (m/s): 12; Direction: to the left \\
\textbf{D.} Magnitude (m/s): 5; Direction: to the right \\

\textbf{Answer:} A \\
\textbf{Explanation:} By the law of conservation of momentum, the total momentum of the system remains zero before and after the throw. Let the skater’s velocity be v
v
. Taking right as positive and left as negative:
[IMAGE:0]

\hrule
\vspace{1em}


\noindent
\textbf{Q1052.} A skier slides down a slope at a constant speed, gradually reaching a stable velocity. When the skier enters smoother snow, the drag force suddenly decreases, and the skier reaches a new stable velocity. Which graph shows how the drag force varies with time?



\textbf{A.} [IMAGE:0] \\
\textbf{B.} [IMAGE:1] \\
\textbf{C.} [IMAGE:2] \\
\textbf{D.} [IMAGE:3] \\

\textbf{Answer:} B \\
\textbf{Explanation:} The variation of drag force is as follows:
As the skier begins to slide down the slope, velocity increases gradually, and so does the drag force (since drag is proportional to the square of velocity).
When drag balances the component of weight along the slope, the skier reaches the first stable velocity, and drag stabilizes.
Upon entering smoother snow, the friction coefficient decreases suddenly, causing a sharp drop in drag force.
Drag force balances the component of weight along the slope again, and the skier reaches a new stable velocity with stabilized drag.

\hrule
\vspace{1em}


\noindent
\textbf{Q1053.} A cart A slides down a smooth incline and collides with a stationary cart B at the bottom. The mass of cart A is twice that of cart B. After the collision, the two carts stick together and continue moving down the incline. The speed of cart A before the collision is v. What is the magnitude and direction of their common velocity after the collision?
[IMAGE:0]



\textbf{A.} Magnitude 0, direction not applicable \\
\textbf{B.} Magnitude v/3, direction up the incline \\
\textbf{C.} Magnitude 2v/3, direction up the incline \\
\textbf{D.} Magnitude 4v/3, direction up the incline \\

\textbf{Answer:} F \\
\textbf{Explanation:} By the conservation of momentum, the total momentum of the system remains constant in the absence of external forces. Before the collision:
Momentum of cart A: 2
m
×
v
=2
mv
(down the incline)
Momentum of cart B: 0 (since it is stationary)
Total momentum of the system: 2
mv
(down the incline)
After the collision, the combined mass of the carts is 2
m
+
m
=3
m
. Let the common velocity after the collision be
u
. Applying the conservation of momentum:
3
m
×
u
=2
mv
Solving for
u
:
u
=2/3
​
v
, direction down the incline.

\hrule
\vspace{1em}


\noindent
\textbf{Q1054.} In space, two spacecraft, A and B, are moving towards each other. Spacecraft A has twice the mass of spacecraft B. Spacecraft A is moving to the right at speed v, while spacecraft B is moving to the left at speed v. They collide and stick together. What is the magnitude and direction of their common velocity after the collision?



\textbf{A.} Magnitude 0,direction not applicable \\
\textbf{B.} [IMAGE:0] \\
\textbf{C.} [IMAGE:1] \\
\textbf{D.} [IMAGE:2] \\

\textbf{Answer:} E \\
\textbf{Explanation:} By the conservation of momentum, the total momentum of the system remains constant in the absence of external forces. Before the collision:
Momentum of spacecraft A: 2
m
×
v
=2
mv
(to the right)
Momentum of spacecraft B:
m
×(
−
v
)=
−
mv
(to the left)
Total momentum of the system: 2
mv
−
mv
=
mv
(to the right)
After the collision, the combined mass of the spacecraft is 2
m
+
m
=3
m
. Let the common velocity after the collision be
u
. Applying the conservation of momentum:
3
m
×
u
=
mv

\hrule
\vspace{1em}


\noindent
\textbf{Q1055.} A metal sphere falls from a high altitude and reaches terminal velocity. Later, it splits into two identical smaller spheres, each with half the original mass and 0.6 times the original cross-sectional area. Which graph shows how the air resistance (drag) force varies with time before and after splitting? (Assume gravitational acceleration and air density remain constant.)



\textbf{A.} [IMAGE:0] \\
\textbf{B.} [IMAGE:1] \\
\textbf{C.} [IMAGE:2] \\
\textbf{D.} [IMAGE:3] \\

\textbf{Answer:} B \\
\textbf{Explanation:} Initially, the air resistance equals the sphere’s weight (mg), stabilizing at terminal velocity. After splitting, each smaller sphere has 0.5m mass and 0.6A cross-sectional area. Using the terminal velocity formula
[IMAGE:0]
the new terminal velocity becomes:
[IMAGE:1]
The reduced terminal velocity and smaller AA lower the drag force. The air resistance must now balance the new weight (0.5mg), causing an abrupt drop to the new equilibrium. The correct graph is option B.

\hrule
\vspace{1em}


\noindent
\textbf{Q1056.} Two skaters, A (mass 60
kg) and B (mass 40
kg), are initially stationary on frictionless ice. Skater A pushes Skater B, causing B to move left at 6
m/s6m/s. Determine the magnitude and direction of Skater A’s velocity.
[IMAGE:0]



\textbf{A.} 4, to the left \\
\textbf{B.} 4, to the right \\
\textbf{C.} 6, to the right \\
\textbf{D.} 3, to the left \\

\textbf{Answer:} B \\
\textbf{Explanation:} By the law of conservation of momentum, the total momentum of the system remains zero before and after the push.
Let Skater A’s velocity be vA. Taking right as positive and left as negative:
[IMAGE:0]
Simplifying:
[IMAGE:1]

\hrule
\vspace{1em}


\noindent
\textbf{Q1057.} A stationary grenade of mass 5
kg on a smooth horizontal surface explodes into two fragments. One fragment (mass 3
kg) moves to the left with 4
m/s. Determine the magnitude and direction of the velocity of the other fragment.



\textbf{A.} Magnitude (m/s): 6; Direction: to the right \\
\textbf{B.} Magnitude (m/s): 4; Direction: to the left \\
\textbf{C.} Magnitude (m/s): 3; Direction: to the right \\
\textbf{D.} Magnitude (m/s): 2; Direction: to the left \\

\textbf{Answer:} A \\
\textbf{Explanation:} By the law of conservation of momentum, the total momentum before the explosion is zero, and it remains zero after the explosion.
Let the velocity of the other fragment
[IMAGE:0]
Taking left as negative and right as positive:
[IMAGE:1]

\hrule
\vspace{1em}


\noindent
\textbf{Q1058.} A raindrop falls freely from a cloud, gradually reaching terminal velocity. When the raindrop enters a denser layer of cloud, the drag force suddenly increases, and the raindrop reaches a new terminal velocity. Which graph shows how the drag force varies with time?



\textbf{A.} [IMAGE:0] \\
\textbf{B.} [IMAGE:1] \\
\textbf{C.} [IMAGE:2] \\
\textbf{D.} [IMAGE:3] \\

\textbf{Answer:} A \\
\textbf{Explanation:} The variation of drag force is as follows:
As the raindrop begins to fall, velocity increases gradually, and so does the drag force (since drag is proportional to the square of velocity).
When drag balances weight, the raindrop reaches the first terminal velocity, and drag stabilizes.
Upon entering the denser cloud layer, the air density increases suddenly, causing a sharp rise in drag force.
Drag force balances weight again, and the raindrop reaches a new terminal velocity with stabilized drag.

\hrule
\vspace{1em}


\noindent
\textbf{Q1059.} A hailstone falls from a cloud and reaches terminal velocity. Later, it partially melts, halving its mass and reducing its cross-sectional area to 0.5 times the original. Which graph shows how the air resistance (drag) force varies with time before and after melting? (Assume gravitational acceleration and air density remain constant.)



\textbf{A.} [IMAGE:0] \\
\textbf{B.} [IMAGE:1] \\
\textbf{C.} [IMAGE:2] \\
\textbf{D.} [IMAGE:3] \\

\textbf{Answer:} B \\
\textbf{Explanation:} Initially, the air resistance equals the hailstone’s weight (mg), stabilizing at terminal velocity. After melting, the mass halves (0.5m), and the cross-sectional area reduces to 0.5 times (0.5A). Using the terminal velocity formula
[IMAGE:0]
the new terminal velocity remains unchanged:
[IMAGE:1]
However, the air resistance must now balance the new weight (0.5mg). Despite the unchanged vt​, the reduced A directly lowers the drag force to 0.5mg. Thus, the resistance drops abruptly to the new equilibrium, corresponding to option B.

\hrule
\vspace{1em}


\noindent
\textbf{Q1060.} A ball A is moving to the right at speed v and collides with a stationary ball B. The mass of ball A is half that of ball B. After the collision, the two balls stick together and continue moving. What is the magnitude and direction of their common velocity after the collision?



\textbf{A.} Magnitude 0, direction not applicable \\
\textbf{B.} Magnitude 1/3v, direction to the left \\
\textbf{C.} Magnitude 2/3v, direction to the left \\
\textbf{D.} Magnitude 4/3v, direction to the left \\

\textbf{Answer:} E \\
\textbf{Explanation:} By the conservation of momentum, the total momentum of the system remains constant in the absence of external forces. Before the collision:
Momentum of ball A:
[IMAGE:0]
Momentum of ball B: 0 (since it is stationary)
Total momentum of the system: 1/2
​
mv (to the right) After the collision, the combined mass of the balls is
[IMAGE:1]
Let the common velocity after the collision be u. Applying the conservation of momentum:
[IMAGE:2]
Solving for u: u=v/3
​
, direction to the right.

\hrule
\vspace{1em}


\noindent
\textbf{Q1061.} Two small balls A and B, each of mass m
m
, are initially at rest on a smooth horizontal surface. Ball A moves to the right with a velocity of 4
m/s, while ball B remains stationary. They undergo a perfectly inelastic collision and stick together. Determine the magnitude and direction of their common velocity after the collision.



\textbf{A.} Magnitude (m/s): 2; Direction: to the right \\
\textbf{B.} Magnitude (m/s): 1; Direction: to the left \\
\textbf{C.} Magnitude (m/s): 2; Direction: to the left \\
\textbf{D.} Magnitude (m/s): 1; Direction: to the right \\

\textbf{Answer:} A \\
\textbf{Explanation:} By the law of conservation of momentum, the total momentum before the collision equals the total momentum after the collision.
Initial momentum:
[IMAGE:0]
After the collision, the combined mass is 2m. Let the common velocity be v
[IMAGE:1]

\hrule
\vspace{1em}


\noindent
\textbf{Q1062.} A metal sphere falls freely from a height into water, gradually reaching terminal velocity. When the sphere enters a more viscous layer of honey, the drag force suddenly increases, and the sphere reaches a new terminal velocity. Which graph shows how the drag force varies with time?



\textbf{A.} [IMAGE:0] \\
\textbf{B.} [IMAGE:1] \\
\textbf{C.} [IMAGE:2] \\
\textbf{D.} [IMAGE:3] \\

\textbf{Answer:} A \\
\textbf{Explanation:} The variation of drag force is as follows:
As the sphere begins to fall, velocity increases gradually, and so does the drag force (since drag is proportional to the square of velocity).
When drag balances weight, the sphere reaches the first terminal velocity, and drag stabilizes.
Upon entering the honey layer, the viscosity of honey increases suddenly, causing a sharp rise in drag force.
Drag force balances weight again, and the sphere reaches a new terminal velocity with stabilized drag.

\hrule
\vspace{1em}


\noindent
\textbf{Q1063.} Two identical carts, A and B, are moving towards each other on a smooth horizontal track. Cart A has three times the mass of cart B. Cart A is moving to the right at speed v, and cart B is moving to the left at speed 2v. They collide and stick together. What is the magnitude and direction of their common velocity after the collision?



\textbf{A.} Magnitude 0, direction not applicable \\
\textbf{B.} Magnitude 1/4v, direction to the left \\
\textbf{C.} Magnitude 1/2v, direction to the left \\
\textbf{D.} Magnitude 3/4v, direction to the left \\

\textbf{Answer:} E \\
\textbf{Explanation:} By the conservation of momentum, the total momentum of the system remains constant in the absence of external forces. Before the collision:
Momentum of cart A: 3m×v=3mv (to the right)
Momentum of cart B: m×(
−
2v)=
−
2mv (to the left)
Total momentum of the system: 3mv
−
2mv=mv (to the right)
After the collision, the combined mass of the carts is 3m+m=4m. Let the common velocity after the collision be u. Applying the conservation of momentum:
4m×u=mv

\hrule
\vspace{1em}


\noindent
\textbf{Q1064.} A raindrop falls from a cloud and reaches terminal velocity. Later, it merges with another raindrop, doubling its mass and increasing its cross-sectional area by 1.5 times. Which graph shows how the air resistance (drag) force varies with time before and after the merger? (Assume gravitational acceleration and air density remain constant.)



\textbf{A.} [IMAGE:0] \\
\textbf{B.} [IMAGE:1] \\
\textbf{C.} [IMAGE:2] \\
\textbf{D.} [IMAGE:3] \\

\textbf{Answer:} A \\
\textbf{Explanation:} Initially, the air resistance equals the raindrop’s weight, stabilizing at terminal velocity. After merging, the mass doubles, and the cross-sectional area increases by 1.5 times. Using the terminal velocity formula
[IMAGE:0]
the mass increase dominates over the area change, leading to a higher terminal velocity. The air resistance must now balance the new weight (2mg), causing an abrupt rise followed by stabilization. The correct graph is option A.

\hrule
\vspace{1em}


\noindent
\textbf{Q1065.} On a smooth air track, there are two gliders, A and B. Glider A has three times the mass of glider B. Glider A is moving to the right at speed v, while glider B is stationary. They undergo a perfectly inelastic collision (stick together after collision). What is the magnitude and direction of their common velocity after the collision?
[IMAGE:0]



\textbf{A.} Magnitude 0, direction not applicable \\
\textbf{B.} Magnitude 1/4
​
v
, direction to the left \\
\textbf{C.} Magnitude 3/4
​
v
, direction to the left \\
\textbf{D.} Magnitude 1/2
​
v
, direction to the left \\

\textbf{Answer:} F \\
\textbf{Explanation:} By the conservation of momentum, the total momentum of the system remains constant in the absence of external forces. Before the collision:
Momentum of glider A: 3
m
×
v
=3
mv
(to the right)
Momentum of glider B:
m
×0=0
Total momentum of the system: 3
mv
(to the right)
After the collision, the combined mass of the gliders is 3
m
+
m
=4
m
. Let the common velocity after the collision be
u
. Applying the conservation of momentum:
4
m
×
u
=3
mv
Solving for
u
:
u
=3
mv/4m
​
=3/4
​
v
, direction to the right.

\hrule
\vspace{1em}


\noindent
\textbf{Q1066.} A stationary ice boat of mass 200
kg is on frictionless ice. Two people are on board: Person A (mass 50
kg) and Person B (mass 50
kg). Person A jumps off the boat to the right with a velocity of 5
m/s relative to the boat. Determine the magnitude and direction of the boat and Person B’s common velocity after the jump.
[IMAGE:0]



\textbf{A.} 1, to the left \\
\textbf{B.} 2, to the right \\
\textbf{C.} 0.5, to the left \\
\textbf{D.} 5, to the right \\

\textbf{Answer:} A \\
\textbf{Explanation:} By conservation of momentum, the total momentum before and after the jump remains zero. Let the velocity of the boat and Person B be v (left as negative, right as positive).
Total mass of boat + Person B: 200+50=250
kg.
Momentum equation:50×5+250×(
−
v
)=0
\implies 
250
−
250
v
=0
\implies 
v
=1m/s

\hrule
\vspace{1em}


\noindent
\textbf{Q1067.} An object falls freely from a height and enters a fluid (such as air or water), gradually reaching terminal velocity. When the object enters a less dense layer of the fluid, the drag force suddenly decreases, and the object reaches a new terminal velocity. Which graph shows how the drag force varies with time?



\textbf{A.} [IMAGE:0] \\
\textbf{B.} [IMAGE:1] \\
\textbf{C.} [IMAGE:2] \\
\textbf{D.} [IMAGE:3] \\

\textbf{Answer:} C \\
\textbf{Explanation:} The variation of drag force is as follows:
As the object begins to fall, velocity increases gradually, and so does the drag force (since drag is proportional to the square of velocity).
When drag balances weight, the object reaches the first terminal velocity, and drag stabilizes.
Upon entering the less dense fluid layer, the drag coefficient decreases suddenly, causing a sharp drop in drag force.
Drag force balances weight again, and the object reaches a new terminal velocity with stabilized drag.

\hrule
\vspace{1em}


\noindent
\textbf{Q1068.} A skater of mass 60
kg is initially stationary on frictionless ice. He throws a ball of mass 5
kg horizontally to the right with a velocity of 12
m/s. Determine the magnitude and direction of the skater’s velocity after throwing the ball.
[IMAGE:0]



\textbf{A.} 1, to the left \\
\textbf{B.} 1, to the right \\
\textbf{C.} 12, to the left \\
\textbf{D.} 5, to the right \\

\textbf{Answer:} A \\
\textbf{Explanation:} By the law of conservation of momentum, the total momentum of the system remains zero before and after the throw. Let the skater’s velocity be v
v
. Taking right as positive and left as negative:
[IMAGE:0]

\hrule
\vspace{1em}


\noindent
\textbf{Q1069.} An object falls freely from a height and enters a fluid (such as air or water), gradually reaching terminal velocity. When the object enters a less dense layer of the fluid, the drag force suddenly decreases, and the object reaches a new terminal velocity. Which graph shows how the drag force varies with time?



\textbf{A.} [IMAGE:0] \\
\textbf{B.} [IMAGE:1] \\
\textbf{C.} [IMAGE:2] \\
\textbf{D.} [IMAGE:3] \\

\textbf{Answer:} B \\
\textbf{Explanation:} The variation of drag force is as follows:
As the object begins to fall, velocity increases gradually, and so does the drag force (since drag is proportional to the square of velocity).
When drag balances weight, the object reaches the first terminal velocity, and drag stabilizes.
Upon entering the less dense fluid layer, the drag coefficient decreases suddenly, causing a sharp drop in drag force.
Drag force balances weight again, and the object reaches a new terminal velocity with stabilized drag.

\hrule
\vspace{1em}


\noindent
\textbf{Q1070.} A spherical object falls freely from a height and enters the air, gradually reaching terminal velocity. When the object enters a denser layer of air, the drag force suddenly increases, and the object reaches a new terminal velocity. Which graph shows how the drag force varies with time?



\textbf{A.} [IMAGE:0] \\
\textbf{B.} [IMAGE:1] \\
\textbf{C.} [IMAGE:2] \\
\textbf{D.} [IMAGE:3] \\

\textbf{Answer:} A \\
\textbf{Explanation:} The variation of drag force is as follows:
As the object begins to fall, velocity increases gradually, and so does the drag force (since drag is proportional to the square of velocity).
When drag balances weight, the object reaches the first terminal velocity, and drag stabilizes.
Upon entering the denser air layer, the drag coefficient increases suddenly, causing a sharp rise in drag force.
Drag force balances weight again, and the object reaches a new terminal velocity with stabilized drag.

\hrule
\vspace{1em}


\noindent
\textbf{Q1071.} A cart A slides down a smooth incline and collides with a stationary cart B at the bottom. The mass of cart A is twice that of cart B. After the collision, the two carts stick together and continue moving down the incline. The speed of cart A before the collision is v. What is the magnitude and direction of their common velocity after the collision?



\textbf{A.} Magnitude 0,direction not applicable \\
\textbf{B.} Magnitude
[IMAGE:0]
,direction up the incline \\
\textbf{C.} Magnitude
[IMAGE:1]
,direction up the incline \\
\textbf{D.} Magnitude
[IMAGE:2]
,direction up the incline \\

\textbf{Answer:} F \\
\textbf{Explanation:} By the conservation of momentum, the total momentum of the system remains constant in the absence of external forces. Before the collision:
Momentum of cart A: 2
m
×
v
=2
mv
(down the incline)
Momentum of cart B: 0 (since it is stationary)
Total momentum of the system: 2
mv
(down the incline)
After the collision, the combined mass of the carts is 2
m
+
m
=3
m
. Let the common velocity after the collision be
u
. Applying the conservation of momentum:
3
m
×
u
=2
mv
Solving for
u
:
u
=2/3
​
v
, direction down the incline.

\hrule
\vspace{1em}


\noindent
\textbf{Q1072.} Find the area of the shaded part in the figure. The three functions are respectively: y=
[IMAGE:0]
y=(x-5)/4
y=-
[IMAGE:1]
[IMAGE:2]



\textbf{A.} 32 \\
\textbf{B.} 83 \\
\textbf{C.} 76 \\
\textbf{D.} 36 \\

\textbf{Answer:} D \\
\textbf{Explanation:} [IMAGE:0]
[IMAGE:1]

\hrule
\vspace{1em}


\noindent
\textbf{Q1073.} Find the area of the shaded part in the figure. The three functions are respectively:y=
[IMAGE:0]
y=(x-3)/2 y=-
[IMAGE:1]
[IMAGE:2]



\textbf{A.} 32/3 \\
\textbf{B.} 8/3 \\
\textbf{C.} 7/6 \\
\textbf{D.} 1/3 \\

\textbf{Answer:} A \\
\textbf{Explanation:} [IMAGE:0]
[IMAGE:1]

\hrule
\vspace{1em}


\noindent
\textbf{Q1074.} Determine the signed area of the plane region enclosed by the continuous curve y = ln x, the x-axis, the vertical lines x = 1/2 and x = 2 (as shown in the figure).
[IMAGE:0]



\textbf{A.} 2.789 \\
\textbf{B.} 17/3 \\
\textbf{C.} [IMAGE:0] \\
\textbf{D.} [IMAGE:1] \\

\textbf{Answer:} F \\
\textbf{Explanation:} [IMAGE:0]
[IMAGE:1]
[IMAGE:2]

\hrule
\vspace{1em}


\noindent
\textbf{Q1075.} Calculate the shaded area enclosed by the two functions in the following figure.
[IMAGE:0]



\textbf{A.} 13 \\
\textbf{B.} 18 \\
\textbf{C.} 76 \\
\textbf{D.} 12 \\

\textbf{Answer:} B \\
\textbf{Explanation:} [IMAGE:0]

\hrule
\vspace{1em}


\noindent
\textbf{Q1076.} Let S denote the directed area of the shaded region in the figure. Then what is the value of S?
[IMAGE:0]



\textbf{A.} [IMAGE:0] \\
\textbf{B.} [IMAGE:1] \\
\textbf{C.} [IMAGE:2] \\
\textbf{D.} [IMAGE:3] \\

\textbf{Answer:} C \\
\textbf{Explanation:} It's the integral value.

\hrule
\vspace{1em}


\noindent
\textbf{Q1077.} What is the area of the plane region enclosed by the curve
[IMAGE:0]
and the lines x=0, y=0?



\textbf{A.} 31/3 \\
\textbf{B.} 88/3 \\
\textbf{C.} 37/3 \\
\textbf{D.} 51/2 \\

\textbf{Answer:} B \\
\textbf{Explanation:} This is a basic problem where the area of the plane region can be determined using a definite integral.
[IMAGE:0]
[IMAGE:1]
[IMAGE:2]
[IMAGE:3]

\hrule
\vspace{1em}


\noindent
\textbf{Q1078.} Which of the following statements about variations in gravitational acceleration is correct?



\textbf{A.} Gravitational acceleration at the top of a high mountain is greater than at sea level. \\
\textbf{B.} Inside the Earth, gravitational acceleration increases with depth. \\
\textbf{C.} Gravitational acceleration at the equator is less than at the poles. \\
\textbf{D.} Gravitational acceleration is independent of Earth's rotation. \\

\textbf{Answer:} C \\
\textbf{Explanation:} Option A is incorrect because gravitational acceleration is smaller at mountain tops than at sea level. Option B is incorrect because inside the Earth, gravitational acceleration first increases and then decreases with depth (assuming uniform density). Option C is correct because gravitational acceleration is smaller at the equator than at the poles. Option D is incorrect because Earth's rotation affects the distribution of gravitational acceleration.

\hrule
\vspace{1em}


\noindent
\textbf{Q1079.} Calculate the area of the shaded region within the red box in the following figure.
[IMAGE:0]



\textbf{A.} 12 \\
\textbf{B.} -4.5 \\
\textbf{C.} 13.5 \\
\textbf{D.} -12 \\

\textbf{Answer:} C \\
\textbf{Explanation:} Explanation:
[IMAGE:0]

\hrule
\vspace{1em}


\noindent
\textbf{Q1080.} Which of the following statements about gravity and gravitational force is correct?



\textbf{A.} Gravity and gravitational force are two different types of forces. \\
\textbf{B.} Gravitational acceleration at Earth's surface is independent of altitude. \\
\textbf{C.} Gravity is a component of gravitational force. \\
\textbf{D.} Gravitational acceleration is exactly the same everywhere on Earth's surface. \\

\textbf{Answer:} C \\
\textbf{Explanation:} Option A is incorrect because gravity is the manifestation of gravitational force at Earth's surface. Option B is incorrect because gravitational acceleration decreases with altitude. Option C is correct because gravity is effectively a component of gravitational force (neglecting Earth's rotation). Option D is incorrect because gravitational acceleration varies with latitude and altitude.

\hrule
\vspace{1em}


\noindent
\textbf{Q1081.} Calculate the shaded area enclosed by the two functions in the following figure.
[IMAGE:0]



\textbf{A.} 32/3 \\
\textbf{B.} -2/3 \\
\textbf{C.} 1/6 \\
\textbf{D.} -1/2 \\

\textbf{Answer:} A \\
\textbf{Explanation:} [IMAGE:0]

\hrule
\vspace{1em}


\noindent
\textbf{Q1082.} As shown in the figure, please calculate the directed area of the shaded region.
[IMAGE:0]



\textbf{A.} 12 \\
\textbf{B.} -12 \\
\textbf{C.} 15 \\
\textbf{D.} -9 \\

\textbf{Answer:} D \\
\textbf{Explanation:} Explanation:
[IMAGE:0]
[IMAGE:1]

\hrule
\vspace{1em}


\noindent
\textbf{Q1083.} Which of the following statements about satellite orbital motion is correct?



\textbf{A.} A satellite in orbit experiences no forces. \\
\textbf{B.} Orbital speed of a satellite is independent of orbital radius. \\
\textbf{C.} Orbital period of a satellite depends on orbital radius. \\
\textbf{D.} The shape of a satellite's orbit must be circular. \\

\textbf{Answer:} C \\
\textbf{Explanation:} Option A is incorrect because satellites are subject to Earth's gravity, which provides centripetal force. Option B is incorrect because orbital speed depends on radius
[IMAGE:0]
Option C is correct because orbital period depends on radius
[IMAGE:1]
Option D is incorrect because orbits can be elliptical or other shapes.

\hrule
\vspace{1em}


\noindent
\textbf{Q1084.} What is the area enclosed by the line x=18 and the curve
[IMAGE:0]
?
[IMAGE:1]



\textbf{A.} 4 \\
\textbf{B.} 8 \\
\textbf{C.} 15 \\
\textbf{D.} 32 \\

\textbf{Answer:} D \\
\textbf{Explanation:} [IMAGE:0]
Area=4*18-A1

\hrule
\vspace{1em}


\noindent
\textbf{Q1085.} What is the area enclosed by the line y=1 and the line y=3 and the curve
[IMAGE:0]
?
[IMAGE:1]



\textbf{A.} 2 \\
\textbf{B.} 9 \\
\textbf{C.} 18 \\
\textbf{D.} 36 \\

\textbf{Answer:} D \\
\textbf{Explanation:} [IMAGE:0]
Area=A1

\hrule
\vspace{1em}


\noindent
\textbf{Q1086.} Which of the following statements about gravitational acceleration is correct?



\textbf{A.} Gravitational acceleration is greatest at the equator. \\
\textbf{B.} Gravitational acceleration is smallest at the poles. \\
\textbf{C.} Gravitational acceleration decreases with increasing altitude. \\
\textbf{D.} Gravitational acceleration depends on the mass of the object. \\

\textbf{Answer:} C \\
\textbf{Explanation:} Options A and B are incorrect because gravitational acceleration is greatest at the poles and smallest at the equator. Option C is correct because gravitational acceleration decreases with altitude. Option D is incorrect because gravitational acceleration depends on Earth's mass and distance, not the object's mass.

\hrule
\vspace{1em}


\noindent
\textbf{Q1087.} A rhombus has diagonals of lengths 3x
cm and 4x
cm. The area of the rhombus is 24
cm
2
. What is the length of the longer diagonal?
[IMAGE:0]



\textbf{A.} 12 cm \\
\textbf{B.} 14 cm \\
\textbf{C.} 16 cm \\
\textbf{D.} 8 cm \\

\textbf{Answer:} D \\
\textbf{Explanation:} [IMAGE:0]
[IMAGE:1]

\hrule
\vspace{1em}


\noindent
\textbf{Q1088.} Which of the following statements about gravitational force is correct?



\textbf{A.} The gravitational force from Earth to a person is greater than that from the person to Earth. \\
\textbf{B.} Gravitational acceleration is the same everywhere on Earth. \\
\textbf{C.} A satellite in orbit does not experience Earth's gravity. \\
\textbf{D.} Gravitational acceleration varies with latitude. \\

\textbf{Answer:} D \\
\textbf{Explanation:} Option A is incorrect because gravitational forces between Earth and a person are equal in magnitude (action-reaction pair). Option B is incorrect because gravitational acceleration varies with latitude and altitude. Option C is incorrect because satellites are still subject to Earth's gravity, which provides centripetal force. Option D is correct because gravitational acceleration changes with latitude.

\hrule
\vspace{1em}


\noindent
\textbf{Q1089.} A cylindrical container has a base radius of (x + 1) centimeters and a height of (x - 2) centimeters. There is a part of liquid in the container, whose volume accounts for 1/3 of the volume of the cylinder. When a cone with the same base area as the cylindrical container is placed with its base on the bottom of the container and its vertex touching the water surface, the vertex of the cone just touches the water surface; the height of the cone is 4. What is the total capacity of this container?
[IMAGE:0]



\textbf{A.} 120π cm³ \\
\textbf{B.} 128π cm³ \\
\textbf{C.} 486π cm³ \\
\textbf{D.} 360π cm³ \\

\textbf{Answer:} C \\
\textbf{Explanation:} Explanation:2(x-2)/3=4 x=8

\hrule
\vspace{1em}


\noindent
\textbf{Q1090.} Which statement about gravity is correct?



\textbf{A.} Gravitational force depends only on mass, not distance. \\
\textbf{B.} Earth’s surface gravity is ~9.8 m/s² but decreases linearly with depth. \\
\textbf{C.} A geostationary satellite’s altitude is determined by Earth’s mass. \\
\textbf{D.} The Moon’s centripetal force is provided by Earth’s gravity. \\

\textbf{Answer:} D \\
\textbf{Explanation:} A is false (force follows inverse-square law). B is false (gravity inside Earth depends on depth and density). C is false (altitude depends on orbital period). D is correct.

\hrule
\vspace{1em}


\noindent
\textbf{Q1091.} A right triangle has legs measuring (x+3) cm and (x
−
1) cm. The area of the triangle is 30
cm
²
. What are the lengths of the two legs?
[IMAGE:0]



\textbf{A.} 10cm and 6cm \\
\textbf{B.} 8cm and 4cm \\
\textbf{C.} 9cm and 5cm \\
\textbf{D.} 10cm and 4cm \\

\textbf{Answer:} A \\
\textbf{Explanation:} Explanation:
(
x-1
)
*(x+3)/2=30

\hrule
\vspace{1em}


\noindent
\textbf{Q1092.} Which statement is incorrect?



\textbf{A.} Earth’s rotation reduces effective gravity at the equator compared to the poles. \\
\textbf{B.} An object weighs slightly less at Mount Everest’s summit than at sea level. \\
\textbf{C.} If Earth stopped rotating, gravitational acceleration at the equator would increase. \\
\textbf{D.} The Moon’s gravitational force on Earth and Earth’s force on the Moon act in opposite directions and not equal. \\

\textbf{Answer:} D \\
\textbf{Explanation:} D is false (forces are equal, opposite, and collinear per Newton’s third law). A, B, C are correct (centrifugal effect and altitude impact).

\hrule
\vspace{1em}


\noindent
\textbf{Q1093.} A shape is formed by drawing a triangle ABC inside the triangle ADE. BC is parallel to DE. The area of triangle ABC is 18 cm², and the area of triangle ADE is 72 cm². BC = s cm, DE = s + 8 cm.
Determine the height of triangle ADE to side DE.
[IMAGE:0]



\textbf{A.} 9cm \\
\textbf{B.} 14cm \\
\textbf{C.} 16cm \\
\textbf{D.} [IMAGE:0] \\

\textbf{Answer:} A \\
\textbf{Explanation:} Area ratio is square of altitude (or side) ratio. Given
[IMAGE:0]
, sides ratio is
[IMAGE:1]
. So,
[IMAGE:2]
. Solving gives s=8
, hence DE=s+8=16cm
. Thus, the height is
[IMAGE:3]
.

\hrule
\vspace{1em}


\noindent
\textbf{Q1094.} In a trapezium ABCD, the parallel sides AD and BC are (x-1)
cm and (2x
+7
)
cm respectively, with a vertical height of x
cm. The area of the trapezium is 36 cm². What is the length of the longer base BC?
[IMAGE:0]



\textbf{A.} 7 cm \\
\textbf{B.} 9 cm \\
\textbf{C.} 11 cm \\
\textbf{D.} 15 cm \\

\textbf{Answer:} D \\
\textbf{Explanation:} [IMAGE:0]

\hrule
\vspace{1em}


\noindent
\textbf{Q1095.} A rectangular flower bed has a length of (x+2) meters and a width of (x
−
1) meters. To enhance the landscape, a 1-meter wide brick path is planned around the flower bed. The area of the brick path is 34 square meters. What are the length and width of the flower bed?
[IMAGE:0]



\textbf{A.} Length 7m, Width 3m \\
\textbf{B.} Length 8m, Width 4m \\
\textbf{C.} Length 9m, Width 5m \\
\textbf{D.} Length 10m, Width 6m \\

\textbf{Answer:} E \\
\textbf{Explanation:} Explanation:34 = 2*[(2*(x+3)+2)/2+(2*(x-2)+2)/2]

\hrule
\vspace{1em}


\noindent
\textbf{Q1096.} Determine the correct statement:



\textbf{A.} The direction of gravitational acceleration always points to Earth’s center. \\
\textbf{B.} An object weighs 1/6 on the Moon’s surface due to the Moon’s smaller mass. \\
\textbf{C.} A satellite in uniform circular motion around Earth experiences zero net force. \\
\textbf{D.} Earth’s gravitational force on an object decreases with altitude. \\

\textbf{Answer:} D \\
\textbf{Explanation:} A is false (direction slightly deviates due to Earth’s rotation). B is false (smaller gravity is due to Moon’s mass, not object’s mass). C is false (centripetal force requires net inward force). D is correct (force follows inverse-square law).

\hrule
\vspace{1em}


\noindent
\textbf{Q1097.} A rectangular swimming pool has a length of (x+1) meters and a width of 1 meters. If the water is drained through an outlet at a rate of 3 cubic meters per second, it takes 8 seconds to empty the pool. What is the length (depth = x-1) of the swimming pool?



\textbf{A.} 1m \\
\textbf{B.} 2m \\
\textbf{C.} 3m \\
\textbf{D.} 4m \\

\textbf{Answer:} F \\
\textbf{Explanation:} Explanation:(x+1)(x-1)=3*8

\hrule
\vspace{1em}


\noindent
\textbf{Q1098.} Which statement is correct?
A. Astronauts in the ISS experience weightlessness because they are far from Earth.
B. Gravitational acceleration decreases linearly with altitude on Earth’s surface.
C. The Moon’s tidal effect on Earth is independent of gravitational force.
D. Earth’s rotation causes gravitational acceleration at the equator to be less than at the poles.



\textbf{A.} The Earth’s gravitational force on the Moon is greater than the Moon’s force on the Earth. \\
\textbf{B.} The gravitational acceleration at Earth’s poles is slightly less than at the equator. \\
\textbf{C.} A satellite orbiting Earth experiences zero gravitational acceleration. \\
\textbf{D.} An object weighs more at higher latitudes than at lower latitudes. \\

\textbf{Answer:} D \\
\textbf{Explanation:} A is false (Newton’s third law ensures equal magnitude). B is false (polar gravity is stronger due to reduced centrifugal effect). C is false (satellites are under gravitational pull). D is correct (closer to Earth’s center at higher latitudes).

\hrule
\vspace{1em}


\noindent
\textbf{Q1099.} An object is pushed up an inclined plane to the top. The base length of the incline is (x
−
1) meters, and the height is (x+6) meters. The pushing force is 10 N, and the work done by the force is 120 J. What is the length of the inclined plane?
[IMAGE:0]



\textbf{A.} 5m \\
\textbf{B.} 7m \\
\textbf{C.} 9m \\
\textbf{D.} 10m \\

\textbf{Answer:} E \\
\textbf{Explanation:} Explanation:
The formula for work done is:
W=F×d
where W is work, F is force, and d is the distance over which the force is applied (the length of the incline).

\hrule
\vspace{1em}


\noindent
\textbf{Q1100.} A shape is formed by drawing a triangle ABC inside the triangle ADE. BC is parallel to DE. The median from A to BC in triangle ABC is 4 cm, and the median from A to DE in triangle ADE is 10 cm. The two medians are collinear. BC = r cm, DE = r + 9 cm.
Find the length of DE.
[IMAGE:0]



\textbf{A.} 12cm \\
\textbf{B.} 15cm \\
\textbf{C.} 18cm \\
\textbf{D.} [IMAGE:0] \\

\textbf{Answer:} B \\
\textbf{Explanation:} Medians ratio equals sides ratio. So,
[IMAGE:0]
. Solving gives r=6
, hence DE=r+9=15cm
.

\hrule
\vspace{1em}


\noindent
\textbf{Q1101.} A light spring has a natural length of 0.60 m and a spring constant of 100 N/m. It is stretched by a force starting from zero and increasing at a constant rate of 0.5 N/s until reaching its maximum value. When the strain energy of the spring is 0.36 J, what is the average power used to stretch the spring?



\textbf{A.} 0.015 W \\
\textbf{B.} 0.020 W \\
\textbf{C.} 0.025 W \\
\textbf{D.} 0.030 W \\

\textbf{Answer:} B \\
\textbf{Explanation:} [IMAGE:0]

\hrule
\vspace{1em}


\noindent
\textbf{Q1102.} Which statements are correct?
1.
The proportionality constants for Coulomb force and gravitational force both depend on the medium.
2.
In a homogeneous medium, the magnitude of Coulomb force decreases.
3. The magnitude of gravitational force is independent of the medium.



\textbf{A.} 2 and 3 \\
\textbf{B.} 1 only \\
\textbf{C.} 1 and 2 \\
\textbf{D.} All of the above \\

\textbf{Answer:} A \\
\textbf{Explanation:} Statement 1 is false (gravitational constant G is independent of medium). Statements 2 and 3 are correct.

\hrule
\vspace{1em}


\noindent
\textbf{Q1103.} A light spring oscillator has a spring constant of 20 N/m. The oscillator is pulled from its equilibrium position by a tension force that starts at zero and increases at a constant rate of 0.60 N/s until it reaches its maximum value. When the force reaches its maximum value, the elastic potential energy of the oscillator is 0.45 J. What is the average power of the work done by the force?



\textbf{A.} 0.18 W \\
\textbf{B.} 0.30 W \\
\textbf{C.} 0.45 W \\
\textbf{D.} 0.60 W \\

\textbf{Answer:} A \\
\textbf{Explanation:} [IMAGE:0]

\hrule
\vspace{1em}


\noindent
\textbf{Q1104.} In a trapezium PQRS, the parallel sides are PQ and RS. PQ =(x
−
4) cm, RS =(x+8) cm and the vertical height QR = x cm.
[IMAGE:0]
The area of the trapezium is 15 cm
2
. What is the length of RS?



\textbf{A.} 9cm \\
\textbf{B.} 2cm \\
\textbf{C.} 11cm \\
\textbf{D.} 3cm \\

\textbf{Answer:} C \\
\textbf{Explanation:} Area=(x-5+x+9)*x/2=15

\hrule
\vspace{1em}


\noindent
\textbf{Q1105.} A shape is formed by drawing a triangle ABC inside the triangle ADE. BC is parallel to DE. BC+DE=21cm
, and AB=4cm, BD=6cm.
Calculate the length of DE
.
[IMAGE:0]



\textbf{A.} 11cm \\
\textbf{B.} 13cm \\
\textbf{C.} 15cm \\
\textbf{D.} [IMAGE:0] \\

\textbf{Answer:} C \\
\textbf{Explanation:} From similarity,
[IMAGE:0]
. BC+DE=1.4DE=21cm
, thus, DE=15cm
.

\hrule
\vspace{1em}


\noindent
\textbf{Q1106.} A light rubber band has an unstretched length of 0.20 m and a spring constant of 30 N/m. The rubber band is stretched by a tension force that starts at zero and increases at a constant rate of 0.40 N/s until it reaches its maximum value. When the force reaches its maximum value, the elastic potential energy of the rubber band is 0.18 J. What is the average power used to stretch the rubber band?



\textbf{A.} 0.020 W \\
\textbf{B.} 0.040 W \\
\textbf{C.} 0.060 W \\
\textbf{D.} 0.080 W \\

\textbf{Answer:} B \\
\textbf{Explanation:} [IMAGE:0]

\hrule
\vspace{1em}


\noindent
\textbf{Q1107.} In a trapezium PQRS, the parallel sides are PQ and RS. PQ =(x
−
2) cm, RS =(x+11) cm and the vertical height QR = x cm.
[IMAGE:0]
The area of the trapezium is 252 cm
2
. What is the length of RS?



\textbf{A.} 9cm \\
\textbf{B.} 10cm \\
\textbf{C.} 11cm \\
\textbf{D.} 12cm \\

\textbf{Answer:} E \\
\textbf{Explanation:} Area=(x-2+x+10)*x/2=252

\hrule
\vspace{1em}


\noindent
\textbf{Q1108.} Determine the correct statements:
1.
Both electric potential energy and gravitational potential energy can be negative.
2.
The zero point of electric potential energy can be arbitrary, but gravitational potential energy must be zero at Earth’s surface.
3. The electric potential energy of a point charge system is inversely proportional to the distance between charges.



\textbf{A.} All of the above \\
\textbf{B.} 1 and 3 \\
\textbf{C.} 1 only \\
\textbf{D.} 2 and 3 \\

\textbf{Answer:} B \\
\textbf{Explanation:} Statement 2 is false (gravitational potential energy’s zero point can be arbitrary). Statement 3 is correct
[IMAGE:0]

\hrule
\vspace{1em}


\noindent
\textbf{Q1109.} A light balloon is filled with gas, with an initial volume of 0.05 m³. The gas pressure relates to volume as
P
=
kV
, where
k
=200Pa/m3. The gas pressure starts at zero and increases at a constant rate of 0.50 Pa/s until it reaches its maximum value. When the pressure reaches its maximum value, the elastic potential energy stored in the balloon is 10.0 J. What is the average power of the work done by the gas during inflation?



\textbf{A.} 0.50 W \\
\textbf{B.} 1.00 W \\
\textbf{C.} 2.00 W \\
\textbf{D.} 5.00 W \\

\textbf{Answer:} C \\
\textbf{Explanation:} [IMAGE:0]

\hrule
\vspace{1em}


\noindent
\textbf{Q1110.} In a trapezium PQRS, the parallel sides are PQ and RS. PQ =(x
−
2) cm, RS =(x+6) cm and the vertical height QR = x cm.
[IMAGE:0]
The area of the trapezium is 99 cm
2
. What is the length of RS?



\textbf{A.} 9cm \\
\textbf{B.} 10cm \\
\textbf{C.} 11cm \\
\textbf{D.} 12cm \\

\textbf{Answer:} E \\
\textbf{Explanation:} Area=(x-2+x+6)*x/2=99

\hrule
\vspace{1em}


\noindent
\textbf{Q1111.} Which statements about conservative forces are correct?
1.
Both Coulomb force and gravitational force are conservative.
2.
Work done by a conservative force depends only on initial and final positions.
3. Friction is a conservative force.



\textbf{A.} 1 and 2 \\
\textbf{B.} 3 only \\
\textbf{C.} 2 and 3 \\
\textbf{D.} All of the above \\

\textbf{Answer:} A \\
\textbf{Explanation:} Statement 3 is false (friction is non-conservative).

\hrule
\vspace{1em}


\noindent
\textbf{Q1112.} Which of the following is the correct unit for electric field intensity?



\textbf{A.} Volt per meter \\
\textbf{B.} Ampere per meter \\
\textbf{C.} Weber per meter \\
\textbf{D.} Newton per meter \\

\textbf{Answer:} A \\
\textbf{Explanation:} The electric field intensity is defined as: E = F/q; where q is the electric charge quantity. The unit of electric field intensity E is v/m.

\hrule
\vspace{1em}


\noindent
\textbf{Q1113.} A light spring has an unstretched length of 0.50 m and a spring constant of 40 N/m. The spring is stretched by a tension force that starts at zero and increases at a constant rate of 0.10 N/s until it reaches its maximum value. When the force reaches its maximum value, the elastic potential energy of the spring is 0.25 J. What is the average power used to stretch the spring?



\textbf{A.} 0.022 W \\
\textbf{B.} 0.040 W \\
\textbf{C.} 0.011 W \\
\textbf{D.} 0.102 W \\

\textbf{Answer:} C \\
\textbf{Explanation:} he elastic potential energy formula is
U
=21
​
kx
2, where
k
is the spring constant and
x
is the displacement.
Given
U
=0.25J,
k
=40N/m, solving for
x
:
[IMAGE:0]
Maximum force
F
max
​
=
kx
=40×0.1118≈4.47N.
[IMAGE:1]

\hrule
\vspace{1em}


\noindent
\textbf{Q1114.} Which of the following is the correct unit for inductance?



\textbf{A.} Farad \\
\textbf{B.} Ampere \\
\textbf{C.} Henry \\
\textbf{D.} Kilogram \\

\textbf{Answer:} C \\
\textbf{Explanation:} The unit of inductance is the henry (Henry). According to the definition of inductance:
L
=Φ/I
​
where Φ is magnetic flux (unit: weber, Wb) and
I
is current (unit: ampere, A). Therefore, the unit of inductance is Wb/A, which is the henry (H).

\hrule
\vspace{1em}


\noindent
\textbf{Q1115.} Which of the following is the correct unit for acceleration?



\textbf{A.} Meter per second \\
\textbf{B.} Meter per second squared \\
\textbf{C.} Meter per second cubed \\
\textbf{D.} Joule \\

\textbf{Answer:} B \\
\textbf{Explanation:} The unit of acceleration is meters per second squared (m/s²). According to the definition of acceleration:
a
=Δ
v
​/
Δ
t
where Δ
v
is change in velocity (unit: meters per second, m/s) and Δ
t
is change in time (unit: second, s). Therefore, the unit of acceleration is m/s².

\hrule
\vspace{1em}


\noindent
\textbf{Q1116.} A spring with a spring constant of 40 N/m is stretched by a force increasing at 0.10 N/s. When the strain energy is 0.08 J, the average power is:



\textbf{A.} 0.004 W \\
\textbf{B.} 0.003 W \\
\textbf{C.} 0.012 W \\
\textbf{D.} 0.016 W \\

\textbf{Answer:} B \\
\textbf{Explanation:} [IMAGE:0]

\hrule
\vspace{1em}


\noindent
\textbf{Q1117.} Which statement is incorrect?
1.
Coulomb force between point charges differs in vacuum and in a medium.
2.
The gravitational constant is independent of the medium’s permittivity.
3.
The directions of Coulomb force and gravitational force are independent of the sign of charge or mass.



\textbf{A.} All are wrong \\
\textbf{B.} 3 only \\
\textbf{C.} 1 and 2 \\
\textbf{D.} 2 and 3 \\

\textbf{Answer:} B \\
\textbf{Explanation:} Statement 3 is false (Coulomb force direction depends on charge signs). Statements 1 and 2 are correct.

\hrule
\vspace{1em}


\noindent
\textbf{Q1118.} Which of the following is the correct unit of speed?



\textbf{A.} Meter per second \\
\textbf{B.} Meter per second squared \\
\textbf{C.} Cubic meter \\
\textbf{D.} Joule \\

\textbf{Answer:} A \\
\textbf{Explanation:} The unit of velocity is meters per second (m/s). According to the definition of velocity:
v
=
d/t
​
where
d
is distance (unit: meter, m) and
t
is time (unit: second, s). Therefore, the unit of velocity is m/s.

\hrule
\vspace{1em}


\noindent
\textbf{Q1119.} A spring with a natural length of 0.25 m and spring constant 200 N/m is stretched by a force increasing at 0.80 N/s. When the strain energy is 0.50 J, the average power is:



\textbf{A.} 0.021 W \\
\textbf{B.} 0.050 W \\
\textbf{C.} 0.028 W \\
\textbf{D.} 0.376 W \\

\textbf{Answer:} C \\
\textbf{Explanation:} [IMAGE:0]

\hrule
\vspace{1em}


\noindent
\textbf{Q1120.} Which statements about Coulomb force and gravitational force are correct?
1.
The direction of Coulomb force depends on the sign of charges, while gravitational force is always attractive.
2.
Both forces obey the inverse-square law.
3. Work done by both forces is path-independent.



\textbf{A.} All of the above \\
\textbf{B.} 1 and 2 \\
\textbf{C.} 3 only \\
\textbf{D.} 2 and 3 \\

\textbf{Answer:} A \\
\textbf{Explanation:} All statements are correct. Coulomb force can be attractive or repulsive, while gravity is always attractive; both follow inverse-square law and are conservative.

\hrule
\vspace{1em}


\noindent
\textbf{Q1121.} A spring with a spring constant of 60 N/m is stretched by a force increasing at 0.30 N/s. When the strain energy is 0.18 J, the average power is:



\textbf{A.} 0.020 W \\
\textbf{B.} 0.012 W \\
\textbf{C.} 0.032 W \\
\textbf{D.} 0.042 W \\

\textbf{Answer:} B \\
\textbf{Explanation:} [IMAGE:0]

\hrule
\vspace{1em}


\noindent
\textbf{Q1122.} Which of the following is the correct unit for power?



\textbf{A.} Pascal \\
\textbf{B.} Watt \\
\textbf{C.} Newton \\
\textbf{D.} Coulomb \\

\textbf{Answer:} B \\
\textbf{Explanation:} The unit of power is the watt (Watt). According to the definition of power:
P
=
E/t
​
where
E
is energy (unit: joule, J) and
t
is time (unit: second, s). Therefore, the unit of power is J/s, which is the watt (W).

\hrule
\vspace{1em}


\noindent
\textbf{Q1123.} A spring with a spring constant of 120 N/m and natural length 0.30 m is stretched by a force increasing at 0.60 N/s. When the strain energy stored is 0.54 J, what is the average power?



\textbf{A.} 0.015 W \\
\textbf{B.} 0.030 W \\
\textbf{C.} 0.028 W \\
\textbf{D.} 0.060 W \\

\textbf{Answer:} C \\
\textbf{Explanation:} [IMAGE:0]

\hrule
\vspace{1em}


\noindent
\textbf{Q1124.} Which of the following is the correct unit for magnetic induction intensity?



\textbf{A.} Ampere per meter \\
\textbf{B.} Joule \\
\textbf{C.} Tesla \\
\textbf{D.} Weber \\

\textbf{Answer:} C \\
\textbf{Explanation:} The unit of magnetic induction intensity is Tesla, and 1T = 1Wb/m
2
.

\hrule
\vspace{1em}


\noindent
\textbf{Q1125.} Which of the following statements about the relationship between electric force and potential energy is correct?
1.
When electric force does positive work, electric potential energy decreases.
2.
When electric force does negative work, electric potential energy increases.
3.
The change in potential energy is equal to the negative of the work done by the electric force (Δ
U
=−
W
).



\textbf{A.} All of the above \\
\textbf{B.} 1 and 2 \\
\textbf{C.} 3 only \\
\textbf{D.} 1 and 3 \\

\textbf{Answer:} A \\
\textbf{Explanation:} All statements are correct. Electric potential energy decreases when electric force does positive work, increases when it does negative work, and the change in potential energy is equal to the negative of the work done by the electric force (Δ
U
=−
W
).

\hrule
\vspace{1em}


\noindent
\textbf{Q1126.} Which of the following is the correct unit for force?



\textbf{A.} Newton \\
\textbf{B.} Kilogram \\
\textbf{C.} Hertz \\
\textbf{D.} Coulomb \\

\textbf{Answer:} A \\
\textbf{Explanation:} The unit of force is the newton (Newton). According to Newton's second law:
F
=
ma
where
m
is mass (unit: kilogram, kg) and
a
is acceleration (unit: meters per second squared, m/s²). Therefore, the unit of force is kg\cdot m/s², which is the newton (N).

\hrule
\vspace{1em}


\noindent
\textbf{Q1127.} A light spring has a natural length of 0.50 m and a spring constant of 80 N/m. It is stretched by a force that starts at zero and increases at a constant rate of 0.40 N/s until reaching its maximum value. When the strain energy of the spring is 0.32 J, what is the average power used to stretch the spring?



\textbf{A.} 0.016 W \\
\textbf{B.} 0.018 W \\
\textbf{C.} 0.064 W \\
\textbf{D.} 0.080 W \\

\textbf{Answer:} B \\
\textbf{Explanation:} [IMAGE:0]

\hrule
\vspace{1em}


\noindent
\textbf{Q1128.} Which of the following is the correct unit for resistance?



\textbf{A.} Joule per coulomb \\
\textbf{B.} Newton per coulomb \\
\textbf{C.} Ohm \\
\textbf{D.} Coulomb per second \\

\textbf{Answer:} C \\
\textbf{Explanation:} The unit of electrical resistance is the ohm (Ohm). According to Ohm's Law:
where
V
is voltage (unit: volt, V) and
I
is current (unit: ampere, A). Therefore, the unit of resistance is volt per ampere (V/A), which is the ohm (Ω).

\hrule
\vspace{1em}


\noindent
\textbf{Q1129.} Which of the following statements about electric field strength and potential is correct?
1.
The direction of electric field strength is always aligned with the direction of the steepest decrease in electric potential.
2.
The unit of electric field strength is volts per meter (V/m).
3.
The unit of electric potential is the volt (V).



\textbf{A.} All of the above \\
\textbf{B.} 1 and 2 \\
\textbf{C.} 3 only \\
\textbf{D.} 1 and 3 \\

\textbf{Answer:} A \\
\textbf{Explanation:} All statements are correct. Electric field strength points in the direction of the steepest potential drop; its unit is volts per meter (V/m); the unit of electric potential is the volt (V).

\hrule
\vspace{1em}


\noindent
\textbf{Q1130.} Which of the following is the correct unit for potential difference (voltage)?



\textbf{A.} Milliampere per Newton \\
\textbf{B.} Ohm per Coulomb \\
\textbf{C.} Coulomb per Newton \\
\textbf{D.} Coulomb per second \\

\textbf{Answer:} E \\
\textbf{Explanation:} From the formula P=VI, voltage V=P/I
​
. The unit of power P is watt (W), and the unit of current I is ampere (A). Therefore, the unit of voltage is watt per ampere (W/A), corresponding to option E.

\hrule
\vspace{1em}


\noindent
\textbf{Q1131.} Which of the following statements about work done by electric force is correct?
1.
Work done by electric force is path-independent and depends only on the initial and final positions.
2.
Work done by electric force can be positive or negative.
3.
The unit of work done by electric force is the joule (J).



\textbf{A.} All of the above \\
\textbf{B.} 1 and 2 \\
\textbf{C.} 3 only \\
\textbf{D.} 1 and 3 \\

\textbf{Answer:} A \\
\textbf{Explanation:} All statements are correct. Electric force is conservative, so work done is path-independent; work can be positive or negative depending on the direction of charge movement relative to the electric field; the unit is the joule (J).

\hrule
\vspace{1em}


\noindent
\textbf{Q1132.} Which of the following is the correct unit for electric current?



\textbf{A.} Milliwatt per volt \\
\textbf{B.} Coulomb per second \\
\textbf{C.} Volt per second \\
\textbf{D.} Coulomb per newton \\

\textbf{Answer:} B \\
\textbf{Explanation:} Explanation: The unit of electric current is the ampere (Ampere). According to the definition of current:
[IMAGE:0]
where Q is the charge (unit: coulomb, C) and t is time (unit: second, s). Therefore, the unit of current is coulomb per second (C/s), which is the ampere (A).

\hrule
\vspace{1em}


\noindent
\textbf{Q1133.} A balloon is released and begins to rise. During the ascent, the balloon experiences a buoyant force that decreases with time (due to decreasing air density with height) and a drag force proportional to its velocity. The balloon eventually reaches a terminal velocity. The graphs below show the variation with time of three quantities (X, Y, and Z):
[IMAGE:0]
Which line in the table correctly identifies the quantities X, Y, and Z?



\textbf{A.} X: Buoyant Force; Y: Drag Force; Z: Kinetic Energy \\
\textbf{B.} X: Drag Force; Y: Buoyant Force; Z: Kinetic Energy \\
\textbf{C.} X: Buoyant Force; Y: Drag Force; Z: Weight \\
\textbf{D.} X: Kinetic Energy; Y: Buoyant Force; Z: Drag Force \\

\textbf{Answer:} F \\
\textbf{Explanation:} Explanation:
Z is Buoyant Force: The buoyant force decreases as the balloon rises due to decreasing air density, approaching a constant value, matching the X graph.
Y is Drag Force: Drag force increases with velocity and approaches a constant value at terminal velocity, matching the Y graph.
X is Weight: Weight is constant and independent of time, matching the horizontal Z graph.

\hrule
\vspace{1em}


\noindent
\textbf{Q1134.} Which of the following statements about electric and gravitational potential energy is correct?
1.
The zero point of electric potential energy can be chosen arbitrarily.
2.
The zero point of gravitational potential energy is usually selected at infinity.
3.
Both electric and gravitational potential energies are scalar quantities.



\textbf{A.} All of the above \\
\textbf{B.} 1 and 2 \\
\textbf{C.} 3 only \\
\textbf{D.} 1 and 3 \\

\textbf{Answer:} A \\
\textbf{Explanation:} All statements are correct. The zero point for electric potential energy can be chosen freely (often at infinity), the zero point for gravitational potential energy is also at infinity, and both are scalar quantities.

\hrule
\vspace{1em}


\noindent
\textbf{Q1135.} The following figure is a schematic diagram of the elevator model
[IMAGE:0]
[1]
When the elevator is accelerating upwards, the pressure exerted by the object on the elevator and the support force exerted by the elevator on the object constitute a pair of action-reaction forces.
[2]
When the elevator is accelerating downward, the pressure exerted by the object on the elevator and the supporting force exerted by the elevator on the object constitute a pair of action-reaction forces.
[3]
When the elevator accelerates, the gravitational force acting on the object block and the supporting force exerted by the elevator on the object block constitute a pair of action-reaction forces.



\textbf{A.} 1 only \\
\textbf{B.} 1 and2 \\
\textbf{C.} 1 and 3 \\
\textbf{D.} 3 only \\

\textbf{Answer:} B \\
\textbf{Explanation:} Gravity and support force are not a pair of action-reaction forces.

\hrule
\vspace{1em}


\noindent
\textbf{Q1136.} An object is immersed in a fluid and begins to sink from rest. During the sinking process, the object experiences a buoyant force that varies with time and a drag force proportional to its velocity. The object eventually reaches a terminal velocity. The graphs below show the variation with time of three quantities (X, Y, and Z):
[IMAGE:0]
Which line in the table correctly identifies the quantities X, Y, and Z?



\textbf{A.} X: Buoyant Force; Y: Drag Force; Z: Kinetic Energy \\
\textbf{B.} X: Drag Force; Y: Buoyant Force; Z: Kinetic Energy \\
\textbf{C.} X: Weight; Y: Drag Force; Z: Buoyant Force \\
\textbf{D.} X: Kinetic Energy; Y: Buoyant Force; Z: Drag Force \\

\textbf{Answer:} C \\
\textbf{Explanation:} Explanation:
Z is Buoyant Force: The buoyant force decreases as the object displaces less fluid, approaching a constant value when fully submerged, matching the X graph.
Y is Drag Force: Drag force increases with velocity and approaches a constant value at terminal velocity, matching the Y graph.
X is Weight: Weight is constant and independent of time, matching the horizontal Z graph.

\hrule
\vspace{1em}


\noindent
\textbf{Q1137.} Which of the following statements about electric and gravitational forces is correct?
1.
The electric force always points from positive to negative charges.
2.
The gravitational force always acts along the line connecting two objects.
3.
Both electric and gravitational forces are conservative forces.



\textbf{A.} All of the above \\
\textbf{B.} 1 and 2 \\
\textbf{C.} 3 only \\
\textbf{D.} 1 and 3 \\

\textbf{Answer:} E \\
\textbf{Explanation:} Statement 1 is incorrect because the direction of the electric force depends on the charge's sign. Statements 2 and 3 are correct: gravitational force acts along the line connecting two objects, and both forces are conservative.

\hrule
\vspace{1em}


\noindent
\textbf{Q1138.} A submersible starts moving from rest in water. It has a constant propelling force, and water resistance increases with its speed. Eventually, it reaches a constant speed. The graphs below show the variation with time of three quantities (E, F, G):
[IMAGE:0]



\textbf{A.} X: Propelling force; Y: Velocity; Z: Mass \\
\textbf{B.} X: Acceleration; Y: Water resistance; Z: Buoyant force \\
\textbf{C.} X: Resultant force; Y: Velocity; Z: Weight \\
\textbf{D.} X: Weight; Y: Velocity;; Z: Resultant force \\

\textbf{Answer:} D \\
\textbf{Explanation:} Analysis: Weight (G) is W=mg, with mass m and g constant. Resultant force (E): Initially Fnet
​
=Fpropel
​
−
f, as water resistance f grows with speed, Fnet
​
decreases to zero at constant speed. Velocity (F) increases from rest until terminal speed is reached.

\hrule
\vspace{1em}


\noindent
\textbf{Q1139.} Which of the following statements about Joule heating is correct?
1.
Joule heat is directly proportional to the square of the current.
2.
Joule heat is directly proportional to resistance.
3.
Joule heat is directly proportional to the time of current flow.



\textbf{A.} All of the above \\
\textbf{B.} 1 and 2 \\
\textbf{C.} 3 only \\
\textbf{D.} 1 and 3 \\

\textbf{Answer:} A \\
\textbf{Explanation:} All statements are correct. The Joule heating formula
[IMAGE:0]
shows that Joule heat is proportional to the square of current, resistance, and time of current flow.

\hrule
\vspace{1em}


\noindent
\textbf{Q1140.} Which of the following statements about the relationship between power, current, and voltage is correct?
1.
Power
P
is directly proportional to the square of the current
I
.
2.
Power
P
is directly proportional to the square of the voltage
V
.
3.
Power
P
is inversely proportional to resistance
R
.



\textbf{A.} All of the above \\
\textbf{B.} 1 and 2 \\
\textbf{C.} 3 only \\
\textbf{D.} 1 and 3 \\

\textbf{Answer:} B \\
\textbf{Explanation:} Statements 1 and 2 are correct. Power formulas
[IMAGE:0]
​
show that power is proportional to the square of current or voltage. Statement 3 is incorrect because the relationship between power and resistance depends on the formula used.

\hrule
\vspace{1em}


\noindent
\textbf{Q1141.} A rocket is launched vertically from the ground. During the launch, the rocket engine provides a constant thrust, while the rocket's mass decreases due to fuel combustion. The rocket eventually reaches a terminal velocity. The graphs below show the variation with time of three quantities (X, Y, and Z):
[IMAGE:0]



\textbf{A.} X: Velocity; Y: Kinetic Energy; Z: Weight \\
\textbf{B.} X: Thrust; Y: Gravitational Potential Energy; Z: Kinetic Energy \\
\textbf{C.} X: Acceleration; Y: Velocity    ; Z: Gravitational Potential Energy \\
\textbf{D.} X: Kinetic Energy; Y: Gravitational Potential Energy; Z: Thrust \\

\textbf{Answer:} E \\
\textbf{Explanation:} The ground is fixed as can be seen in the graph. "1." The gravity of object A and the normal reaction force of the conveyor belt on object A is a balanced forces pair (which is not the same as "action and reaction force pair"). "2." is obviously right. "3." Object A may not necessarily slide. In deed, it depends on the maximum static friction force between the object A and the conveyor belt.

\hrule
\vspace{1em}


\noindent
\textbf{Q1142.} A shape is formed by drawing a triangle ABC inside the triangle ADE. BC is parallel to DE. AB = 8 cm, BC = p cm, DE = p + 6 cm.
[IMAGE:0]
is the bigger root of roots to the equation
[IMAGE:1]
.
Determine the length of AD.
[IMAGE:2]



\textbf{A.} 12cm \\
\textbf{B.} 15cm \\
\textbf{C.} 18cm \\
\textbf{D.} 20cm \\

\textbf{Answer:} D \\
\textbf{Explanation:} p
is the bigger root of roots to the equation
[IMAGE:0]
. So, p=4(yes),p=2(no)
.
Equal angles and BC || DE imply similarity.
So,
[IMAGE:1]
.
Solve
[IMAGE:2]
that AD=20cm

\hrule
\vspace{1em}


\noindent
\textbf{Q1143.} As shown in the figure: An object A with a weight of M = 1 kg is placed on a slope inclined at an angle of 37 degrees to the horizontal plane; the coefficient of friction is 0.5. A vertical upward force F is applied to this object. Find the minimum force F required to make the object A remain stationary on the slope.
[IMAGE:0]



\textbf{A.} [IMAGE:0] \\
\textbf{B.} 1010 \\
\textbf{C.} [IMAGE:1] \\
\textbf{D.} 5 \\

\textbf{Answer:} B \\
\textbf{Explanation:} [IMAGE:0]
The critical condition is satisfied when the maximum friction is occurred; The system is still in equilibrium; Assume the force is F, the normal component of the force is 0.6F; the effective component of F along the slope is:0.8F ; The total normal reaction is: 6-0.6F; Along the direction of the slope, we therefore have:0.5*(6-0.6F)+0.8F = 8; By solving the equation we have B

\hrule
\vspace{1em}


\noindent
\textbf{Q1144.} Which of the following statements about voltage in a parallel circuit is correct?
1.
The voltage across each branch in a parallel circuit is the same.
2.
The total current in a parallel circuit is the sum of the currents through each branch.
3.
The current is greater in the branch with lower resistance in a parallel circuit.



\textbf{A.} All of the above \\
\textbf{B.} 1 and 2 \\
\textbf{C.} 3 only \\
\textbf{D.} 1 and 3 \\

\textbf{Answer:} A \\
\textbf{Explanation:} All statements are correct. In a parallel circuit, the voltage across each branch is equal, the total current is the sum of the branch currents, and the current is greater in branches with lower resistance.

\hrule
\vspace{1em}


\noindent
\textbf{Q1145.} A bicycle starts to move from a stationary state on a horizontal road. It is driven by a constant driving force, and the air resistance increases as the speed increases. The following chart shows the changes of three quantities (X, Y, Z) of the motorcycle over time:
[IMAGE:0]



\textbf{A.} X: acceleration; Y: air resistance; Z: kinetic energy \\
\textbf{B.} X: weight; Y: velocity; Z: acceleration \\
\textbf{C.} X: velocity; Y: driving force; Z: weight \\
\textbf{D.} X: acceleration; Y: velocity; Z: weight \\

\textbf{Answer:} B \\
\textbf{Explanation:} The weight of the motorbike remains constant (mass and gravitational acceleration don’t change, so X is weight). At the start, acceleration (Z) is maximum, decreases as air resistance increases, and becomes zero at terminal velocity. Velocity (Y) increases and approaches terminal velocity.

\hrule
\vspace{1em}


\noindent
\textbf{Q1146.} A shape is formed by drawing a triangle ABC inside the triangle ADE. BC is parallel to DE. The height from A to BC is 3 cm, the height from A to DE is 9 cm, BC = n cm, and DE = n + 4 cm.
Find the length of DE.
[IMAGE:0]



\textbf{A.} 6cm \\
\textbf{B.} 8cm \\
\textbf{C.} 10cm \\
\textbf{D.} [IMAGE:0] \\

\textbf{Answer:} A \\
\textbf{Explanation:} Heights ratio is
[IMAGE:0]
, so sides ratio is also
[IMAGE:1]
. Thus,
[IMAGE:2]
. Solving gives n=2
, hence DE=n+4=6cm
.

\hrule
\vspace{1em}


\noindent
\textbf{Q1147.} As shown in the figure: A body with a mass of M = 2 kg is placed on a slope inclined at an angle of 30 degrees to the horizontal plane; the coefficient of friction is 0.2. The other end of the light rope is vertically suspended by a block of mass N. Find the minimum gravitational force N required to make the object move upward. gravitational acceleration is taken as 10
$𝑚$
/
$𝑠$
[IMAGE:0]



\textbf{A.} [IMAGE:0] \\
\textbf{B.} [IMAGE:1] \\
\textbf{C.} [IMAGE:2] \\
\textbf{D.} 50 \\

\textbf{Answer:} D \\
\textbf{Explanation:} [IMAGE:0]
When the maximum static friction force occurs, the condition is just met. The forces acting on the inclined plane are balanced in the direction of the slope:
[IMAGE:1]
; By solving the equation we have A

\hrule
\vspace{1em}


\noindent
\textbf{Q1148.} Which of the following statements about current in a series circuit is correct?
1.
The current is the same at all points in a series circuit.
2.
The total voltage in a series circuit is the sum of the voltages across each component.
3.
The current decreases where there is higher resistance in a series circuit.



\textbf{A.} All of the above \\
\textbf{B.} 1 and 2 \\
\textbf{C.} 3 only \\
\textbf{D.} 1 and 3 \\

\textbf{Answer:} B \\
\textbf{Explanation:} Statements 1 and 2 are correct. In a series circuit, the current is constant throughout, and the total voltage is the sum of the individual voltages. Statement 3 is incorrect because current remains constant in a series circuit regardless of resistance.

\hrule
\vspace{1em}


\noindent
\textbf{Q1149.} A shape is formed by drawing a triangle ABC inside the triangle ADE. BC is parallel to DE. The perimeter of triangle ABC is 24 cm, BC = m cm, DE = m + 8 cm, and the perimeter of triangle ADE is 40 cm.
Calculate the length of AD+AE.
[IMAGE:0]



\textbf{A.} 14cm \\
\textbf{B.} 16cm \\
\textbf{C.} 18cm \\
\textbf{D.} 120cm \\

\textbf{Answer:} D \\
\textbf{Explanation:} Perimeters of similar triangles are in the ratio of their sides. So,
[IMAGE:0]
. Solving gives m=12
, so DE=m+8=20cm
.
Thus, AD+AE=40-DE=20cm

\hrule
\vspace{1em}


\noindent
\textbf{Q1150.} A plastic ball starts to fall freely from a stationary state and is subject to air resistance proportional to its speed. Eventually, it will reach terminal velocity. Which row in the table can correctly identify the quantities X, Y and Z shown in the chart below?
[IMAGE:0]



\textbf{A.} X: acceleration; Y: velocity; Z: kinetic energy \\
\textbf{B.} X: acceleration; Y: air resistance; Z: weight \\
\textbf{C.} X: kinetic energy; Y: velocity; Z: resultant force \\
\textbf{D.} X: resultant force; Y: velocity; Z: weight \\

\textbf{Answer:} D \\
\textbf{Explanation:} When the plastic ball falls, the initial resultant force is equal to its weight (downward). As velocity increases, air resistance grows until it balances the weight, causing the resultant force to decrease to zero (at terminal velocity).
Thus:
X (Resultant force): Decreases from maximum to zero;
Y (Velocity): Increases from zero and asymptotes to terminal velocity;
Z (Weight): Remains constant because mass is unchanged.

\hrule
\vspace{1em}


\noindent
\textbf{Q1151.} Which of the following statements about the relationship between resistance, current, and voltage is correct?
1.
According to
[IMAGE:0]
​
, when voltage
V
increases, current
I
increases.
2.
According to
[IMAGE:1]
​
, when resistance
R
increases, current
I
decreases.
3.
Resistance
R
is independent of voltage
V
and current
I
.



\textbf{A.} All of the above \\
\textbf{B.} 1 and 2 \\
\textbf{C.} 3 only \\
\textbf{D.} 1 and 3 \\

\textbf{Answer:} A \\
\textbf{Explanation:} Statements 1 and 2 are correct as per Ohm's Law
[IMAGE:0]
​
, where current is directly proportional to voltage and inversely proportional to resistance. Statement 3 is also correct because resistance is an intrinsic property of the conductor and does not depend on voltage or current.

\hrule
\vspace{1em}


\noindent
\textbf{Q1152.} Which statements are correct?
1.
The magnitude of resistance is independent of the voltage across the conductor.
2.
When current flows through a conductor, its resistance decreases due to heating.
The reciprocal of resistance is called conductance.



\textbf{A.} 1 and 3 \\
\textbf{B.} 2 only \\
\textbf{C.} 2 and 3 \\
\textbf{D.} 1 only \\

\textbf{Answer:} A \\
\textbf{Explanation:} Statement 2 is false (resistance may increase with temperature, e.g., in metals).

\hrule
\vspace{1em}


\noindent
\textbf{Q1153.} A shape is formed by drawing a triangle ABC inside the triangle ADE. BC is parallel to DE. The area of triangle ABC is 12 cm², BC = 3w cm, DE = w + 5 cm, and the area of triangle ADE is 48 cm². Determine the length of DE.
[IMAGE:0]



\textbf{A.} 21cm \\
\textbf{B.} 15cm \\
\textbf{C.} 2cm \\
\textbf{D.} 8cm \\

\textbf{Answer:} E \\
\textbf{Explanation:} The ratio of areas of similar triangles is the square of the ratio of their sides. So,
[IMAGE:0]
. Thus,
[IMAGE:1]
. Solving gives w=1
, so DE=w+5=6cm
.

\hrule
\vspace{1em}


\noindent
\textbf{Q1154.} Which is correct about resistance?
1.
The SI unit of resistance is the ohm (Ω).
2.
Resistance of a conductor is independent of temperature.
Ohm’s Law applies to all conductors.



\textbf{A.} 1 only \\
\textbf{B.} 1 and 2 \\
\textbf{C.} 2 and 3 \\
\textbf{D.} All of the above \\

\textbf{Answer:} A \\
\textbf{Explanation:} Statement 2 is false (resistance changes with temperature). Statement 3 is false (Ohm’s Law applies only to ohmic materials).

\hrule
\vspace{1em}


\noindent
\textbf{Q1155.} An object is released from rest on an inclined plane and begins to slide down. During the sliding process, the frictional force acting on the object is constant. The graphs below show the variation with time of three quantities (X, Y, and Z)( Take into account air resistance, etc.)



\textbf{A.} X: Velocity; Y: Kinetic Energy; Z: Gravitational Potential Energy \\
\textbf{B.} X: Resultant Force; Y: Gravitational Potential Energy; Z: Kinetic Energy \\
\textbf{C.} X: Acceleration; Y: Velocity; Z: Gravitational Potential Energy \\
\textbf{D.} X: Kinetic Energy; Y: Gravitational Potential Energy; Z: Resultant Force \\

\textbf{Answer:} F \\
\textbf{Explanation:} Z is Acceleration: The acceleration decreases over time as the object slides down the incline. Although friction is constant, other resistive forces (like air resistance) may increase with velocity, causing acceleration to approach zero, matching the X graph.
Y is Kinetic Energy: Kinetic energy increases with the square of velocity. As velocity approaches a terminal value (when resultant force is zero), kinetic energy approaches a constant value, matching the Y graph.
X is Gravitational Potential Energy: As the object slides down, its height decreases, causing gravitational potential energy to decrease and approach zero (assuming the ground as the zero potential energy level), matching the Z graph.

\hrule
\vspace{1em}


\noindent
\textbf{Q1156.} A shape is formed by drawing a triangle ABC inside the triangle ADE. BC is parallel to DE. AC = 7 cm, BC = z cm, DE = 2z + 1 cm, CE = z + 1 cm.
Find the length of DE.
[IMAGE:0]



\textbf{A.} 11cm \\
\textbf{B.} 13cm \\
\textbf{C.} 15cm \\
\textbf{D.} [IMAGE:0] \\

\textbf{Answer:} C \\
\textbf{Explanation:} Using similarity,
[IMAGE:0]
. Since AE=AC+CE=7+(z+1)=z+8
, we get
[IMAGE:1]
. Solving yields z=7
(negative root -1 is discarded), hence
[IMAGE:2]
cm.

\hrule
\vspace{1em}


\noindent
\textbf{Q1157.} Which statement is incorrect?
1.
A superconductor has zero resistance.
2.
According to
[IMAGE:0]
, resistance is proportional to voltage.
Resistance of a metal conductor usually increases with temperature.



\textbf{A.} All are wrong \\
\textbf{B.} 2 only \\
\textbf{C.} 1 and 3 \\
\textbf{D.} 2 and 3 \\

\textbf{Answer:} B \\
\textbf{Explanation:} Statement 2 is false (resistance is material-dependent, not proportional to voltage). Statements 1 and 3 are correct.

\hrule
\vspace{1em}


\noindent
\textbf{Q1158.} Determine the correct statements:
1.
A conductor’s resistance is determined by its length, cross-sectional area, and material.
2.
If voltage doubles, resistance also doubles.
If current increases, resistance must decrease.



\textbf{A.} All of the above \\
\textbf{B.} 1 and 2 \\
\textbf{C.} 1 only \\
\textbf{D.} 2 and 3 \\

\textbf{Answer:} C \\
\textbf{Explanation:} Statements 2 and 3 are false (resistance is independent of voltage and current).

\hrule
\vspace{1em}


\noindent
\textbf{Q1159.} The parachutist starts to fall from a high altitude in a state of rest. During the fall, the air resistance he/she experiences is proportional to the speed of the parachute. The following chart shows the changes of three quantities (X, Y and Z) over time:
[IMAGE:0]
Which line in the table correctly identifies the quantities X, Y, and Z?



\textbf{A.} X: Velocity; Y: Kinetic Energy; Z: Weight \\
\textbf{B.} X: Resultant Force; Y: Gravitational Potential Energy; Z: Drag Force \\
\textbf{C.} X: Acceleration; Y: Velocity; Z: Drag Force \\
\textbf{D.} X: Kinetic Energy; Y: Gravitational Potential Energy; Z: Weight \\

\textbf{Answer:} F \\
\textbf{Explanation:} Explanation:
X is Weight: Weight is constant and independent of time, matching the horizontal Z graph.
Y is Velocity: Velocity increases during descent but approaches terminal velocity when drag equals weight, matching the Y graph's asymptotic behavior.
Z is Resultant Force: The resultant force (weight minus drag) decreases as velocity increases and approaches zero when drag equals weight, matching the decaying X graph.

\hrule
\vspace{1em}


\noindent
\textbf{Q1160.} A shape is formed by drawing a triangle ABC inside the triangle ADE. BC is parallel to DE. AC = 5 cm, BC = y cm, DE = y + 2 cm, EC = y - 3 cm.
Calculate the length of DE.
[IMAGE:0]



\textbf{A.} 6cm \\
\textbf{B.} 7cm \\
\textbf{C.} 8cm \\
\textbf{D.} 10cm \\

\textbf{Answer:} B \\
\textbf{Explanation:} Since BC || DE, triangles ABC and ADE are similar.
So,
[IMAGE:0]
.
Given
[IMAGE:1]
, we have
[IMAGE:2]
. Solving this gives
[IMAGE:3]
(discarding the negative root -2), so
[IMAGE:4]
.

\hrule
\vspace{1em}


\noindent
\textbf{Q1161.} As shown in the figure: A mass of m=10kg object is placed on a slope inclined at an angle of 37 degrees to the horizontal plane; the coefficient of friction is 0.2. There is an external force acting along the direction of the slope on this object. Find out the minimum force(N) required to make the object remain stationary on the slope. gravitational acceleration is taken as 10
$𝑚$
/
$𝑠$
[IMAGE:0]



\textbf{A.} [IMAGE:0] \\
\textbf{B.} [IMAGE:1] \\
\textbf{C.} [IMAGE:2] \\
\textbf{D.} 50 \\

\textbf{Answer:} D \\
\textbf{Explanation:} [IMAGE:0]
The critical condition is satisfied when the maximum friction is occurred; The system is still in equilibrium; Assume the force is F, the normal component of the force is 0; the effective component of F along the slope is:
F
; The total normal reaction is: 60; Along the direction of the slope, we therefore have:0.2*60+F = 80; By solving the equation we have D

\hrule
\vspace{1em}


\noindent
\textbf{Q1162.} Which statement about Ohm’s Law is correct?
1.
According to
[IMAGE:0]
​
, if voltage increases, resistance decreases.
2.
Resistance is an inherent property of a conductor, independent of voltage and current.
If current is zero, resistance is also zero.



\textbf{A.} All of the above \\
\textbf{B.} 1 and 2 \\
\textbf{C.} 2 only \\
\textbf{D.} 3 only \\

\textbf{Answer:} C \\
\textbf{Explanation:} Statement 1 is false (resistance does not change with voltage). Statement 3 is false (resistance depends on material, not current).

\hrule
\vspace{1em}


\noindent
\textbf{Q1163.} A shape is formed by drawing a triangle ABC inside the triangle ADE. BC is parallel to DE. AB = x-3 cm BC = x - 1 cm DE = x + 3 cm DB = 3 cm.
Calculate the length of DE.
[IMAGE:0]



\textbf{A.} 5cm \\
\textbf{B.} 7cm \\
\textbf{C.} 9cm \\
\textbf{D.} 10cm \\

\textbf{Answer:} E \\
\textbf{Explanation:} AB/AD=BC/DE.
[IMAGE:0]
with
[IMAGE:1]
[IMAGE:2]
.
[IMAGE:3]
.

\hrule
\vspace{1em}


\noindent
\textbf{Q1164.} As the diagram indicates: An object with weight 20N is on a slope with angle 30 degrees to the horizontal direction; The coefficient of friction is 1. An external force is exerted horizontally to the object: Find the minimum force that is required to move the object upwards
[IMAGE:0]



\textbf{A.} [IMAGE:0] \\
\textbf{B.} [IMAGE:1] \\
\textbf{C.} [IMAGE:2] \\
\textbf{D.} [IMAGE:3] \\

\textbf{Answer:} E \\
\textbf{Explanation:} [IMAGE:0]
The critical condition is satisfied when the maximum friction is occurred; The system is still in equilibrium; Assume the force is F, the normal component of the force is 0.5F; the effective component of F along the slope is:
[IMAGE:1]
; The total normal reaction is:
[IMAGE:2]
; Along the direction of the slope, we therefore have:
[IMAGE:3]
; By solving the equation we have E

\hrule
\vspace{1em}


\noindent
\textbf{Q1165.} A solid frustum of a cone with lower base radius R, upper base radius r=R/2, and height h fits inside a hollow cylinder. The cylinder has the same internal radius as the lower base radius of the frustum and a height equal to the height of the frustum. What fraction of the empty space is occupied in the cylinder?



\textbf{A.} [IMAGE:0] \\
\textbf{B.} [IMAGE:1] \\
\textbf{C.} [IMAGE:2] \\
\textbf{D.} [IMAGE:3] \\

\textbf{Answer:} C \\
\textbf{Explanation:} The volume of the frustum
[IMAGE:0]
. The volume of the cylinder
[IMAGE:1]
. The ratio
[IMAGE:2]
.
Thus, the answer is
[IMAGE:3]

\hrule
\vspace{1em}


\noindent
\textbf{Q1166.} An object starts from a state of rest and, under the action of a constant thrust in a fluid, accelerates its motion. It also encounters a resistance that is proportional to its speed in the fluid and eventually reaches the terminal velocity. The following chart shows the changes of three quantities (X, Y, Z) over time:
[IMAGE:0]
Which line in the table correctly identifies X, Y, and Z?



\textbf{A.} X: acceleration; Y: fluid resistance; Z: kinetic energy \\
\textbf{B.} X: acceleration; Y: mass of object; Z: weight \\
\textbf{C.} X: kinetic energy; Y: velocity; Z: potential energy \\
\textbf{D.} X: resultant force; Y: fluid resistance; Z: momentum \\

\textbf{Answer:} E \\
\textbf{Explanation:} The object experiences a constant thrust, but fluid resistance increases with velocity until it balances the thrust. The resultant force (X) decreases to zero (curve declines to zero). Velocity (Y) increases asymptotically toward terminal velocity (curve rises and stabilizes). The object’s mass remains constant, so its weight (Z) is unchanged (horizontal line). Option E is correct. Other options fail because: A’s Z (kinetic energy) should keep increasing; B’s Y (mass) and Z (weight) are constants, but X (acceleration) should decrease; C and D have mismatched physical quantities; F’s X (momentum) should increase, not decrease.

\hrule
\vspace{1em}


\noindent
\textbf{Q1167.} Which statements are correct?
1.
When gravity does negative work, gravitational potential energy increases.
2.
A satellite orbiting Earth has zero gravitational potential energy.
The zero point of gravitational potential energy can be chosen arbitrarily, but is often set at Earth’s surface.



\textbf{A.} 1 and 3 \\
\textbf{B.} 2 only \\
\textbf{C.} 2 and 3 \\
\textbf{D.} 1 only \\

\textbf{Answer:} A \\
\textbf{Explanation:} Statement 2 is false (satellite’s potential energy depends on the reference; it is negative if zero is at infinity).

\hrule
\vspace{1em}


\noindent
\textbf{Q1168.} Which is correct about gravitational potential energy?
1.
Gravitational potential energy belongs to the system of the object and the Earth.
2.
When an object falls from a height, its gravitational potential energy first increases then decreases.
The magnitude of gravitational potential energy is independent of the reference point.



\textbf{A.} 1 only \\
\textbf{B.} 1 and 3 \\
\textbf{C.} 2 and 3 \\
\textbf{D.} All of the above \\

\textbf{Answer:} A \\
\textbf{Explanation:} Statement 2 is false (potential energy decreases continuously during free fall). Statement 3 is false (magnitude depends on reference point).

\hrule
\vspace{1em}


\noindent
\textbf{Q1169.} As the diagram indicates: An object with weight 10N is on a slope with angle 30 degrees to the horizontal direction; The coefficient of friction is 0.25. An external force is exerted horizontally to the object: Find the minimum force that is required to move the object upwards
[IMAGE:0]



\textbf{A.} [IMAGE:0] \\
\textbf{B.} [IMAGE:1] \\
\textbf{C.} [IMAGE:2] \\
\textbf{D.} [IMAGE:3] \\

\textbf{Answer:} D \\
\textbf{Explanation:} [IMAGE:0]
The critical condition is satisfied when the maximum friction is occurred; The system is still in equilibrium; Assume the force is F, the normal component of the force is 0.5F; the effective component of F along the slope is:
[IMAGE:1]
; The total normal reaction is:
[IMAGE:2]
; Along the direction of the slope, we therefore have:
[IMAGE:3]
; By solving the equation we have D

\hrule
\vspace{1em}


\noindent
\textbf{Q1170.} A solid frustum of a cone with lower base radius R, upper base radius r=R/2, and height h fits inside a hollow cylinder. The cylinder has the same internal radius as the lower base radius of the frustum and a height equal to the height of the frustum. What fraction of the space inside the cylinder is occupied by the frustum?



\textbf{A.} [IMAGE:0] \\
\textbf{B.} [IMAGE:1] \\
\textbf{C.} [IMAGE:2] \\
\textbf{D.} [IMAGE:3] \\

\textbf{Answer:} D \\
\textbf{Explanation:} The volume of the frustum
[IMAGE:0]
. The volume of the cylinder
[IMAGE:1]
. The ratio
[IMAGE:2]
.

\hrule
\vspace{1em}


\noindent
\textbf{Q1171.} Which statement is incorrect?
1.
An increase in gravitational potential energy implies work is done against gravity.
2.
An object has the same gravitational potential energy at different heights.
3. The unit of gravitational potential energy is the joule.



\textbf{A.} All are wrong \\
\textbf{B.} 2 only \\
\textbf{C.} 1 and 3 \\
\textbf{D.} 2 and 3 \\

\textbf{Answer:} C \\
\textbf{Explanation:} Statement 2 is incorrect because gravitational potential energy depends on height. Statements 1 and 3 are correct.

\hrule
\vspace{1em}


\noindent
\textbf{Q1172.} A solid regular octahedron with edge length l fits inside a hollow sphere. The sphere has a diameter equal to the distance between two opposite vertices of the octahedron. What fraction of the space inside the sphere is taken up by the octahedron?



\textbf{A.} [IMAGE:0] \\
\textbf{B.} [IMAGE:1] \\
\textbf{C.} [IMAGE:2] \\
\textbf{D.} [IMAGE:3] \\

\textbf{Answer:} C \\
\textbf{Explanation:} The distance between two opposite vertices of a regular octahedron with edge length
[IMAGE:0]
is
[IMAGE:1]
. The volume of the octahedron
[IMAGE:2]
. The volume of the sphere
[IMAGE:3]
. The ratio
[IMAGE:4]
.

\hrule
\vspace{1em}


\noindent
\textbf{Q1173.} Determine the correct statements:
1.
The change in gravitational potential energy depends only on height difference, not the path.
2.
If gravity does positive work, the gravitational potential energy decreases.
3. Gravitational potential energy can be negative.



\textbf{A.} All of the above \\
\textbf{B.} 1 and 2 \\
\textbf{C.} 2 and 3 \\
\textbf{D.} 3 only \\

\textbf{Answer:} A \\
\textbf{Explanation:} All are correct. Gravitational potential energy is path-independent (conservative force); positive work by gravity reduces potential energy; if zero is at infinity, potential energy near Earth is negative.

\hrule
\vspace{1em}


\noindent
\textbf{Q1174.} As the diagram indicates, the spring stiffness of each spring is 12N/mm; the Force at the top is 48N; find the distance of the spring descends:
[IMAGE:0]



\textbf{A.} 0.25mm \\
\textbf{B.} 10mm \\
\textbf{C.} 4mm \\
\textbf{D.} 10m \\

\textbf{Answer:} C \\
\textbf{Explanation:} By parallel connection, the stiffness of the combined spring is 24N/mm, The extension is therefore 4mm.

\hrule
\vspace{1em}


\noindent
\textbf{Q1175.} A solid square pyramid with a square base of side length a
and height h
fits inside a hollow cube. The cube has an edge length equal to the slant height of the pyramid. And
[IMAGE:0]
. What fraction of the space inside the cube is occupied by the pyramid?
[IMAGE:1]



\textbf{A.} [IMAGE:0] \\
\textbf{B.} [IMAGE:1] \\
\textbf{C.} [IMAGE:2] \\
\textbf{D.} [IMAGE:3] \\

\textbf{Answer:} C \\
\textbf{Explanation:} The slant height of the pyramid
[IMAGE:0]
. The volume of the pyramid
[IMAGE:1]
. The volume of the cube
[IMAGE:2]
. , Thus, the ratio
[IMAGE:3]
.

\hrule
\vspace{1em}


\noindent
\textbf{Q1176.} An object starts to fall into the atmosphere from a stationary state and is influenced by gravity and resistance. Eventually, the object will reach a terminal velocity. The following chart shows the changes of three quantities (X, Y and Z) of the object over time:
[IMAGE:0]



\textbf{A.} X: Resultant Force; Y: Buoyant Force; Z: Kinetic Energy \\
\textbf{B.} X: Resultant Force; Y: Velocity; Z: Weight of Object \\
\textbf{C.} X: Buoyant Force; Y: Velocity; Z: Kinetic Energy \\
\textbf{D.} X: Buoyant Force; Y: Resultant Force; Z: Weight of Object \\

\textbf{Answer:} B \\
\textbf{Explanation:} X (Resultant Force): Decreases over time, matching the X graph.
Y (Velocity): Starts at zero and asymptotically approaches terminal velocity, matching the Y graph.
Z (Weight of Object): Remains constant, matching the horizontal Z graph.

\hrule
\vspace{1em}


\noindent
\textbf{Q1177.} Which statement about gravitational potential energy is correct?
1.
When an object is lifted and work is done by an external force, gravitational potential energy increases.
2.
When an object falls freely, gravitational potential energy is converted into kinetic energy.
3.
The zero point of gravitational potential energy must be defined at the Earth’s surface.



\textbf{A.} All of the above \\
\textbf{B.} 1 and 2 \\
\textbf{C.} 3 only \\
\textbf{D.} 1 only \\

\textbf{Answer:} B \\
\textbf{Explanation:} Statement 3 is incorrect because the zero point can be arbitrarily chosen (e.g., at infinity or the ground).

\hrule
\vspace{1em}


\noindent
\textbf{Q1178.} As the diagram indicates, the spring stiffness of each spring is 5N/mm; the weight at the bottom is 20N; find the distance of the weight descends:
[IMAGE:0]



\textbf{A.} 10mm \\
\textbf{B.} 2mm \\
\textbf{C.} 5mm \\
\textbf{D.} 20m \\

\textbf{Answer:} A \\
\textbf{Explanation:} By parallel connection, the stiffness of the combined spring is 40N/mm, the below part is 2.5N/mm; the resultant stiffness is 2N/mm(1/10+1/2.5). The extension is therefore 5mm.

\hrule
\vspace{1em}


\noindent
\textbf{Q1179.} Which of the following statements about the conversion between kinetic and potential energy is correct?
1.
When an object rises, kinetic energy is converted into gravitational potential energy.
2.
When an object falls, gravitational potential energy is converted into kinetic energy.
3. The total of kinetic and potential energy must remain constant.



\textbf{A.} All of the above \\
\textbf{B.} 1 and 2 \\
\textbf{C.} 3 only \\
\textbf{D.} 1 and 3 \\

\textbf{Answer:} B \\
\textbf{Explanation:} For option 3, the total of kinetic and potential energy (mechanical energy) remains constant only in the absence of non-conservative forces. If non-conservative forces (e.g., friction) are present, mechanical energy may decrease.

\hrule
\vspace{1em}


\noindent
\textbf{Q1180.} A solid torus (doughnut - shaped) with inner radius r and outer radius 2r fits inside a hollow cylinder. And its height is 3r. The cylinder has a radius equal to the outer radius of the torus and a height equal to the (outer) diameter of the torus's tube. What fraction of the space inside the cylinder is taken up by the torus?



\textbf{A.} [IMAGE:0] \\
\textbf{B.} [IMAGE:1] \\
\textbf{C.} [IMAGE:2] \\
\textbf{D.} [IMAGE:3] \\

\textbf{Answer:} D \\
\textbf{Explanation:} The answer is D.
The volume of the torus
[IMAGE:0]
. The volume of the cylinder
[IMAGE:1]
. The ratio
[IMAGE:2]
.

\hrule
\vspace{1em}


\noindent
\textbf{Q1181.} As the diagram indicates, the spring stiffness of each spring is 20N/mm; the weight at the bottom is 40N; find the distance of the weight descends:
[IMAGE:0]



\textbf{A.} 1mm \\
\textbf{B.} 2.5mm \\
\textbf{C.} 5mm \\
\textbf{D.} 1.25m \\

\textbf{Answer:} C \\
\textbf{Explanation:} By parallel connection, the stiffness of the combined spring is 40N/mm, the below part is 10N/mm; the resultant stiffness is 8N/mm(1/40+1/10). The extension is therefore 5mm.

\hrule
\vspace{1em}


\noindent
\textbf{Q1182.} Which of the following statements about work and energy is correct?
1.
Work is a measure of energy transformation.
2.
Work is a vector quantity with both magnitude and direction.
3.
The unit of work is the joule (J).



\textbf{A.} All of the above \\
\textbf{B.} 1 and 2 \\
\textbf{C.} 3 only \\
\textbf{D.} 1 and 3 \\

\textbf{Answer:} D \\
\textbf{Explanation:} For option 2, work is a scalar quantity and has magnitude only, not direction.

\hrule
\vspace{1em}


\noindent
\textbf{Q1183.} Which of the following statements about elastic potential energy is correct?
1.
Elastic potential energy is directly proportional to the displacement of the spring.
2.
The formula for elastic potential energy is
[IMAGE:0]
.
3.
Elastic potential energy is always positive.



\textbf{A.} All of the above \\
\textbf{B.} 1 and 2 \\
\textbf{C.} 3 only \\
\textbf{D.} 2 and 3 \\

\textbf{Answer:} D \\
\textbf{Explanation:} For option 1, the formula for elastic potential energy is
[IMAGE:0]
so it is proportional to the square of displacement, not directly proportional to displacement.

\hrule
\vspace{1em}


\noindent
\textbf{Q1184.} A solid regular tetrahedron with edge length a fits inside a hollow cube. The cube has an edge length equal to the edge length of the tetrahedron. What fraction of the space inside the cube is occupied by the tetrahedron?



\textbf{A.} [IMAGE:0] \\
\textbf{B.} [IMAGE:1] \\
\textbf{C.} [IMAGE:2] \\
\textbf{D.} [IMAGE:3] \\

\textbf{Answer:} A \\
\textbf{Explanation:} The height of a regular tetrahedron with edge length
[IMAGE:0]
is
[IMAGE:1]
. The volume of the tetrahedron
[IMAGE:2]
and the volume of the cube
[IMAGE:3]
. The ratio
[IMAGE:4]
.

\hrule
\vspace{1em}


\noindent
\textbf{Q1185.} A 5kg firework explodes in mid-air, splitting into two parts. One part with a mass of 2kg moves horizontally to the left at 10m/s after the explosion. What is the horizontal velocity of the other part?



\textbf{A.} 5 m/s to the right \\
\textbf{B.} 8 m/s to the right \\
\textbf{C.} 4 m/s to the right \\
\textbf{D.} 6.67 m/s to the right \\

\textbf{Answer:} D \\
\textbf{Explanation:} Momentum is conserved during the explosion, with the total momentum before the explosion being zero (as the firework was stationary). After the explosion, the momenta of the two parts are equal in magnitude but opposite in direction
[IMAGE:0]

\hrule
\vspace{1em}


\noindent
\textbf{Q1186.} A solid hemisphere of radius R is inside a hollow cone. The cone has a circular base with radius 2R and a height equal to 3R. What fraction of the space inside the cone is taken up by the hemisphere?



\textbf{A.} [IMAGE:0] \\
\textbf{B.} [IMAGE:1] \\
\textbf{C.} [IMAGE:2] \\
\textbf{D.} [IMAGE:3] \\

\textbf{Answer:} D \\
\textbf{Explanation:} [IMAGE:0]

\hrule
\vspace{1em}


\noindent
\textbf{Q1187.} Which of the following statements about conservation of mechanical energy is correct?
1.
Mechanical energy is conserved only when conservative forces do work.
2.
When mechanical energy is conserved, kinetic and potential energy can be converted into each other.
3.
The law of conservation of mechanical energy applies to all situations.



\textbf{A.} All of the above \\
\textbf{B.} 1 and 2 \\
\textbf{C.} 3 only \\
\textbf{D.} 1 and 3 \\

\textbf{Answer:} B \\
\textbf{Explanation:} For option 3, the law of conservation of mechanical energy applies only when only conservative forces are at work, not in the presence of non-conservative forces.

\hrule
\vspace{1em}


\noindent
\textbf{Q1188.} A 2kg object A is moving to the right at 6m/s and collides with a stationary 3kg object B. After the collision, object A moves to the left at 3m/s. What is the velocity of object B after the collision?



\textbf{A.} 3 m/s \\
\textbf{B.} 5 m/s \\
\textbf{C.} 4 m/s \\
\textbf{D.} 6 m/s \\

\textbf{Answer:} B \\
\textbf{Explanation:} According to the law of conservation of momentum, the total momentum of the system remains constant before and after the collision.
Solving:
[IMAGE:0]

\hrule
\vspace{1em}


\noindent
\textbf{Q1189.} Which of the following statements about kinetic energy is correct?
1.
Kinetic energy is a scalar quantity with magnitude only.
2.
The change in kinetic energy equals the work done by the net external force.
3.
Kinetic energy is directly proportional to the object's velocity.



\textbf{A.} All of the above \\
\textbf{B.} 1 and 2 \\
\textbf{C.} 3 only \\
\textbf{D.} 1 and 3 \\

\textbf{Answer:} B \\
\textbf{Explanation:} For option 3, the kinetic energy formula is
[IMAGE:0]
so kinetic energy is proportional to the square of velocity, not directly proportional to velocity.

\hrule
\vspace{1em}


\noindent
\textbf{Q1190.} A solid right - circular cone with radius r=3R and height h=4R fits inside a hollow cylinder. The cylinder has the same internal radius as the cone and a height equal to the slant height of the cone. What fraction of the space inside the cylinder is occupied by the cone?



\textbf{A.} [IMAGE:0] \\
\textbf{B.} [IMAGE:1] \\
\textbf{C.} [IMAGE:2] \\
\textbf{D.} [IMAGE:3] \\

\textbf{Answer:} D \\
\textbf{Explanation:} [IMAGE:0]

\hrule
\vspace{1em}


\noindent
\textbf{Q1191.} A weather balloon rises at terminal velocity while carrying instruments. Later, the instruments are released, reducing the balloon’s mass to 1/3 of its original and expanding its cross-sectional area to 2.5 times. Which description shows how the downward air resistance (drag) force varies with time after releasing the instruments? (Assume gravitational acceleration and air density remain constant.)



\textbf{A.} Resistance remains constant initially, then gradually decreases after release \\
\textbf{B.} Resistance remains constant initially, then sharply rises and stabilizes at a new value after release \\
\textbf{C.} Resistance increases linearly with time \\
\textbf{D.} Resistance remains constant initially, then sharply drops and stabilizes at a new value after release \\

\textbf{Answer:} D \\
\textbf{Explanation:} Initially, the downward air resistance balances the net upward force (buoyancy minus weight) at terminal velocity. After releasing the instruments, the mass reduces to m/3
and the cross-sectional area increases to 2.5A. Using the terminal velocity formula
[IMAGE:0]
the reduced mass significantly decreases the net downward force (potentially dominated by buoyancy). The downward resistance must now balance a smaller net force. Additionally, the increased A and reduced vt cause the drag force
[IMAGE:1]
to drop abruptly to a new equilibrium. The correct graph is option D.

\hrule
\vspace{1em}


\noindent
\textbf{Q1192.} A 4 kg object is moving with an initial velocity of 6 m/s. It collides with a stationary object of mass 2 kg. After the collision, the 4 kg object moves with a velocity of 0 m/s. What is the velocity of the 2 kg object after the collision? The table is without friction



\textbf{A.} 6 m/s \\
\textbf{B.} 8 m/s \\
\textbf{C.} 12 m/s \\
\textbf{D.} 3.5 m/s \\

\textbf{Answer:} C \\
\textbf{Explanation:} By conservation of momentum; The velocity * mass of the entire system is conserved before and after the collision

\hrule
\vspace{1em}


\noindent
\textbf{Q1193.} As the diagram indicates, the spring stiffness of each spring is 30N/mm; the box weight at the bottom is 36N; find the distance of the spring A descends:
[IMAGE:0]



\textbf{A.} 0.3mm \\
\textbf{B.} 0.6mm \\
\textbf{C.} 1.6mm \\
\textbf{D.} 3mm \\

\textbf{Answer:} B \\
\textbf{Explanation:} By parallel connection, the stiffness of the combined spring is 60N/mm, and the force is equal to the box weight which is 36N. The extension of the spring A is therefore 0.6mm.

\hrule
\vspace{1em}


\noindent
\textbf{Q1194.} Which of the following statements about motion is correct?
1.
An object will definitely remain at rest if no force acts on it.
2.
An object may move with uniform velocity if no force acts on it.
3.
Force is the reason for maintaining an object's motion.



\textbf{A.} All of the above \\
\textbf{B.} 1 and 2 \\
\textbf{C.} 2 and 3 \\
\textbf{D.} 2 only \\

\textbf{Answer:} D \\
\textbf{Explanation:} For options 1 and 3, an object may be at rest or in uniform motion when no force acts on it; force is the cause of changing motion, not maintaining it.

\hrule
\vspace{1em}


\noindent
\textbf{Q1195.} As the diagram indicates, the spring stiffness of each spring is 30N/mm; the weight at the bottom is 36N; find the distance of the weight descends:
[IMAGE:0]



\textbf{A.} 1mm \\
\textbf{B.} 3mm \\
\textbf{C.} 4mm \\
\textbf{D.} 7.5mm \\

\textbf{Answer:} B \\
\textbf{Explanation:} By parallel connection, the stiffness of the combined spring is 60N/mm, the below part is 15N/mm; the resultant stiffness is 12N/mm(1/(1/60+1/15)). The extension is therefore 3mm.

\hrule
\vspace{1em}


\noindent
\textbf{Q1196.} A solid cube of side length s fits perfectly inside a hollow rectangular prism. The rectangular prism has the same internal length and width as the side length of the cube, and its height is equal to the diagonal of the cube's base. What fraction of the space inside the rectangular prism is taken up by the cube?



\textbf{A.} [IMAGE:0] \\
\textbf{B.} [IMAGE:1] \\
\textbf{C.} [IMAGE:2] \\
\textbf{D.} [IMAGE:3] \\

\textbf{Answer:} D \\
\textbf{Explanation:} [IMAGE:0]
.

\hrule
\vspace{1em}


\noindent
\textbf{Q1197.} A snowboarder slides down the slope at a constant speed and gradually reaches a stable speed. When the snowboarder enters the smoother snow area, the resistance suddenly decreases and the snowboarder attains a new stable speed. Which picture shows the variation of resistance with time after the snowboarder achieves the stable speed for the first time?



\textbf{A.} [IMAGE:0] \\
\textbf{B.} [IMAGE:1] \\
\textbf{C.} [IMAGE:2] \\
\textbf{D.} [IMAGE:3] \\

\textbf{Answer:} B \\
\textbf{Explanation:} After entering the gentler slope, the resistance suddenly decreased and the skier accelerated. When the resistance and the component of gravity were balanced again, the resistance remained constant.

\hrule
\vspace{1em}


\noindent
\textbf{Q1198.} As the diagram indicates, the spring stiffness of each spring is 5N/mm; the weight at the bottom is 15N; find the distance of the weight descends:
[IMAGE:0]



\textbf{A.} 1mm \\
\textbf{B.} 2.5mm \\
\textbf{C.} 4mm \\
\textbf{D.} 7.5mm \\

\textbf{Answer:} D \\
\textbf{Explanation:} By parallel connection, the stiffness of the combined spring is 10N/mm, the below part is 2.5N/mm; the resultant stiffness is 2N/mm(1/(1/10+1/2.5)). The extension is therefore 7.5mm.

\hrule
\vspace{1em}


\noindent
\textbf{Q1199.} A metal sphere falls from a high altitude and reaches terminal velocity. Later, it splits into two identical smaller spheres, each with half the original mass and 0.5 times the original cross-sectional area. Which graph shows how the air resistance (drag) force varies with time before and after splitting? (Assume gravitational acceleration and air density remain constant.)



\textbf{A.} [IMAGE:0] \\
\textbf{B.} [IMAGE:1] \\
\textbf{C.} [IMAGE:2] \\
\textbf{D.} [IMAGE:3] \\

\textbf{Answer:} B \\
\textbf{Explanation:} Initially, the air resistance equals the sphere’s weight (mg), stabilizing at terminal velocity. After splitting, each smaller sphere has 0.5m mass and 0.6A cross-sectional area. Using the terminal velocity formula
[IMAGE:0]
the new terminal velocity becomes:
[IMAGE:1]
The reduced terminal velocity and smaller AAlower the drag force. The air resistance must now balance the new weight (0.5mg), causing an abrupt drop to the new equilibrium. The correct graph is option B.

\hrule
\vspace{1em}


\noindent
\textbf{Q1200.} A raindrop freely falls from the cloud layer and gradually reaches its terminal velocity. When the raindrop enters a layer within the cloud with a lower density, the resistance suddenly decreases and the raindrop attains a new terminal velocity. Which figure shows the variation of resistance with time?



\textbf{A.} [IMAGE:0] \\
\textbf{B.} [IMAGE:1] \\
\textbf{C.} [IMAGE:2] \\
\textbf{D.} [IMAGE:3] \\

\textbf{Answer:} B \\
\textbf{Explanation:} After entering the layer with lower density, the resistance drops sharply, causing the raindrops to accelerate. This will lead to an increase in resistance. When the resistance again balances with gravity, it remains constant.

\hrule
\vspace{1em}


\noindent
\textbf{Q1201.} The figure below shows the model of a spring-connected body on a horizontal surface; here, the masses of blocks P and Q are both m; the gravitational acceleration is g; the friction coefficients between the ground and each block are all 0.6; the spring constant k = 2 N/m; a horizontal force F is applied on block P; when P and Q can just move together to the right, find the elongation of the spring.
[IMAGE:0]



\textbf{A.} 0.1mg \\
\textbf{B.} 0.3mg \\
\textbf{C.} 0.4mg \\
\textbf{D.} 0.6mg \\

\textbf{Answer:} B \\
\textbf{Explanation:} At that moment, it was the spring extension when the force exerted by the spring on Q was exactly equal to the maximum static friction force between Q and the ground: 0.6mg/2 = 0.3mg

\hrule
\vspace{1em}


\noindent
\textbf{Q1202.} A solid sphere of radius r fits inside a hollow cylinder. The cylinder has the same internal diameter and length as the diameter of the sphere. What fraction of the empty space inside the cylinder is taken up?



\textbf{A.} [IMAGE:0] \\
\textbf{B.} [IMAGE:1] \\
\textbf{C.} [IMAGE:2] \\
\textbf{D.} [IMAGE:3] \\

\textbf{Answer:} A \\
\textbf{Explanation:} [IMAGE:0]

\hrule
\vspace{1em}


\noindent
\textbf{Q1203.} Which of the following statements about Newton's Third Law is correct?
1.
Action and reaction forces are produced and disappear simultaneously.
2.
Action and reaction forces act on the same object.
3.
Action and reaction forces are equal in magnitude and opposite in direction.



\textbf{A.} All of the above \\
\textbf{B.} 1 and 2 \\
\textbf{C.} 3 only \\
\textbf{D.} 1 and 3 \\

\textbf{Answer:} D \\
\textbf{Explanation:} For option 2, action and reaction forces act on two different interacting objects, not on the same object.

\hrule
\vspace{1em}


\noindent
\textbf{Q1204.} A hailstone falls from a cloud and reaches terminal velocity. Later, it partially melts, halving its mass and reducing its cross-sectional area to 0.49 times the original. Which graph shows how the air resistance (drag) force varies with time before and after melting? (Assume gravitational acceleration and air density remain constant.)



\textbf{A.} [IMAGE:0] \\
\textbf{B.} [IMAGE:1] \\
\textbf{C.} [IMAGE:2] \\
\textbf{D.} [IMAGE:3] \\

\textbf{Answer:} B \\
\textbf{Explanation:} Initially, the air resistance equals the hailstone’s weight (mg), stabilizing at terminal velocity. After melting, the mass halves (0.5m), and the cross-sectional area reduces to 0.49 times (0.49A). Using the terminal velocity formula
[IMAGE:0]
the new terminal velocity remains unchanged:
[IMAGE:1]
However, the air resistance must now balance the new weight (0.5mg). Despite the unchanged vt
​
, the reduced A directly lowers the drag force to 0.5mg. Thus, the resistance drops abruptly to the new equilibrium, corresponding to option B.

\hrule
\vspace{1em}


\noindent
\textbf{Q1205.} The figure below shows a four-link model.
The friction coefficients between each block are all 0.8; the ground is absolutely smooth; the mass of each block is m; the gravitational acceleration is g; now a horizontal force to the right is applied on block D. Find the maximum force F that can be applied to make all the blocks move to the right with the same acceleration together.
[IMAGE:0]



\textbf{A.} 0.1mg \\
\textbf{B.} 0.2mg \\
\textbf{C.} 0.4mg \\
\textbf{D.} 0.8mg \\

\textbf{Answer:} D \\
\textbf{Explanation:} The maximum static friction force acting on block D is: 0.8mg; the maximum static friction forces acting on the other blocks are all greater than this value; therefore, when F is greater than 0.8mg, block D cannot maintain relative rest with the other blocks.

\hrule
\vspace{1em}


\noindent
\textbf{Q1206.} Which of the following statements about force is correct?
1.
Force is a vector quantity with both magnitude and direction.
2.
Force composition follows the parallelogram law.
3.
The effect of a force depends only on its magnitude.



\textbf{A.} All of the above \\
\textbf{B.} 1 and 2 \\
\textbf{C.} 3 only \\
\textbf{D.} 1 and 3 \\

\textbf{Answer:} B \\
\textbf{Explanation:} For option 3, the effect of a force depends not only on magnitude but also on direction and point of application.

\hrule
\vspace{1em}


\noindent
\textbf{Q1207.} A metal ball is dropped freely from a certain height and falls into water, gradually reaching the terminal velocity. Then, the ball is subjected to a force in the direction of velocity for acceleration motion, and finally it reaches a new terminal velocity. Which figure shows the variation of resistance with time after the metal ball reaches the terminal velocity for the first time?



\textbf{A.} [IMAGE:0] \\
\textbf{B.} [IMAGE:1] \\
\textbf{C.} [IMAGE:2] \\
\textbf{D.} [IMAGE:3] \\

\textbf{Answer:} B \\
\textbf{Explanation:} After the ball reaches its terminal velocity for the first time, it is subjected to another force for acceleration. As its speed increases, the resistance also increases. When the resistance increases to the point where it is balanced with the force again, the resistance remains unchanged.

\hrule
\vspace{1em}


\noindent
\textbf{Q1208.} A single water droplet falls from the cloud layer and reaches terminal velocity. Subsequently, it merges with another droplet, doubling its mass and increasing its cross-sectional area by a factor of 1.9. Which figure shows the variation of air resistance (resistance) force with time before and after the merger? (Assume that the gravitational acceleration and air density remain unchanged.)



\textbf{A.} [IMAGE:0] \\
\textbf{B.} [IMAGE:1] \\
\textbf{C.} [IMAGE:2] \\
\textbf{D.} [IMAGE:3] \\

\textbf{Answer:} A \\
\textbf{Explanation:} Initially, the air resistance equals the raindrop’s weight, stabilizing at terminal velocity. After merging, the mass doubles, and the cross-sectional area increases by 1.9 times. Using the terminal velocity formula
[IMAGE:0]
the mass increase dominates over the area change, leading to a higher terminal velocity. The air resistance must now balance the new weight (2mg), causing an abrupt rise followed by stabilization. The correct graph is option A.

\hrule
\vspace{1em}


\noindent
\textbf{Q1209.} As shown in the figure: there is no friction on the ground; force F acts on the left side of object A; the masses of the three objects are 3m, 3m and 4m respectively; find the force between B and C.
[IMAGE:0]



\textbf{A.} 0.1mg \\
\textbf{B.} 0.3mg \\
\textbf{C.} 0.4mg \\
\textbf{D.} 0.8mg \\

\textbf{Answer:} C \\
\textbf{Explanation:} The three bodies have the same acceleration; the acceleration is m;
C has been pushed by B on the B-C boundary; the force is therefore: (4m)*F/(3m+3m+4m)=0.4F.

\hrule
\vspace{1em}


\noindent
\textbf{Q1210.} Which of the following statements about inertia is correct?
1.
Inertia is the property of an object to maintain its state of rest or uniform motion.
2.
The greater the object's speed, the greater its inertia.
3.
Inertia is related to the mass of the object.



\textbf{A.} All of the above \\
\textbf{B.} 1 and 2 \\
\textbf{C.} 3 only \\
\textbf{D.} 1 and 3 \\

\textbf{Answer:} D \\
\textbf{Explanation:} For option 2, inertia depends only on mass and is independent of speed.

\hrule
\vspace{1em}


\noindent
\textbf{Q1211.} An object freely falls from a certain height. When it enters the layer with higher air density, the resistance suddenly increases, and subsequently the gravitational acceleration also increases. Which figure shows the variation of resistance with time?



\textbf{A.} [IMAGE:0] \\
\textbf{B.} [IMAGE:1] \\
\textbf{C.} [IMAGE:2] \\
\textbf{D.} [IMAGE:3] \\

\textbf{Answer:} E \\
\textbf{Explanation:} After the resistance suddenly increases, it causes the object to decelerate. And air resistance is related to the speed. When the resistance is balanced with the gravity, the speed remains constant and the resistance is constant. However, due to the inconsistency of gravity before and after, the resistance at the time of balance when entering the denser layer will increase.

\hrule
\vspace{1em}


\noindent
\textbf{Q1212.} As shown in the figure: there is no friction on the ground; force F acts on the left side of object A; the masses of the three objects are 2m, 3m and 5m respectively; find the force between A and B.
[IMAGE:0]



\textbf{A.} 0.2mg \\
\textbf{B.} 0.8mg \\
\textbf{C.} 0.4mg \\
\textbf{D.} 1mg \\

\textbf{Answer:} B \\
\textbf{Explanation:} The three bodies have the same acceleration; the acceleration is m; B and C have been pushed by A on the A-B boundary; the force is therefore: (3m+5m)*F/(2m+3m+5m).

\hrule
\vspace{1em}


\noindent
\textbf{Q1213.} A rectangular prism has dimensions 2, 4, and 3. What is the perimeter of a triangular ABC (the dashed line triangular in the diagram below)? B is in the center of the bottom face. A and C lie in the vertices of the rectangular prism.
[IMAGE:0]



\textbf{A.} 2\sqrt{}14 \\
\textbf{B.} 2\sqrt{}14+2\sqrt{}5 \\
\textbf{C.} \sqrt{}14+2\sqrt{}5 \\
\textbf{D.} \sqrt{}14+4\sqrt{}5 \\

\textbf{Answer:} B \\
\textbf{Explanation:} AB=BC=\sqrt{}((4+16)/4+9)=\sqrt{}1 4
AC=\sqrt{}(4+16)=2\sqrt{}5
Thus, \sqrt{}14+\sqrt{}14+2\sqrt{}5=2\sqrt{}14+2\sqrt{}5

\hrule
\vspace{1em}


\noindent
\textbf{Q1214.} An object freely falls from a certain height and enters a fluid (such as air or water), gradually reaching the terminal velocity. When the object enters the denser layer of the fluid, the resistance suddenly increases and the object reaches a new terminal velocity. Which figure shows the variation of resistance with time?



\textbf{A.} [IMAGE:0] \\
\textbf{B.} [IMAGE:1] \\
\textbf{C.} [IMAGE:2] \\
\textbf{D.} [IMAGE:3] \\

\textbf{Answer:} A \\
\textbf{Explanation:} The variation of resistance is as follows:
When the object starts to fall, its speed gradually increases and so does the resistance (because resistance is proportional to the square of the speed).
When the resistance is balanced with the gravity, the object reaches the first terminal velocity and the resistance becomes stable.
After entering the fluid layer with higher density, the resistance coefficient suddenly increases, causing the resistance to rise sharply.
When the resistance is balanced with the gravity again, the object reaches a new terminal velocity and the resistance becomes stable once more.

\hrule
\vspace{1em}


\noindent
\textbf{Q1215.} As shown in the figure: the coefficient of friction on the ground is 0.2; force F acts on the left side of object A; the masses of the three objects are all m; find the minimum force F required to make block C receive a horizontal force. gravitational acceleration is taken as 10
$𝑚$
/
$𝑠$
[IMAGE:0]



\textbf{A.} 0.2mg \\
\textbf{B.} 0.6mg \\
\textbf{C.} 0.4mg \\
\textbf{D.} 3mg \\

\textbf{Answer:} C \\
\textbf{Explanation:} F only needs to be greater than the maximum static friction force of A and B.

\hrule
\vspace{1em}


\noindent
\textbf{Q1216.} The picture below shows a person pushing against a wall.
[IMAGE:0]



\textbf{A.} The force exerted by a person on a wall and the force exerted by the wall on the person are a pair of mutual forces. \\
\textbf{B.} There is no counteracting force acting vertically on the human body \\
\textbf{C.} There is no interaction force between the wall and the ground. \\
\textbf{D.} The gravitational force exerted on a person and the supporting force provided by the ground to a person are a pair of mutual forces. \\

\textbf{Answer:} A \\
\textbf{Explanation:} The force exerted by a person on a wall and the force exerted by the wall on the person are a pair of mutual forces.

\hrule
\vspace{1em}


\noindent
\textbf{Q1217.} It is known that a spherical object starts to accelerate in air under a constant force F1, and then moves at a constant speed after reaching a certain speed. Afterwards, the object is subjected to a pulse force F2 in the direction of its velocity. At a certain moment, this force F2 suddenly disappears. It is known that the air resistance is proportional to the velocity. Which of the following diagrams shows the variation of resistance with time?



\textbf{A.} [IMAGE:0] \\
\textbf{B.} [IMAGE:1] \\
\textbf{C.} [IMAGE:2] \\
\textbf{D.} [IMAGE:3] \\

\textbf{Answer:} A \\
\textbf{Explanation:} When F2 disappears, for the object to achieve force balance and come to a rest, it has to resume its original speed.

\hrule
\vspace{1em}


\noindent
\textbf{Q1218.} Which of the following statements about action and reaction forces is correct?
1.
Action and reaction forces are equal in magnitude and opposite in direction.
2.
Action and reaction forces act on the same object.
3.
Action and reaction forces have the same nature.



\textbf{A.} All of the above \\
\textbf{B.} 1 and \\
\textbf{C.} 3 only \\
\textbf{D.} 1 only \\

\textbf{Answer:} E \\
\textbf{Explanation:} For option 2, action and reaction forces act on two different interacting objects, not on the same object.

\hrule
\vspace{1em}


\noindent
\textbf{Q1219.} A cube has sides of length 2. What is the area of a triangular ABC (the dashed line triangular in the diagram below)?
[IMAGE:0]



\textbf{A.} 2\sqrt{}3 \\
\textbf{B.} 2(1+\sqrt{}2+\sqrt{}3) \\
\textbf{C.} 2\sqrt{}3 \\
\textbf{D.} 2\sqrt{}2 \\

\textbf{Answer:} D \\
\textbf{Explanation:} 1/2×2\sqrt{}2×2=2\sqrt{}2

\hrule
\vspace{1em}


\noindent
\textbf{Q1220.} The following picture is a schematic diagram of a rocket.
[IMAGE:0]



\textbf{A.} When
the rocket is not launched; the supporting force exerted by the rocket stand on the rocket and the pressure exerted by the rocket on the stand are a pair of equal and opposite forces. \\
\textbf{B.} The satellite launched by the rocket is in the outer space of the earth where it is not subjected to any force. Therefore, there is no reaction force acting on it. \\
\textbf{C.} The
gravitational force and the thrust of the gas on the rocket are a pair of reaction forces. \\
\textbf{D.} When the rocket is accelerating upwards, there is no force acting in the upward direction of the rocket. \\

\textbf{Answer:} A \\
\textbf{Explanation:} The supporting force exerted by the rocket stand on the rocket and the pressure exerted by the rocket on the stand are a pair of equal and opposite forces.

\hrule
\vspace{1em}


\noindent
\textbf{Q1221.} Which statements are incorrect?
1.
Friction always opposes motion.
2.
A stationary object experiences no forces.
3.
The unit of force is the newton



\textbf{A.} 1 and 2 \\
\textbf{B.} 3 only \\
\textbf{C.} 2 only \\
\textbf{D.} All are wrong \\

\textbf{Answer:} A \\
\textbf{Explanation:} Statement 1 is false (e.g., static friction can aid motion). Statement 2 is false (stationary objects can have balanced forces).

\hrule
\vspace{1em}


\noindent
\textbf{Q1222.} A cube has sides of length 2. What is the perimeter of a triangular ABC (the dashed line triangular in the diagram below)?
[IMAGE:0]



\textbf{A.} 2\sqrt{}3 \\
\textbf{B.} 2(1+\sqrt{}2+\sqrt{}3) \\
\textbf{C.} 2\sqrt{}3 \\
\textbf{D.} 2\sqrt{}2 \\

\textbf{Answer:} B \\
\textbf{Explanation:} 2+2\sqrt{}2+2\sqrt{}3=2(1+\sqrt{}2+\sqrt{}3)

\hrule
\vspace{1em}


\noindent
\textbf{Q1223.} The following picture shows the bullet being fired out.
[IMAGE:0]
[1]
When the bullet is fired; the thrust of the high-pressure gas on the bullet and the recoil force of the bullet on the gun are a pair of action-reaction forces.
[2]
The bullet is in the air; air resistance and the penetration force of the bullet into the air are a pair of reactive forces.
[3]
If the bullet is fired horizontally, there will be no force and reaction force acting on it in the vertical direction.



\textbf{A.} 1 only \\
\textbf{B.} 1 and 2 \\
\textbf{C.} 1 and 3 \\
\textbf{D.} 1
、
2
、
3 \\

\textbf{Answer:} B \\
\textbf{Explanation:} During the flight of the bullet, there is the effect of gravity in the vertical direction as well as some air resistance in the same direction.

\hrule
\vspace{1em}


\noindent
\textbf{Q1224.} Determine the correct statements:
1.
Force is a vector and has direction.
2.
The greater an object’s speed, the greater the force on it.
3.
Rockets launch by utilizing the principle of reaction



\textbf{A.} All of the above \\
\textbf{B.} 1 and 3 \\
\textbf{C.} 2 only \\
\textbf{D.} only \\

\textbf{Answer:} B \\
\textbf{Explanation:} Statement 2 is incorrect because speed does not directly correlate with force (e.g., constant speed implies zero net force).

\hrule
\vspace{1em}


\noindent
\textbf{Q1225.} In this picture, the person is moving forward while the boat is moving backward on the water.
[IMAGE:0]
[1]
The backward frictional force exerted by the heel of a person's foot on the boat and the forward frictional force exerted by the boat on the person are a pair of action-reaction forces.
[2]
The bottom of the ship exerts a backward thrust on the water, and the forward thrust of the water on the bottom of the ship is a pair of reaction forces.
When the boat is at rest, the support force exerted by the boat on the person and the pressure exerted by the person on the boat constitute a pair of action and reaction



\textbf{A.} 1 only \\
\textbf{B.} 1 and2 \\
\textbf{C.} 1 and 3 \\
\textbf{D.} 1
、
2
、
3 \\

\textbf{Answer:} D \\
\textbf{Explanation:} Both are right

\hrule
\vspace{1em}


\noindent
\textbf{Q1226.} A cube has sides of length 5. What is the length of a line joining the midpoint of one face to the midpoint of an adjacent face (the dashed line in the diagram below)?
[IMAGE:0]



\textbf{A.} (5\sqrt{}2)/2 \\
\textbf{B.} (5\sqrt{}5)/2 \\
\textbf{C.} 5 \\
\textbf{D.} (5\sqrt{}6)/2 \\

\textbf{Answer:} A \\
\textbf{Explanation:} [IMAGE:0]

\hrule
\vspace{1em}


\noindent
\textbf{Q1227.} Regarding Newton’s laws, which is correct?
1.
If an object has zero acceleration, the net force on it is zero.
2.
Force is the cause of maintaining an object’s motion.
3.
Action and reaction forces can cancel each other.



\textbf{A.} 1 only \\
\textbf{B.} 1 and 2 \\
\textbf{C.} 2 and 3 \\
\textbf{D.} All are wrong \\

\textbf{Answer:} A \\
\textbf{Explanation:} Statement 2 is incorrect (force changes motion, not maintains it). Statement 3 is false because action-reaction forces act on different objects.

\hrule
\vspace{1em}


\noindent
\textbf{Q1228.} The following figure is a schematic diagram of the elevator model.
[1]
When the elevator is accelerating upwards, the pressure exerted by the object on the elevator and the support force exerted by the elevator on the object constitute a pair of action-reaction forces.
[2]
When the elevator is accelerating downward, the pressure exerted by the object on the elevator and the supporting force exerted by the elevator on the object constitute a pair of action-reaction forces.
[3]
When the elevator accelerates, the gravitational force acting on the object block and the supporting force exerted by the elevator on the object block constitute a pair of action-reaction forces.



\textbf{A.} 1 only \\
\textbf{B.} 1 and2 \\
\textbf{C.} 1 and 3 \\
\textbf{D.} 3 only \\

\textbf{Answer:} B \\
\textbf{Explanation:} Gravity and support force are not a pair of action-reaction forces.

\hrule
\vspace{1em}


\noindent
\textbf{Q1229.} A heat sink dissipates heat through multiple copper cylindrical heat-conducting rods. One surface of the heat sink has a temperature of 90℃, and the other surface is in contact with air at 25℃. The length of each heat-conducting rod is l, and its diameter is d. Given that the cross-sectional area of one heat-conducting rod is A1​, calculate its heat transfer rate Q1​. If another heat-conducting rod has a cross-sectional area of A2​=2A1​ and the same length, what is its heat transfer rate Q2​?



\textbf{A.} 8Q
1 \\
\textbf{B.} 4Q
1 \\
\textbf{C.} 2Q
1 \\
\textbf{D.} 10Q
1 \\

\textbf{Answer:} C \\
\textbf{Explanation:} According to the heat transfer rate formula
[IMAGE:0]
, where
k
is the thermal conductivity,
A
is the cross-sectional area, Δ
T
is the temperature difference, and
L
is the length.
For the first heat-conducting rod:
[IMAGE:1]
For the second heat-conducting rod, since
A
2
​=2
A
1
​ and the length is the same:
[IMAGE:2]
a

\hrule
\vspace{1em}


\noindent
\textbf{Q1230.} A rectangular prism has dimensions 1, 2, and 3. What is the length of a line joining a vertex to the midpoint of the middle opposite edge (the dashed line in the diagram below)?
[IMAGE:0]



\textbf{A.} \sqrt{}7/2 \\
\textbf{B.} \sqrt{}7 \\
\textbf{C.} \sqrt{}14/2 \\
\textbf{D.} [IMAGE:0] \\

\textbf{Answer:} C \\
\textbf{Explanation:} [IMAGE:0]

\hrule
\vspace{1em}


\noindent
\textbf{Q1231.} Which of the following is correct?
1.
The direction of a force affects its effect.
2.
Without external force, an object must remain stationary.
3.
Action and reaction forces act on the same object



\textbf{A.} All of the above \\
\textbf{B.} 1 only \\
\textbf{C.} 1 and 3 \\
\textbf{D.} 2 and 3 \\

\textbf{Answer:} B \\
\textbf{Explanation:} Statement 2 violates Newton’s first law (object may move at constant velocity). Statement 3 is false because action-reaction forces act on different objects.

\hrule
\vspace{1em}


\noindent
\textbf{Q1232.} A physics student is experimenting with gas expansion. A balloon filled with helium is placed in water baths of different temperatures. The balloon has an initial volume of 2 liters at 20°C. The student observes how the volume changes when the balloon is placed in water at different temperatures. Assuming the pressure remains constant, which water temperature will cause the balloon to expand the most?



\textbf{A.} Water Temperature (°C): 30°C; Initial Volume (V₁): 2 liters; Initial Temperature (T₁): 20°C \\
\textbf{B.} Water Temperature (°C): 40°C; Initial Volume (V₁): 2 liters; Initial Temperature (T₁): 20°C \\
\textbf{C.} Water Temperature (°C): 10°C; Initial Volume (V₁): 2 liters; Initial Temperature (T₁): 20°C \\
\textbf{D.} Water Temperature (°C): 50°C; Initial Volume (V₁): 2 liters; Initial Temperature (T₁): 20°C \\

\textbf{Answer:} D \\
\textbf{Explanation:} The relationship between volume and temperature for an ideal gas at constant pressure is given by Charles's Law:
[IMAGE:0]
Option D (50°C) causes the balloon to expand the most because the highest temperature increase leads to the largest volume expansion according to Charles's Law.

\hrule
\vspace{1em}


\noindent
\textbf{Q1233.} A cube has sides of length 4. What is the length of a line joining a vertex to the midpoint of a face diagonal on the opposite face (the dashed line in the diagram below)?
[IMAGE:0]



\textbf{A.} 2\sqrt{}5 \\
\textbf{B.} 2\sqrt{}6 \\
\textbf{C.} 4\sqrt{}5 \\
\textbf{D.} 6 \\

\textbf{Answer:} B \\
\textbf{Explanation:} \sqrt{}((16+16)/4+16)=\sqrt{}2 4=2\sqrt{}6.

\hrule
\vspace{1em}


\noindent
\textbf{Q1234.} Choose the correct statement:
1.
Action and reaction forces are equal in magnitude.
2.
If an object is in equilibrium, its velocity must be zero.
3.
Force can change the motion state of an object



\textbf{A.} All of the above \\
\textbf{B.} 1 and 2 \\
\textbf{C.} 3 only \\
\textbf{D.} 1 only \\

\textbf{Answer:} E \\
\textbf{Explanation:} Statement 2 is incorrect because an object in equilibrium can be moving at constant velocity.

\hrule
\vspace{1em}


\noindent
\textbf{Q1235.} A science class is studying thermal expansion. They have four identical metal rods (same length and cross-sectional area) made of different materials. Each rod is heated from 20°C to 100°C. The coefficient of linear expansion (
α
) for each material is given below.
Which rod will expand the most in length?



\textbf{A.} Material X, Coefficient of Linear Expansion (
α
):
12
×10
−
6
°C
−
1 \\
\textbf{B.} Material Y, Coefficient of Linear Expansion (
α
):
18
×10
−
6
°C
−
1 \\
\textbf{C.} Material Z, Coefficient of Linear Expansion (
α
):
8
×10
−
6
°C
−
1 \\
\textbf{D.} Material W, Coefficient of Linear Expansion (
α
):
15
×10
−
6
°C
−
1 \\

\textbf{Answer:} B \\
\textbf{Explanation:} The change in length (Δ
L
) due to thermal expansion is given by:
[IMAGE:0]
Since all rods have the same original length (
L
) and temperature change (Δ
T
=100−20=80°
C
), the expansion depends only on
α
.
Option B (Material Y) has the highest coefficient of linear expansion (
α
=18×10
−6
°C−1), so it will expand the most.

\hrule
\vspace{1em}


\noindent
\textbf{Q1236.} As the diagram indicates: the ground is frictionless; the force
F
is exerting on the left hand side of the body A; three bodies (A, B and C) have mass
m
,
2m
, and
3m
seperately; Find the force between A and B:
[IMAGE:0]



\textbf{A.} F/2 \\
\textbf{B.} 5F/6 \\
\textbf{C.} 7F/9 \\
\textbf{D.} F/3 \\

\textbf{Answer:} B \\
\textbf{Explanation:} The three bodies have the same acceleration; the acceleration is F/6m; A only has been pushed by force
F
to the right and pushed by "B-C combination" on the A-B boundary to the left;
[IMAGE:0]
the force is therefore: 5F/6.

\hrule
\vspace{1em}


\noindent
\textbf{Q1237.} A cube has sides of length 3. What is the length of a line joining the midpoint of one edge to the midpoint of an opposite edge (the dashed line in the diagram below)?
[IMAGE:0]



\textbf{A.} 6\sqrt{}2 \\
\textbf{B.} 5\sqrt{}2 \\
\textbf{C.} 3\sqrt{}2 \\
\textbf{D.} 2\sqrt{}2 \\

\textbf{Answer:} C \\
\textbf{Explanation:} [IMAGE:0]
.

\hrule
\vspace{1em}


\noindent
\textbf{Q1238.} As the diagram indicates: the ground is frictionless; the force
F
is exerting on the left hand side of the body A; three bodies (A, B and C) have mass
3m
,
2m
, and
m
seperately; Find the force between A and B:
[IMAGE:0]



\textbf{A.} F \\
\textbf{B.} F/2 \\
\textbf{C.} 2F/3 \\
\textbf{D.} F/3 \\

\textbf{Answer:} B \\
\textbf{Explanation:} The three bodies have the same acceleration; the acceleration is F/6m; A only has been pushed by force
F
to the right and pushed by "B-C combination" on the A-B boundary to the left;
[IMAGE:0]
the force is therefore: F/2.

\hrule
\vspace{1em}


\noindent
\textbf{Q1239.} A scientist wants to compare the thermal conductivity of different materials. Four identical rods (same length and cross-sectional area) made of different materials are placed between two reservoirs of water. The temperature of the left reservoir is 90°C, and the temperature of the right reservoir is 30°C. The rods are made of materials with different thermal conductivities.
Which rod will conduct thermal energy at the highest rate?



\textbf{A.} Material: X; Thermal Conductivity (k): 100 W/(m\cdot K) \\
\textbf{B.} Material: Y; Thermal Conductivity (k): 200 W/(m\cdot K) \\
\textbf{C.} Material: Z; Thermal Conductivity (k): 50 W/(m\cdot K) \\
\textbf{D.} Material: W; Thermal Conductivity (k): 300 W/(m\cdot K) \\

\textbf{Answer:} D \\
\textbf{Explanation:} Since all rods have the same A, L, and ΔT, the heat transfer rate depends only on the thermal conductivity k. Option D (Material W) has the highest thermal conductivity (k=300W/(m\cdotpK)), so it will conduct thermal energy at the highest rate.

\hrule
\vspace{1em}


\noindent
\textbf{Q1240.} A science experiment uses different materials to study heat conduction. Four metal rods of the same length (1m) and cross-sectional area (1m²) are connected between two water tanks. The temperature of one tank is fixed at 80°C, and the other tank is at temperature θ. The rods are made of different materials with varying thermal conductivities:
Copper (k=400W/(m\cdotpK))
Aluminum (k=200W/(m\cdotpK))
Iron (k=80W/(m\cdotpK))
Silver (k=420W/(m\cdotpK))
Which rod conducts thermal energy at the highest rate under the given conditions?



\textbf{A.} Material: Copper; Water Temperature (θ): 30°C \\
\textbf{B.} Material: Aluminum; Water Temperature (θ): 40°C \\
\textbf{C.} Material: Iron; Water Temperature (θ): 20°C \\
\textbf{D.} Material: Silver; Water Temperature (θ):50°C \\

\textbf{Answer:} D \\
\textbf{Explanation:} The rate of heat transfer (Q/t) through conduction is given by Fourier's Law:
[IMAGE:0]
Since all rods have the same A, L, and the temperature difference ΔT is determined by the water temperatures:
[IMAGE:1]
Calculate the ratio of heat conduction rates:
[IMAGE:2]
Option D (Silver) has the highest heat transfer rate because silver has the highest thermal conductivity (k=420), which outweighs its smaller temperature difference.

\hrule
\vspace{1em}


\noindent
\textbf{Q1241.} As the diagram indicates: the ground is frictionless; the force
F
is exerting on the left hand side of the body A; three bodies are with mass
m
; Find the force between A and B:
[IMAGE:0]



\textbf{A.} F \\
\textbf{B.} F/2 \\
\textbf{C.} 2F/3 \\
\textbf{D.} F/3 \\

\textbf{Answer:} C \\
\textbf{Explanation:} The three bodies have the same acceleration; the acceleration is F/3m; A only has been pushed by force
F
to the right and pushed by "B-C combination" on the A-B boundary to the left;
[IMAGE:0]
the force is therefore: 2F/3.

\hrule
\vspace{1em}


\noindent
\textbf{Q1242.} A liquid flows through a pipe with a certain flow rate.
The pipe is surrounded by a fluid at a higher temperature, causing heat to be transferred to the liquid via convection.
Which liquid will experience the highest rate of thermal energy transfer?



\textbf{A.} Liquid: Water; Mass Flow Rate (kg/s): 5; Specific Heat (J/kg\cdot °C): 4186; Temperature Difference (°C): 20 \\
\textbf{B.} Liquid: Ethanol; Mass Flow Rate (kg/s): 3; Specific Heat (J/kg\cdot °C): 2440; Temperature Difference (°C): 25 \\
\textbf{C.} Liquid: Mercury; Mass Flow Rate (kg/s): 2; Specific Heat (J/kg\cdot °C): 139; Temperature Difference (°C): 30 \\
\textbf{D.} Liquid: Glycerin; Mass Flow Rate (kg/s): 4; Specific Heat (J/kg\cdot °C): 2410; Temperature Difference (°C): 15 \\

\textbf{Answer:} A \\
\textbf{Explanation:} The rate of thermal energy transfer due to convection depends on the mass flow rate of the liquid, its specific heat capacity, and the temperature difference between the liquid and the surrounding fluid. The formula for the rate of thermal energy transfer is:
[IMAGE:0]

\hrule
\vspace{1em}


\noindent
\textbf{Q1243.} As the diagram indicates: the ground is frictionless; the force F=5N
is exerting on the left hand side of the body A; three bodies are with mass m=2kg
; Find the force between B and C (round to one decimal place):
[IMAGE:0]



\textbf{A.} 5.0 \\
\textbf{B.} 2.5 \\
\textbf{C.} 3.3 \\
\textbf{D.} 1.7 \\

\textbf{Answer:} D \\
\textbf{Explanation:} The three bodies have the same acceleration; the acceleration is F/3m; the third one only has been pushed by B on the B-C boundary; the force is therefore:
[IMAGE:0]
.

\hrule
\vspace{1em}


\noindent
\textbf{Q1244.} A metal sphere is placed in a large container of hot water at 80°C. The sphere is made of a material with thermal conductivity k, mass m, specific heat capacity c, and initial temperature θ.
Which sphere will reach the water temperature fastest?



\textbf{A.} Material (k): Aluminum (200); Mass m (kg): 1; Specific Heat c (J/kg\cdot °C): 897; Initial Temp θ (°C): 10 \\
\textbf{B.} Material (k): Iron (80); Mass m (kg): 3; Specific Heat c (J/kg\cdot °C): 450; Initial Temp θ (°C): 25 \\
\textbf{C.} Material (k): Copper (400); Mass m (kg): 2; Specific Heat c (J/kg\cdot °C): 385; Initial Temp θ (°C): 20 \\
\textbf{D.} Material (k): Lead (35); Mass m (kg): 0.5; Specific Heat c (J/kg\cdot °C): 128; Initial Temp θ (°C): 15 \\

\textbf{Answer:} C \\
\textbf{Explanation:} The rate at which the sphere heats up depends on the thermal energy conducted into it and its heat capacity. The key factors are:
Thermal Conductivity (k): Higher k means faster heat transfer.
Heat Capacity (m\cdot c): Lower heat capacity means the sphere heats up faster for the same energy input.
The effective rate of heating is proportional to:
[IMAGE:0]

\hrule
\vspace{1em}


\noindent
\textbf{Q1245.} A object with zero intial velocity and zero initial acceleration is placed on a horizontal conveyor belt; find the correct statement:
[IMAGE:0]
The gravity of object A and the normal reaction force of the conveyor belt on object A is an action and reaction force pair;
When the conveyor belt moves at a constant speed, the gravity and the normal reaction force of the conveyor belt on the object are balanced;
If the conveyor belt suddenly accelerates, the object will slide backward relative to the conveyor belt;



\textbf{A.} 1 only \\
\textbf{B.} 1 and 2 \\
\textbf{C.} 2 and 3 \\
\textbf{D.} 2 only \\

\textbf{Answer:} D \\
\textbf{Explanation:} The ground is fixed as can be seen in the graph. "1." The gravity of object A and the normal reaction force of the conveyor belt on object A is a balanced forces pair (which is not the same as "action and reaction force pair"). "2." is obviously right. "3." Object A may not necessarily slide. In deed, it depends on the maximum static friction force between the object A and the conveyor belt.

\hrule
\vspace{1em}


\noindent
\textbf{Q1246.} The object A is placed on a circular disc rotating at a constant speed without sliding between them; the disc is rough and the object rotate with the disc simultaneously; find the correct statement:
[IMAGE:0]
The gravity of object A and the normal reaction force of the disc on object A is an action and reaction force pair;
When the disc rotates at a constant speed, the gravity and the normal reaction force of the disc on the object are balanced;
If the disc suddenly stops rotating, the object will continue to move in the direction of the radius of the disk;



\textbf{A.} 1 only \\
\textbf{B.} 1 and 2 \\
\textbf{C.} 2 and 3 \\
\textbf{D.} 2 only \\

\textbf{Answer:} D \\
\textbf{Explanation:} The ground is fixed as can be seen in the graph. "1." The gravity of object A and the normal reaction force of the disc on object A is a balanced forces pair (which is not the same as "action and reaction force pair"). "2." is obviously right. "3." If the disc suddenly stops rotating, the object will continue to move in the tangential direction.

\hrule
\vspace{1em}


\noindent
\textbf{Q1247.} Two tanks of water are connected by a solid cylindrical metal rod. The rod is insulated. One tank contains water at 100°C, and the other tank contains water at temperature θ.
Which of the following conditions results in the highest rate of thermal energy conduction along the rod?



\textbf{A.} Materials (k): Copper (400); Length (m): 1;
Diameter (m): 0.02; Temperature θ (°C): 30 \\
\textbf{B.} Materials (k): Aluminum (200); Length (m): 0.5;
Diameter (m): 0.01; Temperature θ (°C): 40 \\
\textbf{C.} Materials (k): Steel (50); Length (m): 2;
Diameter (m): 0.03; Temperature θ (°C): 20 \\
\textbf{D.} Materials (k): Glass (10); Length (m): 0.25;
Diameter (m): 0.04; Temperature θ (°C): 50 \\

\textbf{Answer:} A \\
\textbf{Explanation:} The formula for the heat conduction rate is:
[IMAGE:0]
k represents the thermal conductivity of the material.
A represents the cross-sectional area.
Δ
T represents the temperature difference.
L represents the length.

\hrule
\vspace{1em}


\noindent
\textbf{Q1248.} The object A with zero velocity and zero acceleration is hanging from a spring; find the correct statement:
[IMAGE:0]
The gravity of object A and the tension force of the spring on object A is an action and reaction force pair;
When the object is in equilibrium, the magnitude of gravity is equal to the tension force of the spring on the object;
If the spring is now cut, the object will fall freely;



\textbf{A.} 1 only \\
\textbf{B.} 1 and 2 \\
\textbf{C.} 2 and 3 \\
\textbf{D.} 3 only \\

\textbf{Answer:} C \\
\textbf{Explanation:} The ground is fixed as can be seen in the graph. "1." The gravity of object A and the tension force of the spring on object A is a balanced forces pair (which is not the same as "action and reaction force pair"). "2." is obviously right. "3." is obviously right.

\hrule
\vspace{1em}


\noindent
\textbf{Q1249.} The object A with zero velocity and zero acceleration is placed on an inclined plane; find the correct statement:
[IMAGE:0]
The gravity of object A and the normal reaction force of the inclined plane on object A is a balanced forces pair;
The magnitude of normal component of gravity is equal to the magnitude of normal reaction force of the inclined plane on the object;
If the inclined plane is now a smooth surface with no friction; the object will slide down the plane;



\textbf{A.} 1 only \\
\textbf{B.} 2 and 3 \\
\textbf{C.} 1 and 3 \\
\textbf{D.} 3 only \\

\textbf{Answer:} B \\
\textbf{Explanation:} The inclined plane is fixed as can be seen in the graph. "1." the two forces have different directions. "2." is obviously right. "3." is obviously right because of the component of gravity (parallel to the inclined plane).

\hrule
\vspace{1em}


\noindent
\textbf{Q1250.} The object A with zero velocity and zero acceleration is placed on the ground; find the correct statement:
[IMAGE:0]
The gravity of A and the normal reaction by the ground to A is an action and reaction force pair;
The magnitude of the gravity equals the normal reaction force from the ground to the object when in equilibrium;
The normal reaction by the ground to A can be a negative;



\textbf{A.} 1 only \\
\textbf{B.} 1 and 2 \\
\textbf{C.} 1 and 3 \\
\textbf{D.} 2 only \\

\textbf{Answer:} D \\
\textbf{Explanation:} The ground is fixed as can be seen in the graph. "1." The gravity of A and the normal reaction by the ground to A is a balanced forces pair (which is not the same as "an action and reaction force pair"). "2." is obviously right. "3." is obviously wrong.

\hrule
\vspace{1em}


\noindent
\textbf{Q1251.} Two tanks of water are connected by a solid cylindrical copper bar of length l and diameter d. The bar is insulated.One tank contains water at 90 °C and the other tank
contains water at temperature θ.
Under which of the following circumstances can the two water tanks reach thermal equilibrium the fastest?



\textbf{A.} L: 1m; D:
[IMAGE:0]
;
Water tmerpature: 40 degrees \\
\textbf{B.} L: 1m; D:
[IMAGE:1]
;
Water tmerpature: 53 degrees \\
\textbf{C.} L: 7m; D:
[IMAGE:2]
;
Water tmerpature: 40 degrees \\
\textbf{D.} L: 14m; D:
[IMAGE:3]
;
Water tmerpature: 20 degrees \\

\textbf{Answer:} B \\
\textbf{Explanation:} The heat transfer rate is proportional to the temperature difference and the cross-sectional area; inversely proportional to the length between;
w
hen the length of the rod and the cross-sectional area are equal, the smaller the temperature difference is, the shorter the time required to reach equilibrium will be.

\hrule
\vspace{1em}


\noindent
\textbf{Q1252.} As the diagram indicates: An object with weight 6N is on a slope with angle 45 degrees to the horizontal direction; The coefficient of friction is 0.5 and the maximum static friction
[IMAGE:0]
. An external force F
is exerted to the object: Find the minimum force that is required to move the object downwards
[IMAGE:1]



\textbf{A.} [IMAGE:0] \\
\textbf{B.} [IMAGE:1] \\
\textbf{C.} [IMAGE:2] \\
\textbf{D.} [IMAGE:3] \\

\textbf{Answer:} B \\
\textbf{Explanation:} [IMAGE:0]
[IMAGE:1]
, so the friction must be maximum static friction
[IMAGE:2]
[IMAGE:3]

\hrule
\vspace{1em}


\noindent
\textbf{Q1253.} Two tanks of water are connected by a solid cylindrical copper bar of length l and diameter d. The bar is insulated.One tank contains water at 90 °C and the other tank
contains water at temperature θ.
For which of the following conditions is thermal energy conducted along the bar at the Highest rate?



\textbf{A.} L:
1
m
; D:
[IMAGE:0]
;
Water tmerpature: 40 degrees \\
\textbf{B.} L:
2
m
; D:
[IMAGE:1]
;
Water tmerpature: 43 degrees \\
\textbf{C.} L:
8
m
; D:
[IMAGE:2]
;
Water tmerpature: 40 degrees \\
\textbf{D.} L:
15
m
; D:
[IMAGE:3]
;
Water tmerpature: 20 degrees \\

\textbf{Answer:} A \\
\textbf{Explanation:} The heat transfer rate is proportional to the temperature difference and the cross-sectional area; inversely proportional to the length between.

\hrule
\vspace{1em}


\noindent
\textbf{Q1254.} A cube has sides of length 3. What is the length of a line joining the midpoint of one edge to the midpoint of an opposite edge (the dashed line in the diagram below)?
[IMAGE:0]



\textbf{A.} 6\sqrt{}2 \\
\textbf{B.} 5\sqrt{}2 \\
\textbf{C.} 3\sqrt{}2 \\
\textbf{D.} 2\sqrt{}2 \\

\textbf{Answer:} C \\
\textbf{Explanation:} [IMAGE:0]
.

\hrule
\vspace{1em}


\noindent
\textbf{Q1255.} A rectangular prism has dimensions 2, 4, and 3. What is the length of a line joining a vertex to the center of the opposite face (the dashed line in the diagram below)?
[IMAGE:0]



\textbf{A.} \sqrt{}14 \\
\textbf{B.} \sqrt{}17 \\
\textbf{C.} \sqrt{}21 \\
\textbf{D.} \sqrt{}29 \\

\textbf{Answer:} A \\
\textbf{Explanation:} \sqrt{}((4+16)/4+9)=\sqrt{}1 4

\hrule
\vspace{1em}


\noindent
\textbf{Q1256.} As the diagram indicates: An object with weight 6N is on a slope with angle 45 degrees to the horizontal direction; The coefficient of friction is 0.5. An external force
[IMAGE:0]
is exerted to the object: Find the minimum force that is required to move the object upwards
[IMAGE:1]



\textbf{A.} [IMAGE:0] \\
\textbf{B.} [IMAGE:1] \\
\textbf{C.} [IMAGE:2] \\
\textbf{D.} [IMAGE:3] \\

\textbf{Answer:} C \\
\textbf{Explanation:} [IMAGE:0]
[IMAGE:1]
[IMAGE:2]

\hrule
\vspace{1em}


\noindent
\textbf{Q1257.} The densities of two metals P and Q are
[IMAGE:0]
and
[IMAGE:1]
respectively. What is the density of an alloy made from equal volumes of metals P and Q (with the total volume remaining unchanged)?



\textbf{A.} [IMAGE:0] \\
\textbf{B.} [IMAGE:1] \\
\textbf{C.} [IMAGE:2] \\
\textbf{D.} [IMAGE:3] \\

\textbf{Answer:} C \\
\textbf{Explanation:} The alloy's density is
[IMAGE:0]
since the total mass is the sum of individual masses (
[IMAGE:1]
) divided by the doubled volume (
[IMAGE:2]
).

\hrule
\vspace{1em}


\noindent
\textbf{Q1258.} The densities of two metals P and Q are
[IMAGE:0]
and
[IMAGE:1]
respectively. What is the density of an alloy made from equal masses of metals P and Q (with the total volume remaining unchanged)?



\textbf{A.} [IMAGE:0] \\
\textbf{B.} [IMAGE:1] \\
\textbf{C.} [IMAGE:2] \\
\textbf{D.} [IMAGE:3] \\

\textbf{Answer:} D \\
\textbf{Explanation:} The alloy's density is
[IMAGE:0]
because the total mass is doubled while the combined volume adds the individual volumes (
[IMAGE:1]
).

\hrule
\vspace{1em}


\noindent
\textbf{Q1259.} A steel cylinder with a volume of 0.1m3 contains oxygen with a density of 8kg/m3. During welding, 4 of 1 of the oxygen is used. Then, 0.6kg of another gas with a density of 12kg/m3 is added. Assuming no leakage occurs, what is the density of the gas mixture in the cylinder?



\textbf{A.} [IMAGE:0] \\
\textbf{B.} [IMAGE:1] \\
\textbf{C.} [IMAGE:2] \\
\textbf{D.} [IMAGE:3] \\

\textbf{Answer:} D \\
\textbf{Explanation:} Initial oxygen mass:
[IMAGE:0]
.
Remaining oxygen after usage:
Used oxygen =
[IMAGE:1]
.
Remaining oxygen =
[IMAGE:2]
.
Total mass after adding gas:
Added gas mass =
[IMAGE:3]
.
Total mass =
[IMAGE:4]
.
Density of the mixture:
Total volume remains
[IMAGE:5]
.
[IMAGE:6]
.
(Note that the information of the density of "another gas" is useless)

\hrule
\vspace{1em}


\noindent
\textbf{Q1260.} A container filled with water has a total mass of 450g. When a 200g small stone is placed into the container, water overflows. After removing the excess water and measuring again, the total mass becomes 550g. What is the density of the small stone?



\textbf{A.} [IMAGE:0] \\
\textbf{B.} [IMAGE:1] \\
\textbf{C.} [IMAGE:2] \\
\textbf{D.} [IMAGE:3] \\

\textbf{Answer:} C \\
\textbf{Explanation:} The volume of small stone:
[IMAGE:0]
the density of small stone:
[IMAGE:1]

\hrule
\vspace{1em}


\noindent
\textbf{Q1261.} A cube has sides of length 2. What is the length of a line joining a vertex to the midpoint of one of the opposite edges (the dashed line in the diagram below)?
[IMAGE:0]



\textbf{A.} \sqrt{}5 \\
\textbf{B.} \sqrt{}6 \\
\textbf{C.} 2\sqrt{}2 \\
\textbf{D.} 3 \\

\textbf{Answer:} D \\
\textbf{Explanation:} \sqrt{}(4+4+1)=3.

\hrule
\vspace{1em}


\noindent
\textbf{Q1262.} As the diagram indicates: An object with weight 6N is on a slope with angle 45 degrees to the horizontal direction; The coefficient of friction is 0.5. An external force
[IMAGE:0]
is exerted to the object: Find the minimum force that is required to move the object upwards
[IMAGE:1]



\textbf{A.} 1N \\
\textbf{B.} 3N \\
\textbf{C.} 9N \\
\textbf{D.} 18N \\

\textbf{Answer:} D \\
\textbf{Explanation:} [IMAGE:0]
[IMAGE:1]
[IMAGE:2]

\hrule
\vspace{1em}


\noindent
\textbf{Q1263.} As the diagram indicates: An object with weight 6N is on a slope with angle 30 degrees to the horizontal direction; The coefficient of friction is 0.6. An external force is exerted horizontally to the object: Find the minimum force that is required to move the object upwards
[IMAGE:0]



\textbf{A.} [IMAGE:0] \\
\textbf{B.} [IMAGE:1] \\
\textbf{C.} [IMAGE:2] \\
\textbf{D.} [IMAGE:3] \\

\textbf{Answer:} D \\
\textbf{Explanation:} [IMAGE:0]
The critical condition is satisfied when the maximum friction is occurred; The system is still in equilibrium; Assume the force is F, the normal component of the force is 0.5F; the effective component of F along the slope is:
[IMAGE:1]
; The total normal reaction is:
[IMAGE:2]
; Along the direction of the slope, we therefore have:
[IMAGE:3]
[IMAGE:4]

\hrule
\vspace{1em}


\noindent
\textbf{Q1264.} A cube has sides of unit length. What is the length of a line joining a vertex to the center of cube (the dashed line in the diagram below)?
[IMAGE:0]



\textbf{A.} [IMAGE:0] \\
\textbf{B.} [IMAGE:1] \\
\textbf{C.} [IMAGE:2] \\
\textbf{D.} [IMAGE:3] \\

\textbf{Answer:} C \\
\textbf{Explanation:} [IMAGE:0]
.

\hrule
\vspace{1em}


\noindent
\textbf{Q1265.} Two liquids P and Q can be mixed together in any proportion. The density of liquid P is
[IMAGE:0]
and the density of liquid Q is
[IMAGE:1]
. A volume
[IMAGE:2]
of liquid P and a volume
[IMAGE:3]
of liquid Q are mixed together to create a chemical reaction which makes total volume increase to
[IMAGE:4]
. What is the density of the mixture?



\textbf{A.} [IMAGE:0] \\
\textbf{B.} [IMAGE:1] \\
\textbf{C.} [IMAGE:2] \\
\textbf{D.} [IMAGE:3] \\

\textbf{Answer:} E \\
\textbf{Explanation:} The density is total mass divided by total volume which is E. (pay attention to the difference of "increase by" and "increase to")

\hrule
\vspace{1em}


\noindent
\textbf{Q1266.} Two liquids P and Q can be mixed together in any proportion. The density of liquid P is
[IMAGE:0]
and the density of liquid Q is
[IMAGE:1]
. A volume
[IMAGE:2]
of liquid P and a volume
[IMAGE:3]
of liquid Q are mixed together to create a chemical reaction which makes total volume increase by
[IMAGE:4]
. What is the density of the mixture?



\textbf{A.} [IMAGE:0] \\
\textbf{B.} [IMAGE:1] \\
\textbf{C.} [IMAGE:2] \\
\textbf{D.} [IMAGE:3] \\

\textbf{Answer:} F \\
\textbf{Explanation:} The density is total mass divided by total volume which is F. (pay attention to the difference of "increase by" and "increase to")

\hrule
\vspace{1em}


\noindent
\textbf{Q1267.} Two liquids P and Q can be mixed together in any proportion. The density of liquid P is
[IMAGE:0]
and the density of liquid Q is
[IMAGE:1]
. A volume
[IMAGE:2]
of liquid P and a volume
[IMAGE:3]
of liquid Q are mixed together to produce a volume that is equal to
[IMAGE:4]
. What is the density of the mixture?



\textbf{A.} [IMAGE:0] \\
\textbf{B.} [IMAGE:1] \\
\textbf{C.} [IMAGE:2] \\
\textbf{D.} [IMAGE:3] \\

\textbf{Answer:} D \\
\textbf{Explanation:} The density is total mass divided by total volume which is
[IMAGE:0]
. Thus, the answer is D.

\hrule
\vspace{1em}


\noindent
\textbf{Q1268.} Two liquids P and Q can be mixed together in any proportion. The density of liquid P is
[IMAGE:0]
and the density of liquid Q is
[IMAGE:1]
. A volume
[IMAGE:2]
of liquid P and a volume
[IMAGE:3]
of liquid Q are mixed together to produce a volume that is equal to
[IMAGE:4]
. What is the density of the mixture?



\textbf{A.} [IMAGE:0] \\
\textbf{B.} [IMAGE:1] \\
\textbf{C.} [IMAGE:2] \\
\textbf{D.} [IMAGE:3] \\

\textbf{Answer:} B \\
\textbf{Explanation:} The density is total mass divided by total volume which is B.

\hrule
\vspace{1em}


\noindent
\textbf{Q1269.} Two liquids P and Q can be mixed together in any proportion. The density of liquid P is
[IMAGE:0]
and the density of liquid Q is
[IMAGE:1]
. A volume
[IMAGE:2]
of liquid P and a volume
[IMAGE:3]
of liquid Q are mixed together to produce a volume that is equal to
[IMAGE:4]
. What is the density of the mixture?



\textbf{A.} [IMAGE:0] \\
\textbf{B.} [IMAGE:1] \\
\textbf{C.} [IMAGE:2] \\
\textbf{D.} [IMAGE:3] \\

\textbf{Answer:} B \\
\textbf{Explanation:} The density is total mass divided by total volume which is B.

\hrule
\vspace{1em}


\noindent
\textbf{Q1270.} Two liquids P and Q can be mixed together in any proportion. The density of liquid P is
[IMAGE:0]
and the density of liquid Q is
[IMAGE:1]
. A volume
[IMAGE:2]
of liquid P and a volume
[IMAGE:3]
of liquid Q are mixed together to produce a volume that is equal to
[IMAGE:4]
. What is the density of the mixture?



\textbf{A.} [IMAGE:0] \\
\textbf{B.} [IMAGE:1] \\
\textbf{C.} [IMAGE:2] \\
\textbf{D.} [IMAGE:3] \\

\textbf{Answer:} C \\
\textbf{Explanation:} The density is total mass divided by total volume which is C.

\hrule
\vspace{1em}


\noindent
\textbf{Q1271.} As the diagram indicates: An object with weight 6N is on a slope with angle 30 degrees to the horizontal direction; The coefficient of friction is 0.4. An external force is exerted horizontally to the object: Find the minimum force that is required to move the object upwards
[IMAGE:0]



\textbf{A.} [IMAGE:0] \\
\textbf{B.} [IMAGE:1] \\
\textbf{C.} [IMAGE:2] \\
\textbf{D.} [IMAGE:3] \\

\textbf{Answer:} A \\
\textbf{Explanation:} [IMAGE:0]
The critical condition is satisfied when the maximum friction is occurred; The system is still in equilibrium; Assume the force is F, the normal component of the force is 0.5F; the effective component of F along the slope is:
[IMAGE:1]
; The total normal reaction is:
[IMAGE:2]
; Along the direction of the slope, we therefore have:
[IMAGE:3]
[IMAGE:4]

\hrule
\vspace{1em}


\noindent
\textbf{Q1272.} During the soldering process, precise control of the soldering iron tip temperature is crucial. To maintain a constant tip temperature, a soldering iron is equipped with a temperature sensor and feedback control system. Assuming the tip (mass 2.0 g, copper) needs to be kept at 250°C while the ambient temperature is 20°C, and the heat exchange rate between the tip and the environment is 0.2 W/°C (i.e., the tip loses 0.5 W of heat for every 1°C above the ambient temperature)
Calculate the thermal power required from the soldering iron to maintain the tip at 250°C.



\textbf{A.} 10W \\
\textbf{B.} 11.5W \\
\textbf{C.} 23W \\
\textbf{D.} 46W \\

\textbf{Answer:} D \\
\textbf{Explanation:} Temperature difference:
[IMAGE:0]
Heat loss power:
[IMAGE:1]
Thermal power required is equal to the hear loss power, which is 115W
.

\hrule
\vspace{1em}


\noindent
\textbf{Q1273.} A soldering iron needs to adjust the tip temperature for welding different materials. To quickly reach the required welding temperature, a designer develops a new tip material whose specific heat capacity varies with temperature (relationship:
[IMAGE:0]
, where t is the temperature in °C). When the soldering iron heats this new tip (mass 1.0 g) with a thermal power of 30 W, the tip's temperature rises from 20°C to 200°C in 30 s.
Calculate the heat transferred to the surrounding environment during this process.



\textbf{A.} 111.6J \\
\textbf{B.} 240J \\
\textbf{C.} 788.4J \\
\textbf{D.} 1200J \\

\textbf{Answer:} C \\
\textbf{Explanation:} Heat provided by the soldering iron:
[IMAGE:0]
Heat absorbed by the tip (using average specific heat capacity because of the linearity of
[IMAGE:1]
equation):
Average specific heat capacity:
[IMAGE:2]
Heat absorbed:
[IMAGE:3]
Heat transferred to the environment:
[IMAGE:4]

\hrule
\vspace{1em}


\noindent
\textbf{Q1274.} During continuous operation, the copper tip of a soldering iron (mass 1.8 g, specific heat capacity
[IMAGE:0]
) experiences process of repeated heating and cooling. In a specific soldering task, the tip is first heated to 300°C from the room temperature(20°C), then rapidly cooled to 50°C, then heated again to 250°C, and finally cooled to room temperature (20°C). Assuming each heating and cooling process is linear and takes 10 s. The thermal power is 50 W when it comes to rising up the temperature of the copper tip.
Calculate the total heat transferred to the surrounding environment during the entire process.



\textbf{A.} 320J \\
\textbf{B.} 480J \\
\textbf{C.} 640J \\
\textbf{D.} 1200J \\

\textbf{Answer:} F \\
\textbf{Explanation:} Heat absorbed during first heating:
[IMAGE:0]
Heat lost during first cooling:
[IMAGE:1]
Heat absorbed during second heating:
[IMAGE:2]
Heat lost during second cooling:
[IMAGE:3]
Total heat transferred to the environment:
[IMAGE:4]
.
PS: If write down the
[IMAGE:5]
equation at the very beginning, the calculation process will become easier.

\hrule
\vspace{1em}


\noindent
\textbf{Q1275.} To improve soldering efficiency, a designer coats the surface of a copper tip (mass 2.0 g, specific heat capacity
[IMAGE:0]
) with a special material that significantly enhances the tip's thermal conductivity but also increases its heat capacity by 20%. When the soldering iron heats the tip with a thermal power of 50 W, the tip's temperature rises by 220°C in 8 s.
Calculate the heat lost to the environment. after coating.



\textbf{A.} 188.8J \\
\textbf{B.} 288.8J \\
\textbf{C.} 640J \\
\textbf{D.} 1200J \\

\textbf{Answer:} A \\
\textbf{Explanation:} Heat provided by the soldering iron:
[IMAGE:0]
Heat absorbed by the tip:
[IMAGE:1]
Heat lost to the environment:
[IMAGE:2]

\hrule
\vspace{1em}


\noindent
\textbf{Q1276.} During the soldering process, the copper tip of a soldering iron (mass 2.5g
, specific heat capacity
[IMAGE:0]
) accumulates heat due to prolonged use. When the soldering iron stops heating, the tip begins to cool and its temperature drops by
[IMAGE:1]
in 30s
. Assuming all the heat lost by the tip is absorbed by the surrounding environment and the cooling rate is constant
Calculate the heat lost by the tip per second during cooling.



\textbf{A.} 6J/s \\
\textbf{B.} 60J/s \\
\textbf{C.} 180J/s \\
\textbf{D.} 1000J/s \\

\textbf{Answer:} A \\
\textbf{Explanation:} Total heat lost by the tip:
[IMAGE:0]
Heat lost per second:
[IMAGE:1]
.

\hrule
\vspace{1em}


\noindent
\textbf{Q1277.} A soldering iron is equipped with an aluminum tip of mass 3.0 g (specific heat capacity of aluminum =
[IMAGE:0]
). When the soldering iron heats the tip with a thermal power of 40 W, the tip's temperature rises to a certain temperature in 30 s. However, due to a heat sink on the tip's surface, some heat is lost to the environment. If only 85% of the heat provided by the heating power is actually absorbed by the tip.
Calculate the heat lost to the environment.



\textbf{A.} 120J \\
\textbf{B.} 160J \\
\textbf{C.} 180J \\
\textbf{D.} 1200J \\

\textbf{Answer:} C \\
\textbf{Explanation:} Heat provided by the soldering iron:
[IMAGE:0]
Heat absorbed by the tip:
[IMAGE:1]
Heat lost to the environment:
[IMAGE:2]

\hrule
\vspace{1em}


\noindent
\textbf{Q1278.} A soldering iron has a copper tip of mass 2.5g
The tip is heated with 30W
of thermal power. In 60s
, the temperature of the tip increases by
[IMAGE:0]
.
How much energy is transferred from the tip to the surroundings in this time? (specific heat capacity of copper =
[IMAGE:1]
).



\textbf{A.} 320J \\
\textbf{B.} 480J \\
\textbf{C.} 640J \\
\textbf{D.} 1200J \\

\textbf{Answer:} E \\
\textbf{Explanation:} By the conservation of energy; during this time; the energy to heat the tip minus the energy dissipated into the air equals the energy to raise the temperature of the tip: therefore
[IMAGE:0]
.

\hrule
\vspace{1em}


\noindent
\textbf{Q1279.} A soldering iron has a copper tip of mass 1.0g
The tip is heated with 20W
of thermal power. In 45s
, the temperature of the tip increases by
[IMAGE:0]
.
How much energy is transferred from the tip to the surroundings in this time? (specific heat capacity of copper =
[IMAGE:1]
).



\textbf{A.} 320J \\
\textbf{B.} 480J \\
\textbf{C.} 640J \\
\textbf{D.} 700J \\

\textbf{Answer:} E \\
\textbf{Explanation:} By the conservation of energy; during this time; the energy to heat the tip minus the energy dissipated into the air equals the energy to raise the temperature of the tip: therefore
[IMAGE:0]
.

\hrule
\vspace{1em}


\noindent
\textbf{Q1280.} The circuit shown in the diagram contains six resistors and an ideal volt ammeter. And the resistance of one of the six resistors is unknown (xΩ) . The reading scope on the ammeter is 4A. What is the power dissipated in the "unknown resistor" (xΩ)
[IMAGE:0]



\textbf{A.} 0W \\
\textbf{B.} 4W \\
\textbf{C.} 6W \\
\textbf{D.} 8W \\

\textbf{Answer:} F \\
\textbf{Explanation:} 12-4×1.5=6V
6V/4A=1.5Ω
Then, 1/x+1/3=1/1.5, x=3Ω.
Thus, $6^2$/3=12W.

\hrule
\vspace{1em}


\noindent
\textbf{Q1281.} A soldering iron has a copper tip of mass
2.0g.
The tip is heated with 27W
of thermal power. In 40s
, the temperature of the tip increases by
[IMAGE:0]
.
How much energy is transferred from the tip to the surroundings in this time? (specific heat capacity of copper =
[IMAGE:1]
).



\textbf{A.} 320J \\
\textbf{B.} 760J \\
\textbf{C.} 780J \\
\textbf{D.} 880J \\

\textbf{Answer:} B \\
\textbf{Explanation:} By the conservation of energy; during this time; the energy to heat the tip minus the energy dissipated into the air equals the energy to raise the temperature of the tip: therefore
[IMAGE:0]
.

\hrule
\vspace{1em}


\noindent
\textbf{Q1282.} In a sealed container, the temperature of a gas remains constant. When the volume is 4 m
3
, the pressure is 6000 Pa. If the volume expands to 16 m
3
, what is the new pressure?



\textbf{A.} 4000 Pa \\
\textbf{B.} 3000 Pa \\
\textbf{C.} 2000 Pa \\
\textbf{D.} 1000 Pa \\

\textbf{Answer:} F \\
\textbf{Explanation:} By Boyle’s law, at constant temperature, the pressure PP of a gas is inversely proportional to its volume
[IMAGE:0]

\hrule
\vspace{1em}


\noindent
\textbf{Q1283.} A soldering iron has a copper tip of mass 2.0g.
The tip is heated with 20W
of thermal power. In 33s
, the temperature of the tip increases by
[IMAGE:0]
.
How much energy is transferred from the tip to the surroundings in this time? (specific heat capacity of copper =
[IMAGE:1]
).



\textbf{A.} 320J \\
\textbf{B.} 500J \\
\textbf{C.} 640J \\
\textbf{D.} 1200J \\

\textbf{Answer:} B \\
\textbf{Explanation:} By the conservation of energy; during this time; the energy to heat the tip minus the energy dissipated into the air equals the energy to raise the temperature of the tip: therefore
[IMAGE:0]
.

\hrule
\vspace{1em}


\noindent
\textbf{Q1284.} A circuit contains a variable resistor whose resistance R can be adjusted. When the resistance is 10Ω, the current flowing through the circuit is 2A. Assuming the voltage of the power supply remains constant, what is the current when the resistance is increased to 50Ω?



\textbf{A.} 0.5A \\
\textbf{B.} 1.0A \\
\textbf{C.} 1.5A \\
\textbf{D.} 2.0A \\

\textbf{Answer:} F \\
\textbf{Explanation:} According to Ohm's Law, the current I is inversely proportional to the resistance R:
[IMAGE:0]
where V is the constant voltage of the power supply. Given R=10Ω and I=2A, the voltage is:
[IMAGE:1]
When the resistance is increased to 40Ω, the current becomes:
[IMAGE:2]

\hrule
\vspace{1em}


\noindent
\textbf{Q1285.} Mike heats ice cubes and observes the physical changes of ice. During this process, he measures and draws a graph of temperature versus time, as shown in the figure below. Based on the figure, which of the following analysis is correct?
[IMAGE:0]



\textbf{A.} The BC segment in the figure represents the melting process of ice. \\
\textbf{B.} Ice has an uncertain melting point, indicating that ice is not a crystal. \\
\textbf{C.} The water temperature rises slowly, indicating that the specific heat &capacity of water is smaller than that of ice. \\
\textbf{D.} The temperature of boiling water remains constant, indicating that boiling &does not require heat absorption. \\

\textbf{Answer:} A \\
\textbf{Explanation:} A. The AB segment in the figure represents the temperature rise of ice, while the BC segment represents the melting process of ice; hence, A is correct.
B. From the figure, it can be seen that the temperature of ice remains constant at 0\circ C, indicating that ice is a crystal; hence, B is incorrect.
C. Since the mass of ice and water is the same and they are heated by the same alcohol lamp, the slow rise in water temperature indicates that the specific heat capacity of water is larger than that of ice; hence, C is incorrect.
D. The temperature of boiling water remains constant, but it requires continuous heat absorption. If heating is stopped, boiling will also stop; hence, D is incorrect.

\hrule
\vspace{1em}


\noindent
\textbf{Q1286.} A glass of water is placed in a refrigerator; with a mass of 0.2 kg and initial temperature of 10 degrees; Find the equilibrium temperature: The latent heat of ice is 300kJ/kg; the specific heat capacity of water is 2.09 kJ/(kg\cdot degree); the specific heat capacity of water is 4200 kJ/(kg\cdot degree); Under standard atmospheric pressure , water can completely freeze and the freezing temperature (i.e., the ice point) of the water is -20 degrees Celsius.
Calculate the amount of heat that needs to be released for the water in this cup to turn into ice at 0 degrees Celsius.



\textbf{A.} 1631.42KJ \\
\textbf{B.} 7200.68KJ \\
\textbf{C.} 8468.36KJ \\
\textbf{D.} 8520.00KJ \\

\textbf{Answer:} C \\
\textbf{Explanation:} Assume that the equilibrium temperature is x; the heat released by the water equals the heat absorbs the ice:
[IMAGE:0]
PS: some formulas are listed as follows
[IMAGE:1]
[IMAGE:2]
[IMAGE:3]

\hrule
\vspace{1em}


\noindent
\textbf{Q1287.} A cylinder contains a fixed amount of ideal gas. When the piston is at the midpoint, the volume of the gas is 500cm
3
and the pressure is 1.5atm. If the piston is slowly compressed, reducing the gas volume to 10cm
3,
what is the new pressure of the gas, assuming the temperature remains constant?



\textbf{A.} 10atm \\
\textbf{B.} 75atm \\
\textbf{C.} 20atm \\
\textbf{D.} 25atm \\

\textbf{Answer:} B \\
\textbf{Explanation:} According to Boyle's Law, at constant temperature, the pressure of an ideal gas is inversely proportional to its volume:
[IMAGE:0]
[IMAGE:1]

\hrule
\vspace{1em}


\noindent
\textbf{Q1288.} A glass of water is placed in a refrigerator; with a mass of 0.4 kg and initial temperature of 5 degrees; Find the equilibrium temperature: The latent heat of ice is 300kJ/kg; the specific heat capacity of ice is 2.09 kJ/(kg\cdot degree); the specific heat capacity of water is 4200 kJ/(kg\cdot degree); Under standard atmospheric pressure , water can completely freeze and the freezing temperature (i.e., the ice point) of the water is 0 degrees Celsius.
Calculate the amount of heat that needs to be released for the water in this cup to turn into ice at 0 degrees Celsius.



\textbf{A.} 1630KJ \\
\textbf{B.} 3600KJ \\
\textbf{C.} 4200KJ \\
\textbf{D.} 8520KJ \\

\textbf{Answer:} D \\
\textbf{Explanation:} Assume that the equilibrium temperature is x; the heat released by the water equals the heat absorbs the ice:
[IMAGE:0]
PS: some formulas are listed as follows
[IMAGE:1]
[IMAGE:2]
[IMAGE:3]

\hrule
\vspace{1em}


\noindent
\textbf{Q1289.} The circuit shown in the diagram contains six resistors and an ideal voltmeter. And the resistance of one of the six resistors is unknown (xΩ) . The reading scope on the voltmeter is 5.5V. What is the power dissipated in the "unknown resistor" (xΩ)?
[IMAGE:0]



\textbf{A.} 0W \\
\textbf{B.} 2W \\
\textbf{C.} 4W \\
\textbf{D.} 6W \\

\textbf{Answer:} F \\
\textbf{Explanation:} The voltmeter records the potential difference between the measured points, which is 10V and 12×3/(1+x+3)V. The difference is 7V.
So 12×3/(1+x+3)=10-5.5=4.5, x=4Ω.
Thus, (12/(1+4+3) )^2×4=9W

\hrule
\vspace{1em}


\noindent
\textbf{Q1290.} In a hydraulic system, the force F remains constant. Piston A has an area of 0.1 m
2
and generates a pressure of 5000 Pa. If replaced by Piston B with an area of 0.4 m
2,
what is the new pressure?



\textbf{A.} 2000 Pa \\
\textbf{B.} 2500 Pa \\
\textbf{C.} 4000 Pa \\
\textbf{D.} 5000 Pa \\

\textbf{Answer:} F \\
\textbf{Explanation:} Inverse Relationship
:
Pressure PP is inversely proportional to the area A
[IMAGE:0]
Substituting P1=5000 Pa,
A
1​=0.2m
2
, A2=1 m
2
:
[IMAGE:1]

\hrule
\vspace{1em}


\noindent
\textbf{Q1291.} A piece of ice undergoes the following three processes:
Ice at -20°C is heated to 0°C, absorbing heat
;
Ice at 0°C melts into water at 0°C, absorbing heat
[IMAGE:0]
;
Water at 20°C is heated to 40°C, absorbing
[IMAGE:1]
heat
[IMAGE:2]
.
It is known that the specific heat capacity of ice is less than that of water. The mass remains constant throughout the entire process. Which of the following is true?



\textbf{A.} [IMAGE:0] \\
\textbf{B.} [IMAGE:1] \\
\textbf{C.} [IMAGE:2] \\
\textbf{D.} [IMAGE:3] \\

\textbf{Answer:} C \\
\textbf{Explanation:} To compare the heat absorbed in each process, we need to use the formulas for heat absorption:
Heat absorbed by ice from -20°C to 0°C:
[IMAGE:0]
Heat absorbed by ice melting at 0°C:
[IMAGE:1]
where
[IMAGE:2]
is the latent heat of fusion for ice.
Heat absorbed by water from 20°C to 40°C:
[IMAGE:3]
Given that the specific heat capacity of ice
[IMAGE:4]
is less than that of water
[IMAGE:5]
, we have:
[IMAGE:6]
Thus,
[IMAGE:7]
.
The latent heat of fusion
[IMAGE:8]
for ice is generally much larger than the specific heat capacities
[IMAGE:9]
and
[IMAGE:10]
multiplied by the temperature change. Therefore:
[IMAGE:11]
Combining these results, we get:
[IMAGE:12]

\hrule
\vspace{1em}


\noindent
\textbf{Q1292.} The gravitational force F between the Earth and the Moon is inversely proportional to the square of the distance r between them. When the Earth and Moon are 3.84×10
5
km apart, the force is 2×10
20
N. What is the distance between them when the gravitational force decreases to 1.25×10
19
N?



\textbf{A.} 1.92×10
5
km \\
\textbf{B.} 3.84×10
5
km \\
\textbf{C.} 5.76×10
5
km \\
\textbf{D.} 7.68×10
5
km \\

\textbf{Answer:} F \\
\textbf{Explanation:} [IMAGE:0]

\hrule
\vspace{1em}


\noindent
\textbf{Q1293.} The ice is submerged into a glass of water; the 1.0kg ice is at -14 degrees; 0.2 kg water is at 5 degrees; Find the equilibrium temperature: The latent heat of ice is 300kJ/kg; the specific heat capacity of ice is 2.09 kJ/(kg\cdot degree); the specific heat capacity of water is 4200 kJ/(kg\cdot degree); The ice can completely melt when the temperature between 0 degrees and 5 degrees; the melting point of ice under standard atmospheric pressure is 0 degrees:



\textbf{A.} 5.23 \\
\textbf{B.} 8.7 \\
\textbf{C.} 7.7 \\
\textbf{D.} 0.87 \\

\textbf{Answer:} E \\
\textbf{Explanation:} Assume that the equilibrium temperature is x; the heat released by the water equals the heat absorbs the ice:
[IMAGE:0]
.
[IMAGE:1]
[IMAGE:2]
[IMAGE:3]
PS: some formulas are listed as follows
[IMAGE:4]
[IMAGE:5]
[IMAGE:6]

\hrule
\vspace{1em}


\noindent
\textbf{Q1294.} The ice is submerged into a glass of water; the 1.0kg ice is at -60 degrees; 1 kg water is at 5 degrees; Find the equilibrium temperature: The latent heat of ice is 300kJ/kg; the specific heat capacity of ice is 2.09 kJ/(kg\cdot degree); the specific heat capacity of water is 4200 kJ/(kg\cdot degree); The ice can completely melt when the temperature between 0 degrees and 5 degrees; the melting point of ice under standard atmospheric pressure is 0 degrees:



\textbf{A.} 5.23 \\
\textbf{B.} 10.00 \\
\textbf{C.} 3.12 \\
\textbf{D.} 2.45 \\

\textbf{Answer:} D \\
\textbf{Explanation:} Assume that the equilibrium temperature is x; the heat released by the water equals the heat absorbs the ice:
[IMAGE:0]
.
[IMAGE:1]
[IMAGE:2]
PS: some formulas are listed as follows
[IMAGE:3]
[IMAGE:4]
[IMAGE:5]

\hrule
\vspace{1em}


\noindent
\textbf{Q1295.} A car starts from a stationary state and undergoes uniform acceleration with an acceleration of a = 2 meters/second² for 6 seconds. Then it moves at a constant speed until the total time reaches 15 seconds. If the total distance covered is 144 meters, what is the constant speed after the uniform acceleration stage?



\textbf{A.} 5 m/s \\
\textbf{B.} 8 m/s \\
\textbf{C.} 10 m/s \\
\textbf{D.} 12 m/s \\

\textbf{Answer:} D \\
\textbf{Explanation:} Explanation:
Constant speed stage speed: 2*6 = 12 m/s

\hrule
\vspace{1em}


\noindent
\textbf{Q1296.} The ice is submerged into a glass of water; the 1.0kg ice is at 0 degrees; 1 kg water is at 6 degrees; Find the equilibrium temperature: The latent heat of ice is 300kJ/kg; the specific heat capacity of water is 4200 kJ/Kg*degree; The ice can completely melt when the temperature between 0 degrees and 6 degrees; the melting point of ice under standard atmospheric pressure is 0 degrees:



\textbf{A.} 5.2 \\
\textbf{B.} 2.5 \\
\textbf{C.} 3.0 \\
\textbf{D.} 10.0 \\

\textbf{Answer:} C \\
\textbf{Explanation:} Assume that the equilibrium temperature is x; the heat released by the water equals the heat absorbs the ice:
[IMAGE:0]
.
[IMAGE:1]
[IMAGE:2]
where
[IMAGE:3]
corresponds to the process of ice at 0 degrees turning into water at 0 degrees,
[IMAGE:4]
corresponds to the process of the water formed from the ice absorbing heat, and
[IMAGE:5]
corresponds to the process of the original water in the glass releasing heat.

\hrule
\vspace{1em}


\noindent
\textbf{Q1297.} The circuit shown in the diagram contains six resistors and an ideal digital ammeter. And one of the six resistors is variable (xΩ) whose scope is from 1Ω to 3Ω. What is the reading scope on the ammeter?
[IMAGE:0]



\textbf{A.} from 0A to 3A \\
\textbf{B.} from 1A to 4A \\
\textbf{C.} from 2.3A to 4A \\
\textbf{D.} from 3.7A to 4.3A \\

\textbf{Answer:} G \\
\textbf{Explanation:} from 12/(1.5+1.5)=4A to 12/(0.75+1.5)=5.3A.

\hrule
\vspace{1em}


\noindent
\textbf{Q1298.} The electrostatic force F between two point charges is inversely proportional to the square of the distance r between them. When the charges are 3m apart, the force is 12N. What is the distance between the charges when the force becomes 6.75N?



\textbf{A.} 0.5m \\
\textbf{B.} 1m \\
\textbf{C.} 2m \\
\textbf{D.} 2.5m \\

\textbf{Answer:} F \\
\textbf{Explanation:} By Coulomb's Law, the electrostatic force is inversely proportional to the square of the distance:
[IMAGE:0]
where k is a constant. When F=12N and r=3m: k=F
\cdot 
r
2
=12×32 =108
For
F
=108N:
[IMAGE:1]

\hrule
\vspace{1em}


\noindent
\textbf{Q1299.} The ice is submerged into a glass of water; the 2.0kg ice is at 0 degrees; 1 kg water is at 25 degrees; Find the equilibrium temperature: The latent heat of ice is 300kJ/kg; the specific heat capacity of water is 4200 kJ/Kg*degree; The ice can completely melt when the temperature between 0 degrees and 25 degrees; the melting point of ice under standard atmospheric pressure is 0 degrees:



\textbf{A.} 5.6 \\
\textbf{B.} 7.4 \\
\textbf{C.} 6.6 \\
\textbf{D.} 8.3 \\

\textbf{Answer:} D \\
\textbf{Explanation:} Assume that the equilibrium temperature is x; the heat released by the water equals the heat absorbs the ice:
[IMAGE:0]
.
[IMAGE:1]
[IMAGE:2]
[IMAGE:3]
where
[IMAGE:4]
corresponds to the process of ice at 0 degrees turning into water at 0 degrees,
[IMAGE:5]
corresponds to the process of the water formed from the ice absorbing heat, and
[IMAGE:6]
corresponds to the process of the original water in the glass releasing heat.

\hrule
\vspace{1em}


\noindent
\textbf{Q1300.} The ice is submerged into a glass of water; the 0.5kg ice is at 0 degrees; 0.5 kg water is at 18 degrees; Find the equilibrium temperature: The latent heat of ice is 300kJ/kg; the specific heat capacity of water is 4200 kJ/Kg*degree; The ice can completely melt when the temperature between 0 degrees and 18 degrees; the melting point of ice under standard atmospheric pressure is 0 degrees:



\textbf{A.} 5.6 \\
\textbf{B.} 7.4 \\
\textbf{C.} 3.6 \\
\textbf{D.} 9.0 \\

\textbf{Answer:} D \\
\textbf{Explanation:} Assume that the equilibrium temperature is x; the heat released by the water equals the heat absorbs the ice:
[IMAGE:0]
.
[IMAGE:1]
[IMAGE:2]
where
[IMAGE:3]
corresponds to the process of ice at 0 degrees turning into water at 0 degrees,
[IMAGE:4]
corresponds to the process of the water formed from the ice absorbing heat, and
[IMAGE:5]
corresponds to the process of the original water in the glass releasing heat.

\hrule
\vspace{1em}


\noindent
\textbf{Q1301.} The velocity v of an object is inversely proportional to the square root of time t. When v=12m/s, t=4s. What is the value of t when v=3m/s?



\textbf{A.} 16/9s \\
\textbf{B.} 4s \\
\textbf{C.} 9/16s \\
\textbf{D.} 27/14s \\

\textbf{Answer:} F \\
\textbf{Explanation:} Velocity v is inversely proportional to the square root of time t, which can be expressed as
[IMAGE:0]
where k is a constant. When v=12m/s and t=4s, substituting into the equation gives:
[IMAGE:1]
When v=24m/s, substituting into the equation gives:
[IMAGE:2]

\hrule
\vspace{1em}


\noindent
\textbf{Q1302.} The ice is submerged into a glass of water; the 0.5kg ice is at 0 degrees; 1 kg water is at 16 degrees; Find the equilibrium temperature: The latent heat of ice is 300kJ/kg; the specific heat capacity of water is 4200 kJ/Kg*degree; The ice can completely melt when the temperature between 0 degrees and 16 degrees; the melting point of ice under standard atmospheric pressure is 0 degrees:



\textbf{A.} 5.6 \\
\textbf{B.} 7.4 \\
\textbf{C.} 10 \\
\textbf{D.} 10.6 \\

\textbf{Answer:} D \\
\textbf{Explanation:} Assume that the equilibrium temperature is x; the heat released by the water equals the heat absorbs the ice:
[IMAGE:0]
.
[IMAGE:1]
[IMAGE:2]
[IMAGE:3]
where 0.5*300
corresponds to the process of ice at 0 degrees turning into water at 0 degrees,
[IMAGE:4]
corresponds to the process of the water formed from the ice absorbing heat, and
[IMAGE:5]
corresponds to the process of the original water in the glass releasing heat.

\hrule
\vspace{1em}


\noindent
\textbf{Q1303.} The circuit shown in the diagram contains six resistors and an ideal volt ammeter. And one of the six resistors is variable (xΩ) whose maximum is 6Ω.
What is the reading scope on the  voltmeter?
[IMAGE:0]



\textbf{A.} from 1.5V to 3.5V \\
\textbf{B.} from 1V to 4V \\
\textbf{C.} from 2V to 4V \\
\textbf{D.} from 4V to 7V \\

\textbf{Answer:} D \\
\textbf{Explanation:} from
[IMAGE:0]
to
[IMAGE:1]
.

\hrule
\vspace{1em}


\noindent
\textbf{Q1304.} The circuit shown in the diagram contains six resistors and an ideal digital ammeter.
What is the reading on the ammeter?
[IMAGE:0]



\textbf{A.} 0V \\
\textbf{B.} 2V \\
\textbf{C.} 4V \\
\textbf{D.} 6V \\

\textbf{Answer:} C \\
\textbf{Explanation:} [IMAGE:0]
.

\hrule
\vspace{1em}


\noindent
\textbf{Q1305.} The circuit shown in the diagram contains six resistors and an ideal digital ammeter.
What is the reading on the ammeter?
[IMAGE:0]



\textbf{A.} 0V \\
\textbf{B.} 2V \\
\textbf{C.} 4V \\
\textbf{D.} 6V \\

\textbf{Answer:} B \\
\textbf{Explanation:} [IMAGE:0]

\hrule
\vspace{1em}


\noindent
\textbf{Q1306.} The quantities a and b are positive. aa is inversely proportional to the square root of b. When a=4, b=16. What is the value of b when a=0.64?



\textbf{A.} 16 \\
\textbf{B.} 49 \\
\textbf{C.} 81 \\
\textbf{D.} 625 \\

\textbf{Answer:} D \\
\textbf{Explanation:} Since
$𝑎$
is inversely proportional to the square root of b, the relationship is:
[IMAGE:0]
[IMAGE:1]

\hrule
\vspace{1em}


\noindent
\textbf{Q1307.} The circuit shown in the diagram contains six resistors and an ideal digital voltmeter.
What is the reading on the voltmeter?
[IMAGE:0]



\textbf{A.} 0V \\
\textbf{B.} 2V \\
\textbf{C.} 4V \\
\textbf{D.} 6V \\

\textbf{Answer:} C \\
\textbf{Explanation:} The voltmeter records the potential difference between the measured points, which is 13V and 9V. The difference is 4V.

\hrule
\vspace{1em}


\noindent
\textbf{Q1308.} The circuit shown in the diagram contains six resistors and an ideal digital voltmeter.
What is the reading on the voltmeter?
[IMAGE:0]



\textbf{A.} 1V \\
\textbf{B.} 2V \\
\textbf{C.} 4V \\
\textbf{D.} 6V \\

\textbf{Answer:} A \\
\textbf{Explanation:} The voltmeter records the potential difference between the measured points, which is 10V and 9V. The difference is 1V.

\hrule
\vspace{1em}


\noindent
\textbf{Q1309.} The quantities
x
and y are positive. x is inversely proportional to the square of
y
. When x=3, y=4. What is the value of
y
when x=1.92?



\textbf{A.} 1 \\
\textbf{B.} 2 \\
\textbf{C.} 5 \\
\textbf{D.} 16 \\

\textbf{Answer:} C \\
\textbf{Explanation:} Explanation: Since x is inversely proportional to the square of y, the relationship is:
[IMAGE:0]
[IMAGE:1]

\hrule
\vspace{1em}


\noindent
\textbf{Q1310.} A 5 kg object is initially at rest on a smooth horizontal surface. Starting at t=0, the object is subjected to a horizontal force that varies with time as follows:
From 0 to 0.05 s, the force increases linearly from 0 to 10 N;
From 0.05 to 0.10 s, the force remains constant at 10 N;
From 0.10 to 0.15 s, the force decreases linearly back to 0.
What is the kinetic energy of the object at t=0.0.075 s?



\textbf{A.} 0 J \\
\textbf{B.} 0.50 J \\
\textbf{C.} 1.25 J \\
\textbf{D.} 0.025 J \\

\textbf{Answer:} D \\
\textbf{Explanation:} The kinetic energy is equal to the work done by the force, which is the integral of the force over time.
From 0 to 0.05 s: Force increases linearly from 0 to 10 N (triangle).
From 0.05 to 0.10 s: Force remains constant at 10 N (rectangle).
From 0.10 to 0.15 s: Force decreases linearly from 10 N to 0 (triangle).

\hrule
\vspace{1em}


\noindent
\textbf{Q1311.} The circuit shown in the diagram contains six resistors and an ideal digital voltmeter.
What is the reading on the voltmeter?
[IMAGE:0]



\textbf{A.} 0V \\
\textbf{B.} 2.4V \\
\textbf{C.} 4V \\
\textbf{D.} 5.2V \\

\textbf{Answer:} C \\
\textbf{Explanation:} The voltmeter records the potential difference between the measured points, which is 10V and 6V. The difference is 4V.

\hrule
\vspace{1em}


\noindent
\textbf{Q1312.} An object of mass 5 kg is at rest at time t=0. A resultant force acts on the object in a constant direction. The magnitude of the resultant force acting on the object varies with time as follows:
The force starts at 0 N, increases to 10 N at t=0.1 s and remains constant until t=0.2 s, then increases linearly to 20 N, and finally decreases linearly to 0 N at t=0.3 s.
[IMAGE:0]
What is the kinetic energy of the object at time t=0.1 s?



\textbf{A.} 0.025 J \\
\textbf{B.} 2.34 J \\
\textbf{C.} 4.02 J \\
\textbf{D.} 3.06 J \\

\textbf{Answer:} A \\
\textbf{Explanation:} The kinetic energy equals the work done by the force; the integration of force over time equals the change in momentum. The graph shows that the force increases from 0 N to 10 N at t=0.1 s and remains constant until t=0.2 s, then increases linearly to 20 N, and finally decreases linearly to 0 N at t=0.3 s.

\hrule
\vspace{1em}


\noindent
\textbf{Q1313.} An archaeologist discovers an ancient charcoal sample, which contains only 1/8 of the carbon-14 found in modern carbon. Given that the half-life of carbon-14 is 5730 years, how old is the charcoal approximately?



\textbf{A.} 5730 years \\
\textbf{B.} 11460 years \\
\textbf{C.} 17190 years \\
\textbf{D.} 22920 years \\

\textbf{Answer:} C \\
\textbf{Explanation:} The decay of carbon-14 follows the half-life rule. When the amount of carbon-14 is reduced to 1/8 of its original amount, it indicates that 3 half-lives have passed (since 231​=81​). Each half-life is 5730 years, so the total time is:
3×5730=17190years

\hrule
\vspace{1em}


\noindent
\textbf{Q1314.} An object of mass 5 kg is at rest at time t=0. A resultant force acts on the object in a constant direction. The magnitude of the resultant force acting on the object varies with time as follows:
The force starts at 0 N, increases to 100 N at t=0.1 s, and then decreases linearly to 0 N at t=0.2 s.
[IMAGE:0]
What is the kinetic energy of the object at time t=0.1 s?



\textbf{A.} 0J \\
\textbf{B.} 5J \\
\textbf{C.} 2.5J \\
\textbf{D.} 2J \\

\textbf{Answer:} C \\
\textbf{Explanation:} The kinetic energy equals the work done by the force; the integration of force over time equals the change in momentum. The graph shows that the force increases from 0 N to 100 N at t=0.1 s.
[IMAGE:0]
[IMAGE:1]

\hrule
\vspace{1em}


\noindent
\textbf{Q1315.} The circuit shown in the diagram contains six resistors and an ideal digital voltmeter.
What is the reading on the voltmeter?
[IMAGE:0]



\textbf{A.} 0V \\
\textbf{B.} 2V \\
\textbf{C.} 4V \\
\textbf{D.} 6V \\

\textbf{Answer:} A \\
\textbf{Explanation:} The voltmeter records the potential difference between the measured points, which is 10V and 10V. The difference is 0V.

\hrule
\vspace{1em}


\noindent
\textbf{Q1316.} A rocket of mass 10 kg is launched with a varying thrust. The thrust varies with time as follows:
The thrust starts at 0 N, increases to 200 N at t=0.1 s, and then decreases linearly to 0 N at t=0.2 s.
Ignoring air resistance and other forces, what is the kinetic energy of the rocket at t=0.1 s?



\textbf{A.} 0J \\
\textbf{B.} 1J \\
\textbf{C.} 2J \\
\textbf{D.} 3J \\

\textbf{Answer:} F \\
\textbf{Explanation:} The kinetic energy equals the work done by the force; the integration of force over time equals the change in momentum. The graph shows that the thrust increases from 0 N to 200 N at t=0.1 s and then decreases linearly to 0 N at t=0.2 s.
[IMAGE:0]
[IMAGE:1]
[IMAGE:2]

\hrule
\vspace{1em}


\noindent
\textbf{Q1317.} A circuit contains a fixed capacitor Y, a fixed resistor X, and a variable resistor W. The power supply has no internal resistance.
The resistance of W decreases. What is the process of the charge stored in Y at steady state?
[IMAGE:0]



\textbf{A.} decreases \\
\textbf{B.} stays constant \\
\textbf{C.} increases \\
\textbf{D.} uncertain \\

\textbf{Answer:} A \\
\textbf{Explanation:} The charge stored in a capacitor is directly proportional to the voltage across it (fixed capacitor). When the resistance of W decreases, the voltage across Y decreases, resulting in less stored charge.

\hrule
\vspace{1em}


\noindent
\textbf{Q1318.} A 2 kg object is initially at rest on a smooth horizontal surface. Starting at t=0, the object is subjected to a horizontal force that varies with time as follows:
From 0 to 0.6 s, the force increases linearly from 0 to 10 N;
From 0.6 to 1.2 s, the force remains constant at 10 N;
From 1.2to 1.6s, the force decreases linearly back to 0
[IMAGE:0]
What is the kinetic energy of the object at t=0.8 s?



\textbf{A.} 36J \\
\textbf{B.} 72J \\
\textbf{C.} 14J \\
\textbf{D.} 26J \\

\textbf{Answer:} A \\
\textbf{Explanation:} The kinetic energy is equal to the work done by the force, which is the integral of the force over time (the area under the force-time graph).

\hrule
\vspace{1em}


\noindent
\textbf{Q1319.} Iodine-131 (I-131) is a commonly used radioactive isotope in the medical field. Iodine-131 has an atomic number of 53 and a mass number of 131. After undergoing beta decay, iodine-131 transforms into another element. What is the product of iodine-131 after beta decay?



\textbf{A.} Xenon-131 (Xe-131) \\
\textbf{B.} Cesium-131 (Cs-131) \\
\textbf{C.} Tellurium-131 (Te-131) \\
\textbf{D.} Strontium-131 (Sr-131) \\

\textbf{Answer:} A \\
\textbf{Explanation:} Beta decay involves the conversion of a neutron into a proton within the nucleus, while releasing an electron (beta particle). As a result, beta decay increases the atomic number by 1 while keeping the mass number unchanged.

\hrule
\vspace{1em}


\noindent
\textbf{Q1320.} A circuit contains a fixed capacitor Y, a fixed resistor X, and a variable resistor W. The power supply has no internal resistance.
The resistance of W decreases. What is the process of the charge stored in Y at steady state?
[IMAGE:0]



\textbf{A.} decreases \\
\textbf{B.} stays constant \\
\textbf{C.} increases \\
\textbf{D.} uncertain \\

\textbf{Answer:} C \\
\textbf{Explanation:} The charge stored in a capacitor is directly proportional to the voltage across it (fixed capacitor). When the resistance of W decreases, the voltage across Y increases, resulting in more stored charge.

\hrule
\vspace{1em}


\noindent
\textbf{Q1321.} A ball A with a mass of 2 kilograms is moving at a speed of 8 meters/second and collides elastically with a stationary ball B whose mass is 4 kilograms. After the collision, the speed of ball A becomes -2 meters/second (in the opposite direction). What is the speed of ball B after the collision? (The result should be rounded to two decimal places.)



\textbf{A.} 1.0m/s \\
\textbf{B.} 2.0m/s \\
\textbf{C.} 3.0m/s \\
\textbf{D.} 6.0m/s \\

\textbf{Answer:} E \\
\textbf{Explanation:} In an elastic collision, both momentum and kinetic energy are conserved. Using the conservation of momentum:
[IMAGE:0]
Since kinetic energy is conserved, the answer is correct. The velocity of ball B after the collision is 4 m/s, corresponding to option E.

\hrule
\vspace{1em}


\noindent
\textbf{Q1322.} A circuit contains two fixed resistors, X and Y, and a variable resistor W. The variable power supply(PS) has no internal resistance.
How can the most probable changes in resistance and power supply make it: the power dissipated in X decreases and the power dissipated in Y decreases?
[IMAGE:0]



\textbf{A.} PS: stays constant,W: decreases \\
\textbf{B.} PS: increases,W: increases \\
\textbf{C.} PS: stays constant,W: stays constant \\
\textbf{D.} PS: decreases,W: stays constant \\

\textbf{Answer:} D \\
\textbf{Explanation:} When the resistance of W is increased, The W and Y in total has a larger resistance; X owns less voltage; W and Y owns more voltage.
When the resistance of W is decreased, The W and Y in total has a smaller resistance; X owns more voltage; W and Y owns less voltage.
When the power supply is increased, X owns more voltage; W and Y owns more voltage.
When the power supply is decreased, X owns more voltage; W and Y owns more voltage.
Thus, "PS: increases, W: impossible", "PS: stays constant, W: impossible", "PS: decreases, W: any".

\hrule
\vspace{1em}


\noindent
\textbf{Q1323.} When an electron in a hydrogen atom transitions from the 3rd energy level to the 2nd energy level, what is the wavelength of the emitted photon? The Rydberg constant for hydrogen is R
H
​=1.097×107m−1.
Balmer formula:
[IMAGE:0]



\textbf{A.} 656 nm \\
\textbf{B.} 486 nm \\
\textbf{C.} 434 nm \\
\textbf{D.} 397 nm \\

\textbf{Answer:} A \\
\textbf{Explanation:} Using the Balmer formula for hydrogen, the wavelength of the emitted photon when an electron transitions from a higher energy level to a lower one is given by:
[IMAGE:0]
where
n
1
​=2 and
n
2​
=3.
Thus, the wavelength is:
[IMAGE:1]

\hrule
\vspace{1em}


\noindent
\textbf{Q1324.} An object of mass 6 kg is at rest at time = 0 s. A resultant force acts on the object in a constant direction.The magnitude of the resultant force acting on the object varies with time as shown by the graph.
[IMAGE:0]
What is the kinetic energy of the object at time = 6 s?



\textbf{A.} 4J \\
\textbf{B.} 11J \\
\textbf{C.} 16J \\
\textbf{D.} 10J \\

\textbf{Answer:} G \\
\textbf{Explanation:} The kinetic energy equals the work done by the force; the integration of force on time equals the change in momentum; mv; therefore the terminal K.E. can be found:1/2 mv2.

\hrule
\vspace{1em}


\noindent
\textbf{Q1325.} A circuit contains two fixed resistors, X and Y, and a variable resistor W. The variable power supply(PS) has no internal resistance.
How can the most probable changes in resistance and power supply make it: the power dissipated in X increases and the power dissipated in Y decreases?
[IMAGE:0]



\textbf{A.} PS: decreases,W: decreases \\
\textbf{B.} PS: decreases,W: increases \\
\textbf{C.} PS: stays constant,W: stays constant \\
\textbf{D.} PS: increases,W: stays constant \\

\textbf{Answer:} A \\
\textbf{Explanation:} When the resistance of W is increased, The W and Y in total has a larger resistance; X owns less voltage; W and Y owns more voltage.
When the resistance of W is decreased, The W and Y in total has a smaller resistance; X owns more voltage; W and Y owns less voltage.
When the power supply is increased, X owns more voltage; W and Y owns more voltage.
When the power supply is decreased, X owns more voltage; W and Y owns more voltage.
Thus, "PS: impossible, W: increases", "PS: impossible, W: stays constant", "PS: any, W: decreases".

\hrule
\vspace{1em}


\noindent
\textbf{Q1326.} An object of mass 3 kg is at rest at time = 0 s. A resultant force acts on the object in a constant direction.The magnitude of the resultant force acting on the object varies with time as shown by the graph.
[IMAGE:0]
What is the kinetic energy of the object at time = 2 s?



\textbf{A.} 96J \\
\textbf{B.} 81J \\
\textbf{C.} 12J \\
\textbf{D.} 125J \\

\textbf{Answer:} F \\
\textbf{Explanation:} The kinetic energy equals the work done by the force; the integration of force on time equals the change in momentum; mv; therefore the terminal K.E. can be found:1/2 mv
2
.

\hrule
\vspace{1em}


\noindent
\textbf{Q1327.} An object with a mass of 1 kilogram has a velocity of 1 m/s at time t = 0 seconds. A constant force in a certain direction acts on this object. The variation of the magnitude
of the force acting on this object with time is shown in the figure.
[IMAGE:0]
What is the kinetic energy of the object at time = 0.10 s?



\textbf{A.} 1.2J \\
\textbf{B.} 0.81J \\
\textbf{C.} 3.6J \\
\textbf{D.} 1.25J \\

\textbf{Answer:} F \\
\textbf{Explanation:} The kinetic energy equals the work done by the force; the integration of force on time equals the change in momentum; mv=1*1+mv
1
; therefore the terminal K.E. can be found:1/2 mv
2
.

\hrule
\vspace{1em}


\noindent
\textbf{Q1328.} A circuit contains two fixed resistors, X and Y, and a variable resistor W. The constant power supply has no internal resistance.
The resistance of W increases or decreases due to the set of positive integers.
[IMAGE:0]
is the
[IMAGE:1]
element of set of positive integers. If
[IMAGE:2]
is a prime number, W decreases 1 Ω, otherwise, W increases 1Ω. Let
[IMAGE:3]
from 5 to 10. What happens to the power dissipated in W and Y after the whole process?
[IMAGE:4]



\textbf{A.} W: decreases,Y: decreases \\
\textbf{B.} W: decreases,Y: stays constant \\
\textbf{C.} W: decreases,Y: increases \\
\textbf{D.} W: increases,Y: decreases \\

\textbf{Answer:} G \\
\textbf{Explanation:} When the resistance of W is increased (5,7 are prime numbers while 6,8,9,10 are not prime numbers, -2+4>0), X and Y owns less voltage; W owns more voltage. And the resistance of W is also increased and the resistance of Y is constant.

\hrule
\vspace{1em}


\noindent
\textbf{Q1329.} An object of mass 4 kg is at rest at time = 0 s. A resultant force acts on the object in a constant direction.The magnitude of the resultant force acting on the object varies with time as shown by
the graph.
What is the kinetic energy of the object at time = 0.10 s?



\textbf{A.} 0J \\
\textbf{B.} 0.313J \\
\textbf{C.} 1J \\
\textbf{D.} 0.125J \\

\textbf{Answer:} D \\
\textbf{Explanation:} The kinetic energy equals the work done by the force; the integration of force on time equals the change in momentum; mv; therefore the terminal K.E. can be found:1/2 mv2.

\hrule
\vspace{1em}


\noindent
\textbf{Q1330.} A cyclist travels at a constant speed of
[IMAGE:0]
. A second cyclist starts from rest and accelerates at
[IMAGE:1]
. However, after 10 seconds, the second cyclist's bike breaks down and decelerate at
[IMAGE:2]
. Will the second cyclist ever catch up to the first cyclist, and if so, when?



\textbf{A.} Yes, at t = 90 s \\
\textbf{B.} Yes, at t = 30 s \\
\textbf{C.} No \\
\textbf{D.} Yes, at t = 60 s \\

\textbf{Answer:} C \\
\textbf{Explanation:} Distance traveled by the first cyclist in
[IMAGE:0]
:
[IMAGE:1]
.
Distance traveled by the second cyclist in
[IMAGE:2]
:
[IMAGE:3]
.
Speed of the second cyclist after
[IMAGE:4]
:
[IMAGE:5]
.
Let
[IMAGE:6]
be the time after
[IMAGE:7]
when the second cyclist catches up (If Yes).
[IMAGE:8]
[IMAGE:9]
[IMAGE:10]
.

\hrule
\vspace{1em}


\noindent
\textbf{Q1331.} A circuit contains two fixed resistors, X and Y, and a variable resistor W. The constant power supply has no internal resistance.
The resistance of W decreases 1 kΩ and then increases 1100Ω. What happens to the power dissipated in W and Y after the whole process?



\textbf{A.} W: decreases,Y: decreases \\
\textbf{B.} W: decreases,Y: stays constant \\
\textbf{C.} W: decreases,Y: increases \\
\textbf{D.} W: increases,Y: decreases \\

\textbf{Answer:} I \\
\textbf{Explanation:} When the resistance of W is increased (-1000+1100>0), The W and Y in total has a larger resistance; X owns less voltage; W and Y owns more voltage. And the resistance of W is also increased and the resistance of Y is constant. Thus, the answer is I.

\hrule
\vspace{1em}


\noindent
\textbf{Q1332.} A cyclist travels at a constant speed of
[IMAGE:0]
. A second cyclist starts from rest and accelerates at
[IMAGE:1]
. However, after 20 seconds, the second cyclist's bike breaks down and decelerate at
[IMAGE:2]
. Will the second cyclist ever catch up to the first cyclist, and if so, when is the second time?



\textbf{A.} Yes, at t = 80 s \\
\textbf{B.} Yes, at t = 30 s \\
\textbf{C.} No \\
\textbf{D.} Yes, at t = 60 s \\

\textbf{Answer:} A \\
\textbf{Explanation:} Distance traveled by the first cyclist in
[IMAGE:0]
:
[IMAGE:1]
.
Distance traveled by the second cyclist in
[IMAGE:2]
:
[IMAGE:3]
.
Speed of the second cyclist after
[IMAGE:4]
:
[IMAGE:5]
.
Let
[IMAGE:6]
be the time after
[IMAGE:7]
when the second cyclist catches up (If Yes).
[IMAGE:8]
[IMAGE:9]
[IMAGE:10]
[IMAGE:11]
or
[IMAGE:12]
[IMAGE:13]

\hrule
\vspace{1em}


\noindent
\textbf{Q1333.} A boat is traveling upstream at
[IMAGE:0]
relative to the water. A second boat moving with the water starts from the same point and accelerates downstream at
[IMAGE:1]
relative to the water when the first boat passes. If the river current is 3m/s
, how long will it take for the distance of the second boat to be bigger than the the distance of first boat?



\textbf{A.} 20 s \\
\textbf{B.} 40 s \\
\textbf{C.} 60 s \\
\textbf{D.} 80 s \\

\textbf{Answer:} B \\
\textbf{Explanation:} Adjust for the initial current:
[IMAGE:0]
,
[IMAGE:1]
[IMAGE:2]
,
[IMAGE:3]

\hrule
\vspace{1em}


\noindent
\textbf{Q1334.} The light intensity is reduced by 50% after passing through a medium. A student incorrectly calculates the original intensity by increasing the reduced value by 60%, resulting in an error of 45 lux. What is the correct original intensity?



\textbf{A.} 225 lux \\
\textbf{B.} 120 lux \\
\textbf{C.} 150 lux \\
\textbf{D.} 180 lux \\

\textbf{Answer:} A \\
\textbf{Explanation:} Let the original intensity be L.
[IMAGE:0]

\hrule
\vspace{1em}


\noindent
\textbf{Q1335.} A circuit contains two fixed resistors, X and Y, and a variable voltage source V. The internal resistance of the voltage source is negligible.
The voltage of V increases 10V then decreases 9.9V. What happens to the power dissipated in X and in Y after the whole process?
[IMAGE:0]



\textbf{A.} X: decreases,Y: decreases \\
\textbf{B.} X: decreases,Y: stays constant \\
\textbf{C.} X: decreases,Y: increases \\
\textbf{D.} X: increases,Y: decreases \\

\textbf{Answer:} F \\
\textbf{Explanation:} When the voltage of V increases (10-9.9>0), the power dissipated in both X and Y increases because power is proportional to the square of the voltage.

\hrule
\vspace{1em}


\noindent
\textbf{Q1336.} The power of a machine is reduced by 35% due to friction. A technician incorrectly restores the original power by increasing the reduced value by 30%, resulting in an error of 21 W. What is the correct original power?



\textbf{A.} 140 W \\
\textbf{B.} 163 W \\
\textbf{C.} 182 W \\
\textbf{D.} 200 W \\

\textbf{Answer:} E \\
\textbf{Explanation:} Let the original power be P.
[IMAGE:0]

\hrule
\vspace{1em}


\noindent
\textbf{Q1337.} Two runners start from the same point. Runner A runs at a constant speed of
[IMAGE:0]
. Runner B starts from rest and accelerates at
[IMAGE:1]
for the first 8
seconds, then runs at a constant speed. How long does it take for Runner B to catch up to Runner A for the first time, and how far have they traveled?



\textbf{A.} 18s, 90m \\
\textbf{B.} 18s, 104m \\
\textbf{C.} 14s, 75m \\
\textbf{D.} 10s, 60m \\

\textbf{Answer:} B \\
\textbf{Explanation:} Distance traveled by Runner A in
[IMAGE:0]
:
[IMAGE:1]
. Distance traveled by Runner B in
[IMAGE:2]
:
[IMAGE:3]
.
Speed of Runner B after
[IMAGE:4]
:
[IMAGE:5]
.
Let
[IMAGE:6]
be the time after
[IMAGE:7]
when Runner B catches up. The equation is
[IMAGE:8]
. Solving gives
[IMAGE:9]
, so the total time is t=10+8=18s
and the distance is d=6×18=104m
.

\hrule
\vspace{1em}


\noindent
\textbf{Q1338.} The intensity of a sound wave is reduced by 40% due to obstacles. A student incorrectly calculates the original intensity by increasing the reduced value by 50%, resulting in an error of 24 W/m². What is the correct original intensity?



\textbf{A.} 120 W/m² \\
\textbf{B.} 150 W/m² \\
\textbf{C.} 180 W/m² \\
\textbf{D.} 200 W/m² \\

\textbf{Answer:} E \\
\textbf{Explanation:} Let the original intensity be I.
[IMAGE:0]

\hrule
\vspace{1em}


\noindent
\textbf{Q1339.} In a circuit, the voltage is reduced by 20% due to increased load. An engineer incorrectly restores the original voltage by increasing the reduced value by 50%, resulting in an error of 12 V. What is the correct original voltage?



\textbf{A.} 160V \\
\textbf{B.} 80V \\
\textbf{C.} 10
0
V \\
\textbf{D.} 60V \\

\textbf{Answer:} D \\
\textbf{Explanation:} Let the original voltage be V. After a 20% reduction, it becomes 0.8V. The engineer’s calculation 0.8V×1.5=1.2V 1.2V-V=12
V=60V

\hrule
\vspace{1em}


\noindent
\textbf{Q1340.} The elastic potential energy of a spring is reduced by 30% due to temperature drop. A student incorrectly calculates the original energy by increasing the reduced value by 40%, resulting in an error of 42 J. What is the correct original elastic potential energy?



\textbf{A.} 150 J \\
\textbf{B.} 180 J \\
\textbf{C.} 200 J \\
\textbf{D.} 240 J \\

\textbf{Answer:} C \\
\textbf{Explanation:} Let the original energy be E. After a 30% reduction, it becomes 0.7E. The student calculates
0.7E×1.4=0.98E.The error is E-0.98E=0.02E=42,solving E=2100.

\hrule
\vspace{1em}


\noindent
\textbf{Q1341.} A car is traveling at a constant speed of
[IMAGE:0]
on a straight road. A motorcycle starts from rest and accelerates at
[IMAGE:1]
. The motorcycle continues to accelerate until it reaches a maximum speed of
[IMAGE:2]
, after which it decelerate at
[IMAGE:3]
until it reaches the speed
[IMAGE:4]
. How many times will the motorcycle meet the car, and at what times?



\textbf{A.} 1 tine, at t = 20 s \\
\textbf{B.} 2 times, at t = 20 s and t = 50 s \\
\textbf{C.} 1 time, at t = 50 s \\
\textbf{D.} 2 times, at t = 10 s and t = 30 s \\

\textbf{Answer:} D \\
\textbf{Explanation:} acceleration process:
[IMAGE:0]
,
[IMAGE:1]
deceleration process:
[IMAGE:2]
,
[IMAGE:3]
, and it needs the motorcycle decelerates to
[IMAGE:4]
, which is bigger than 5m/s
.

\hrule
\vspace{1em}


\noindent
\textbf{Q1342.} A voltage is reduced by 25% due to increased resistance. A student incorrectly restores the original voltage by increasing the reduced value by 20%, resulting in an error of 9 V. What is the correct original voltage?



\textbf{A.} 36 V \\
\textbf{B.} 90 V \\
\textbf{C.} 60 V \\
\textbf{D.} 72 V \\

\textbf{Answer:} B \\
\textbf{Explanation:} Let the original voltage be V
V
. After a 25% reduction, it becomes 0.75V
[IMAGE:0]
[IMAGE:1]

\hrule
\vspace{1em}


\noindent
\textbf{Q1343.} A circuit contains two fixed resistors, X and Y, and a variable resistor W. The power supply has no internal resistance.The resistance of W decreases 10 Ω and then increases 12Ω. What happens to the current passing over X and Y after the whole process?
[IMAGE:0]



\textbf{A.} X: decreases,Y: decreases \\
\textbf{B.} X: decreases,Y: stays constant \\
\textbf{C.} X: decreases,Y: increases \\
\textbf{D.} X: increases,Y: decreases \\

\textbf{Answer:} D \\
\textbf{Explanation:} When the resistance of W is decreased (-10+9=-1<0), The W and Y in total has a smaller resistance; X owns more voltage; Y owns less voltage. Because of the constant value of resistance of X and Y, the current passing over X increases and the current passing over Y decreases.

\hrule
\vspace{1em}


\noindent
\textbf{Q1344.} The power of a machine is reduced by 40% due to losses. A technician incorrectly restores the original power by increasing the reduced value by 50%, resulting in an error of 24 W. What is the correct original power?



\textbf{A.} 120 W \\
\textbf{B.} 150 W \\
\textbf{C.} 180 W \\
\textbf{D.} 200 W \\

\textbf{Answer:} E \\
\textbf{Explanation:} Analysis
:
Let the original power be P After a 40% reduction, it becomes 0.6P. The technician incorrectly calculates 0.6P×1.5=0.9P. The error is P
−
0.9P=0.1P=24, solving P=240.

\hrule
\vspace{1em}


\noindent
\textbf{Q1345.} A boat is traveling upstream at
[IMAGE:0]
relative to the water. A second boat starts from the same point and accelerates downstream at
[IMAGE:1]
relative to the water when the first boat passes. If the river current is 3m/s
, how long will it take for the speed of the second boat to be bigger than the the distance of first boat?



\textbf{A.} 20s \\
\textbf{B.} 40s \\
\textbf{C.} 60s \\
\textbf{D.} 80s \\

\textbf{Answer:} A \\
\textbf{Explanation:} Adjust for the initial current:
[IMAGE:0]
,
[IMAGE:1]
[IMAGE:2]
,
[IMAGE:3]

\hrule
\vspace{1em}


\noindent
\textbf{Q1346.} The velocity of an object is reduced by 25% due to friction. A student incorrectly calculates the original velocity by increasing the reduced value by 30%, resulting in an error of 18 m/s. What is the correct original velocity?



\textbf{A.} 180 m/s \\
\textbf{B.} 192 m/s \\
\textbf{C.} 210 m/s \\
\textbf{D.} 225 m/s \\

\textbf{Answer:} E \\
\textbf{Explanation:} Analysis
:
[IMAGE:0]
v−0.975v=0.025v=18
v
=720

\hrule
\vspace{1em}


\noindent
\textbf{Q1347.} A car is traveling at a constant speed of
[IMAGE:0]
on a straight road. It overtakes a truck moving at
[IMAGE:1]
in the same direction. At the moment the car overtakes the truck, the truck starts to accelerate at
[IMAGE:2]
. How long will it take for the truck to overtake the car again?



\textbf{A.} 10s \\
\textbf{B.} 16.67s \\
\textbf{C.} 25s \\
\textbf{D.} 33.33s \\

\textbf{Answer:} C \\
\textbf{Explanation:} [IMAGE:0]
,
[IMAGE:1]

\hrule
\vspace{1em}


\noindent
\textbf{Q1348.} In a circuit, the current is reduced by 30% due to increased resistance. An engineer incorrectly restores the original current by increasing the reduced value by 40%, resulting in an error of 12 A. What is the correct original current?



\textbf{A.} 100 A \\
\textbf{B.} 120 A \\
\textbf{C.} 150 A \\
\textbf{D.} 600 A \\

\textbf{Answer:} D \\
\textbf{Explanation:} Analysis:
[IMAGE:0]
[IMAGE:1]
[IMAGE:2]

\hrule
\vspace{1em}


\noindent
\textbf{Q1349.} A circuit contains two fixed resistors, X and Y, and a variable resistor W. The power supply has no internal resistance.
The resistance of W decreases 10 Ω and then increases 12Ω. What happens to the current passing over X and Y after the whole process?
[IMAGE:0]



\textbf{A.} X: decreases,Y: decreases \\
\textbf{B.} X: decreases,Y: stays constant \\
\textbf{C.} X: decreases,Y: increases \\
\textbf{D.} X: increases,Y: decreases \\

\textbf{Answer:} C \\
\textbf{Explanation:} When the resistance of W is increased (-10+12=2>0), The W and Y in total has a larger resistance; X owns less voltage; Y owns more voltage. Because of the constant value of resistance of X and Y, the current passing over X decreases and the current passing over Y increases.

\hrule
\vspace{1em}


\noindent
\textbf{Q1350.} In an experiment, the spring constant of a spring is reduced by 20% due to temperature increase. A student incorrectly calculates the original constant by increasing the reduced value by 50%, resulting in an error of 12 N/m. What is the correct original spring constant?



\textbf{A.} 60 N/m \\
\textbf{B.} 95 N/m \\
\textbf{C.} 10 N/m \\
\textbf{D.} 25 N/m \\

\textbf{Answer:} A \\
\textbf{Explanation:} Let the original spring constant be k. After a 20% reduction, it becomes 0.8k. The student incorrectly calculates it as 0.8k×1.5=1.2k, 1.2k-k=12N/m
k=60N/m

\hrule
\vspace{1em}


\noindent
\textbf{Q1351.} A man is cycling along a straight horizontal road at a constant speed of
[IMAGE:0]
. He passes a boy who is cycling at
[IMAGE:1]
in the same direction. When the man is level with the boy, the boy begins to accelerate at a constant rate of
[IMAGE:2]
. The boy maintains this constant acceleration and the man continues at constant speed until the boy passes the man. What is the time interval between the two instances when the man and the boy are level?



\textbf{A.} 5s \\
\textbf{B.} 10s \\
\textbf{C.} 22.5s \\
\textbf{D.} 35s \\

\textbf{Answer:} E \\
\textbf{Explanation:} [IMAGE:0]
[IMAGE:1]
.

\hrule
\vspace{1em}


\noindent
\textbf{Q1352.} A man is cycling along a straight horizontal road at a constant speed of
[IMAGE:0]
. He passes a boy who is cycling at
[IMAGE:1]
in the same direction. When the man is level with the boy, the boy begins to accelerate at a constant rate of
[IMAGE:2]
. The boy maintains this constant acceleration and the man continues at constant speed until the boy passes the man. What is the time interval between the two instances when the man and the boy are level?



\textbf{A.} 5s \\
\textbf{B.} 10s \\
\textbf{C.} 20s \\
\textbf{D.} 35s \\

\textbf{Answer:} B \\
\textbf{Explanation:} [IMAGE:0]
[IMAGE:1]
.

\hrule
\vspace{1em}


\noindent
\textbf{Q1353.} A man is cycling along a straight horizontal road at a constant speed of
[IMAGE:0]
. He passes a boy who is cycling at
[IMAGE:1]
in the same direction. When the man is level with the boy, the boy begins to accelerate at a constant rate of
[IMAGE:2]
. The boy maintains this constant acceleration and the man continues at constant speed until the boy passes the man. What is the time interval between the two instances when the man and the boy are level?



\textbf{A.} 5s \\
\textbf{B.} 10s \\
\textbf{C.} 22.5s \\
\textbf{D.} 40s \\

\textbf{Answer:} B \\
\textbf{Explanation:} [IMAGE:0]
[IMAGE:1]
.

\hrule
\vspace{1em}


\noindent
\textbf{Q1354.} An advanced proton cyclotron increases particle energy by ΔE per revolution while experiencing the
[IMAGE:0]
time radiative energy loss
[IMAGE:1]
, where N is the number of revolutions. After N revolutions, what's its kinetic energy?



\textbf{A.} [IMAGE:0] \\
\textbf{B.} [IMAGE:1] \\
\textbf{C.} [IMAGE:2] \\
\textbf{D.} [IMAGE:3] \\

\textbf{Answer:} D \\
\textbf{Explanation:} Solution:
[IMAGE:0]
[IMAGE:1]

\hrule
\vspace{1em}


\noindent
\textbf{Q1355.} A circuit contains two fixed resistors, X and Y, and a variable resistor W. The power supply has no internal resistance.
The resistance of W decreases 10 Ω and then increases 8Ω. What happens to the power dissipated in X and in Y after the whole process?
[IMAGE:0]



\textbf{A.} X:decreases, Y:decreases \\
\textbf{B.} X:decreases, Y:stays constant \\
\textbf{C.} X:decreases, Y:increases \\
\textbf{D.} X: increases,Y: decreases \\

\textbf{Answer:} D \\
\textbf{Explanation:} When the resistance of W is decreased totally (-10+8=-2<0), The W and Y in total has a smaller resistance; X owns more voltage; Y owns less voltage.

\hrule
\vspace{1em}


\noindent
\textbf{Q1356.} A bungee jumper (m) falls distance L before cord (2k) stretches, with air resistance F. What's maximum stretch distance?



\textbf{A.} [IMAGE:0] \\
\textbf{B.} [IMAGE:1] \\
\textbf{C.} [IMAGE:2] \\
\textbf{D.} [IMAGE:3] \\

\textbf{Answer:} A \\
\textbf{Explanation:} Solution: Energy conservation: mg(L+x) = kx² + FL.
x² - mgx/k + (FL/k - mgL/k) = 0
[IMAGE:0]
[IMAGE:1]
(mg>F in this process.)
(PS: the option C is a disturbance term. )

\hrule
\vspace{1em}


\noindent
\textbf{Q1357.} A charged particle (2q) with mass 2m accelerates through potential difference V while experiencing constant drag force F. What's its final speed after moving with distance d?



\textbf{A.} [IMAGE:0] \\
\textbf{B.} [IMAGE:1] \\
\textbf{C.} [IMAGE:2] \\
\textbf{D.} [IMAGE:3] \\

\textbf{Answer:} D \\
\textbf{Explanation:} Solution: 2qV = mv² + Fd.

\hrule
\vspace{1em}


\noindent
\textbf{Q1358.} A small steel ball of mass m is released from the top of a semi-circular ramp of radius r as shown in the diagram:
[IMAGE:0]
After being released, the ball moves around the semi-circle to the lowest point at position P and then rises to a maximum height on the other side at position Q before falling down again. Assume that the friction force acting on the ball has a constant magnitude whilst the ball is moving. What is the equation of angle
[IMAGE:1]
of the ball with half kinetic energy (of the ball as it passes the position P at the second time) after passing through the P twice (gravitational field strength = g) (for example, the angle of Q position is 45°)



\textbf{A.} [IMAGE:0] \\
\textbf{B.} [IMAGE:1] \\
\textbf{C.} [IMAGE:2] \\
\textbf{D.} [IMAGE:3] \\

\textbf{Answer:} A \\
\textbf{Explanation:} Solution:
The loss in the kinetic energy is proportional to the distance of the path. And the frictional loss from original position to Q position is equal to
[IMAGE:0]
. Thus, the kinetic energy of the ball as it first passes position P is
[IMAGE:1]
. And the kinetic energy of the ball as it second passes position P is
[IMAGE:2]
.
[IMAGE:3]
[IMAGE:4]
[IMAGE:5]
[IMAGE:6]

\hrule
\vspace{1em}


\noindent
\textbf{Q1359.} A satellite in circular orbit at altitude h experiences atmospheric drag force F=
[IMAGE:0]
. The radius of the Earth is R. What's its orbital speed after completing half an orbit?



\textbf{A.} [IMAGE:0] \\
\textbf{B.} [IMAGE:1] \\
\textbf{C.} [IMAGE:2] \\
\textbf{D.} [IMAGE:3] \\

\textbf{Answer:} B \\
\textbf{Explanation:} Solution:
For a satellite in a circular orbit around the Earth, the total mechanical energy
[IMAGE:0]
is the sum of its kinetic energy
[IMAGE:1]
and potential energy
[IMAGE:2]
:
[IMAGE:3]
,
[IMAGE:4]
,
[IMAGE:5]
. In this case,
[IMAGE:6]
.
Since the satellite is in a stable circular orbit, the centripetal force provided by gravity is equal to the required force for circular motion:
[IMAGE:7]
Substituting this back into the expression for the total mechanical energy, we get:
[IMAGE:8]
This is the initial orbital energy of the satellite.
After losing this energy, the final orbital energy of the satellite is:
[IMAGE:9]
The final orbital speed
[IMAGE:10]
can be found by equating the final kinetic energy to the final orbital energy (neglecting changes in potential energy due to the small change in altitude):
[IMAGE:11]
Solving for
[IMAGE:12]
, we get:
[IMAGE:13]
[IMAGE:14]
This matches option A.

\hrule
\vspace{1em}


\noindent
\textbf{Q1360.} A spring (k) launches a block (m) across a rough surface (μ). If compressed distance d, what's proportion of the block's stopping distance to the compressed distance d?



\textbf{A.} [IMAGE:0] \\
\textbf{B.} [IMAGE:1] \\
\textbf{C.} [IMAGE:2] \\
\textbf{D.} [IMAGE:3] \\

\textbf{Answer:} D \\
\textbf{Explanation:} Solution: Spring energy ½kd² = friction work μmgx.
[IMAGE:0]

\hrule
\vspace{1em}


\noindent
\textbf{Q1361.} A roller coaster car starts from rest at height H, completes a vertical loop of diameter D (H>D), with constant friction force f. The friction before the roller coaster car entering the ring can be neglected. What's its speed at the top of the loop?



\textbf{A.} [IMAGE:0] \\
\textbf{B.} [IMAGE:1] \\
\textbf{C.} [IMAGE:2] \\
\textbf{D.} [IMAGE:3] \\

\textbf{Answer:} B \\
\textbf{Explanation:} Solution: R=D/2
Energy loss = f×πR. Apply conservation: mgH = ½mv² + mg2R + f×πR.
Thus,
[IMAGE:0]

\hrule
\vspace{1em}


\noindent
\textbf{Q1362.} A block slides down a 60° incline of length L with coefficient of kinetic friction μ. If released from rest, what's its speed at the bottom?



\textbf{A.} [IMAGE:0] \\
\textbf{B.} [IMAGE:1] \\
\textbf{C.} [IMAGE:2] \\
\textbf{D.} [IMAGE:3] \\

\textbf{Answer:} D \\
\textbf{Explanation:} Solution: Net acceleration = gsin60° - μgcos60° = \sqrt{}3/2g - 0.5μg. And
[IMAGE:0]
.

\hrule
\vspace{1em}


\noindent
\textbf{Q1363.} A solid pyramid has a height of 160 m and a square base. The material density is 2600 kg/m³. If atmospheric pressure increases by 20 kPa, by how much will the average pressure increase?



\textbf{A.} 10 kPa \\
\textbf{B.} 20 kPa \\
\textbf{C.} 30 kPa \\
\textbf{D.} 40 kPa \\

\textbf{Answer:} B \\
\textbf{Explanation:} Pressure Formula:
[IMAGE:0]
The change in average pressure is equal to the change in atmospheric pressure. Therefore, if atmospheric pressure increases by 20 kPa, the average pressure will also increase by 20 kPa.

\hrule
\vspace{1em}


\noindent
\textbf{Q1364.} A submarine reduces its mass by ejecting water, initially at a constant rate and then slowing down. If the thrust is constant, how does the acceleration magnitude change?



\textbf{A.} First increases at a constant rate, then at a decreasing rate \\
\textbf{B.} Always increases at a constant rate \\
\textbf{C.} First increases at an increasing rate, then at a decreasing rate \\
\textbf{D.} Not changing \\

\textbf{Answer:} C \\
\textbf{Explanation:} The mass loss rate starts constant and later slows. From  a=F/m, decreasing mass causes acceleration to increase. Initially, with constant mass loss, da/dt=F/$m^2$ \cdot |dm/dt|
grows (increasing rate). Later, as mass loss slows, da/dt grows more slowly (decreasing rate). Thus, acceleration first increases at an increasing rate, then at a decreasing rate.

\hrule
\vspace{1em}


\noindent
\textbf{Q1365.} A pendulum bob of mass 0.5m is released from height h above its lowest point. The string encounters constant air resistance force F during its swing. What is the bob's speed at the lowest point (at first time)?



\textbf{A.} [IMAGE:0] \\
\textbf{B.} [IMAGE:1] \\
\textbf{C.} [IMAGE:2] \\
\textbf{D.} [IMAGE:3] \\

\textbf{Answer:} B \\
\textbf{Explanation:} Solution: Energy loss = F×2πh/4 = F×πh/2. Apply energy conservation: 0.5mgh = ½×0.5mv² + F×πh/2.

\hrule
\vspace{1em}


\noindent
\textbf{Q1366.} A solid cone has a height of 240 m and a circular base. The material density is 2200 kg/m³, and atmospheric pressure is 100 kPa. If the base area is doubled, how will the average pressure change?



\textbf{A.} Increase \\
\textbf{B.} Decrease \\
\textbf{C.} Remain the same \\
\textbf{D.} Cannot be determined \\

\textbf{Answer:} C \\
\textbf{Explanation:} Pressure Formula:
[IMAGE:0]
The base area cancels out in the pressure formula, so changing the base area does not affect the final pressure. Therefore, doubling the base area will not change the average pressure.

\hrule
\vspace{1em}


\noindent
\textbf{Q1367.} A small steel ball of mass m is released from the top of a semi-circular ramp of radius r as shown in the diagram:
After being released, the ball moves around the semi-circle to the lowest point at position P and then rises to a maximum height on the other side at position Q before falling down again. Assume that the friction force acting on the ball has a constant magnitude whilst the ball is moving. What is the additional energy E after it passes position P at the second time to make the ball achieve the same height of Q on the left side? (gravitational field strength = g)



\textbf{A.} [IMAGE:0] \\
\textbf{B.} [IMAGE:1] \\
\textbf{C.} [IMAGE:2] \\
\textbf{D.} [IMAGE:3] \\

\textbf{Answer:} D \\
\textbf{Explanation:} Solution:
The loss in the kinetic energy is proportional to the distance of the path. And the frictional loss from original position to Q position is equal to
[IMAGE:0]
. Thus, the kinetic energy of the ball as it passes position P at the first time is
[IMAGE:1]
. And the kinetic energy of the ball as it passes position P at the second time is
[IMAGE:2]
.
[IMAGE:3]
[IMAGE:4]
[IMAGE:5]

\hrule
\vspace{1em}


\noindent
\textbf{Q1368.} A shape is formed by drawing a triangle ABC inside the triangle ADE. BC is parallel to DE. The area of triangle ABC is 18 cm², and the area of triangle ADE is 72 cm². BC = s cm, DE = s + 8 cm.
Determine the height of triangle ADE to side DE.



\textbf{A.} 9cm \\
\textbf{B.} 14cm \\
\textbf{C.} 16cm \\
\textbf{D.} [IMAGE:0] \\

\textbf{Answer:} A \\
\textbf{Explanation:} Area ratio is square of altitude (or side) ratio. Given
[IMAGE:0]
, sides ratio is
[IMAGE:1]
. So,
[IMAGE:2]
. Solving gives s=8
, hence DE=s+8=16cm
. Thus, the height is
[IMAGE:3]
.

\hrule
\vspace{1em}


\noindent
\textbf{Q1369.} A shape is formed by drawing a triangle ABC inside the triangle ADE. BC is parallel to DE. The median from A to BC in triangle ABC is 4 cm, and the median from A to DE in triangle ADE is 10 cm. The two medians are collinear. BC = r cm, DE = r + 9 cm.
Find the length of DE.



\textbf{A.} 12cm \\
\textbf{B.} 15cm \\
\textbf{C.} 18cm \\
\textbf{D.} [IMAGE:0] \\

\textbf{Answer:} B \\
\textbf{Explanation:} Medians ratio equals sides ratio. So,
[IMAGE:0]
. Solving gives r=6
, hence DE=r+9=15cm
.

\hrule
\vspace{1em}


\noindent
\textbf{Q1370.} A truck continuously loads cargo, increasing its mass at a constant rate. If the engine provides constant thrust, how does the truck's acceleration magnitude change?



\textbf{A.} Increasing at an increasing rate \\
\textbf{B.} Increasing at a constant rate \\
\textbf{C.} Increasing at a decreasing rate \\
\textbf{D.} Not changing \\

\textbf{Answer:} G \\
\textbf{Explanation:} The truck's mass mm increases at a constant rate (dm/dt=k) and thrust FF is constant. From a=F/m, acceleration decreases as mass increases. The derivative da/dt=-Fk/$m^2$  shows that the rate of decrease slows over time because mm grows. Thus, acceleration decreases at a decreasing rate.

\hrule
\vspace{1em}


\noindent
\textbf{Q1371.} A solid rectangular prism and a solid cylinder have the same base area of 100 m². The prism has a height of 120 m and a density of 2400 kg/m³; the cylinder has a height of 180 m and a density of 2000 kg/m³. Atmospheric pressure is 100 kPa. What is the sum of their average pressures?



\textbf{A.} 8200 kPa \\
\textbf{B.} 6900 kPa \\
\textbf{C.} 6680 kPa \\
\textbf{D.} 7100 kPa \\

\textbf{Answer:} C \\
\textbf{Explanation:} [IMAGE:0]

\hrule
\vspace{1em}


\noindent
\textbf{Q1372.} A shape is formed by drawing a triangle ABC inside the triangle ADE. BC is parallel to DE. BC+DE=21cm
, and AB=4cm, BD=6cm.
Calculate the length of DE.



\textbf{A.} 11cm \\
\textbf{B.} 13cm \\
\textbf{C.} 15cm \\
\textbf{D.} [IMAGE:0] \\

\textbf{Answer:} C \\
\textbf{Explanation:} From similarity,
[IMAGE:0]
. BC+DE=1.4DE=21cm
, thus, DE=15cm
.

\hrule
\vspace{1em}


\noindent
\textbf{Q1373.} A ball decelerates uniformly from +28.0m/s
to +14.0m/s
in 0.001s
, then accelerates back to +20.0m/s
in 0.006s
.
What is the total displacement during contact?



\textbf{A.} 0.100m \\
\textbf{B.} 0.123m \\
\textbf{C.} 0.132m \\
\textbf{D.} 0.231m \\

\textbf{Answer:} B \\
\textbf{Explanation:} Stage 1:
[IMAGE:0]
. Stage 2:
[IMAGE:1]
. Total
[IMAGE:2]
.

\hrule
\vspace{1em}


\noindent
\textbf{Q1374.} A shape is formed by drawing a triangle ABC inside the triangle ADE. BC is parallel to DE. AB = 8cm, BC = p cm, DE = p + 6 cm. P
is the bigger root of roots to the equation
[IMAGE:0]
.
Determine the length of AD.



\textbf{A.} 12cm \\
\textbf{B.} 15cm \\
\textbf{C.} 18cm \\
\textbf{D.} 20cm \\

\textbf{Answer:} D \\
\textbf{Explanation:} P
is the bigger root of roots to the equation
[IMAGE:0]
. So, p=4(yes),p=2(no)
.
Equal angles and BC || DE imply similarity.
So,
[IMAGE:1]
.
Solve
[IMAGE:2]
that AD=20cm

\hrule
\vspace{1em}


\noindent
\textbf{Q1375.} A ball (v=12.0m/s)
compresses a racket string by 0.04m
before rebounding at 8.0m/s.
What is the peak deceleration?



\textbf{A.} [IMAGE:0] \\
\textbf{B.} [IMAGE:1] \\
\textbf{C.} [IMAGE:2] \\
\textbf{D.} [IMAGE:3] \\

\textbf{Answer:} A \\
\textbf{Explanation:} Energy loss implies non-constant force, but assuming average deceleration:
[IMAGE:0]
.

\hrule
\vspace{1em}


\noindent
\textbf{Q1376.} A shape is formed by drawing a triangle ABC inside the triangle ADE. BC is parallel to DE. The height from A to BC is 3 cm, the height from A to DE is 9 cm, BC = n cm, and DE = n + 4 cm.
Find the length of DE.



\textbf{A.} 6cm \\
\textbf{B.} 8cm \\
\textbf{C.} 10cm \\
\textbf{D.} [IMAGE:0] \\

\textbf{Answer:} A \\
\textbf{Explanation:} Heights ratio is
[IMAGE:0]
, so sides ratio is also
[IMAGE:1]
. Thus,
[IMAGE:2]
. Solving gives n=2
, hence DE=n+4=6cm
.

\hrule
\vspace{1em}


\noindent
\textbf{Q1377.} A topspin ball slows horizontally from 30.0m/s
to 24.0m/s
while gaining 6.0m/s
downward due to spin. Contact time is 0.00425s
.
What is the net acceleration?



\textbf{A.} [IMAGE:0] \\
\textbf{B.} [IMAGE:1] \\
\textbf{C.} [IMAGE:2] \\
\textbf{D.} [IMAGE:3] \\

\textbf{Answer:} C \\
\textbf{Explanation:} [IMAGE:0]
,
[IMAGE:1]
. Net
[IMAGE:2]
.
[IMAGE:3]
.

\hrule
\vspace{1em}


\noindent
\textbf{Q1378.} A solid pyramid has a height of 150 m and a square base. The average pressure on the ground is 450 kPa, and atmospheric pressure is 100 kPa. What is the density of the pyramid material?



\textbf{A.} 500 kg/m³ \\
\textbf{B.} 600 kg/m³ \\
\textbf{C.} 700 kg/m³ \\
\textbf{D.} 800 kg/m³ \\

\textbf{Answer:} C \\
\textbf{Explanation:} [IMAGE:0]

\hrule
\vspace{1em}


\noindent
\textbf{Q1379.} The ratio of mass of two solid spheres is 9:4, the ratio of density of these spheres is 2:3. Find the ratio of the radius of the first sphere and the diameter of the second sphere.



\textbf{A.} 3:1 \\
\textbf{B.} 2:3 \\
\textbf{C.} 3:2 \\
\textbf{D.} 3:4 \\

\textbf{Answer:} D \\
\textbf{Explanation:} The volume ratio is 27:8, the radius ratio is 3 :2. Therefore, the ratio of the radius of the first sphere and the diameter of the second sphere is 3:4 (aha, other option are all misleading term.)

\hrule
\vspace{1em}


\noindent
\textbf{Q1380.} A tennis ball (v=25.0m/s)
penetrates a net, slowing to 5.0m/s
over 0.030m.
What is the deceleration magnitude?



\textbf{A.} [IMAGE:0] \\
\textbf{B.} [IMAGE:1] \\
\textbf{C.} [IMAGE:2] \\
\textbf{D.} [IMAGE:3] \\

\textbf{Answer:} E \\
\textbf{Explanation:} [IMAGE:0]
.

\hrule
\vspace{1em}


\noindent
\textbf{Q1381.} A leaking balloon experiences a constant thrust as it releases gas. If the gas release rate accelerates, how does the balloon's acceleration magnitude change?



\textbf{A.} Increasing at an increasing rate \\
\textbf{B.} Increasing at a constant rate \\
\textbf{C.} Increasing at a decreasing rate \\
\textbf{D.} Not changing \\

\textbf{Answer:} E \\
\textbf{Explanation:} The balloon's mass m
m
decreases with an accelerating rate
[IMAGE:0]
.
[IMAGE:1]
decreasing mass causes acceleration to increase. However, if the mass loss rate itself increases, the second derivative of acceleration
[IMAGE:2]
becomes negative, leading to a decreasing rate of acceleration increase. This contradiction suggests a design flaw. A corrected scenario might involve increasing mass or adjusted conditions.

\hrule
\vspace{1em}


\noindent
\textbf{Q1382.} A shape is formed by drawing a triangle ABC inside the triangle ADE. BC is parallel to DE. The perimeter of triangle ABC is 24cm, BC = m cm, DE = m + 8cm, and the perimeter of triangle ADE is 40 cm.
Calculate the length of AD+AE.



\textbf{A.} 14cm \\
\textbf{B.} 16cm \\
\textbf{C.} 18cm \\
\textbf{D.} 20cm \\

\textbf{Answer:} D \\
\textbf{Explanation:} Perimeters of similar triangles are in the ratio of their sides. So,
[IMAGE:0]
. Solving gives m=12
, so DE=m+8=20cm.
Thus, AD+AE=40-DE=20cm
,

\hrule
\vspace{1em}


\noindent
\textbf{Q1383.} The ratio of mass of two solid spheres is 9:4, the ratio of density of these spheres is 2:3. Find the ratio of the radius of these two solid spheres.



\textbf{A.} 4:9 \\
\textbf{B.} 2:3 \\
\textbf{C.} 3:2 \\
\textbf{D.} 32:81 \\

\textbf{Answer:} C \\
\textbf{Explanation:} The volume ratio is 27:8, the radius ratio is therefore: 3 :2.

\hrule
\vspace{1em}


\noindent
\textbf{Q1384.} A tennis ball approaches at 29.0m/s
horizontally. The racket applies a constant upward force, adding a vertical velocity of 5.0m/s
while reducing horizontal speed to 17.0m/s
. The contact lasts 0.005s
.
What is the magnitude of the net acceleration?



\textbf{A.} [IMAGE:0] \\
\textbf{B.} [IMAGE:1] \\
\textbf{C.} [IMAGE:2] \\
\textbf{D.} [IMAGE:3] \\

\textbf{Answer:} A \\
\textbf{Explanation:} [IMAGE:0]
,
[IMAGE:1]
. Net
[IMAGE:2]
. Acceleration
[IMAGE:3]
.

\hrule
\vspace{1em}


\noindent
\textbf{Q1385.} A shape is formed by drawing a triangle ABC inside the triangle ADE. BC is parallel to DE. The area of triangle ABC is 12 cm², BC = 3w cm, DE = w + 5 cm, and the area of triangle ADE is 48 cm².
Determine the length of DE.



\textbf{A.} 21cm \\
\textbf{B.} 15cm \\
\textbf{C.} [IMAGE:0] \\
\textbf{D.} 8cm \\

\textbf{Answer:} E \\
\textbf{Explanation:} The ratio of areas of similar triangles is the square of the ratio of their sides. So,
[IMAGE:0]
. Thus,
[IMAGE:1]
. Solving gives w=1
, so DE=w+5=6cm
.

\hrule
\vspace{1em}


\noindent
\textbf{Q1386.} The ratio of radius of first solid spheres and diameter of second solid spheres is 3:4, the ratio of density of these spheres is 2:3. Find the ratio of the mass of these two solid spheres.



\textbf{A.} 27:12 \\
\textbf{B.} 27:16 \\
\textbf{C.} 27:32 \\
\textbf{D.} 27:64 \\

\textbf{Answer:} A \\
\textbf{Explanation:} The volume ratio is 81:8, the mass ratio is therefore: 27 :4. The option D is a misleading term (mix up radius and diameter).

\hrule
\vspace{1em}


\noindent
\textbf{Q1387.} A tennis ball strikes the court surface at 18.0m/s
at a
[IMAGE:0]
angle. The bounce reverses the vertical velocity component and reduces the horizontal speed by 20%
. The contact time is 0.02s
.
What is the magnitude of the average acceleration during impact?
(
[IMAGE:1]
)



\textbf{A.} [IMAGE:0] \\
\textbf{B.} [IMAGE:1] \\
\textbf{C.} [IMAGE:2] \\
\textbf{D.} [IMAGE:3] \\

\textbf{Answer:} B \\
\textbf{Explanation:} Vertical velocity changes from
[IMAGE:0]
to
[IMAGE:1]
(
[IMAGE:2]
), while horizontal velocity changes from
[IMAGE:3]
to
[IMAGE:4]
(
[IMAGE:5]
). Total
[IMAGE:6]
. Acceleration
[IMAGE:7]
.

\hrule
\vspace{1em}


\noindent
\textbf{Q1388.} A shape is formed by drawing a triangle ABC inside the triangle ADE.
BC is parallel to DE. AC = 7cm, BC = z cm, DE = 2z + 1cm, CE = z + 1cm.



\textbf{A.} 11cm \\
\textbf{B.} 13cm \\
\textbf{C.} 15cm \\
\textbf{D.} [IMAGE:0] \\

\textbf{Answer:} C \\
\textbf{Explanation:} Using similarity,
[IMAGE:0]
. Since
[IMAGE:1]
, we get
[IMAGE:2]
. Solving yields
[IMAGE:3]
(negative root -1 is discarded), hence
[IMAGE:4]
cm.

\hrule
\vspace{1em}


\noindent
\textbf{Q1389.} The ratio of radius of two solid spheres is 3:4, the ratio of density of these spheres is 2:3. Find the ratio of the mass of these two solid spheres.



\textbf{A.} 4:9 \\
\textbf{B.} 27:16 \\
\textbf{C.} 27:32 \\
\textbf{D.} 16:81 \\

\textbf{Answer:} C \\
\textbf{Explanation:} The volume ratio is 81:64, the mass ratio is therefore: 27 :64.

\hrule
\vspace{1em}


\noindent
\textbf{Q1390.} A tennis ball travelling at 24.0m/s
is hit by a racket. As a result of the impact, the ball returns back along its original path having undergone a change in velocity of  36.0m/s
. The acceleration of the ball whilst in contact with the racket is constant with magnitude
[IMAGE:0]
.
What is the total distance travelled by the ball whilst in contact with the racket?



\textbf{A.} 0.00cm \\
\textbf{B.} 4.80cm \\
\textbf{C.} 7.50cm \\
\textbf{D.} 15.0cm \\

\textbf{Answer:} C \\
\textbf{Explanation:} The initial velocity is 24.0m/s
; the terminal one is -12.0m/s
, the process is not symmetrical in time;
the first half is 24.0*24.0/4800/2=0.06m=6.00cm
.
the second half is 12.0*12.0/4800/2=0.060m=1.50cm
.
Thus, the answer is 7.5cm
.

\hrule
\vspace{1em}


\noindent
\textbf{Q1391.} The ratio of radius of two solid spheres is 4:3, the ratio of density of these spheres is 1:3. Find the ratio of the mass of these two solid spheres.



\textbf{A.} 1:9 \\
\textbf{B.} 16:27 \\
\textbf{C.} 8:9 \\
\textbf{D.} 4:81 \\

\textbf{Answer:} E \\
\textbf{Explanation:} The volume ratio is 64:27, the mass ratio is therefore: 64:81.

\hrule
\vspace{1em}


\noindent
\textbf{Q1392.} A solid cylinder and a solid cone have the same height of 200 m and the same base area. The density of the cylinder is 2500 kg/m³, and the density of the cone is 3000 kg/m³. Atmospheric pressure is 100 kPa. What is the difference in average pressure between the two?



\textbf{A.} 1200 kPa \\
\textbf{B.} 2200 kPa \\
\textbf{C.} 3200 kPa \\
\textbf{D.} 4100 kPa \\

\textbf{Answer:} E \\
\textbf{Explanation:} [IMAGE:0]

\hrule
\vspace{1em}


\noindent
\textbf{Q1393.} A shape is formed by drawing a triangle ABC inside the triangle ADE.
BC is parallel to DE. AC = 5 cm, BC = y cm, DE = y + 2 cm, EC = y - 3 cm.
Calculate the length of DE.



\textbf{A.} 6cm \\
\textbf{B.} 7cm \\
\textbf{C.} 8cm \\
\textbf{D.} 10cm \\

\textbf{Answer:} B \\
\textbf{Explanation:} Since BC || DE, triangles ABC and ADE are similar.
So,
[IMAGE:0]
.
Given
[IMAGE:1]
, we have
[IMAGE:2]
. Solving this gives
[IMAGE:3]
(discarding the negative root -2), so
[IMAGE:4]
.

\hrule
\vspace{1em}


\noindent
\textbf{Q1394.} A transverse wave propagates in a medium with a wave speed of 15 m/s, a frequency of 3 Hz, and an amplitude of 5 cm. During the wave's propagation, a particle starts vibrating from the equilibrium position. After 3 periods, what is the maximum distance of the particle from the equilibrium position?



\textbf{A.} 5 cm \\
\textbf{B.} 10 cm \\
\textbf{C.} 20 cm \\
\textbf{D.} 30 cm \\

\textbf{Answer:} A \\
\textbf{Explanation:} The particle is only vibrating in the direction perpendicular to the propagation direction (the option C is a misleading term). Thus, it can be the amplitude which is 5 cm.

\hrule
\vspace{1em}


\noindent
\textbf{Q1395.} A tennis ball travelling at 10.0m/s
is hit by a racket. As a result of the impact, the ball returns back along its original path having undergone a change in velocity of 40.0m/s
. The acceleration of the ball whilst in contact with the racket is constant with magnitude
[IMAGE:0]
.
What is the total distance travelled by the ball whilst in contact with the racket?



\textbf{A.} 0.00cm \\
\textbf{B.} 5.00cm \\
\textbf{C.} 9.60cm \\
\textbf{D.} 10.0cm \\

\textbf{Answer:} D \\
\textbf{Explanation:} The initial velocity is 10.0m/s
; the terminal one is -30m/s
, the process is not symmetrical in time; the first half is 10.0*10.0/5000/2=0.01m=1cm
; the second half is 30.0*30.0/5000/2=0.04m=9cm
. Thus, the answer is 10cm
.

\hrule
\vspace{1em}


\noindent
\textbf{Q1396.} A satellite adjusts its orbit in space by ejecting fuel at a constant rate. If the engine applies a constant force, how does the magnitude of the satellite's acceleration change?



\textbf{A.} Increasing at an increasing rate \\
\textbf{B.} Increasing at a constant rate \\
\textbf{C.} Increasing at a decreasing rate \\
\textbf{D.} Not changing \\

\textbf{Answer:} A \\
\textbf{Explanation:} The satellite's mass mm decreases over time(dm/dt<0) while the force FF is constant. From a=F/m decreasing mass causes acceleration to increase. Since the mass loss rate is constant, the derivative da/dt=F/$m^2$ \cdot |dm/dt| grows as mm decreases. Thus, acceleration increases at an increasing rate.

\hrule
\vspace{1em}


\noindent
\textbf{Q1397.} A shape is formed by drawing a triangle ABC inside the triangle ADE. BC is parallel to DE. AB = x-3 cm BC = x - 1 cm DE = x + 3 cm DB = 3 cm.
Calculate the length of DE.



\textbf{A.} 5cm \\
\textbf{B.} 7cm \\
\textbf{C.} 9cm \\
\textbf{D.} 10cm \\

\textbf{Answer:} E \\
\textbf{Explanation:} AB/AD=BC/DE.
[IMAGE:0]
with
[IMAGE:1]
[IMAGE:2]
.
[IMAGE:3]
.

\hrule
\vspace{1em}


\noindent
\textbf{Q1398.} A tennis ball travelling at
[IMAGE:0]
is hit by a racket. As a result of the impact, the ball returns back along its original path having undergone a change in velocity of
[IMAGE:1]
. The acceleration of the ball whilst in contact with the racket is constant with magnitude
[IMAGE:2]
.
What is the total distance travelled by the ball whilst in contact with the racket?



\textbf{A.} 2.00cm \\
\textbf{B.} 2.50cm \\
\textbf{C.} 3.00cm \\
\textbf{D.} 14.4cm \\

\textbf{Answer:} B \\
\textbf{Explanation:} The initial velocity is 10.0m/s
; the terminal one is -10.0m/s
, the process is symmetrical in time; the first half is 10.0*10.0/4000/2=0.025m=1.25cm
. Thus, the answer is 2.50cm
.

\hrule
\vspace{1em}


\noindent
\textbf{Q1399.} A mechanical wave propagates on a string with a wave speed of 12 m/s, a frequency of 3 Hz, and an amplitude of 4 cm. If the wave travels in the positive x-direction, and at a certain moment, the particle at x=0 is at the equilibrium position and vibrating upwards, what position is the particle at x=2 meters in?



\textbf{A.} Maximum displacement and vibrating upwards \\
\textbf{B.} Maximum displacement and vibrating downwards \\
\textbf{C.} Equilibrium position and vibrating upwards \\
\textbf{D.} Equilibrium position and vibrating downwards \\

\textbf{Answer:} D \\
\textbf{Explanation:} [IMAGE:0]
[IMAGE:1]
, which means half the period. Thus, the answer is "Equilibrium position and vibrating downwards".

\hrule
\vspace{1em}


\noindent
\textbf{Q1400.} A tennis ball travelling at 11.0m/s
is hit by a racket. As a result of the impact, the ball returns back along its original path having undergone a change in velocity of 22.0m/s
. The acceleration of the ball whilst in contact with the racket is constant with magnitude
[IMAGE:0]
.
What is the total distance travelled by the ball whilst in contact with the racket?



\textbf{A.} 2.00cm \\
\textbf{B.} 4.00cm \\
\textbf{C.} 9.00cm \\
\textbf{D.} 14.4cm \\

\textbf{Answer:} A \\
\textbf{Explanation:} The initial velocity is 11.0m/s
; the terminal one is -11.0m/s
, the process is symmetrical in time; the first half is 11.0*11.0/6000/2=0.01m=1.00cm
. Thus, the answer is 2.00cm
.

\hrule
\vspace{1em}


\noindent
\textbf{Q1401.} A solid frustum of a cone with lower base radius R
, upper base radius
[IMAGE:0]
, and height
[IMAGE:1]
fits inside a hollow cylinder. The cylinder has the same internal radius as the lower base radius of the frustum and a height equal to the height of the frustum. What fraction of the empty space is occupied in the cylinder?



\textbf{A.} [IMAGE:0] \\
\textbf{B.} [IMAGE:1] \\
\textbf{C.} [IMAGE:2] \\
\textbf{D.} [IMAGE:3] \\

\textbf{Answer:} C \\
\textbf{Explanation:} The volume of the frustum
[IMAGE:0]
. The volume of the cylinder
[IMAGE:1]
. The ratio
[IMAGE:2]
.
Thus, the answer is
[IMAGE:3]

\hrule
\vspace{1em}


\noindent
\textbf{Q1402.} Liquid fuel causes effective mass to oscillate as m(t) = m₀ - kt + 0.001εcos(ωt) with constant thrust.
Acceleration behavior?



\textbf{A.} It is increasing at an increasing rate. \\
\textbf{B.} It is increasing at a constant rate. \\
\textbf{C.} It is increasing at a decreasing rate. \\
\textbf{D.} It is not changing. \\

\textbf{Answer:} G \\
\textbf{Explanation:} a(t) = F/[m(t)] shows oscillation can affect the acceleration which depends on the parameter settings of the mass equation, even the 0.001 is a small number.

\hrule
\vspace{1em}


\noindent
\textbf{Q1403.} A solid frustum of a cone with lower base radius R
, upper base radius
[IMAGE:0]
, and height
[IMAGE:1]
fits inside a hollow cylinder. The cylinder has the same internal radius as the lower base radius of the frustum and a height equal to the height of the frustum. What fraction of the space inside the cylinder is occupied by the frustum?



\textbf{A.} [IMAGE:0] \\
\textbf{B.} [IMAGE:1] \\
\textbf{C.} [IMAGE:2] \\
\textbf{D.} [IMAGE:3] \\

\textbf{Answer:} D \\
\textbf{Explanation:} The volume of the frustum
[IMAGE:0]
. The volume of the cylinder
[IMAGE:1]
. The ratio
[IMAGE:2]
.

\hrule
\vspace{1em}


\noindent
\textbf{Q1404.} A transverse wave propagates in a medium with a wavelength of 8 meters, a frequency of 4 Hz, and an amplitude of 6 cm. During the wave's propagation, a particle starts vibrating from its maximum displacement. How long will it take for the particle to return to its maximum displacement (in the same direction) for the first time?



\textbf{A.} 0.125 seconds \\
\textbf{B.} 0.25
seconds \\
\textbf{C.} 0.5
seconds \\
\textbf{D.} 1
seconds \\

\textbf{Answer:} B \\
\textbf{Explanation:} [IMAGE:0]

\hrule
\vspace{1em}


\noindent
\textbf{Q1405.} A solid regular octahedron with edge length
[IMAGE:0]
fits inside a hollow sphere. The sphere has a diameter equal to the distance between two opposite vertices of the octahedron. What fraction of the space inside the sphere is taken up by the octahedron?



\textbf{A.} [IMAGE:0] \\
\textbf{B.} [IMAGE:1] \\
\textbf{C.} [IMAGE:2] \\
\textbf{D.} [IMAGE:3] \\

\textbf{Answer:} C \\
\textbf{Explanation:} The distance between two opposite vertices of a regular octahedron with edge length
[IMAGE:0]
is
[IMAGE:1]
. The volume of the octahedron
[IMAGE:2]
. The volume of the sphere
[IMAGE:3]
. The ratio
[IMAGE:4]
.

\hrule
\vspace{1em}


\noindent
\textbf{Q1406.} A solid square pyramid with a square base of side length a
and height h
fits inside a hollow cube. The cube has an edge length equal to the slant height of the pyramid. And
[IMAGE:0]
. What fraction of the space inside the cube is occupied by the pyramid?



\textbf{A.} [IMAGE:0] \\
\textbf{B.} [IMAGE:1] \\
\textbf{C.} [IMAGE:2] \\
\textbf{D.} [IMAGE:3] \\

\textbf{Answer:} C \\
\textbf{Explanation:} The slant height of the pyramid
[IMAGE:0]
. The volume of the pyramid
[IMAGE:1]
. The volume of the cube
[IMAGE:2]
. , Thus, the ratio
[IMAGE:3]
.

\hrule
\vspace{1em}


\noindent
\textbf{Q1407.} A transverse travels 10m within a period 5s. The amplitude of the wave is 10m. Find the distance travelled by one of the particles in the waveform during 16s.



\textbf{A.} 32m \\
\textbf{B.} 64m \\
\textbf{C.} 100m \\
\textbf{D.} 128m \\

\textbf{Answer:} D \\
\textbf{Explanation:} The particle is only vibrating in the direction perpendicular to the propagation direction (the option A is a misleading term); In each cycle, the object passes 4 amplitudes; Therefore
[IMAGE:0]
.

\hrule
\vspace{1em}


\noindent
\textbf{Q1408.} A solid torus (doughnut - shaped) with inner radius
[IMAGE:0]
and outer radius
[IMAGE:1]
fits inside a hollow cylinder. And its height is
[IMAGE:2]
. The cylinder has a radius equal to the outer radius of the torus and a height equal to the (outer) diameter of the torus's tube. What fraction of the space inside the cylinder is taken up by the torus?



\textbf{A.} [IMAGE:0] \\
\textbf{B.} [IMAGE:1] \\
\textbf{C.} [IMAGE:2] \\
\textbf{D.} [IMAGE:3] \\

\textbf{Answer:} D \\
\textbf{Explanation:} The volume of the torus
[IMAGE:0]
. The volume of the cylinder
[IMAGE:1]
. The ratio
[IMAGE:2]
.

\hrule
\vspace{1em}


\noindent
\textbf{Q1409.} A transverse travels 10m within a period 4s. The amplitude of the wave is 5m. Find the distance travelled by one of the particles in the waveform during 16s.



\textbf{A.} 40m \\
\textbf{B.} 100m \\
\textbf{C.} 80m \\
\textbf{D.} 200m \\

\textbf{Answer:} C \\
\textbf{Explanation:} The particle is only vibrating in the direction perpendicular to the propagation direction; In each cycle, the object passes 4 amplitudes; Therefore
[IMAGE:0]
.

\hrule
\vspace{1em}


\noindent
\textbf{Q1410.} Rocket's exhaust velocity decreases linearly with time (
[IMAGE:0]
,
[IMAGE:1]
) while mass flow rate is constant as k
. The initial mass of rocket is
[IMAGE:2]
.
[IMAGE:3]
is a constant.
Check the acceleration behavior?



\textbf{A.} Quadratic increase \\
\textbf{B.} Linear increase \\
\textbf{C.} Step function \\
\textbf{D.} Quadratic decrease \\

\textbf{Answer:} B \\
\textbf{Explanation:} Thrust formula:
[IMAGE:0]
(momentum theorem)
Given conditions:
[IMAGE:1]
(constant mass flow rate)
[IMAGE:2]
(exhaust velocity increases linearly with time)
Therefore, thrust:
[IMAGE:3]
(linear increase)
Rocket mass:
[IMAGE:4]
(linear decrease)
Acceleration:
[IMAGE:5]
When
[IMAGE:6]
, the acceleration exhibits precisely linear growth.

\hrule
\vspace{1em}


\noindent
\textbf{Q1411.} A sled sliding on ice continuously collects snow, increasing its mass. If the engine applies a constant thrust, how does the magnitude of the sled's acceleration change?



\textbf{A.} Increasing at an increasing rate \\
\textbf{B.} Increasing at a constant rate \\
\textbf{C.} Increasing at a decreasing rate \\
\textbf{D.} Not changing \\

\textbf{Answer:} G \\
\textbf{Explanation:} By Newton's second law, a=F/m. The sled's mass mm increases over time (dm/dt>0) while the thrust FF is constant, so acceleration decreases. Since mass increases at a constant rate (m=m0+kt), the derivative da/dt=-Fk/($m_0$+kt)^2  shows that the rate of decrease slows over time. Thus, acceleration decreases at a decreasing rate.

\hrule
\vspace{1em}


\noindent
\textbf{Q1412.} A solid prism of height 200 m has a triangular base. The density of the material is 2200 kg/m³. Atmospheric pressure is 100 kPa. What is the average pressure on the ground under the prism?



\textbf{A.} 4500 kPa \\
\textbf{B.} 5400 kPa \\
\textbf{C.} 6400 kPa \\
\textbf{D.} 7400 kPa \\

\textbf{Answer:} A \\
\textbf{Explanation:} Explanation:
Volume Calculation: Volume of a prism = base area × height.
Mass Calculation: Mass = volume × density.
Weight Calculation: Weight = mass × gravitational field strength (
g
=10N/kg).
Total Pressure Calculation: Average pressure = weight/base area
​
+atmospheric pressure.

\hrule
\vspace{1em}


\noindent
\textbf{Q1413.} A solid regular tetrahedron with edge length a
fits inside a hollow cube. The cube has an edge length equal to the edge length of the tetrahedron. What fraction of the space inside the cube is occupied by the tetrahedron?



\textbf{A.} [IMAGE:0] \\
\textbf{B.} [IMAGE:1] \\
\textbf{C.} [IMAGE:2] \\
\textbf{D.} [IMAGE:3] \\

\textbf{Answer:} A \\
\textbf{Explanation:} The height of a regular tetrahedron with edge length a
is
[IMAGE:0]
. The volume of the tetrahedron
[IMAGE:1]
and the volume of the cube
[IMAGE:2]
. The ratio
[IMAGE:3]
.

\hrule
\vspace{1em}


\noindent
\textbf{Q1414.} A 5kg firework explodes in mid-air, splitting into two parts. One part with a mass of 2kg moves horizontally to the left at 10m/s after the explosion. What is the horizontal velocity of the other part?



\textbf{A.} 5 m/s to the right \\
\textbf{B.} 8 m/s to the right \\
\textbf{C.} 4 m/s to the right \\
\textbf{D.} 6.67 m/s to the right \\

\textbf{Answer:} D \\
\textbf{Explanation:} Momentum is conserved during the explosion, with the total momentum before the explosion being zero (as the firework was stationary). After the explosion, the momenta of the two parts are equal in magnitude but opposite in direction.
[IMAGE:0]

\hrule
\vspace{1em}


\noindent
\textbf{Q1415.} A rokect with mass
[IMAGE:0]
. For first half of fuel (total mass is
[IMAGE:1]
): burns at rate R with thrust F. For second half: burns at 2R with thrust 1.8F.
What describes acceleration?



\textbf{A.} Constant throughout \\
\textbf{B.} Jumps up at midpoint \\
\textbf{C.} Jumps down at midpoint \\
\textbf{D.} Always increasing \\

\textbf{Answer:} B \\
\textbf{Explanation:} First phase:
[IMAGE:0]
with final value
[IMAGE:1]
Second phase:
[IMAGE:2]
, whose numerator is suddently increasing and denominator is decreasing faster.

\hrule
\vspace{1em}


\noindent
\textbf{Q1416.} A solid hemisphere of radius
[IMAGE:0]
is inside a hollow cone. The cone has a circular base with radius
[IMAGE:1]
and a height equal to
[IMAGE:2]
. What fraction of the space inside the cone is taken up by the hemisphere?



\textbf{A.} [IMAGE:0] \\
\textbf{B.} [IMAGE:1] \\
\textbf{C.} [IMAGE:2] \\
\textbf{D.} [IMAGE:3] \\

\textbf{Answer:} D \\
\textbf{Explanation:} [IMAGE:0]
.

\hrule
\vspace{1em}


\noindent
\textbf{Q1417.} A 2kg object A is moving to the right at 6m/s and collides with a stationary 3kg object B. After the collision, object A moves to the left at 3m/s. What is the velocity of object B after the collision?



\textbf{A.} 3 m/s \\
\textbf{B.} 5 m/s \\
\textbf{C.} 4 m/s \\
\textbf{D.} 6 m/s \\

\textbf{Answer:} B \\
\textbf{Explanation:} According to the law of conservation of momentum, the total momentum of the system remains constant before and after the collision.
Solving:
[IMAGE:0]

\hrule
\vspace{1em}


\noindent
\textbf{Q1418.} A solid cylinder of height 300 m has a circular base. The density of the material is 2800 kg/m³. Atmospheric pressure is 100 kPa. What is the average pressure on the ground under the cylinder?



\textbf{A.} 8500 kPa \\
\textbf{B.} 9400 kPa \\
\textbf{C.} 1040 kPa \\
\textbf{D.} 1140 kPa \\

\textbf{Answer:} A \\
\textbf{Explanation:} 1.
Explanation:
Volume Calculation: Volume of a cylinder = base area × height.
2.
Mass Calculation: Mass = volume × density.
3.
Weight Calculation: Weight = mass × gravitational field strength (
g
=10N/kg).
4.
Total Pressure Calculation: Average pressure = aweight/base are
​
+atmospheric pressure.

\hrule
\vspace{1em}


\noindent
\textbf{Q1419.} A solid right - circular cone with radius r=3R
and height h=4R
fits inside a hollow cylinder. The cylinder has the same internal radius as the cone and a height equal to the slant height of the cone. What fraction of the space inside the cylinder is occupied by the cone?



\textbf{A.} [IMAGE:0] \\
\textbf{B.} [IMAGE:1] \\
\textbf{C.} [IMAGE:2] \\
\textbf{D.} [IMAGE:3] \\

\textbf{Answer:} D \\
\textbf{Explanation:} [IMAGE:0]
.

\hrule
\vspace{1em}


\noindent
\textbf{Q1420.} A rocket (mass is
[IMAGE:0]
) adjusts its thrust to always equal 10% of its instantaneous fuel mass (F = 0.2M). Fuel burns at constant rate(
[IMAGE:1]
, where
[IMAGE:2]
is a constant, and the inital mass of fuel is
[IMAGE:3]
).
How does acceleration behave?



\textbf{A.} Constant at 0.1 m/s² \\
\textbf{B.} Increases \\
\textbf{C.} Decreases \\
\textbf{D.} Proportional to 1/m \\

\textbf{Answer:} C \\
\textbf{Explanation:} Based on acceleration-force equation, it has
initial state:
[IMAGE:0]
operating state:
[IMAGE:1]
it can be proved that
[IMAGE:2]

\hrule
\vspace{1em}


\noindent
\textbf{Q1421.} A 4 kg object is moving with an initial velocity of 6 m/s. It collides with a stationary object of mass 2 kg. After the collision, the 4 kg object moves with a velocity of 0 m/s. What is the velocity of the 2 kg object after the collision? The table is without friction



\textbf{A.} 6 m/s \\
\textbf{B.} 8 m/s \\
\textbf{C.} 12 m/s \\
\textbf{D.} 3.5 m/s \\

\textbf{Answer:} C \\
\textbf{Explanation:} By conservation of momentum; The velocity * mass of the entire system is conserved before and after the collision

\hrule
\vspace{1em}


\noindent
\textbf{Q1422.} A solid cube of side length
[IMAGE:0]
fits perfectly inside a hollow rectangular prism. The rectangular prism has the same internal length and width as the side length of the cube, and its height is equal to the diagonal of the cube's base. What fraction of the space inside the rectangular prism is taken up by the cube?



\textbf{A.} [IMAGE:0] \\
\textbf{B.} [IMAGE:1] \\
\textbf{C.} [IMAGE:2] \\
\textbf{D.} [IMAGE:3] \\

\textbf{Answer:} E \\
\textbf{Explanation:} [IMAGE:0]
.

\hrule
\vspace{1em}


\noindent
\textbf{Q1423.} A charged sphere moves on a smooth icy surface, experiencing a magnetic force proportional to its speed but in the opposite direction (with the magnetic field perpendicular to the ice). How does the magnitude of the sphere’s acceleration change?



\textbf{A.} It is increasing at an increasing rate. \\
\textbf{B.} It is increasing at a constant rate. \\
\textbf{C.} It is increasing at a decreasing rate. \\
\textbf{D.} It is not changing. \\

\textbf{Answer:} G \\
\textbf{Explanation:} The magnetic force
F
=−
kv
(where
k
is a constant), so acceleration
a
=
F
/
m
=−
kv
/
m
. The magnitude of acceleration is |
a
|=
kv
/
m
. As the sphere slows down due to the opposing force, speed
v
decreases over time. This causes |
a
| to decrease as well. The rate of decrease in
v
depends on
a
, leading to a scenario similar to exponential decay. The rate at which |
a
| decreases also diminishes over time, meaning acceleration decreases at a decreasing rate, making option G correct.

\hrule
\vspace{1em}


\noindent
\textbf{Q1424.} A rocket adjusts its thrust to always equal 20% of its instantaneous total mass (F = 0.2m). Fuel burns at constant rate.
How does acceleration behave?



\textbf{A.} Constant at 0.1 m/s² \\
\textbf{B.} Increases linearly \\
\textbf{C.} Decreases exponentially \\
\textbf{D.} Constant at 0.2/m \\

\textbf{Answer:} D \\
\textbf{Explanation:} a = F/m = 0.2m/m = 0.2 m/s² always.

\hrule
\vspace{1em}


\noindent
\textbf{Q1425.} A solid rectangular prism of height 120 m has a rectangular base. The density of the material is 2500 kg/m³. Atmospheric pressure is 100 kPa. What is the average pressure on the ground under the prism?



\textbf{A.} 80 kPa \\
\textbf{B.} 180 kPa \\
\textbf{C.} 20 kPa \\
\textbf{D.} 380 kPa \\

\textbf{Answer:} E \\
\textbf{Explanation:} Volume Calculation: Volume of a rectangular prism = base area × height.
Mass Calculation: Mass = volume × density.
Weight Calculation: Weight = mass × gravitational field strength (
g
=10N/kg).
Total Pressure Calculation: Average pressure = weight/base area
​
+atmospheric pressure.
[IMAGE:0]

\hrule
\vspace{1em}


\noindent
\textbf{Q1426.} A solid sphere of radius r fits inside a hollow cylinder. The cylinder has the same internal diameter and length as the diameter of the sphere. What fraction of the empty space inside the cylinder is taken up?



\textbf{A.} [IMAGE:0] \\
\textbf{B.} [IMAGE:1] \\
\textbf{C.} [IMAGE:2] \\
\textbf{D.} [IMAGE:3] \\

\textbf{Answer:} A \\
\textbf{Explanation:} [IMAGE:0]
.

\hrule
\vspace{1em}


\noindent
\textbf{Q1427.} An object is horizontally launched with an initial velocity of 10 m/s, and air resistance is neglected. What is the vertical component of the object's velocity at the end of the fifth second? gravitational acceleration is taken as 10
$𝑚$
/
$𝑠$.



\textbf{A.} 15 m/s \\
\textbf{B.} 30
m/s \\
\textbf{C.} 50 m/s \\
\textbf{D.} 20 m/s \\

\textbf{Answer:} D \\
\textbf{Explanation:} In horizontal projectile motion, the vertical component of velocity is only affected by gravitational acceleration. Using the SUVAT equation: v = u + at Where:
• v is the final vertical velocity
• u is the initial vertical velocity (0 m/s, as the object is horizontally launched)
• a is the gravitational acceleration (10 m/s²)
• t is the time (5 seconds) Substituting the values: v = 0 + 10 * 5 = 50 m/s Thus, the vertical component of velocity at the end of the fifth second is 50 m/s.

\hrule
\vspace{1em}


\noindent
\textbf{Q1428.} An object is dropped from a height with an initial velocity of 0. It falls for 3 seconds with a gravitational acceleration of 10 m/s
²
. What is the velocity of the object at the end of these 3 seconds?



\textbf{A.} 15m/s \\
\textbf{B.} 30m \\
\textbf{C.} 25m \\
\textbf{D.} 20m \\

\textbf{Answer:} B \\
\textbf{Explanation:} Using the SUVAT equation: v = u + at Where:
• v is the final velocity
• u is the initial velocity (0 m/s)
• a is the acceleration (10 m/s²)
• t is the time (3 seconds) Substituting the values: v = 0 + 10 * 3 = 30 m/s Thus, the velocity at the end of 3 seconds is 30 m/s

\hrule
\vspace{1em}


\noindent
\textbf{Q1429.} A rocket travelling in space is burning its fuel at a decreasing rate
[IMAGE:0]
, where
[IMAGE:1]
is inital rate of burning fuel,
[IMAGE:2]
denotes time and
[IMAGE:3]
is a constant. By expelling the burnt fuel through a nozzle, the engine is applying a constant force to the rocket.
What is happening to the magnitude of the velocity of the rocket?



\textbf{A.} It is increasing at an increasing rate. \\
\textbf{B.} It is increasing at a constant rate. \\
\textbf{C.} It is increasing at a decreasing rate. \\
\textbf{D.} It is not changing. \\

\textbf{Answer:} A \\
\textbf{Explanation:} The purposive force is a constant; the mass is decreasing.
PS: Though the fuel consumption is at a decreasing rate, the mass is still decreasing.
Thus, the acceleration is therefore increasing;
the jerk(rate of change of acceleration) is
[IMAGE:0]
, which means the acceleration increasing is at a contant rate in time.

\hrule
\vspace{1em}


\noindent
\textbf{Q1430.} A rectangular prism has dimensions 2, 4, and 3. What is the perimeter of a triangular ABC (the dashed line triangular in the diagram below)? B is in the center of the bottom face. A and C lie in the vertices of the rectangular prism.



\textbf{A.} [IMAGE:0] \\
\textbf{B.} [IMAGE:1] \\
\textbf{C.} [IMAGE:2] \\
\textbf{D.} [IMAGE:3] \\

\textbf{Answer:} B \\
\textbf{Explanation:} [IMAGE:0]
[IMAGE:1]
Thus,
[IMAGE:2]

\hrule
\vspace{1em}


\noindent
\textbf{Q1431.} A solid cone of height 420 m has a circular base. The density of the material is 2100 kg/m³. Atmospheric pressure is 100 kPa. What is the average pressure on the ground under the cone?



\textbf{A.} 98 kPa \\
\textbf{B.} 108 kPa \\
\textbf{C.} 198 kPa \\
\textbf{D.} 980 kPa \\

\textbf{Answer:} G \\
\textbf{Explanation:} Explanation:
Volume Calculation: Volume of a cone = 1/3
​
×base area×height.
Mass Calculation: Mass = volume × density.
Weight Calculation: Weight = mass × gravitational field strength (
g
=10N/kg).
Total Pressure Calculation: Average pressure = weight/base area
​
+atmospheric pressure.
[IMAGE:0]

\hrule
\vspace{1em}


\noindent
\textbf{Q1432.} A balloon ascends at a constant speed while releasing gas, causing its mass to decrease linearly. Assuming buoyant force and drag force are both constant and independent of speed, how does the magnitude of the balloon’s acceleration change?



\textbf{A.} It is increasing at an increasing rate. \\
\textbf{B.} It is increasing at a constant rate. \\
\textbf{C.} It is increasing at a decreasing rate. \\
\textbf{D.} It is not changing. \\

\textbf{Answer:} B \\
\textbf{Explanation:} At constant speed, net force is zero: buoyant force
B
equals the sum of weight
mg
and drag force
f
. As mass
m
decreases linearly, weight
mg
decreases, leading to a net force
Fnet
​=
B
−
mg
−
f
. Since
B
and
f
are constant,
Fnet
​ increases linearly as
m
decreases. By Newton’s second law
a
=
Fnet
​/
m
, both
Fnet
​ and
m
change linearly. Substituting
Fnet
​=
B
−
mg
−
f
into
a
, we get
a
=(
B
−
mg
−
f
)/
m
=(
B
−
f
)/
m
−
g
. As
m
decreases linearly, (
B
−
f
)/
m
increases linearly, making
a
increase linearly. Thus, acceleration increases at a constant rate, making option B correct.

\hrule
\vspace{1em}


\noindent
\textbf{Q1433.} A cube has sides of length 2. What is the area of a triangular ABC (the dashed line triangular in the diagram below)?



\textbf{A.} [IMAGE:0] \\
\textbf{B.} [IMAGE:1] \\
\textbf{C.} [IMAGE:2] \\
\textbf{D.} [IMAGE:3] \\

\textbf{Answer:} D \\
\textbf{Explanation:} [IMAGE:0]
.

\hrule
\vspace{1em}


\noindent
\textbf{Q1434.} A car starts from 3 m/s and accelerates uniformly at 4 m/s² for 5 seconds. What is the final velocity of the car?



\textbf{A.} 20m/s \\
\textbf{B.} 17m/s \\
\textbf{C.} 25m/s \\
\textbf{D.} 23m/s \\

\textbf{Answer:} D \\
\textbf{Explanation:} Using the SUVAT equation: v=u+at
Where:
• v is the final velocity
• u is the initial velocity (3 m/s)
• a is the acceleration (4 m/s²)
• t is the time (5 seconds)
Substituting the values: v=3+(4)*(5)=23 m/s Thus, the final velocity is 23 m/s.

\hrule
\vspace{1em}


\noindent
\textbf{Q1435.} A rocket travelling in space is burning its fuel at a constant rate. By expelling the burnt fuel through a nozzle, the engine is applying a constant force to the rocket.
What is happening to the magnitude of the distance of the rocket?



\textbf{A.} It is dncreasing at an increasing rate. \\
\textbf{B.} It is dncreasing at a constant rate. \\
\textbf{C.} It is dncreasing at a decreasing rate. \\
\textbf{D.} It is not changing. \\

\textbf{Answer:} E \\
\textbf{Explanation:} The purposive force is a constant; the mass is decreasing; the acceleration is therefore increasing; so the velocity is creasing at an increasing rate and the distance is creasing at an increasing rate.

\hrule
\vspace{1em}


\noindent
\textbf{Q1436.} A cube has sides of length 2. What is the perimeter of a triangular ABC (the dashed line triangular in the diagram below)?



\textbf{A.} [IMAGE:0] \\
\textbf{B.} [IMAGE:1] \\
\textbf{C.} [IMAGE:2] \\
\textbf{D.} [IMAGE:3] \\

\textbf{Answer:} B \\
\textbf{Explanation:} [IMAGE:0]

\hrule
\vspace{1em}


\noindent
\textbf{Q1437.} A rocket with a mass of
[IMAGE:0]
is flying through the air, and at a certain point, its velocity is
[IMAGE:1]
in a horizontal direction, with its fuel about to run out. The rocket explodes into two pieces at a certain moment. One piece, with a mass of
[IMAGE:2]
flies off in the direction opposite to
[IMAGE:3]
with a velocity of
[IMAGE:4]
. Determine the velocity
[IMAGE:5]
of the other piece after the explosion.



\textbf{A.} 4m/s \\
\textbf{B.} 8m/s \\
\textbf{C.} 12m/s \\
\textbf{D.} 15m/s \\

\textbf{Answer:} C \\
\textbf{Explanation:} The momentum of the system is conserved, because the internal forces are much greater than the gravitational force.
By conservation of momentum, The velocity * mass of the entire system is conserved before and after the collision.
[IMAGE:0]
[IMAGE:1]
Thus,
[IMAGE:2]
. Pay attention to the direction, or the option B will be a disturbance term.

\hrule
\vspace{1em}


\noindent
\textbf{Q1438.} A cube has sides of length 5. What is the length of a line joining the midpoint of one face to the midpoint of an adjacent face (the dashed line in the diagram below)?



\textbf{A.} [IMAGE:0] \\
\textbf{B.} [IMAGE:1] \\
\textbf{C.} [IMAGE:2] \\
\textbf{D.} [IMAGE:3] \\

\textbf{Answer:} A \\
\textbf{Explanation:} [IMAGE:0]
.

\hrule
\vspace{1em}


\noindent
\textbf{Q1439.} A car travels on a straight road with its traction force decreasing linearly over time until it reaches a constant value. Assuming constant resistance, how does the magnitude of the car’s acceleration change?



\textbf{A.} It is increasing at an increasing rate. \\
\textbf{B.} It is increasing at a constant rate. \\
\textbf{C.} It is increasing at a decreasing rate. \\
\textbf{D.} It is not changing. \\

\textbf{Answer:} F \\
\textbf{Explanation:} Traction force
F
(
t
) decreases linearly, and resistance
f
is constant. Net force
Fnet
​
=
F
(
t
)
−
f
. When
F
(
t
)>
f
, net force is positive, and acceleration
a
=
Fnet
​
/
m
decreases linearly as
F
(
t
) decreases, meaning the magnitude of acceleration (positive) is decreasing at a constant rate. When
F
(
t
)<
f
, net force becomes negative, and deceleration
a
=(
f
−
F
(
t
))/
m
increases linearly as
F
(
t
) continues to decrease. However, if the question focuses on the deceleration phase, the magnitude of acceleration (now negative, so its magnitude is |
a
|=(
f
−
F
(
t
))/
m
) increases linearly. But typically, such questions might only consider the acceleration phase before
F
(
t
) equals resistance. Thus, during the acceleration phase, the magnitude of acceleration decreases at a constant rate, making option F correct.

\hrule
\vspace{1em}


\noindent
\textbf{Q1440.} A rocket travelling in space is burning its fuel at a constant rate. By expelling the burnt fuel through a nozzle, the engine is applying a constant force to the rocket.
What is happening to the magnitude of the velocity of the rocket?



\textbf{A.} It is dncreasing at an increasing rate. \\
\textbf{B.} It is dncreasing at a constant rate. \\
\textbf{C.} It is dncreasing at a decreasing rate. \\
\textbf{D.} It is not changing. \\

\textbf{Answer:} E \\
\textbf{Explanation:} The purposive force is a constant; the mass is decreasing; the acceleration is therefore increasing; so the velocity is creasing at an increasing rate.

\hrule
\vspace{1em}


\noindent
\textbf{Q1441.} A rectangular prism has dimensions 1, 2, and 3. What is the length of a line joining a vertex to the midpoint of the middle opposite edge (the dashed line in the diagram below)?



\textbf{A.} [IMAGE:0] \\
\textbf{B.} [IMAGE:1] \\
\textbf{C.} [IMAGE:2] \\
\textbf{D.} [IMAGE:3] \\

\textbf{Answer:} C \\
\textbf{Explanation:} [IMAGE:0]

\hrule
\vspace{1em}


\noindent
\textbf{Q1442.} A solid pyramid with a height of 20 m has a square base. The density of the material is 1500 kg/m³. Atmospheric pressure is 100 kPa. What is the average pressure on the ground under the pyramid?



\textbf{A.} 100 kPa \\
\textbf{B.} 200 kPa \\
\textbf{C.} 300 kPa \\
\textbf{D.} 400 kPa \\

\textbf{Answer:} B \\
\textbf{Explanation:} [IMAGE:0]

\hrule
\vspace{1em}


\noindent
\textbf{Q1443.} A future vehicle of mass 500 kg travels in a straight line along a horizontal road, as shown in the acceleration/deceleration–time graph.
What is the average resultant force acting on the vehicle over the time for which it is accelerating / decelerating?



\textbf{A.} -380N \\
\textbf{B.} 420N \\
\textbf{C.} -960N \\
\textbf{D.} -7500N \\

\textbf{Answer:} E \\
\textbf{Explanation:} What is the average resultant force acting on the vehicle over the time for which it is accelerating?
The vehicle is accelerating from 0s to 10s (all the time because acceleration is bigger than zero, which may be a trap).
0s~5s:
[IMAGE:0]
5s~10s:
[IMAGE:1]
Thus, the acceleration in average is therefore
[IMAGE:2]
;
[IMAGE:3]
. (D option is a disturbance term)

\hrule
\vspace{1em}


\noindent
\textbf{Q1444.} A 5 kg object is moving with an initial velocity of 6 m/s. It collides with a stationary object of mass 50 kg. After the collision, the 5 kg object moves with a velocity of -1 m/s. What is the velocity of the 3 kg object after the collision? The table is without friction.



\textbf{A.} 0.3m/s \\
\textbf{B.} 0.5m/s \\
\textbf{C.} 0.7m/s \\
\textbf{D.} 1m/s \\

\textbf{Answer:} C \\
\textbf{Explanation:} By conservation of momentum, The velocity * mass of the entire system is conserved before and after the collision.
Pay attention to the direction, or the option B will be a disturbance term.

\hrule
\vspace{1em}


\noindent
\textbf{Q1445.} A cube has sides of length 4. What is the length of a line joining a vertex to the midpoint of a face diagonal on the opposite face (the dashed line in the diagram below)?



\textbf{A.} [IMAGE:0] \\
\textbf{B.} [IMAGE:1] \\
\textbf{C.} [IMAGE:2] \\
\textbf{D.} 6 \\

\textbf{Answer:} B \\
\textbf{Explanation:} [IMAGE:0]
.

\hrule
\vspace{1em}


\noindent
\textbf{Q1446.} An electromagnetic wave propagates in a vacuum. The distance between adjacent wave crests is 5 cm. What is the frequency of this electromagnetic wave? Given that the speed of light is:3.0x10
8
m/s



\textbf{A.} 2GHz \\
\textbf{B.} 1GHz \\
\textbf{C.} 3GHz \\
\textbf{D.} 5GHz \\

\textbf{Answer:} C \\
\textbf{Explanation:} The frequency of electromagnetic waves is equal to the speed of light divided by the wavelength; the distance between wave crests is 5 cm, then the wavelength is 10 cm.

\hrule
\vspace{1em}


\noindent
\textbf{Q1447.} A crane lifts a heavy load with constant power. As the load’s speed increases, what happens to the magnitude of its acceleration?



\textbf{A.} It is increasing at an increasing rate. \\
\textbf{B.} It is increasing at a constant rate. \\
\textbf{C.} It is increasing at a decreasing rate. \\
\textbf{D.} It is not changing. \\

\textbf{Answer:} G \\
\textbf{Explanation:} Power
P
=
Fv
, so with constant power, as speed
v
increases, the tension force
F
=
P
/
v
decreases. By Newton’s second law
F
−
mg
=
ma
, as
F
decreases, the net force
F
−
mg
decreases, causing acceleration
a
=(
F
−
mg
)/
m
to decrease. However, since
F
=
P
/
v
, the rate at which
F
decreases slows down as
v
increases (because
v
is in the denominator), leading to a deceleration in the decrease of acceleration. Thus, acceleration decreases at a decreasing rate, making option G correct.

\hrule
\vspace{1em}


\noindent
\textbf{Q1448.} A cube has sides of length 3. What is the length of a line joining the midpoint of one edge to the midpoint of an opposite edge (the dashed line in the diagram below)?



\textbf{A.} [IMAGE:0] \\
\textbf{B.} [IMAGE:1] \\
\textbf{C.} [IMAGE:2] \\
\textbf{D.} [IMAGE:3] \\

\textbf{Answer:} C \\
\textbf{Explanation:} [IMAGE:0]
.

\hrule
\vspace{1em}


\noindent
\textbf{Q1449.} A 5 kg object is moving with an initial velocity of 6 m/s. It collides with a stationary object of mass 2 kg. After the collision, the 5 kg object moves with a velocity of 4 m/s. What is the velocity of the 3 kg object after the collision? The table is without friction.



\textbf{A.} 3m/s \\
\textbf{B.} 5m/s \\
\textbf{C.} 6.7m/s \\
\textbf{D.} 8m/s \\

\textbf{Answer:} B \\
\textbf{Explanation:} By conservation of momentum, The velocity * mass of the entire system is conserved before and after the collision.

\hrule
\vspace{1em}


\noindent
\textbf{Q1450.} A transverse travels 1m within a period 2s. The amplitude of the wave is 2m. The initial amplitude of the first particle on the waveform is -2m; find out the time when it reaches the crest for the first time.



\textbf{A.} 2s \\
\textbf{B.} 1s \\
\textbf{C.} 5s \\
\textbf{D.} 6s \\

\textbf{Answer:} B \\
\textbf{Explanation:} The particle starts at the trough of the wave and reaches the crest in half of the period.

\hrule
\vspace{1em}


\noindent
\textbf{Q1451.} A car of mass 700 kg travels in a straight line along a horizontal road, as shown in the deceleration–time graph.
What is the average resultant force acting on the car over the time for which it is decelerating?



\textbf{A.} -380N \\
\textbf{B.} -420N \\
\textbf{C.} -840N \\
\textbf{D.} -1190N \\

\textbf{Answer:} C \\
\textbf{Explanation:} By the gradient of the graph in the linear region; the velocity of 5s is
[IMAGE:0]
; the acceleration in average is therefore
[IMAGE:1]
;
[IMAGE:2]
.

\hrule
\vspace{1em}


\noindent
\textbf{Q1452.} A rectangular prism has dimensions 2, 4, and 3. What is the length of a line joining a vertex to the center of the opposite face (the dashed line in the diagram below)?



\textbf{A.} [IMAGE:0] \\
\textbf{B.} [IMAGE:1] \\
\textbf{C.} [IMAGE:2] \\
\textbf{D.} [IMAGE:3] \\

\textbf{Answer:} A \\
\textbf{Explanation:} [IMAGE:0]

\hrule
\vspace{1em}


\noindent
\textbf{Q1453.} A car goes to a rest with constant speed 50 m/s. At a certain moment, it decelerates uniformly at -2 m/s² until it stops. Then, it continues to accelerate uniformly at 3m/s² to 30 m/s. What is the total of the deceleration-acceleration process.



\textbf{A.} 10s \\
\textbf{B.} 25s \\
\textbf{C.} 35s \\
\textbf{D.} 45s \\

\textbf{Answer:} C \\
\textbf{Explanation:} 1.deceleration process
Using the SUVAT equation:
[IMAGE:0]
, where:
[IMAGE:1]
is the final velocity (0 m/s),
[IMAGE:2]
is the initial velocity (50 m/s),
[IMAGE:3]
is the acceleration (-2 m/s²), and
[IMAGE:4]
is the time.
Substituting the values:
[IMAGE:5]
Thus,
[IMAGE:6]
2.acceleration process
Using the SUVAT equation:
[IMAGE:7]
, where:
[IMAGE:8]
is the final velocity (30 m/s),
[IMAGE:9]
is the initial velocity (0 m/s),
[IMAGE:10]
is the acceleration (3 m/s²), and
[IMAGE:11]
is the time.
Substituting the values:
[IMAGE:12]
Thus,
[IMAGE:13]
So, the answer is
[IMAGE:14]

\hrule
\vspace{1em}


\noindent
\textbf{Q1454.} A future vehicle of mass 500 kg travels in a straight line along a horizontal road, as shown in the acceleration–time graph.
What is the final kinetic energy of the vehicle?



\textbf{A.} 50J \\
\textbf{B.} 125J \\
\textbf{C.} 150J \\
\textbf{D.} 360000J \\

\textbf{Answer:} F \\
\textbf{Explanation:} The vehicle is accelerating from 0s to 10s (all the time because acceleration is not zero).
0s~5s:
[IMAGE:0]
5s~10s:
[IMAGE:1]
Thus,
[IMAGE:2]

\hrule
\vspace{1em}


\noindent
\textbf{Q1455.} A hot air balloon ascending releases sandbags at a constant rate, causing its total mass to decrease uniformly. Given that the buoyant force remains constant and the net force on the balloon is also constant, how does the magnitude of the balloon’s acceleration change?



\textbf{A.} It is increasing at an increasing rate. \\
\textbf{B.} It is increasing at a constant rate. \\
\textbf{C.} It is increasing at a decreasing rate. \\
\textbf{D.} It is not changing. \\

\textbf{Answer:} B \\
\textbf{Explanation:} By Newton’s second law
F
=
ma
, with constant net force
F
and mass
m
decreasing linearly over time, acceleration
a
=
F
/
m
increases linearly as mass decreases. Thus, acceleration increases at a constant rate, making option B correct.

\hrule
\vspace{1em}


\noindent
\textbf{Q1456.} A cube has sides of unit length. What is the length of a line joining a vertex to the center of cube (the dashed line in the diagram below)?



\textbf{A.} [IMAGE:0] \\
\textbf{B.} [IMAGE:1] \\
\textbf{C.} [IMAGE:2] \\
\textbf{D.} [IMAGE:3] \\

\textbf{Answer:} C \\
\textbf{Explanation:} [IMAGE:0]
.

\hrule
\vspace{1em}


\noindent
\textbf{Q1457.} A transverse travels 1m within a period 2s. The amplitude of the wave is 10m. Find the distance travelled by one of the particles in the waveform during 10s.



\textbf{A.} 200m \\
\textbf{B.} 240m \\
\textbf{C.} 50m \\
\textbf{D.} 210m \\

\textbf{Answer:} A \\
\textbf{Explanation:} The particle is only vibrating in the direction perpendicular to the propagation direction; In each cycle; the object passes 4 amplitudes; Therefore 5*10*4=200m

\hrule
\vspace{1em}


\noindent
\textbf{Q1458.} A car goes to a rest with constant speed 50m/s. At a certain moment, it decelerates uniformly at -2 m/s² for 10 s. What is the final velocity of the car?



\textbf{A.} 10m/s \\
\textbf{B.} 30m/s \\
\textbf{C.} 40m/s \\
\textbf{D.} 70m/s \\

\textbf{Answer:} B \\
\textbf{Explanation:} Using the SUVAT equation:
[IMAGE:0]
, where:
[IMAGE:1]
is the final velocity,
[IMAGE:2]
is the initial velocity (50 m/s),
[IMAGE:3]
is the acceleration (-2 m/s²), and
[IMAGE:4]
is the time (10 seconds).
Substituting the values:
[IMAGE:5]
Thus, the final velocity is 30 m/s. And option D is a disturbance term (which means acceleration process).

\hrule
\vspace{1em}


\noindent
\textbf{Q1459.} A future vehicle of mass 600 kg travels in a straight line along a horizontal road, as shown in the acceleration–time graph.
What is the final velocity of the vehicle?



\textbf{A.} 25m/s \\
\textbf{B.} 50m/s \\
\textbf{C.} 100m/s \\
\textbf{D.} 120m/s \\

\textbf{Answer:} D \\
\textbf{Explanation:} The vehicle is accelerating from 0s to 10s (all the time because acceleration is not zero).
0s~5s:
[IMAGE:0]
5s~10s:
[IMAGE:1]

\hrule
\vspace{1em}


\noindent
\textbf{Q1460.} A transverse travels 10000m within a period 1s. The amplitude of the wave is 6m. Find the distance travelled by one of the particles in the waveform during 10s.



\textbf{A.} 140m \\
\textbf{B.} 240m \\
\textbf{C.} 50m \\
\textbf{D.} 200m \\

\textbf{Answer:} B \\
\textbf{Explanation:} The particle is only vibrating in the direction perpendicular to the propagation direction; In each cycle; the object passes 4 amplitudes; Therefore 10*6*4=240m

\hrule
\vspace{1em}


\noindent
\textbf{Q1461.} A car starts from rest and accelerates uniformly at 1 m/s² for 1 min. What is the final velocity of the car?



\textbf{A.} 1m/s \\
\textbf{B.} 30m/s \\
\textbf{C.} 60m/s \\
\textbf{D.} 90m/s \\

\textbf{Answer:} C \\
\textbf{Explanation:} Using the SUVAT equation:
[IMAGE:0]
, where: v is the final velocity,
[IMAGE:1]
is the initial velocity (0 m/s, as the car starts from rest), a is the acceleration (1 m/s²), and t is the time (1minute = 60 seconds). Substituting the values:
[IMAGE:2]
Thus, the final velocity is 60 m/s. And option A is a disturbance term.

\hrule
\vspace{1em}


\noindent
\textbf{Q1462.} A future vehicle of mass 400 kg travels in a straight line along a horizontal road, as shown in the acceleration–time graph.
What is the average resultant force acting on the vehicle over the time for which it is accelerating?



\textbf{A.} 380N \\
\textbf{B.} 420N \\
\textbf{C.} 550N \\
\textbf{D.} 5090N \\

\textbf{Answer:} G \\
\textbf{Explanation:} The vehicle is accelerating from 0s to 10s (all the time because acceleration is not zero).
0s~5s:
[IMAGE:0]
5s~10s:
[IMAGE:1]
Thus, the acceleration in average is therefore
[IMAGE:2]
;
[IMAGE:3]
.

\hrule
\vspace{1em}


\noindent
\textbf{Q1463.} A cube has sides of length 2. What is the length of a line joining a vertex to the midpoint of one of the opposite edges (the dashed line in the diagram below)?



\textbf{A.} [IMAGE:0] \\
\textbf{B.} [IMAGE:1] \\
\textbf{C.} [IMAGE:2] \\
\textbf{D.} 3 \\

\textbf{Answer:} D \\
\textbf{Explanation:} [IMAGE:0]
.

\hrule
\vspace{1em}


\noindent
\textbf{Q1464.} The ratio of the radii of two uniformly charged spherical bodies in two separate vacuums is 1:2; the ratio of the surface charge densities is 3:2; find the ratio of the charges (the charge is distributed on the surface of the spheres).



\textbf{A.} 1:9 \\
\textbf{B.} 3:4 \\
\textbf{C.} 5:6 \\
\textbf{D.} 3:8 \\

\textbf{Answer:} D \\
\textbf{Explanation:} The Surface area ratio is :1:4, the charges ratio is therefore: 3:18

\hrule
\vspace{1em}


\noindent
\textbf{Q1465.} A car of mass 700 kg travels in a straight line along a horizontal road, as shown in the speed–time graph.
What is the average resultant force acting on the car over the time for which it is accelerating?



\textbf{A.} 380N \\
\textbf{B.} 420N \\
\textbf{C.} 550N \\
\textbf{D.} 1290N \\

\textbf{Answer:} E \\
\textbf{Explanation:} The terminal velocity is
[IMAGE:0]
; the acceleration in average is therefore
[IMAGE:1]
;
[IMAGE:2]
.

\hrule
\vspace{1em}


\noindent
\textbf{Q1466.} A car of mass 820 kg travels in a straight line along a horizontal road, as shown in the distance–time graph.
What is the average resultant force acting on the car over the time for which it is accelerating?



\textbf{A.} 380N \\
\textbf{B.} 1148N \\
\textbf{C.} 1150N \\
\textbf{D.} 1190N \\

\textbf{Answer:} B \\
\textbf{Explanation:} By the gradient of the graph in the linear region; the terminal velocity is
[IMAGE:0]
; the acceleration in average is therefore
[IMAGE:1]
;
[IMAGE:2]
.

\hrule
\vspace{1em}


\noindent
\textbf{Q1467.} The height ratio of two cones with the same base area is 2:3; the density ratio of the two cones is 1:2; find the mass ratio of the two.



\textbf{A.} 1:9 \\
\textbf{B.} 3:4 \\
\textbf{C.} 1:6 \\
\textbf{D.} 2:1 \\

\textbf{Answer:} E \\
\textbf{Explanation:} The volume of a cone is 1/3 times the base area multiplied by the height. height ratio of two cones with the same base area is 2:3; the density ratio of the two cones is 1:2;so mass ratio is 1:3

\hrule
\vspace{1em}


\noindent
\textbf{Q1468.} A car of mass 380 kg travels in a straight line along a horizontal road.
The car accelerates non-uniformly from rest for 5.0 seconds and then moves at constant speed, as shown in the distance–time graph.
What is the average resultant force acting on the car over the time for which it is accelerating?



\textbf{A.} 380N \\
\textbf{B.} 420N \\
\textbf{C.} 500N \\
\textbf{D.} 1200N \\

\textbf{Answer:} A \\
\textbf{Explanation:} By the gradient of the graph in the linear region; the terminal velocity is
[IMAGE:0]
; the acceleration in average is therefore
[IMAGE:1]
;
[IMAGE:2]
.

\hrule
\vspace{1em}


\noindent
\textbf{Q1469.} A cube has sides of unit length. What is the length of a line joining a vertex to the center of cube (the dashed line in the diagram below)?



\textbf{A.} [IMAGE:0] \\
\textbf{B.} [IMAGE:1] \\
\textbf{C.} [IMAGE:2] \\
\textbf{D.} [IMAGE:3] \\

\textbf{Answer:} C \\
\textbf{Explanation:} [IMAGE:0]
.

\hrule
\vspace{1em}


\noindent
\textbf{Q1470.} The ratio of the radii of two hollow thin spherical shells is 1:2, and the ratio of their surface densities is 2:3. The ratio of the masses of these two spherical shells is:



\textbf{A.} 1:9 \\
\textbf{B.} 3:4 \\
\textbf{C.} 1:6 \\
\textbf{D.} 2:1 \\

\textbf{Answer:} C \\
\textbf{Explanation:} The Surface area ratio is :1:4, the mass ratio is therefore: 2:12

\hrule
\vspace{1em}


\noindent
\textbf{Q1471.} A car of mass 710 kg travels in a straight line along a horizontal road.
The car accelerates non-uniformly from rest for 5.0 seconds and then moves at constant speed, as shown in the distance–time graph.
What is the average resultant force acting on the car over the time for which it is accelerating?



\textbf{A.} 380N \\
\textbf{B.} 420N \\
\textbf{C.} 426N \\
\textbf{D.} 1200N \\

\textbf{Answer:} C \\
\textbf{Explanation:} By the gradient of the graph in the linear region; the terminal velocity is
[IMAGE:0]
; the acceleration in average is therefore
[IMAGE:1]
;
[IMAGE:2]
.

\hrule
\vspace{1em}


\noindent
\textbf{Q1472.} The ratio of radius of two solid spheres is 3:1, the ratio of density of these spheres is 1:2. Find the ratio of the mass of these two solid spheres.



\textbf{A.} 1:9 \\
\textbf{B.} 3:4 \\
\textbf{C.} 1:18 \\
\textbf{D.} 2:1 \\

\textbf{Answer:} E \\
\textbf{Explanation:} The volume ratio is 27:1, the mass ratio is therefore: 27:2

\hrule
\vspace{1em}


\noindent
\textbf{Q1473.} A car of mass 500 kg travels in a straight line along a horizontal road.
The car accelerates non-uniformly from rest for 5.0 seconds and then moves at constant speed, as shown in the distance–time graph.
What is the average resultant force acting on the car over the time for which it is accelerating?



\textbf{A.} 320N \\
\textbf{B.} 480N \\
\textbf{C.} 1000N \\
\textbf{D.} 1200N \\

\textbf{Answer:} C \\
\textbf{Explanation:} By the gradient of the graph in the linear region; the terminal velocity is
[IMAGE:0]
; the acceleration in average is therefore
[IMAGE:1]
;
[IMAGE:2]
.

\hrule
\vspace{1em}


\noindent
\textbf{Q1474.} The ratio of radius of two solid spheres is 3:2, the ratio of density of these spheres is 1:3. Find the ratio of the mass of these two solid spheres



\textbf{A.} 1:9 \\
\textbf{B.} 1:27 \\
\textbf{C.} 9:8 \\
\textbf{D.} 2:81 \\

\textbf{Answer:} C \\
\textbf{Explanation:} The volume ratio is 3:2, the mass ratio is therefore: 1:18

\hrule
\vspace{1em}


\noindent
\textbf{Q1475.} Which statement about the center of gravity is correct?



\textbf{A.} The center of gravity must be inside the object \\
\textbf{B.} The center of gravity is the equivalent point where gravity acts \\
\textbf{C.} Gravitational acceleration is always 9.8 m/s² on Earth \\
\textbf{D.} The center of gravity must be inside the object  and  The center of gravity is the equivalent point where gravity acts \\

\textbf{Answer:} B \\
\textbf{Explanation:} Incorrect. The center of gravity can be outside the object (e.g., a ring’s center of gravity is at its center).
Correct. The center of gravity is the equivalent point of gravitational action.
Incorrect. Gravitational acceleration varies with latitude and altitude

\hrule
\vspace{1em}


\noindent
\textbf{Q1476.} Which one of the following about nuclear stability is true?



\textbf{A.} Nuclei with an even number of protons and an even number of neutrons are always unstable \\
\textbf{B.} The stability of a nucleus is only determined by the number of protons \\
\textbf{C.} Magic numbers of protons and neutrons make a nucleus more stable \\
\textbf{D.} All light nuclei are unstable \\

\textbf{Answer:} C \\
\textbf{Explanation:} Nuclei with
odd
numbers of protons and neutrons (2, 8, 20, 28, 50, 82, 126) have closed shells and are more stable

\hrule
\vspace{1em}


\noindent
\textbf{Q1477.} Which statement about nuclear fusion is correct?



\textbf{A.} Nuclear fusion occurs at room temperature and pressure \\
\textbf{B.} The fuel for nuclear fusion is hard to obtain and store \\
\textbf{C.} The energy released in a nuclear fusion reaction is less than that in a nuclear fission reaction \\
\textbf{D.} Nuclear fusion is the process that powers the sun and other stars \\

\textbf{Answer:} D \\
\textbf{Explanation:} The sun and stars generate energy through nuclear fusion of hydrogen into helium

\hrule
\vspace{1em}


\noindent
\textbf{Q1478.} Which of the following about nuclear fission is true?



\textbf{A.} Nuclear fission can only be induced by neutrons \\
\textbf{B.} In a nuclear fission reaction,the total mass of the products is greater than the mass of the reactant \\
\textbf{C.} The energy released in a nuclear fission reaction is mainly in the form of kinetic energy of the fission fragments \\
\textbf{D.} All heavy nuclei are easy to fission \\

\textbf{Answer:} C \\
\textbf{Explanation:} The fission fragments have a lot of kinetic energy, which is the main form of energy released

\hrule
\vspace{1em}


\noindent
\textbf{Q1479.} Which statement about nuclear forces is correct?



\textbf{A.} The electromagnetic force is responsible for holding the nucleus together \\
\textbf{B.} The strong nuclear force acts only between protons \\
\textbf{C.} The range of the strong nuclear force is much larger than the range of the electromagnetic force \\
\textbf{D.} The weak nuclear force is responsible for beta - decay \\

\textbf{Answer:} D \\
\textbf{Explanation:} Beta - decay is a result of the weak nuclear force

\hrule
\vspace{1em}


\noindent
\textbf{Q1480.} Which one of the following about nuclear isotopes is true?



\textbf{A.} All isotopes of an element are stable \\
\textbf{B.} Isotopes of an element have the same chemical properties but different physical properties \\
\textbf{C.} The number of neutrons in an isotope determines its atomic number \\
\textbf{D.} Radioactive isotopes cannot be used in medical applications \\

\textbf{Answer:} B \\
\textbf{Explanation:} Isotopes have the same number of protons (same atomic number), so they have similar chemical properties

\hrule
\vspace{1em}


\noindent
\textbf{Q1481.} Which statement about nuclear energy production is correct?



\textbf{A.} Nuclear power plants use nuclear fusion to generate electricity \\
\textbf{B.} Breeder reactors are designed to produce less fissile material than they consume \\
\textbf{C.} The waste from nuclear power plants is not radioactive \\
\textbf{D.} Nuclear energy is a renewable energy source \\

\textbf{Answer:} E \\
\textbf{Explanation:} Control rods are used to control the rate of the nuclear chain reaction by absorbing neutrons

\hrule
\vspace{1em}


\noindent
\textbf{Q1482.} Which of the following about radioactive decay is true?



\textbf{A.} Radioactive decay follows a second - order reaction kinetics \\
\textbf{B.} The decay constant of a radioactive substance is independent of temperature and pressure \\
\textbf{C.} After one half - life,all the radioactive nuclei in a sample will have decayed \\
\textbf{D.} Gamma rays are emitted during alpha - decay \\

\textbf{Answer:} B \\
\textbf{Explanation:} The decay constant is a characteristic property of the radioactive substance and is not affected by external conditions

\hrule
\vspace{1em}


\noindent
\textbf{Q1483.} Which statement about nuclear structure is correct?



\textbf{A.} The nucleus of an atom contains only protons \\
\textbf{B.} The strong nuclear force is weaker than the electromagnetic force at short distances within the nucleus \\
\textbf{C.} The binding energy per nucleon of a nucleus is highest for medium - sized nuclei \\
\textbf{D.} The number of protons in a nucleus determines its chemical properties,but not its stability \\

\textbf{Answer:} C \\
\textbf{Explanation:} Medium - sized nuclei have the most stable structure, leading to the highest binding energy per nucleon

\hrule
\vspace{1em}


\noindent
\textbf{Q1484.} Which one of the following statements about nuclear reactions is true?



\textbf{A.} In a nuclear fusion reaction,heavy nuclei combine to form lighter nuclei \\
\textbf{B.} The energy released in a nuclear fission reaction comes from the conversion of mass into energy according to
[IMAGE:0] \\
\textbf{C.} A nuclear chain reaction can only occur in a nuclear bomb \\
\textbf{D.} All radioactive isotopes have the same half - life \\

\textbf{Answer:} B \\
\textbf{Explanation:} The famous Einstein's mass - energy equivalence formula explains the energy source in nuclear fission

\hrule
\vspace{1em}


\noindent
\textbf{Q1485.} Which one of the following statements about nuclear physics is wrong?



\textbf{A.} The process of emission of a gamma ray from a nucleus is called gamma decay \\
\textbf{B.} The half-life of a radioactive substance is the time taken for half of its nuclei to decay \\
\textbf{C.} The number of neutrons in a nucleus is its mass number minus its atomic number (proton number) \\
\textbf{D.} When a nucleus emits a beta particle,its proton number changes \\

\textbf{Answer:} D \\
\textbf{Explanation:} Number of particles is conserved since the only change is between neutrons and protons

\hrule
\vspace{1em}


\noindent
\textbf{Q1486.} A force F=10 N is applied to an object at an angle of 30
\circ 
above the horizontal. Resolve this force into its horizontal and vertical components and determine their magnitudes



\textbf{A.} 4
$𝑁$
; 6N \\
\textbf{B.} [IMAGE:0]
$𝑁$; 5N \\
\textbf{C.} 5
$𝑁$; 5N \\
\textbf{D.} [IMAGE:1]
$𝑁$
;
5N \\

\textbf{Answer:} D \\
\textbf{Explanation:} According to the decomposition of forces, the horizontal component of force is
[IMAGE:0]
$𝑁$
; the vertical component of force is 5N

\hrule
\vspace{1em}


\noindent
\textbf{Q1487.} On a horizontal ground, an object is acted upon by two forces. The first force F
1
has a magnitude of 4N and makes an angle of 30° with the horizontal direction; the second force F
2
has a magnitude of
[IMAGE:0]
$𝑁$
and points horizontally to the left. Find the magnitude of the resultant force of these two forces and the angle between it and the horizontal direction



\textbf{A.} 2
$𝑁$
; 0
° \\
\textbf{B.} [IMAGE:0]
$𝑁$
; 90
° \\
\textbf{C.} 2
$𝑁$
;
90
° \\
\textbf{D.} [IMAGE:1]
$𝑁$
; 60
° \\

\textbf{Answer:} C \\
\textbf{Explanation:} According to the synthesis and decomposition of forces, it can be known that F
2
cancels out the horizontal rightward component of F
1
, and the vertical component of F1 is 2N in magnitude

\hrule
\vspace{1em}


\noindent
\textbf{Q1488.} F
1
is with magnitude 30 N, pointing south-ward; F
2
is with magnitude 40 N, pointing west-ward; Find the magnitude of the component of the resulting force



\textbf{A.} 40
$𝑁$ \\
\textbf{B.} [IMAGE:0]
$𝑁$ \\
\textbf{C.} 50
$𝑁$ \\
\textbf{D.} [IMAGE:1]
$𝑁$ \\

\textbf{Answer:} C \\
\textbf{Explanation:} According to the Pythagorean theorem, the magnitude of the resultant force is 50N

\hrule
\vspace{1em}


\noindent
\textbf{Q1489.} In a Cartesian coordinate system, an object starts at the origin. It first moves along the vector
A
=(2m,3m), then moves along the vector
B
=(4m,5m). Find the magnitude of the object's final displacement vector
R
relative to the origin



\textbf{A.} 9m \\
\textbf{B.} 2m \\
\textbf{C.} 8m \\
\textbf{D.} 4m \\

\textbf{Answer:} E \\
\textbf{Explanation:} According to the rules of vector addition; the magnitude of the resultant Displacement is 10m

\hrule
\vspace{1em}


\noindent
\textbf{Q1490.} An object is acted upon by two concurrent forces. One force
F
1​​=(3N,0N) points in the positive x-direction, and the other force
F
2​​=(0N,4N) points in the positive y-direction. Find the magnitude of the resultant force
F



\textbf{A.} 5
N \\
\textbf{B.} 3N \\
\textbf{C.} 0N \\
\textbf{D.} 4
N \\

\textbf{Answer:} A \\
\textbf{Explanation:} According to the rules of vector addition; the magnitude of the resultant force is 5N

\hrule
\vspace{1em}


\noindent
\textbf{Q1491.} [IMAGE:0]
[IMAGE:1]
These are two vectors representing two forces as shown above, they both exert on one object with mass 1kg; Find the acceleration of the object



\textbf{A.} 4
$𝑚$
/
$𝑠$
2 \\
\textbf{B.} 0
$𝑚$
/
$𝑠$
2 \\
\textbf{C.} 2.5
$𝑚$
/
$𝑠$
2 \\
\textbf{D.} 2
$𝑚$
/
$𝑠$
2 \\

\textbf{Answer:} B \\
\textbf{Explanation:} The magnitude of the resultant force is 0N, by equation F=ma; m=1kg; acceleration is 0N

\hrule
\vspace{1em}


\noindent
\textbf{Q1492.} Resultant force and force component
[IMAGE:0]
is with magnitude
[IMAGE:1]
N, pointing northward;
[IMAGE:2]
is with magnitude
[IMAGE:3]
N, pointing east-ward; Find the component of the resulting force along the direction on the bearing of 60 degrees



\textbf{A.} 2N \\
\textbf{B.} -2N \\
\textbf{C.} [IMAGE:0] \\
\textbf{D.} 4N \\

\textbf{Answer:} C \\
\textbf{Explanation:} The resultant itself is not on the direction required but on the bearing of 30 degrees, so the answer is not
[IMAGE:0]
. And the answer is
[IMAGE:1]

\hrule
\vspace{1em}


\noindent
\textbf{Q1493.} Resultant force and force component
[IMAGE:0]
is with magnitude
[IMAGE:1]
N, pointing northward;
[IMAGE:2]
is with magnitude
[IMAGE:3]
N, pointing east-ward; Find the component of the resulting force along the direction on the bearing of 30 degrees



\textbf{A.} 2N \\
\textbf{B.} -2N \\
\textbf{C.} [IMAGE:0] \\
\textbf{D.} 4N \\

\textbf{Answer:} A \\
\textbf{Explanation:} The resultant itself is on the direction required, the magnitude is also:
[IMAGE:0]

\hrule
\vspace{1em}


\noindent
\textbf{Q1494.} Resultant force and force component
[IMAGE:0]
is with magnitude +10N, pointing northward;
[IMAGE:1]
is with magnitude +10N, pointing east-ward; Find the component of the resulting force along the direction on the bearing of 225 degrees.



\textbf{A.} [IMAGE:0] \\
\textbf{B.} [IMAGE:1] \\
\textbf{C.} [IMAGE:2] \\
\textbf{D.} [IMAGE:3] \\

\textbf{Answer:} D \\
\textbf{Explanation:} The resultant itself is on the direction required, the magnitude is also:
[IMAGE:0]

\hrule
\vspace{1em}


\noindent
\textbf{Q1495.} Vectoral addition
[IMAGE:0]
These are two vectors representing two forces, they both exert on one object with mass 2kg; Find the acceleration of the object



\textbf{A.} [IMAGE:0] \\
\textbf{B.} [IMAGE:1] \\
\textbf{C.} [IMAGE:2] \\
\textbf{D.} [IMAGE:3] \\

\textbf{Answer:} E \\
\textbf{Explanation:} The magnitude of the resultant force is
[IMAGE:0]
, by equation F=ma; acceleration is
[IMAGE:1]

\hrule
\vspace{1em}


\noindent
\textbf{Q1496.} Vectoral addition
[IMAGE:0]
These are two vectors representing two forces, they both exert on one object with mass 2kg; Find the acceleration of the object



\textbf{A.} [IMAGE:0] \\
\textbf{B.} [IMAGE:1] \\
\textbf{C.} [IMAGE:2] \\
\textbf{D.} [IMAGE:3] \\

\textbf{Answer:} B \\
\textbf{Explanation:} The magnitude of the resultant force is
[IMAGE:0]
, by equation F=ma; acceleration is
[IMAGE:1]

\hrule
\vspace{1em}


\noindent
\textbf{Q1497.} Vectoral addition:
[IMAGE:0]
These are two vectors representing two forces, they both exert on one object with mass 2kg; Find the acceleration of the object:



\textbf{A.} [IMAGE:0] \\
\textbf{B.} [IMAGE:1] \\
\textbf{C.} [IMAGE:2] \\
\textbf{D.} [IMAGE:3] \\

\textbf{Answer:} D \\
\textbf{Explanation:} The magnitude of the resultant force is
[IMAGE:0]
, by equation F=ma; acceleration is
[IMAGE:1]

\hrule
\vspace{1em}


\end{document}
